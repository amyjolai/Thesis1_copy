%% I may change the way this is done in a future version, 
%%  but given that some people needed it, if you need a different degree title 
%%  (e.g. Master of Science, Master in Science, Master of Arts, etc)
%%  uncomment the following 3 lines and set as appropriate (this *has* to be before \maketitle)
%% \makeatletter
%% \renewcommand {\@degree@string} {Master of Things}
%% \makeatother

\title{Computational Study of the Ageing Processes in Nitrocellulose}
\author{Amy J. Lai}
\department{Department of Chemistry}

\maketitle
\makedeclaration

%\cleardoublepage\phantomsection
\addcontentsline{toc}{chapter}{Abstract}
\begin{abstract} % 300 word limit
%My research is about stuff.
%It begins with a study of some stuff, and then some other stuff and things.
The dominant degradation pathways of \ac{NC} were investigated, using \ac{QM} methods to probe the primary denitration routes, followed by key secondary reactions.  %by looking at the explored degradation routes in alternative nitrate esters, and tested for \ac{NC} a truncated model for the wider polymer
The polymer structure was truncated in order to facilitate \ac{DFT} studies into the mechanistic details of denitration at individuate nitrate sites. 
Comparison of monomer, dimer and trimer units of the polymer using \ac{QTAIM} topology analysis of interaction sites, analysis of the \ac{ESP} and charges showed that the most suitable model for study of the decomposition reactions was \textbeta-glucopyranose monomer, bi-capped with methoxy groups. 
%This was determined via analysis of the \ac{QTAIM} \ac{BCP} between the capping groups and the wider monomer, when changing from hydroxyl and methoxy homogeneous or mixed groups at the C1 and C4 sites on the ring. 
The model was nitrated at the C2 position, to mimic the most stable nitration site \cite{Shukla2012a}.
%first site of nitration or last site of denitration. %REF
The primary thermolytic and hydrolytic denitration routes were explored using \ac{TS} searches and \ac{PES} scans. It was found that the thermolytic behaviour of the \ac{NC} denitration step matched the energy profile of other nitrate esters \cite{Tsyshevsky2013}. 
Protonation at the bridging oxygen site of the nitrate was found to be the most likely to lead to denitration. It was not possible to isolate a \ac{TS} for the hydrolytic reaction, though a number of coordination schemes were tested. 
Secondary processes following initial denitration were examined. Ethyl nitrate was used as a test system before extension to the monomer. Different reaction pathways for decomposition, with forward propagation of the evolved species to the reaction step, revealed that \ce{^{.}NO2} was the most likely cause for the the experimentally observed autocatalytic rate of degradation. 
\end{abstract}

%\addcontentsline{toc}{chapter}{Impact Statement}%%%%%%%%%%%%%%%%%%
%\begin{impact}
%This study examines the degradation properties of \ac{NC}, shedding light on the mechanistic details of decomposition that are yet currently either unknown, or are unclear with contradictory views in existing literature.  
%\ac{NC} is used in a massive variety of products, with utilisation in household, industrial, military and medicinal applications. 
%
%Improved understanding of the fundamental chemistry in the degradation of \ac{NC} could benefit in the following key impact areas: \\%Despite its long history, the majority of information on the subject is only based on ag
%
%\noindent\textbf{Knowledge \& collective benefit} %(for the common good) and expertise and education
%\begin{itemize}
%\item Broaden understanding and depth of knowledge in the area of \ac{NC} degradation, with the view to validate or reject conflicting schemes in literature. 
%%eludcidate for reserarch and commmon knowedlge
%%\noindent 
%\item Improve conservation procedures for existing and legacy \ac{NC} products, such as cinematographic film, artworks and historical munitions with better understanding of the handling procedures for these aged products. % \cite{Service1999}
%\end{itemize}
%\noindent\textbf{Environmental}
%%With increasing oil prices and forecasts of a future lack of availability, renewable non-petrochemical-based alternatives to materials synthesis could become more important. 
%
%%\item 
%\noindent Design next generation \ac{NC} products more environmentally friendly, in terms of durability and recycle-ability.
%
%%\item 
%\noindent Improve and clean up industrial processes for more environmentally benign production process.
%%It could mean that \ac{NC} 	\textendash which is not derived from petrochemicals \textendash\  can help fight the plastic / waste disposal problem. %It's non-toxic to humans (CHECK) and so in its degradation, could also help fight the micro-plastics ingestion problem, in the long term. (basically, polysaccharides, like PLA etc, could assist in the long term transition from petrochemical plastics to biodegradability. Though \ac{NC} isn't necessary biodegradable without specific microbes found in the lab yet - better understanding of its breakdown could facility an easier process for this.)
%
%%\item 
%\noindent Design appropriate \ac{NC} disposal methods, other than the existing harsh chemical or incineration treatment, with opportunity to feed back into other processes, such as the production of fertiliser. %insert into/  modify industrial process to re-formulate into other products, such as fertilisers etc. with leftover \ac{NC} etc with opportunity to streamline to more environmentally friendly methods,
%
%
%\noindent\textbf{Safety}
%
%%\item 
%\noindent Guidelines on safety and usage could be refined based on detailed knowledge of the degradation reactions, instead of current guidelines based on more crude, aggregate experimental observations.
%%with better understanding of the actual reactions that take place, instead of a broad-brush style guidelines based on historical \/ less precise experimental observations.
%%Current decomposition mechanisms are speculated based on aggregate experimental measurements giving a crude understanding of the degradation patterns in the material.
%
%\noindent\textbf{Industrial \& Commercial}
%
%%\item 
%\noindent More adaptability in the production of \ac{NC} and its formulation into products, in the case that the variety of source cellulose feedstock changes due to supply chain issues. 
%
%%\item 
%\noindent Streamline industrial processes to specific reactivity requirements, based on detailed mechanistic considerations, leading to cost saving.%  if we know what can be trimmed - commerical benefits by cutting waste, optimising work flows and tailoring processes to exact instead of wide-margin .
%
%\noindent\textbf{Innovation}
%
%%\item 
%\noindent Create new products, not limited by previous assumptions about the shelf life and reactivity. % etc (obviously this is a big generalisation. IT might be that the current industrial standards are ok / not too far off. But at least this might mean better/ more tailored / cheaper stabiliser / plasticiser formulations can be used, or things that are more readily available and more options, in case certain things become unavailable.) 
%
%%\item 
%\noindent Facilitate the design of products for applications that were previously not known that the material was suitable for.% we didn't know it was suitable for, before
%
%%\end{itemize}
%\end{impact}

%%\cleardoublepage\phantomsection
%\addcontentsline{toc}{chapter}{Acknowledgements}
%\begin{acknowledgements}
%The completion of this body of work %the hours behind it, emotional rollercoasters etc
% could not have happened if it were not for the people who dedicated time guiding it to completion. \\
%Firstly, thank you to Professor Nora de Leeuw for the opportunity to undertake this study, and for granting me the chance to visit conferences and research exchanges.
%
%%, and for taking time to check on me periodically, even with your busy schedule.
%A huge thanks to Professor Graham Worth for taking on a precarious t all his time and dedication spent %on guiding a lost PhD student
%Dr David Santos Carballal %for driving the study on the phase diagram stuff
%Professor David Scanlon for his assistance in navigating a difficult % taking me on in group meetings. Always asking how I am when we bump in the corridor. Always being real. Looking out for PhD students, and actually caring.
%Professor Ivan Parkin %again, for just caring 
%Professor Jim Anderson for being interested, taking the time to meet with me and always responding to my %pleas for help
%questions.
%Dr Zhimei Du
%The office crew - for keeping me alive with boba and love. 
% Special shoutout to the people who read chapters - Emilia, James, Jordan for reading
% Gabiee for always bein there and looking out for me, all these years. Been with me the whole way. 
% And Filipe too, for letting me have Gabie time, and opening the doors to their home to me too. 
%Both for looking after me so well!
%Dani - for doing what he could, for the encouragement, and being patient and understanding. 
%Lisa Patrick - being there during my insanity, offering to read
%Paul and Hafsa too
%Sarah
%Jordan, for housing me, encouraging me, being there for me
% Jona
% Olivia and Fred for letting me in their home. Fred for the banging playlists
% \newpage 
% Now that everyone else has given up on reading, this is to acknowledge the real MVP's of this PhD, without whom my days would have greyer and dryer (and by that I mean dry of bubble tea). \\
% The crew at 205, of whom I will be the last to graduate:\\
% Dr Abdul Rashidi + open mic\\
% Dr Lisa Richards + south africa trip\\
% Dr Qamreen Parker + boba crew\\
% Dr Emilia Olsson + reading\\
% Dr Simon Austin + cheese\\
% %Dr James Pegg + dinosaur memes\\
% Kit & Ceridwen
%Stephen in student records - for turning my PhD extension application around, so quickly. 
%
%We acknowledge the support of the Supercomputing Wales project, which is part-funded by the European Regional Development Fund (ERDF) via Welsh Government.
%\end{acknowledgements}


\setcounter{tocdepth}{3} 
% Setting this higher means you get contents entries for more minor section headers.

\tableofcontents

%\cleardoublepage\phantomsection
\addcontentsline{toc}{chapter}{List of Figures}
\listoffigures

%\cleardoublepage\phantomsection
\addcontentsline{toc}{chapter}{List of Tables}
\listoftables

