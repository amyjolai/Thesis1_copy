% I may change the way this is done in a future version, 
%  but given that some people needed it, if you need a different degree title 
%  (e.g. Master of Science, Master in Science, Master of Arts, etc)
%  uncomment the following 3 lines and set as appropriate (this *has* to be before \maketitle)
% \makeatletter
% \renewcommand {\@degree@string} {Master of Things}
% \makeatother

\title{Computational Study of the Ageing Processes in Nitrocellulose}
\author{Amy J. Lai}
\department{Department of Chemistry}

\maketitle
\makedeclaration

%%\cleardoublepage\phantomsection
%\addcontentsline{toc}{chapter}{Abstract}
%\begin{abstract} % 300 word limit
%My research is about stuff.
%It begins with a study of some stuff, and then some other stuff and things.
%There is a 300-word limit on your abstract.
%\end{abstract}

\addcontentsline{toc}{chapter}{Impact Statement}
\begin{impact} % 300 word limit
% "With increasing oil prices and forecasts of a future lack of availability, renewable non-petrochemical-based alternatives to materials synthesis could become more important. Polysaccharides are the products of a natural carbon-capture process, photosynthesis, followed by further biosynthetic modifications. Some are produced on a very large scale in nature, and some have industrial relevance with, for example, materials and food applications, either in their native or chemically modified forms. This review covers methods for the chemical modification of polysaccharides. The topic of general modification of polysaccharides has been reviewed previously [1], and several more specific reviews are referenced later. In this review, I have limited myself to discussing the synthesis of modifications whereby the polymeric chain remains intact—or at least while degradation may take place to some extent, the products are still polysaccharides. The conversion of polysaccharides into small molecules has been reviewed elsewhere [2, 3] and is not covered here.

% Chemical modification can change the character of the polysaccharides, for example, rendering them hydrophobic [4]. Some such processes, such as the formation of cellulose esters (including nitrocellulose, celluloid, cellulose acetate), are very well known and have been carried out at an industrial level for more than a hundred years. The object of this review is not to cover such well-known processes in detail but rather to describe published results of current research and the state-of-the-art in polysaccharide derivatisation. Neither have I gone into details about the possible applications of the products but have focussed on aspects related to reactivity and chemical structure. The modifications are presented classified by reaction type."\\


\textbf{Some of the following will belong in the motivation section, rather than the IMPACT section.}\\

- \ac{NC} has been used in many products and its important to know how they are ageing, especially for those with a propellant application.

- create new products, not limited by previous assumptions about the shelf life / reactivity etc (obviously this is a big generalisation. IT might be that the current industrial standards are ok / not too far off. But at least this might mean better/ more tailored / cheaper stabiliser / plasticiser formulations can be used, or things that are more readily available and more options, in case certain things become unavailable.) 

- This means that next generation products could be more environmentally friendly / industrial process is more environmentally friendly, easier to dispose of, or more recyclable.
It could mean that \ac{NC}, which is not derived from petrochemicals / plastics, can help fight the plastic / waste disposal problem. It's non-toxic to humans (CHECK) and so in its degradation, could also help fight the micro-plastics ingestion problem, in the long term. (basically, polysaccharides, like PLA etc, could assist in the long term transition from petrochemical plastics to biodegradability. Though \ac{NC} isn't necessary biodegradable without specific microbes found in the lab yet - better understanding of its breakdown could facility an easier process for this.)

- current industrial guidelines on safety and usage could be refined with better understanding of the actual reactions that take place, instead of a broad-brush style guidelines based on historical \/ less precise experimental observations

- More agility in the production of \ac{NC}, when the source cellulose feedstock change. 

- More appropriate \ac{NC} disposal methods, with opportunity to streamline to more environmentally friendly methods, or insert into/  modify industrial process to re-formulate into other products, such as fertilisers etc. with leftover \ac{NC} etc


- table of the property/sector vs benefit. (See Tab. 1) This may or may not be relevant. ASK MAX about this one closer to the time. 

\begin{table}[h!]
%  \begin{center}
    \caption{The impact of improving understanding of \ac{NC} ageing processes}
    \label{tab:table1}
%    \begin{tabular}{l|r} % <-- Alignments: 1st column left, 2nd middle and 3rd right, with vertical lines in between
    \begin{tabular}{|p{5cm}|p{8cm}|}
    \hline
      \textbf{\ac{NC} Application} & \textbf{Benefit}\\
      \hline
      Medicine: Spray on bandages in health & Better predict lifetime of the product, and shed light on what stabilisers to use to extend usability time\\
      \hline
      Commercial \& Defence: Ejection propellant in aircraft & Improve formulations, increase lifetime and heat resistivity of propellant mixture, decrease costs. \\
      \hline
      3 & 23.113231 \\
    \hline
    \end{tabular}
%  \end{center}
\end{table}

In this study I do not present a complete solution to the myriad interlinked degradation reactions that are accountable for the decomposition behaviour of \ac{NC}. Rather, this work elucidates the initial steps on the path towards full understanding of the ageing processes of \ac{NC}

There is a 500-word limit on your impact statement.
\end{impact}

%\cleardoublepage\phantomsection
\addcontentsline{toc}{chapter}{Acknowledgements}
\begin{acknowledgements}
The completion of this body of work %the hours behind it, emotional rollercoasters etc
 could not have happened if it were not for the people who dedicated time guiding it to completion. \\
Firstly, thank you to Professor Nora de Leeuw for the opportunity to undertake this study, and for granting me the chance to visit conferences and research exchanges.

%, and for taking time to check on me periodically, even with your busy schedule.
A huge thanks to Professor Graham Worth for taking on a precarious t all his time and dedication spent %on guiding a lost PhD student
Dr David Santos Carballal %for driving the study on the phase diagram stuff
Professor David Scanlon for his assistance in navigating a difficult % taking me on in group meetings. Always asking how I am when we bump in the corridor. Always being real. Looking out for PhD students, and actually caring.
Professor Ivan Parkin %again, for just caring 
Professor Jim Anderson for being interested, taking the time to meet with me and always responding to my %pleas for help
questions.
Dr Zhimei Du

 
% \newpage 
% Now that everyone else has given up on reading, this is to acknowledge the real MVP's of this PhD, without whom my days would have greyer and dryer (and by that I mean dry of bubble tea). \\
% The crew at 205, of whom I will be the last to graduate:\\
% Dr Abdul Rashidi\\
% Dr Lisa Richards\\
% Dr Qamreen Parker\\
% Dr Emilia Olsson\\
% Dr Simon Austin\\
% %Dr James Pegg

\end{acknowledgements}


\setcounter{tocdepth}{3} 
% Setting this higher means you get contents entries for
%  more minor section headers.

\tableofcontents

%\cleardoublepage\phantomsection
\addcontentsline{toc}{chapter}{List of Figures}
\listoffigures

%\cleardoublepage\phantomsection
\addcontentsline{toc}{chapter}{List of Tables}
\listoftables

