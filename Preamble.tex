%%% I may change the way this is done in a future version, 
%%%  but given that some people needed it, if you need a different degree title 
%%%  (e.g. Master of Science, Master in Science, Master of Arts, etc)
%%%  uncomment the following 3 lines and set as appropriate (this *has* to be before \maketitle)
%%% \makeatletter
%%% \renewcommand {\@degree@string} {Master of Things}
%%% \makeatother
%
%\title{Computational Study of the Ageing Processes in Nitrocellulose}
\title{Determining the dominant degradation mechanisms in Nitrocellulose \added{[CORRECTIONS]}}
\author{Amy J. Lai}
\department{Department of Chemistry}

\maketitle
\makedeclaration

%\cleardoublepage\phantomsection
% Apparently, no references in abstract
\addcontentsline{toc}{chapter}{Abstract}
\begin{abstract} % 300 word limit
\added{Nitrocellulose (\acs{NC}) is the base component for many modern day propellants and explosives, as well as for everyday items such as printing inks, paint and lacquer coatings. Despite its early beginnings as the first man-made plastic, % winning a medal at the third World's Fair in London in 1862, 
the decomposition pathways from the bulk material to the products observed from its ambient ageing are still not fully understood. 
Knowledge of these processes are of critical importance when considering the conservation of \acs{NC} artefacts, refinement of product formulations, predictions of shelf life and safety improvements. }

In this study, the dominant degradation pathways of \acs{NC} were investigated using \ac{QM} methods to probe \added{the mechanisms leading to the initial cleavage of nitrate groups from the cellulosic backbone.} 
\added{The \acs{NC} structure was truncated from a polymer chain to monomer, dimer and trimer units}. Density functional theory methods (\acs{DFT}) were used to study the mechanistic detail at individual nitrate sites. %s of denitration
Comparison of differently sized units using the \ac{QTAIM}, % topology analysis%of interaction sites
analysis of the \ac{ESP} surface and partial charges showed that the most suitable approximation for study of the decomposition reactions was the \textbeta-glucopyranose monomer, bi-capped with methoxy groups. 
%This was determined via analysis of the \ac{QTAIM} \ac{BCP} between the capping groups and the wider monomer, when changing from hydroxyl and methoxy homogeneous or mixed groups at the C1 and C4 sites on the ring. 
%The model was nitrated at the C2 position, to mimic the most stable nitration site \cite{Shukla2012a}.
%first site of nitration or last site of denitration. %REF
% you can tighten up this language - more specific pls

The primary thermolytic and hydrolytic denitration routes were explored using \ac{TS} searches and \ac{PES} scans. 
It was found that the thermolytic behaviour of the \acs{NC} denitration step matched that of a well studied nitrate ester, \acf{PETN}. %\cite{Tsyshevsky2013}. 
The hydrolytic scheme for nitrate cleavage was studied, finding that protonation at the bridging oxygen site was the most likely to lead to denitration. It was not possible to isolate a \ac{TS} for the hydrolytic reaction, though a number of % water coordination / complexation? Or possible TS geometries? of the nitrate was found to be
coordination schemes were tested. 

Key secondary processes beyond nitrate cleavage were examined to determine the fate of nitrogen in the system and the cause of the transition from a first order reaction rate to autocatalytic decomposition. %, after an incubation period.  %by looking at the explored degradation routes in alternative nitrate esters, and tested for \acs{NC} a truncated model for the wider polymer
%Secondary processes following initial denitration were examined. 
The energies of reactions in three different decomposition schemes proposed in literature were compared. 
Ethyl nitrate was used as a test system before extension to the \acs{NC} monomer. 
\added{New reaction pathways for decomposition were constructed using the reactions posed in the literature studies. %, with forward propagation of the evolved species to the next reaction step. 
The new schemes revealed that \rad\ce{NO2} was the most likely cause for the experimentally observed autocatalytic rate of degradation. }%, although further study into the relative rates of reaction are required. }
\end{abstract}
%
\addcontentsline{toc}{chapter}{Impact Statement}%%%%%%%%%%%%%%%%%%
\begin{impact}
This study examines the degradation properties of \acs{NC}, shedding light on the mechanistic details of decomposition that are yet currently either unknown, or are unclear %with contradictory views 
in existing literature.  
\acs{NC} is used in an extensive variety of products, with utilisation in household, industrial, military and medicinal applications. 
Improved understanding of the fundamental chemistry in the degradation of \acs{NC} could benefit the following key impact areas: \\%Despite its long history, the majority of information on the subject is only based on ag

\noindent\textbf{Knowledge \& collective benefit} %(for the common good) and expertise and education
\begin{itemize}
\item Broaden understanding and depth of knowledge in the area of \acs{NC} degradation, with the view to validate or deconflict competing schemes in literature. 
%eludcidate for reserarch and commmon knowedlge clarify
%\noindent 
\item Improve conservation practices for existing and legacy \acs{NC} products of cultural value, such as cinematographic film, artworks and historical munitions with better information on environmental conditions most impacting the decomposition pathways. 
%most damaging 
%with better understanding of the preservation and handling procedures for these aged products. % \cite{Service1999}
\end{itemize}
\noindent\textbf{Environmental}
%With increasing oil prices and forecasts of a future lack of availability, renewable non-petrochemical-based alternatives to materials synthesis could become more important. 

\begin{itemize}
%\noindent 
\item Inform %more effective 
improved \acs{NC} disposal methods, avoiding existing harsh chemical and incineration treatments, with opportunity to feed back into other manufacture streams such as the production of fertiliser. %insert into/  modify industrial process to re-formulate into other products, such as fertilisers etc. with leftover \acs{NC} etc with opportunity to streamline to more environmentally friendly methods,

%\noindent 
\item Facilitate the design of next generation \acs{NC} products with lower environmental impact, in terms of durability and recycle-ability, with cleaner %production
industrial processes. 
% in a more environmentally friendly way

%\item 
%\noindent Improve and clean up industrial processes for more environmentally benign production process.
%It could mean that \acs{NC} 	\textendash which is not derived from petrochemicals \textendash\  can help fight the plastic / waste disposal problem. %It's non-toxic to humans (CHECK) and so in its degradation, could also help fight the micro-plastics ingestion problem, in the long term. (basically, polysaccharides, like PLA etc, could assist in the long term transition from petrochemical plastics to biodegradability. Though \acs{NC} isn't necessary biodegradable without specific microbes found in the lab yet - better understanding of its breakdown could facility an easier process for this.)

\end{itemize}
\noindent\textbf{Safety}
\begin{itemize}
\item Refine guidelines on safety and usage, based on detailed knowledge of the degradation mechanisms, improving upon current practices based on aggregate experimental observations. %cruder
%with better understanding of the actual reactions that take place, instead of a broad-brush style guidelines based on historical \/ less precise experimental observations.
%Current decomposition mechanisms are speculated based on aggregate experimental measurements giving a crude understanding of the degradation patterns in the material.
\end{itemize}

\noindent\textbf{Industrial \& commercial}
\begin{itemize}
\item Improve versatility and adaptability in \acs{NC} production and its product formulations, in view of varying cellulose feedstock sources due to supply chain fluctuation. 

\item Streamline industrial processes to specific reactivity requirements, based on detailed mechanistic considerations regarding shelf life and interaction with other components, leading to cost saving and better performance of final product.%  if we know what can be trimmed - commerical benefits by cutting waste, optimising work flows and tailoring processes to exact instead of wide-margin.
\end{itemize}

\noindent\textbf{Innovation}
\begin{itemize}
\item Allow the design of new \acs{NC} products, not limited by 
%previous assumptions 
crude understanding about the material shelf life and reactivity, leading to novel applications. % etc (obviously this is a big generalisation. IT might be that the current industrial standards are ok / not too far off. But at least this might mean better/ more tailored / cheaper stabiliser / plasticiser formulations can be used, or things that are more readily available and more options, in case certain things become unavailable.) 

%\item 
%\item Facilitate the design of products for applications that were previously not known that the material was suitable for.% we didn't know it was suitable for, before
\end{itemize}
%\end{itemize}
\end{impact}

%\cleardoublepage\phantomsection
\addcontentsline{toc}{chapter}{Acknowledgements}
\begin{acknowledgements}

Firstly, thank you to Prof. Nora de Leeuw for the opportunity to undertake this course of study, and for the encouragement and support in attending the conferences and research exchanges that enriched my work. 

\textit{Enormous} thanks goes to Prof. Graham Worth without whose support this thesis would not have come to be. 
I can't express enough my gratitude for the huge patience, guidance and empathy you've shown during this process. 

Further thanks goes to Prof. David Scanlon for adopting me into group meetings and for generally being cool, and to both you and Prof. Ivan Parkin for your guidance. 
Thanks to Prof. Jim Anderson for offering your organic chemistry wisdom when curly arrows got the better of me. 
Thank you to Dr Zhimei Du for being ever-present and ready to assist with any difficulties I had throughout my time at UCL. 

A big thank you to Dr Cornie van Sittert and the group for hosting me in Potch, and to Tanie Hestelle, Kerneels and Petra for making me feel absolutely welcome. 
Kevin, Louise, Dani, Daniel and Tjaart - we had the best chemistry department road trip.
Thank you to Dr David Santos-Carballal for always being generous with scientific advice, and always cheerful, and to Dr Anthony Nash for offering guidance in the early years, which set the scene for the final product. 


Thank you to my friend Gabie, who has been there from the \textit{beginning}, and always made sure I was still tied to planet Earth, especially when things were starting to look bit hairy. 
More thanks are owed to Lisa Patrick, Sharon, Matt \& Chris from TYC journal club, Paul \& Hafsa, Sarah French, Jordan, Will \& Valerie (my ChemLads crew), Miriam Price, Anne S{\"u}ffel \& Danni CM UCL Barbell circa. 2017, Jona (and Olivia \& Freddie and family), my flatmates Sev, Joe \& Naomi, and the Cabinet Office crew, for being great people who had a positive impact on this whole experience and helped me get through this. There are many more still to thank - now with all this free time I'll be able to thank you all in person. 

And finally... a HUGE shoutout to the Boba+Cookie crew. You guys the real MVPs. Without you and our shared post-lunch diabetic comas I wouldn't have made it. Qam, we will finally go to Bang Bang, Emilia, we will all get the train and visit you once it's allowed again, and Abdul, we're waiting for your EP release in Japan - now that it's immortalised in my thesis it \textit{has} to happen. Thank you all for the pep talks, good company and for keeping the fire going. Last one out - I'll catch you on the other side. \#WeMadeIt 


%The completion of this body of work %the hours behind it, emotional rollercoasters etc
% could not have happened if it were not for the people who dedicated time guiding it to completion. \\
%Firstly, thank you to Professor Nora de Leeuw for the opportunity to undertake this study, and for granting me the chance to visit conferences and research exchanges.
%
%, and for taking time to check on me periodically, even with your busy schedule.
%A huge thanks to Professor Graham Worth for taking on a precarious t all his time and dedication spent %on guiding a lost PhD student
%Dr David Santos Carballal %for driving the study on the phase diagram stuff
%Professor David Scanlon for his assistance in navigating a difficult % taking me on in group meetings. Always asking how I am when we bump in the corridor. Always being real. Looking out for PhD students, and actually caring.
%Professor Ivan Parkin %again, for just caring 
%Professor Jim Anderson for being interested, taking the time to meet with me and always responding to my %pleas for help
%questions.
%Dr Zhimei Du
%The office crew - for keeping me alive with boba and love. 
% Special shoutout to the people who read chapters - Emilia, James, Jordan for reading
% Gabiee for always bein there and looking out for me, all these years. Been with me the whole way. 
% And Filipe too, for letting me have Gabie time, and opening the doors to their home to me too. 
%Both for looking after me so well!
%Lisa Patrick - being there during my insanity, offering to read
%Paul and Hafsa too
%Sarah
%Jordan, for housing me, encouraging me, being there for me
% Jona
% Olivia and Fred for letting me in their home. Fred for the banging playlists
% \newpage 
% Now that everyone else has given up on reading, this is to acknowledge the real MVP's of this PhD, without whom my days would have greyer and dryer (and by that I mean dry of bubble tea). \\
% The crew at 205, of whom I will be the last to graduate:\\
% Dr Abdul Rashidi + open mic\\
% Dr Lisa Richards + south africa trip\\
% Dr Qamreen Parker + boba crew\\
% Dr Emilia Olsson + reading\\
% Dr Simon Austin + cheese\\
% %Dr James Pegg + dinosaur memes\\
% Kit & Ceridwen
%Stephen in student records - for turning my PhD extension application around, so quickly. 
%
%We acknowledge the support of the Supercomputing Wales project, which is part-funded by the European Regional Development Fund (ERDF) via Welsh Government.
\end{acknowledgements}


\setcounter{tocdepth}{3} 
% Setting this higher means you get contents entries for more minor section headers.

\tableofcontents

%\cleardoublepage\phantomsection
\addcontentsline{toc}{chapter}{List of Figures}
\listoffigures

%\cleardoublepage\phantomsection
\addcontentsline{toc}{chapter}{List of Tables}
\listoftables

