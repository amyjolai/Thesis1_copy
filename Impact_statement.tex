\addcontentsline{toc}{chapter}{Impact Statement}
\begin{impact} 

This study examines the degradation properties of \ac{NC}, shedding light on the mechanistic details of decomposition that are yet currently either unknown, or are unclear, at times due to contradictory views in existing literature.  
\ac{NC} is used in a massive variety of products, with utilisation in household, industrial, military and medicinal applications. Improved understanding of the fundamental chemistry in the degradtion of \ac{NC} could benefit in the following key impact areas: %Despite its long history, the majority of information on the subject is only based on ag

\begin{itemize}

\textbf{Knowledge \& common benefit} %(for the common good) and expertise and education

%eludcidate for reserarch and commmon knowedlge
\item Improve conservation procedures for existing and legacy \ac{NC} products, such as cinematographic film, artworks, with better understanding of the handling procedures for these aged products \cite{Service1999}

\textbf{Environmental}
%With increasing oil prices and forecasts of a future lack of availability, renewable non-petrochemical-based alternatives to materials synthesis could become more important. 

\item Design next generation \ac{NC} products more environmentally friendly, in terms of durability and recycle-ability.

\item Improve and clean up industrial processes for more environmentally benign production process.
%It could mean that \ac{NC} 	\textendash which is not derived from petrochemicals \textendash\  can help fight the plastic / waste disposal problem. %It's non-toxic to humans (CHECK) and so in its degradation, could also help fight the micro-plastics ingestion problem, in the long term. (basically, polysaccharides, like PLA etc, could assist in the long term transition from petrochemical plastics to biodegradability. Though \ac{NC} isn't necessary biodegradable without specific microbes found in the lab yet - better understanding of its breakdown could facility an easier process for this.)

\item Design appropriate \ac{NC} disposal methods, other than the existing harsh chemical or incineration treatment, with opportunity to feed back into other processes, such as the production of fertiliser. %insert into/  modify industrial process to re-formulate into other products, such as fertilisers etc. with leftover \ac{NC} etc with opportunity to streamline to more environmentally friendly methods,


\textbf{Safety}

\item Guidelines on safety and usage could be refined based on detailed knowledge of the degradation reactions, instead of current guidelines based on more crude, aggregate experimental observations.
%with better understanding of the actual reactions that take place, instead of a broad-brush style guidelines based on historical \/ less precise experimental observations.
%Current decomposition mechanisms are speculated based on aggregate experimental measurements giving a crude understanding of the degradation patterns in the material.

\textbf{Industrial \& Commercial}

\item More adaptability in the production of \ac{NC} and its formulation into products, in the case that the variety of source cellulose feedstock changes due to supply chain issues. 

\item Streamline industrial processes to specific reactivity requirements, based on detailed mechanistic considerations, leading to cost saving.%  if we know what can be trimmed - commerical benefits by cutting waste, optimising work flows and tailoring processes to exact instead of wide-margin .

\textbf{Innovation}

\item Create new products, not limited by previous assumptions about the shelf life and reactivity. % etc (obviously this is a big generalisation. IT might be that the current industrial standards are ok / not too far off. But at least this might mean better/ more tailored / cheaper stabiliser / plasticiser formulations can be used, or things that are more readily available and more options, in case certain things become unavailable.) 

\item Facilitate the design of products for applications that were previously not known that the material was suitable for.% we didn't know it was suitable for, before

\end{itemize}

% "With increasing oil prices and forecasts of a future lack of availability, renewable non-petrochemical-based alternatives to materials synthesis could become more important. Polysaccharides are the products of a natural carbon-capture process, photosynthesis, followed by further biosynthetic modifications. Some are produced on a very large scale in nature, and some have industrial relevance with, for example, materials and food applications, either in their native or chemically modified forms. This review covers methods for the chemical modification of polysaccharides. The topic of general modification of polysaccharides has been reviewed previously [1], and several more specific reviews are referenced later. In this review, I have limited myself to discussing the synthesis of modifications whereby the polymeric chain remains intact—or at least while degradation may take place to some extent, the products are still polysaccharides. The conversion of polysaccharides into small molecules has been reviewed elsewhere [2, 3] and is not covered here.

% Chemical modification can change the character of the polysaccharides, for example, rendering them hydrophobic [4]. Some such processes, such as the formation of cellulose esters (including nitrocellulose, celluloid, cellulose acetate), are very well known and have been carried out at an industrial level for more than a hundred years. The object of this review is not to cover such well-known processes in detail but rather to describe published results of current research and the state-of-the-art in polysaccharide derivatisation. Neither have I gone into details about the possible applications of the products but have focussed on aspects related to reactivity and chemical structure. The modifications are presented classified by reaction type."\\

%%Table of the property/sector vs benefit. (See Tab. 1) This may or may not be relevant. Ask MAX about this one closer to the time. 
%
%\begin{table}[htp]
%%  \begin{center}
%    \caption{The impact of improving understanding of \ac{NC} ageing processes}
%    \label{tab:table1}
%%    \begin{tabular}{l|r} % <-- Alignments: 1st column left, 2nd middle and 3rd right, with vertical lines in between
%    \begin{tabular}{|p{5cm}|p{8cm}|}
%    \hline
%      \textbf{\ac{NC} Application} & \textbf{Benefit}\\
%      \hline
%      Medicine: Spray on bandages in health & Better predict lifetime of the product, and shed light on what stabilisers to use to extend usability time\\
%      \hline
%      Commercial \& Defence: Ejection propellant in aircraft & Improve formulations, increase lifetime and heat resistivity of propellant mixture, decrease costs. \\
%      \hline
%  %    3 & 23.113231 \\
%%    \hline
%    \end{tabular}
%%  \end{center}
%\end{table}

%In this study I do not present a complete solution to the myriad interlinked degradation reactions that are accountable for the decomposition behaviour of \ac{NC}. Rather, this work elucidates the initial steps on the path towards full understanding of the ageing processes of \ac{NC}
