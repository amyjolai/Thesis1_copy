\chapter{Diffusion of hydrolysis products}
\label{chapterlabel5}

\section{Introduction}
\ac{NC}  of 12.5\% nitrogen and below is widely used in industry "for many paints, lacquers, varnishes, packaging films, printing inks, adhesives and leather and fabric finishes". %REF hsg135 - Storage and handling of industrial nitrocellulose.pdf
\ac{NC} formulations above 12.6\% nitrogen by mass are considered Class 1 explosives. 

In this chapter, the diffusion of aforementioned decomposition products through \ac{NC} matrices with maximum percentage nitration of 14.4\%, and 12.6\% nitration will be examined. This is in order to compare the circulation of the species released during decomposition in two %industrailly relevant
different grades of \ac{NC}. Though the maximum nitration level is not attainable during the standard \ac{NC} synthesis procedures,
%(involving nitric and sulfuric acid mixtures, though it can almost be achieved by other, more intensive means not industrially viable - check literature on this one. Tbh it's too unstable for industrial applications anyways)
 %REF the 13.6% grade, if not mentioned earlier
 % This is waffle. Please fix. 
 and thus not used industrially, understanding of the diffusion processes in the highest level of nitration will be pertinent to the ageing behaviour associated with \ac{NC} of nitration levels above that of 12.6\% % the stuff used in propellants, etc etc. 

\section{Computational details}
\section{Fully nitrated NC}
\subsection{Diffusion of \ce{NO.}}
%Or more NO radical tbh
\subsection{Diffusion of \ce{N2O}}
%split into H+ and NO3-
\subsection{Diffusion in the presence of water}
Realistically \ac{NC} will be stored in water. 25\% water 
%REF http://www.inchem.org/documents/icsc/icsc/eics1560.htm 
% and https://cameochemicals.noaa.gov/chemical/19286 and https://us.vwr.com/assetsvc/asset/en_US/id/17987600/contents

% \section{NC of 13.5\% nitrogen}
% \subsection{Diffusion of \ce{NO2+}}
% Or more NO radical tbh
%Apart from fully nitrated, these will have a mixture of all nitration states

% \subsection{Diffusion of \ce{NO3-}}
% split into H+ and NO3-
% \subsection{Diffusion in the presence of water}

\section{NC of 12.5\% nitrogen}
\subsection{Diffusion of \ce{NO.}}
Or more NO radical tbh
If we do do NO2+ we want to see it interact with water, more than anything.
As for the radical, cite Lisa's work for the fully nitrated. You can still do the radical with the less that fully nitrated. Also could repeat what Lisa did, since she may have included plasticiser and you don't.
\subsection{Diffusion of \ce{N2O}}
split into H+ and NO3-
\subsection{Diffusion in the presence of water}

\section{Summary}