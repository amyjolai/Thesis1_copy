\chapter{Theory and Implementation}
\label{chapterlabel2}
\graphicspath{ {./Theory_pics/} }
%References:
% Jensen
% Cramer
% Leach
%
%The theoretical basis for methods used in this thesis and more widely in computational chemistry is described in this chapter.
% include a physical constants page, with the numbers and roundings I used

% The rate at which you move through the topics is probably too slow. Can skip bits. They'll get it (although I'm not sure I will.)
\section{Electronic structure methods}
%Intro to the history spiel. What it is, where did it come from. 
%Wave particle duality. 
% Dodgy first sentence, change it. 
Electronic structure methods apply the principles of quantum mechanics to the evaluation of electron position and movement, thereby allowing chemists to derive the properties and interactions of molecules. 
Despite the long history of research and use of \ac{NC} in industry, experimental analysis has failed to distinguish the fine mechanistic details of its decomposition. This is partly owed to the variation arising from biodiverse \ac{NC} source materials, combined with the complexity due to the interplay of many different %myriad
and simultaneous degradation interactions. Electronic structure methods provide a means to %avenue for exploration/ opportunity 
untangle the individual facets of decomposition. %and study each in . 

At the most fundamental level, the wave function ($\Psi$) holds the description of a quantum system. %Need a linking sentence here - What is the wave function, and why do we need to look at the probablility densiy instead?
In a non-relativistic system, % So what happens in a relativistic system?
the probability of a particle possessing a given momentum, or residing in a particular location, is given by the probability density. This can be obtained by  multiplication of $\Psi$ with its complex conjugate,  $|\Psi^{2}|$. %Born interpretation. Ref? 
%Make sure you know what a complex conjugate is.
Integration of $|\Psi^{2}|$ over a region of space returns the probability that a system will be found within, called the Born interpretation. %Ref
%, such that the integral of over a region of space returns the probability that a system will be found within. %in that region.$|\Psi^{*}\Psi|$ 
%In Born's statistical interpretation in non-relativistic quantum mechanics,[8][9][10] the squared modulus of the wave function, |ψ|2, is a real number interpreted as the probability density of measuring a particle's being detected at a given place – or having a given momentum – at a given time, and possibly having definite values for discrete degrees of freedom. The integral of this quantity, over all the system's degrees of freedom, must be 1 in accordance with the probability interpretation. This general requirement that a wave function must satisfy is called the normalization condition. Since the wave function is complex valued, only its relative phase and relative magnitude can be measured—its value does not, in isolation, tell anything about the magnitudes or directions of measurable observables; one has to apply quantum operators, whose eigenvalues correspond to sets of possible results of measurements, to the wave function ψ and calculate the statistical distributions for measurable quantities.
%From Wikipedia: https://en.wikipedia.org/wiki/Wave_function
%Read the page, it's pretty good. 
Values of $\Psi$ are chosen to be orthonormal; % that is, orthogonal and normalised, such that integration of ...
integrating $|\Psi^{2}|$ over all space gives the probability of 1:
\begin{equation}
\braket{\Psi_{i}|\Psi_{j}} = \delta_{ij}
\label{equ:born}
\end{equation}
%The above equation just means an integral of Psi_i and Psi_j over 3D Cartesian space. Or something to that effect.  
where all states are represented by $i$ and $j$, and:
\begin{center}
$\delta_{ij}=0$ for $i\neq j$\\
$\delta_{ij}=1$ for $i=j$
%  δij is the Kronecker delta 
% Read Cramer page 107 for more clarfication
the integral is one.
\end{center}
%What does the delta refer to??
%Therefore, for $|\Psi^{2}|$ that has been normalised, the integral over all space is equal to 1, indicating that the probability of finding the system in the space is equal to 1.  %[REF] 
Operators acting on $\Psi$ yield the observable properties of the system. 
% How, and why?
The operator returning the energy of the system is called the the Hamiltonian operator ($\mathbf{H}$).
Erwin Schr\"{o}dinger %first 
proposed his equation in 1926, describing a quantum system using its wave function \cite{schrodinger1926}. %Check this is accurate. Does the equation use the wave function to describe the (energy of the) system, or does the equation just describe the wave function itself?
%The time-independent %Schr\"{o}dinger 
Schr\"{o}dinger's time-independent equation is:
%Unless you know the long form... given in short-form: %(equation \ref{equ:schrodinger}). NB MAKE
%\subsection{The Schr\"{o}dinger Equation} time independent schrodingers
\begin{equation}
%\centering
\mathbf{H}\Psi=E\Psi
\label{equ:schrodinger}
\end{equation}
and the energy of the system is given by the expectation value of the Hamiltonian operator:
%The observable, A, of any operator, Aˆ, is given by its expectation value,
%https://pureadmin.qub.ac.uk/ws/portalfiles/portal/137543092/thesis.pdf pg24
% can be evaluated
\begin{equation} 
E=\braket{\Psi|\mathbf{H}|\Psi} 
\end{equation}
%MAKE SURE YOU KNOW WHAT THIS MEANS
%where $\mathbf{H}$ is the Hamiltonian operator and $E$ is the energy ofm the system. $\mathbf{H}$ is called the eigenvalue, $\Psi$ the eigenfunction and $E$ is scalar. 
where the Hamiltonian operator $\mathbf{H}$ is an eigenvalue of %the eigenfunction, 
the wave function 
$\Psi$, and $E$ is a scalar denoting the energy of the system. 
A given system may have many acceptable values for $\Psi$, each with an associated value for $E$.
%(Cramer pg 107)

%SEPARATION OF THE wave function:\\
%The $\Psi$ turns to $\psi$, and $\partial$ turns to d, after separation of terms. I don't really know why we changed the notation. But basically, once you start working with the time indepedent Schr, then switch to $\psi$, only. \\
%[Note, time-\textbf{dependent} Schrodinger in one dimension ($x$), and time-independent potential energy for a single particle.] \\
%The wave function evolves in time $\Psi(r_{1}, r_{2}....t)$  according to:
%\begin{equation}
%\textit{i}\hbar \frac{\partial \Psi}{\partial t} = H\Psi
%\end{equation}
%
%One dimension with time dependency:
%\begin{equation}
%H\Psi = - \frac{\hbar^{2}}{2m}\frac{\partial^{2} \Psi}{\partial x^{2}} + V(x)\Psi = \textit{i}\hbar \frac{\partial \Psi}{\partial t}
%\end{equation}
%
%Separation of variables:
%\begin{equation}
%\Psi(x,t) = \psi(x)\theta(t)
%\end{equation}
%
%Now:
%\begin{equation}
% - \frac{\hbar^{2}}{2m}\theta \frac{d^{2} \psi}{d x^{2}} + V(x)\psi \theta = \textit{i}\hbar \psi \frac{d\theta}{dt}
%\end{equation}
%
%Simplifying by dividing by $\psi \theta$:
%\begin{equation}
% - \frac{\hbar^{2}}{2m}\frac{1}{\psi} \frac{d^{2} \psi}{d x^{2}} + V(x)= \textit{i}\hbar \frac{1}{\theta} \frac{d\theta}{dt}
%\end{equation}
%
%Perhaps straight away reduce it down to atomic units. ie. When working in \ac{au}, the general form of the Hamiltonian is given by: And delete the line later about it getting reduced to 1. 
%\textbackslash underbrace\{x+y\}\]\{|A|\
The general form of the Hamiltonian is given by: 
\begin{equation}
\mathbf{H}= - \sum \frac{\hbar^{2}}{2m_e}\nabla_i^{2} - \sum \frac{\hbar^{2}}{2m_k}\nabla_k^{2} - \sum_{i} \sum_{k} \frac{e^{2}Z_{k}}{r_{ik}} + \sum_{i<j} \frac{e^{2}}{r_{ij}} + \sum_{k<l} \frac{e^{2}Z_{k}Z_{l}}{r_{kl}}
\label{equ:hamiltonian}
\end{equation}
where all electrons are represented by $i$ and $j$, and all nuclei by $k$ and $l$ \cite{Cramer2004}. 
%where electrons are labelled $i=1,2,3.....j$
%$i \rightarrow j$ encompasses all electrons and $k \rightarrow l$ all nuclei. 
$\hbar$ %= \frac{h}{2\pi}$
is the reduced Planck's constant ($\hbar = \frac{h}{2\pi} = 1.055\times10^{-34}$), 
$m_e$ is the mass of an electron, %Numbers and ref for all these please
$m_k$ is the mass of the nucleus $k$, 
$e$ is the charge of an electron, 
%$\mathsf{e}$ is the charge of an electron %Change to sans serif, to differentiate from the exponential symbol? If so, change the equation too. 
$Z_k$ is the atomic number of $k$ and %$r_{xy}$ is the distance between particles $x$ and $y$.
$r_{ik}$ is the distance between particles $i$ and $k$. 
When using \ac{au}, %the electronic mass, charge and reduced Planck's constant 
the value of $e$, $m_e$ and $\hbar$ are reduced to 1. %Give the simplified equation here? %see Leah pg 29 for physical constants
$\nabla^{2}$ refers to the Laplacian operator, which describes the divergence of the gradient of a field. % It describes how the gradient (the gradient of a function of space) diverges over a function. Field or function? I guess if its a 3D plane function, you can call it a field.
%Laplacian of the electron density, is therefore the rate of change / divergence of the electron density, over that bit of space / cartesian coord?
% +ve is electron density deficient, -ve is electron density rich
%See the wiki: https://en.wikipedia.org/wiki/Laplace_operator
In Cartesian space, this is defined as the sum of the second derivatives of the gradient with respect to each of the three dimensions ($x$,$y$,$z$): % Does it refer to the change in electronic position, in this case? Do I even need to mention much about Laplacian?
%
\begin{equation}
\nabla_i^{2} = \frac{\partial^{2}}{\partial x_{i}^{2}} + \frac{\partial^{2}}{\partial y_{i}^{2}} + \frac{\partial^{2}}{\partial z_{i}^{2}}
\label{equ:laplace}
\end{equation}
%Maybe I should label the equation itself
The first and second terms of equation (\ref{equ:hamiltonian}) correspond to the kinetic energy of the electrons and the nuclei, respectively. Electron-nuclear attraction is described by the third term; %followed by the interelectronic and internuclear repulsive terms. 
the fourth term describes inter-electronic repulsion and the final term the inter-nuclear repulsion. 
The final three potential energy terms are identical to their expression in classical mechanics. %the same as in classical mechanics. 
%analogous with classical mechanics.
%identical to their expression / representation in classical mechanics. 
% attractive and repulsive terms are% potential energy terms that are
% described as in classical mechanics. 
%coloumbic interactions
%
%Probably don't need this section ----------------------------------------------------------------%
The kinetic energy terms %are 
can be expressed as the eigenvalues of the kinetic energy operator ($\mathbf{T}$):
\begin{equation}
\mathbf{T}=-\frac{\hbar^{2}}{2m}\nabla^{2}
\label{equ:KE}
\end{equation}
% NO, keep it generic so you can apply it to the nuclear one (term 2, too)
%\begin{equation}
%\mathbf{T}=-\frac{\hbar^{2}}{2m}\sum^{N}_{i=1}\nabla^{2}_{i}
%\label{equ:KE}
%\end{equation}
%Don't know really know why being able to do the above, is important. 
The total, non-relativistic Hamiltonian %Equation \ref{equ:hamiltonian}
can therefore be written in terms of the kinetic energy and potential energy operators:
\begin{equation}
\mathbf{H} = \mathbf{T}_{e} + \mathbf{T}_{N} + \mathbf{V}_{e-N}+ \mathbf{V}_{e-e} + \mathbf{V}_{N-N}
\label{equ:Htot}
\end{equation}
where the terms are as they were in equation \ref{equ:hamiltonian}. $\mathbf{T}_{e}$ corresponds to the kinetic energy of the electrons, $\mathbf{T}_{N}$ the kinetic energy of the nuclei, $\mathbf{V}_{e-N}$ the coulombic interaction between electron and nuclei, $\mathbf{V}_{e-e}$ the electron-electron repulsion and $\mathbf{V}_{N-N}$ the nuclear-nuclear interaction. 
%the nuclear-nuclear coulomb interaction. 
%, the first two terms denoting the kinetic energy of the electrons and nuclei, and the latter three terms detailing the electron-nuclear interactions.
%Was there really any point in rewriting it?
%The Hamiltonian for the full system is represented in terms of the kinetic energy and potential energy operators. 
%$\hat{H}$
%
%Dirac’s equations [REF] present the most complete description of a whole N-electron system by accounting fully for special relativity in the context of quantum mechanics, however many layers of approximation must be adopted in order to arrive at a model that is pliable within computational limitations. In the context of quantum electronic structure calculations it is usually appropriate to exclude the contribution of relativistic effects and nuclear motion, thus applying the Born-Oppenheimer approximation. 
%This presents the time independent Schrödinger’s equation \ref{equ:schrodinger}, which the Hartree-Fock approximation and many other electronic structure wave function based methods aim to solve. 
%
%The Hamiltonian operator (Eq. 3) acting on the wave function produces the electronic energy of a many electron system, where each of the terms in the Hamiltonian correspond to kinetic energies of the electrons and nuclei, the attraction of the electrons to the nuclei, and the inter-electronic and inter-nuclear repulsions.7
%
%-------------------------------------------------------------------------------------------------------------------------------------------------------------------------------------------------%
\subsection{Born-Oppenheimer approximation}
%This maybe should be a paragraph rather than a whole multi-page section
%Get the original reference - might be libo `s reference [27]
In a real system, the motion of elections and nuclei are coupled. Electron density flows dynamically in response to the change in nuclear position and repulsion from other electrons. The correlated motion of particles is described by the pairwise attractive and repulsive terms of the Schr\"{o}dinger equation. However, this interdependency makes defining a wave function difficult. 
% results in a wavefuction that is difficult to define. 
%of such a system 
%this representation is unworkable owing to the complexity of considering the interdependent movement of all particles in the system. 
Relative to electronic motion, nuclei move far more slowly, owing to their much greater mass (the mass of a proton is around 1836 times larger than that of the electron). %REF The relaxation time of electrons may be treated as instantaneous, when comparing against that of the nuclei. 
%This difference in electronic and nuclear motion leads
%The difference is such, that the 
Nuclear positions therefore appear essentially stationary when compared to that of the electrons. Exploiting this property, the 
%Owing to this difference in relative  
Born-Oppenheimer approximation fixes the nuclear positions. In this way, the motion of electrons and nuclei can be decoupled, and the electronic properties of the system may be calculated for the given nuclear coordinates. Dependency on the nuclear kinetic energy term ($\mathbf{T}_{N}$) %of the Hamiltonian 
is removed, as the nuclei are frozen. % becomes then independent of electronic motion; 
%The electron-nuclear interaction term ($\mathbf{V}_{e-N}$) disappears, and 
The nuclear-nuclear repulsive term ($\mathbf{V}_{N-N}$) becomes a constant for the specified geometry.  
Equation \ref{equ:hamiltonian} is reduced to its electronic components and nuclear constants which in atomic units can be written as:
%
%\begin{equation}
%\mathbf{H} = \mathbf{T}_{e} + \mathbf{V}_{e-N}+ \mathbf{V}_{e-e} + \mathbf{V}_{N-N}
%\label{equ:BO}
%\end{equation}
% With AU:
\begin{equation}
%\mathbf{H}= - \frac{1}{2} \sum \nabla_i^{2} - \sum_{i} \sum_{k} \frac{Z_{k}}{r_{ik}} + \sum_{i<j} \frac{1}{r_{ij}} + \mathbf{V}_{N-N}
\mathbf{H} = \mathbf{T}_{e} + \mathbf{V}_{e-N}+ \mathbf{V}_{e-e} + \mathbf{V}_{N-N}
\label{equ:hamiltonian_BO}
\end{equation}
% Probably don't need this section ----------------------------------------------------------%
%Or in terms of the operators:
%\begin{equation}
%\mathbf{H} = \mathbf{T}_{e} + \mathbf{V}_{e-N}+ \mathbf{V}_{e-e} + \mathbf{V}_{N-N}
%\label{equ:hamiltonian_BO_op}
%\end{equation}
%
and the electronic terms can be collected into one term, to simplify notation:
\begin{equation}
\mathbf{H}= \mathbf{H}_{el}+ \mathbf{V}_{N-N}
\label{equ:hamiltonian_BO_simp}
\end{equation}
%
The Schr\"{o}dinger’s equation can now be written in terms of only the electronic coordinates: 
%CHECK THIS IS ACCURATE
%
\begin{equation}
(\mathbf{H}_{el} + \mathbf{V}_{N-N})
%Should the E have the el? When H_el and VNN are added, surely it gives the BO approx for E rather than just E_el? But then the wavefucntion is el too.. hmmm
%If I do change it, make sure to propagate through subsequent paragraph
%\Psi_{el}(\mathbf{q}_{i};\mathbf{q}_{k})=E(\mathbf{q}_{i};\mathbf{q}_{k})
\Psi_{el}(\mathbf{q}_{i};\mathbf{q}_{k})=E_{el}(\mathbf{q}_{i};\mathbf{q}_{k})
\label{equ:elec_schro}
\end{equation}
%where $\mathbf{H}_{el}$ is the pure electronic Hamiltonian, that is, only including the electronic kinetic energy , electron-nuclear attractive term, and electron-electron repulsion term from the total Hamiltonian given in \ref{equ:hamiltonian}. $\mathbf{V}_{N-N}$ is the nuclear-nuclear repulsion, constant for the fixed positions. %And so can be an additive term at the end. 
where the electronic coordinates are given by $\mathbf{q}_{i}$, the stationary nuclear positions by $\mathbf{q}_{k}$ and 
$E_{el}$ is the electronic energy of the system. The values of $\mathbf{q}_{i}$ are independent variables, whereas the values of $\mathbf{q}_{k}$ are parameters. %Set parameters WAIT DID I EVEN COPY THIS RIGHT, WHAT DOES IT MEAN AND AM I MISSING A \Psi
% SO HOW DO WE CALCULATE Eel?
%significance of this? caveats of this?
% and thus appear following a semicolon rather than a comma in the variable list for \Psi
% so that the  Schr\"{o}dinger’s equation can be solved for the electrons alone. 

%CHECK this section
Given the example of a diatomic molecule, a potential energy curve can be obtained by calculating the value of $E_{el}$ at different inter-nuclear distances.
%a particular inter-nuclear separation, increasing the separation distance and again calculating the energy. 
%(\ref{elements of physical chemistry})
A series of these calculations %along different fixed nuclear separations 
generates a potential energy profile, allowing identification of an equilibrium bond length at the minimum of the curve. 
Calculation of $E_{el}$ for all possible nuclear coordinates in a system of three or more atoms facilitates the construction of a hypersurface %on which all nuclear motion occurs, 
%with 
on which the potential energy is defined by the nuclear geometry, called a \ac{PES}. % obtain transition structures for reaction involving a change in geometry, at the peak between two troughs.  
%This last section may be a bit dodgy/ inaccurate/ CHECK
Exploration of the \ac{PES} allows for discovery of global and local minimum energy structures, intermediate products on a reaction coordinate and transition states, through scrutiny of the %second derivatives of the gradient 
%maxima and minima 
at a particular set of nuclear coordinates. 
%The matrix of second partial derivatives is referred to as the Hessian matrix. 
%Contained within a Hessian matrix
%
%If $  x_0  $  is a critical point and $ 
%f^{\prime\prime}(x_0) < 0  $  , then is a local maximum.
%If $  x_0  $  is a critical point and $ 
%f^{\prime\prime}(x_0) > 0  $  , then is a local minimum.
%If the second derivative is positive at a point, the graph is concave up. If the second derivative is positive at a critical point, then the critical point is a local minimum.
%If the second derivative is negative at a point, the graph is concave down. If the second derivative is negative at a critical point, then the critical point is a local maximum.
%An inflection point marks the transition from concave up and concave down. The second derivative will be zero at an inflection point.

Molecular structure theories adopt the Born-Oppenheimer approximation for its effective simplification of the coupled nuclear-electronic motion problem, in addition to its %relative
accuracy; this assumption works well for ground state molecules and only introduces very small errors. %CHECK THIS, and what happens for excited ones?? / refs / examples
The model breaks down in the situation where there are multiple \ac{PES} close in energy %proximity 
to one another, %In what kind of systems?
or even intersecting. %CHECK / ref/ give examples
%See Jensen for clarification
In these cases the coupled equations must be considered. 
However for the work within this study, the Born-Oppenheimer approximation is successfully applied for all calculations involving electronic structure determination. 

%-------------------------------------------------------------------------------------------------------------------------------------------------------------------------------------------------
\subsection{Slater determinants}
% A method of expressing the wave function, such that it obeys Pauli's
%Summed up as - antisymmetry and 
% A paragrah to talk about how exchange is the difficult bit to solve?
In a system of multiple electrons, each electron is indistinguishable. If the positions of two electrons are swapped, the distribution of electron density in the system remains the same. %\cite{Leach2001}. 
The Pauli exclusion principle states that no two identical fermions, such as electrons, may simultaneously occupy the same quantum state within the same system. When considering an atom with two or more electrons, this means that none may have the same set of quantum numbers. %say what they are?
As a result, for two equivalent electrons, the wave function of the system is antisymmetric with respect to the exchange of their coordinates: 
%must therefore change sign upon the interchange of their coordinates
%permutation of their coordinates. %is this an imprecise statement?
%https://www.chem.tamu.edu/rgroup/hughbanks/courses/673/handouts/antisymmetry.pdf
\begin{equation}
\Psi(1,2..i,...j..N) = - \Psi(1,2..i,...j..N)
\end{equation}

%or
%\begin{equation}
%\Phi(\mathbf{X}_{1},\mathbf{X}_{2}) = - Phi(\mathbf{X}_{2},\mathbf{X}_{1})
%\end{equation} 
%These antisymmetry 
This requirement is fulfilled by expressing the wave function as a Slater determinant, which changes sign with permutation of the coordinates of two electrons. 
%What is a determinant anyway
In the case of a multi-electronic system, the generalised Slater determinant for $N$ total electrons is as follows: 
\begin{equation}
\psi_{SD} = \frac{1}{\sqrt{N!}}
\begin{vmatrix}
\chi_{1}(1) 	& \chi_{2}(1) 	&\cdots 	& \chi_{N}(1) \\
\chi_{1}(2) 	& \chi_{2}(2) 	&\cdots 	& \chi_{N}(2) \\
\vdots 		& \vdots 			& \ddots 	& \vdots 		   \\ 
\chi_{1}(N) & \chi_{2}(N) 	&\cdots 	& \chi_{N}(N) 
\end{vmatrix}
\end{equation}\\
%\frac{1}{\sqrt{N!}} is a normalisation factor - Mats linder thesis 2012
where $\chi_{N}$ represents single electron wave functions, or spin-orbitals \cite{Slater1929}. %which are orthonormal
In the context of a molecule, the single electron wave functions are molecular orbitals. 
Rows are labelled by the coordinates of each electron: (1), (2) $\cdots$ ($N$), whereas each column uses a different orbital function: $\chi_{1}$, $\chi_{2} \cdots$, $\chi_{N}$.
%
%Rows are labelled by the coordinates of each electron: (1), (2) $\cdots$ ($N$)
%Rows are labelled by the coordinates of each electron: $\textbf{x}_{1}$,$ \textbf{x}_{2}\cdots \textbf{x}_{N}$, whereas each column uses a different orbital function: $\chi_{1}$, $\chi_{2} \cdots$, $\chi_{N}$.
%The orbitals increase along the columns, whilst electron coordinates run down the rows. 
%If the labels of (1) and (2) are exchanged, the rows of the determinant are exchanged
%If the labels of $\textbf{x}_{1}$ and $\textbf{x}_{2}$ are exchanged, 
If the labels of (1) and (2) are exchanged, 
the rows of the determinant are exchanged; a general property of determinants is that the interchange of two rows leads to a change of sign. The expanded form of the determinant ($\psi_{SD}$) will have the opposite sign when a pair of electronic coordinates are switched, by switching rows within the determinant, thereby fulfilling the antisymmetry requirement. 
In the dis-allowed case of two equivalent electrons occupying the same spin-orbital, two columns would be identical \cite{Dykstra1994}. The evaluation of the determinant would then be zero, indicating that the probability of two electrons with identical spin occupying the same orbital was zero. 
%Application of the Slater determinant therefore fulfils the Pauli exclusion principle. 
%-------------------------------------------------------------------------------------------------------------------------------------------------------------------------------------------------
\subsection{Variational principle}
\label{sect:variational_principle}
%https://www.diva-portal.org/smash/get/diva2:548763/FULLTEXT01.pdf
In order to obtain the ground state energy of a system, %They aren't really looking for the ground state, in an excited state / ground state sense. But rather a placement of electrions, sense. Not sure if this terminology is correct here...
the wave function giving the lowest energy must be found. %Within the Born-Oppenheimer approximation, 
This corresponds to the electronic configuration with lowest value of $E_{el}$. Difficulty then arises, as ground state energy cannot be computed exactly. 
The variational theorem states that the calculated energy of any guess wave function can only be greater than or equal to the real ground-state energy ($E_{0}$) of the system. % The concept of \''lower bound'' is a property of wave functions? How is this derived, and where does it stem from?
This provides a criterion for selection of the best guess wave function, as the energy is always bounded from below, where: 
%make sure you know what this means and whether the hat is appropriate
%\begin{equation}
%E_{exact} \leq {\langle \Psi | \hat{H} | \Psi \rangle}
%\end{equation}
\begin{equation}
E = \langle \Psi | \mathbf{H} | \Psi \rangle \geq E_{0}
%E_{0} \leq {\langle \Psi | \mathbf{H} | \Psi \rangle}
% Normalised
%E_{el} = \frac{\langle \psi | \mathbf{H} | \psi \rangle}{\langle \psi | \psi \rangle} \leq E_{0} 
\end{equation}
for a normalised wave function. % in the Born-Oppenheimer approximation. 
Thus, when choosing between different trial wave functions, the solution with the lowest energy is the one closest to the exact value. % The best wave function is therefore when the energy is at a minimum and the first derivative of the energy is equal to zero (\textdelta E = 0.) 

%Thus, when choosing a trial wavefunction, the expectation value (${\langle \Psi | \mathbf{H} | \Psi \rangle}$) is minimised ...
%This is called the variational principle, whereby the lowest energy 
%%Jensen pg 87
%The energy of an approximate wave function can be calculated as the expectation
%value of the Hamiltonian operator, divided by the norm of the wave function.
%\begin{equation}
%E_{el} = \frac{\langle \psi | \mathbf{H} | \psi \rangle}{\langle \psi | \psi \rangle}
%\end{equation}
%For a normalized wave function the denominator is 1, and therefore $E_{el} = {\langle \psi | \mathbf{H} | \psi \rangle}$.
%
%-------------------------------------------------------------------------------------------------------------------------------------------------------------------------------------------------
\subsection{Hartree-Fock self-consistent field method}
In practice, equation \ref{equ:elec_schro} can only be solved \textit{exactly} in very few circumstances; no exact solutions can be found for problems involving three or more interacting particles, such as in the case of a helium atom possessing two electrons and one proton. %VIVA: Why? - the correlated interaction between the partices which cannot be calculated exactly. 
For systems of complexity greater than one electron, further approximations must be made. %like what?

The \ac{HF} approximation was the first practically applicable method for calculation of the ground-state energy of atoms with fixed nuclear positions.  
%Need some preamble here, how did we get from slater and BO, to Hartree's method? 
%Not sure what this means, nor which they refer to:
% The product form of an electronic wave function is only obtainable \textit{via} and approximation. 
%The electronic Hamiltonian (equation \ref\label{equ:hamiltonian_BO_op}) is not separable for individual electrons. 
%This is prevented by the electron-electron repulsion term/ operator - an exact wave function cannot be determined as a product of the one electron function (from Slater's ?). 
%The electron-electron repulsion term 
%As the schrodingers equation cannot be solved analytically for systems with more than one electrion, 
%The \ac{SCF} method iteratively determines a set of molecular orbitals that minimises the energy of the wave function, treating the interaction of each electron as in a static field of all other electrons in the system.
%
The \ac{SCF} method was proposed by Hartree in 1928 \cite{Hartree1928,Hartree1929}. 
%https://repository.upenn.edu/cgi/viewcontent.cgi?article=3638&context=edissertations
%The method is an independent-particle model. %Each electron  % all electronic interactions treated are instead averaged into a static field. 
$N$ electrons are treated as individual particles occupying single-electron spin orbitals and move independently of the dynamics of any other fermions in the system. 
%%without the effect of the surrounding fermions. 
%This is also referred to as the Mean-Field approximation; 
The effective interaction of one electron with all other fermions is averaged and applied as a static external field, in the form of a spherical potential around the electron (called the mean-field approximation). 
In this way, the $N$-body problem is reduced to a 1-body problem. 
%This assumption places a static spherical potential aroun	d the electron, derived from the average effect of the nucleus and other electrons in the system. 
%(On the Coloumb replusion. Cramer pg 126)
The approximation neglects exchange in the electron-electron interaction; %describing the energy of electrons as individual particles moving independently of the dynamics of any other fermions in the system. 
the calculated Hartree wavefunction alone does not include any contribution from electron correlation, and incorrectly implies that the electrons are distinguishable. %if you don't know why this is, omit it. 
%NOTE the difference between exchange - the non-classical effect of electrons of differing spin, and correlation, which is due to the instantaneous interactions that are not captured by the static models. (And which every other contribution to the difference to the ``real'' energy is lumped in.)
Fock developed this idea by introducing Slater determinant wave functions \cite{Fock1930}. 
The effects of exchange on the coulombic repulsion were incorporated, achieved by taking the trial wave function as a single Slater determinant. %again, if you don't know how or why this is, omit it. 

% trial wave functions ($\Psi_{T}$)
For an electron in orbital $\chi_{i}$ in the field of nuclei $M$ and other electrons $\chi_{j}$, the Hamiltonian operator is comprised of three terms, corresponding to the three contributions to the energy. 
The core Hamiltonian operator, $\mathbf{H}^{core}$ %(1)$ 
comprises of the kinetic energy of each electron, and the electron-nuclear interaction: % `` the energy of each electron moving in the field of the bar nuclei'')
%
\begin{equation}
\mathbf{H}^{core}(1)= - \frac{1}{2} \nabla^{2}_{1} - \sum^{M}_{A=1}\frac{Z_{A}}{r_{1A}}
\end{equation}
%
where %$\mathbf{H}^{core}$ is the core Hamiltonian operator, ($M$ are all the nuclei)
%$A$ is the nucleus / normalisation constant?, 
$Z_{A}$ is the nuclear charge and $r_{1A}$ is the %normalised electron-nuclear separation. 
separation of electron (1) with nucleus A. %halved to remove duplicity / double counting 
In a mono-electronic system, this would be the only operator present. 

The coulomb operator, $\mathbf{J}_{j}$ %(1)$ average coulomb potential
corresponds to the averaged interaction potential between each pair of electrons in the same orbital, and with other electrons in other orbitals $\chi_{j}$:
%electron-electron repulsion
%
\begin{equation}
\mathbf{J}_{j}(1)=\int d\tau_{2}\chi_{j}(2)\frac{1}{r_{12}}\chi_{j}(2)
\end{equation}
%(Now electrons are the big numbers and the little i and j's are the orbitals)
where $d\tau_{i}$ indicates the integration is over the spatial and spin coordinates of electron $i$, and $r_{12}$ is the distance between the two electrons. 
%corresponding to the averaged interaction potential due to an electron in orbital $\chi_{j}$.

The exchange operator $\mathbf{K}_{j}$ %(1)$  non-local exchange potential
, is only non-zero for electrons with the same spin, arising due to the antisymmetry of the wavefunction: % / Slater dterminant - Leach pg 46, also see ``repulsive contribution'' on pg 206. As two nuclei move closer, electrons of the same spin move out of the way, causing a depletion of electron density in the internuclear space, thus increasing the internuclear repulsion due to deshielding. At short distances the interaction energy increases as $1/r$, at large distances the energy decays exponentially as $e^{-2r/a_{0}}$ however ($a_{0}$ is the Bohr radius). 
\begin{equation}
\mathbf{K}_{j}(1)\chi_{i}(1)=\Big[\int d\tau_{2}\chi_{j}(2)\frac{1}{r_{12}}\chi_{j}(2)\Big]\chi_{j}(1)
\end{equation}
defined in terms of its effect when acting on $\chi_{i}$. 
% \ac{HF} 
The Hamiltonian operator % for electron in $\chi_{i}$ 
written in terms of its core, coulomb and exchange contributions is as follows: %K is negative because of antisymmetry
\begin{equation}
\Big[\mathbf{H}^{core}(1)+\sum^{N}_{j=1}\{\mathbf{J}_{j}(1)-\mathbf{K}_{j}(1)\}\Big]\chi_{i}(1)=\sum^{N}_{j=1}\varepsilon_{ij}\chi_{j}(1)
\end{equation}
%WHAT is \varepsilon_{ij}?
and can be simplified to:
\begin{equation}
\textbf{F}_{i}\chi_{i}=\sum_{j}\varepsilon_{ij}\chi_{j}
\end{equation}
where $\textbf{F}_{i}$ is the Fock operator, and $\varepsilon_{ij}$ is the energy of orbital $\chi_{j}$. 
%  $\varepsilon$ is the diagonal matrix of orbital energies. Each of its elements $\varepsilon_{i}$ is the energy of  orbital $\chi_{i}$.
%The approximation, therefore, lacks any electron correlation (other than in the average sense) and may lead to large errors in the energy:
%for example, band gaps in some semiconductors and bond-length in some molecules are badly
%described. https://pureadmin.qub.ac.uk/ws/portalfiles/portal/137543092/thesis.pdf
The operator is a one electron Hamiltonian for an electron in a multi-electron system. 
For ``closed shell'' problems where there are no unpaired electrons, the operator has the form:
\begin{equation}
\textbf{F}_{i}(1)=\mathbf{H}^{core}(1)+\sum^{N/2}_{j=1}\{2\mathbf{J}_{j}(1)-\mathbf{K}_{j}(1)\}
\end{equation}
%Localised orbitals and setting the langrange multipliers (\varepsilon) to 0 unless i = j :
%Applying Slater's rearranging for localised orbitals
%Rearranging to the standard eigenvalue form:
%%(Of which I'm 100% taking as a black box, here)
%\begin{equation}
%\textbf{F}_{i}\chi_{i}=\varepsilon_{i}\chi_{i}
%\end{equation}
%In the first step, guess wave functions for the occupied \ac{MO}s
%The core Hamiltonian 
$\mathbf{H}^{core}$ consisting of the kinetic energy terms can be solved exactly, the electron-electron repulsion $\mathbf{J}_{j}$ must apply the mean-field approximation but the exchange component $\mathbf{K}_{j}$ is solved iteratively. %CHECK - is J solved iteratively too?
Starting with an initial guess %solutions 
wave functions for the occupied orbitals $\chi_{i}$, solution of the one-electron \ac{HF} eigenvalue equations generates a new set of orbitals. 
%Starting with an initial guess of the wave functions for occupied orbitals, solution of the one-electron Schr\"{o}dinger equation generates a set of orbitals.
% minimising the energy of the wave function. %CHECK
%Trial wavefunctions / solutions to \chi_{i} are chosen and used for \ac{HF} eigenvalue equations obtained. These then used to calculate the coulomb and exchange opertors.  Solve HF equations, gnerating new set of solutions for \chi_{i}, which are then used in the next iteration. This is repeated with each iteration refining to a lower energy until results for every electron is the same as the previous iteration /converges. 
Applying the variational principle, the spin orbitals are varied to minimise %to find the lowest 
the energy. This process propagates using the newly generated orbitals for the next optimisation, until the difference between the final solution and its previous iteration falls within an acceptable threshold and is ``self-consistent''. 

%The lowest possible energy is the \ac{HF} approximation for the ground state energy, ignoring electron correlation. 
%Numerical solutions of the Schr\"{o}dinger equation present the best atomic orbitals \cite{Atkins2011} %Check this statement. 
%The \ac{HF} method starts
%
%\begin{equation}
%E=\frac{\int\psi^{*}(\textbf{x}) H \psi(\textbf{x})d\textbf{x}}{\int\psi^{*}(\textbf{x}) \psi(\textbf{x})d\textbf{x}}
%\label{equ:HF}
%\end{equation}
%-------------------------------------------------------------------------------------------------------------------------------------------------------------------------------------------------%%[PLEASE REVAMP ALL OF THIS AND FILL IN DETAIL]
\subsection{Open shell systems}
The forced pairing of electrons of opposing spin into a shared orbital is referred to as the \textit{restricted} scheme (figure \ref{fig:RHF_ROHF_UHF}). %, (e.g. R\acs{B3LYP}).  You haven't described functionals yet
For closed shell systems, this treatment is appropriate. For species with unpaired electron spin such as in transition metal complexes or radicals, an alternative model allowing singly occupied orbitals must be adopted.
The \textit{restricted-open} %e.g.RO\asc{B3LYP}, 
scheme maintains electron pairing within orbitals except in the case of the \ac{HOMO}, which is singly occupied.
An alternative model is the \textit{unrestricted} scheme, %(U\acs{B3LYP}), 
where all electrons are unpaired and reside in their own orbitals. % (Figure 9). 
A caveat of the unrestricted model is its susceptibility to spin contamination, which has consequences at large bond separations. This artificial mixing of spin states leads to a lowering of the obtained energies when compared to the restricted variants \cite{Menon2008}.
% where the bond has not completely broken.
%Explain what spin contamination is, and whether we have to deal with it here. It was not super prevalent in our study - whilst we did looking at bond breaking, the energy at large separations was only used as a quantitative indication that a reaction was likely to take place / give a suggestion of the geometry of the product complex. 

\begin{figure}[h]
\centering
\includegraphics[width=0.6\linewidth]{Jensen_pg120}
\caption[Restricted \acf{HF}, \acf{ROHF} and \acf{UHF} methods of calculation.]{The electron ordering schemes corresponding to the restricted \acf{HF}, \acf{ROHF} and \acf{UHF} methods of calculation for closed and open shell systems \cite{Jensen2007}.}
\label{fig:RHF_ROHF_UHF}
\end{figure}
% CHECK add in later - SPIN CONTAMINATION
%For the unrestricted case, the $\alpha$ and $\beta$ electrons are in different orbitals, leading to two molecular orbital expansion coefficients : 
%\begin{equation}
%\psi_{i}^{\alpha}=\sum^{K}_{\mu=1} c_{\mu i}^{\alpha} \chi_{\mu}
%\end{equation}
%\begin{equation}
%\psi_{i}^{\beta} =\sum^{K}_{\mu=1} c_{\mu i}^{\beta} \chi_{\mu}
%\end{equation}
%
%Spin contamination arises as whilst the two sets of orbitals 
%\begin{figure}[h]
%\centering
%\begin{modiagram}
%\atom{left}{
%1s = { 0		; pair} ,
%2s = { 0.5	; pair}
%%2p = { 1		; pair}
%}
%%\atom{left}{
%%1s,2s,3s}
%%\atom{right}{
%%1s = { 0; pair} ,
%%2s = { 1; pair}}
%%\Energyaxis[title=Energy]
%\end{modiagram}
%\caption{}
%\end{figure}
%-------------------------------------------------------------------------------------------------------------------------------------------------------------------------------------------------
\subsection{Electron correlation}
The energy difference between the real energy and the result obtained from \ac{HF} is called the correlation energy, $E_{corr.}$. % \textit{correlation energy}.
\begin{equation}
E_{corr.}=E_{exact}-E_{HF}
\end{equation}
The $E_{corr.}$ term can be divided into two components. The static correlation component arises as a system cannot be fully described by a single set of \ac{MO}s, %from the imperfect of a single set of molecular orbitals in describing a system,
and the dynamic correlation contribution derives from the neglect of instantaneous electron repulsion interactions. % of the same spin, beyond exchange. 
The latter includes the description of instantaneous dipolar interactions, leading to van der Waals forces, which are lost when the electron repulsion terms are averaged. 
%when non-bonding interactions are involved, such as in hydrogen bonding systems and during bond fission where the dissociating species are at a large separation, instantaneous dipolar effects. van der Waals. 
Post-\ac{HF} methods such as perturbation theory and coupled-cluster techniques aim to %redress this dis %close this gap, 
account for the difference by inclusion of the contribution from correlation as an additive term, or \textit{via} multi-electron wave functions. %NOT multi reference - this is for excited states / CI and REF
However, these methods become prohibitively expensive with increasing numbers of electrons, such that the system size is limited to small molecules for calculations of high accuracy. 
The high computational demand associated with %working with
handling a many-electron wave function is circumvented in \ac{DFT}, by expression of the total energy in terms of electron density. % ($\rho(r)$). 
%they use configuration state functions (CSF), a symmetry-adapted linear combination of Slater determinants
%\begin{modiagram}
%\EnergyAxiss[title=E]
%\end{modiagram}
%
%This missing term is small at equlibrium bond lengths (why?) but its absence contributes a large error to evaluation of the energy w
%For example, at least one quarter of the strength of hydrogen bonds between water molecules are due to correlation interactions [37].%
%libo's thesis
%\subsection{Post-Hartree-Fock methods}
%
%\subsubsection{MP2}
%-------------------------------------------------------------------------------------------------------------------------------------------------------------------------------------------------
\section{Density functional theory}
%FIX THIS PARAGRAPH
There are two approaches for solving the Schr\"{o}dinger equation for a polyatomic system with many electrons. %Obtaining the energy, if this first sentence is imprecise for DFT
\textit{Ab initio} methods generate solutions from ``first principles'', without information gained from experimental results. 
By contrast, \textit{semi-empirical} methods deal with parameters fitted to experimental quantities, such as enthalpies of formation or dipole moments. %ref if you can  
%Solutions to the Schrödinger equation using reference
%
%See here for a nice sum of the weaknesses: Some Fundamental Issues in Ground-State Density, Perdew 2009 (https://pubs.acs.org/doi/10.1021/ct800531s)
%Functional Theory: A Guide for the Perplexed
%
%Scaling: https://pureadmin.qub.ac.uk/ws/portalfiles/portal/137543092/thesis.pdf
% MP2 scaling $O(N^{5})$ DFT scales between $O(N^{4})$ and $O(N)$
%Linear scaling DFT can be used for millions of atoms (but with accuracy payoff, obvs)
\ac{DFT} derives from the Thomas-Femi-Dirac model, whereby electron correlation is modelled via functionals of the electron density, $\rho(r)$. %38,39  REF What is this model
Electron density has reduced dimensionality when compared with the wave function, %How 
and can be obtained both \textit{via} theoretical methods as well as experimentally, such as through x-ray diffraction. %REF
%https://www.diva-portal.org/smash/get/diva2:548763/FULLTEXT01.pdf
Currently, it forms the most widely used approach for \ac{QM} problems. %REF? family of protocols
%, with the \acs{B3LYP} functional considered the gold-standard / lots of paramterised data bases etc.
%\ac{DFT} provides an avenue for the calculation of the total energy of the system, avoiding the complicated many-electron wave function.
% CHECK the below statement, its a bit dodgy: https://www.diva-portal.org/smash/get/diva2:548763/FULLTEXT01.pdf %protocols
%Useless trivia
%All known \ac{DFT} methods make use of parameters extracted from empirical expressions; whilst the formalism is exact, it is not strictly \textit{ab initio}. 
When compared to \ac{HF} and post-\ac{HF} methods, \ac{DFT} provides increased computational efficiency. Modern hybrid functionals are able to produce results on the order of \acs{MP2} accuracy, utilising only the resource of a \ac{HF} calculation. %ref 17
The \ac{NC} monomers, dimer and trimer models examined in this study consist of between 30 - 75 atoms; \ac{DFT} provided the best pay-off between accuracy and efficiency for application on a system of this size. 

\subsection{Hohenberg-Kohn formalism}
Modern \ac{DFT} is based on two fundamental theorems proposed by Hohenberg and Kohn in 1964 \cite{Hohenberg1964}. %REF 
The first theorem states that for the ground state of system, there exists a unique energy and non-degenerate electron density. The density can therefore be used to determine the Hamiltonian of a system, thereby also describing its ground state energy $E[\rho(r)]$%$E_{g}$
, wave function and other properties of the system. The energy is a functional of the density: %ref and other properties
%https://pureadmin.qub.ac.uk/ws/portalfiles/portal/137543092/thesis.pdf
\begin{equation}
%E[\rho]=\mathbf{T_{e}[\rho]+\mathbf{V}_{e-e}[\rho]+\mathbf{V}_{e-N}[\rho]
%E_{g}
E[\rho(r)]=\int \rho(r) V(r) \,d(r) + F[\rho(r)]
\label{equ:DFT}
\end{equation}
where $V(r)$ is the external potential, with the first term of the equation arising from the interaction of electrons with $V(r)$ (usually a coulombic attraction between electrons and nuclei). $F[\rho(r)]$ is a universal functional of the density, representing the total kinetic energy and electron-electron repulsion. %Bowlers notes
It is not possible to explicitly express $F[\rho(r)]$ in terms of $\rho(r)$, so its exact form is not known. 
%, $V(r)$ is the external potential uniquely determined by $\rho(r)$.  NO, that is V_(ext), silly. vext(r) is the potential representing the total nuclear attraction of the electrons in addition to any other external potential
%where the energy of the system $E[\rho]$ can be obtained \textit{via} the sum of the kinetic energy of the electrons $\mathbf{T_{e}[\rho]$, electron-electron repulsion $\mathbf{V}_{e-e}[\rho]$ and electron-nucleus attraction $\mathbf{V}_{e-N}[\rho]$ derived as a function of the electron density. 
%https://pdfs.semanticscholar.org/b0de/c00cd81492b681c897249672ee8f42c79d34.pdf 

The second theorem states that the ground state energy can be obtained \textit{via} minimisation of $E[\rho(r)]$. Since equation \ref{equ:DFT} gives the exact energy of the original Hamiltonian, by applying the variational principle, the lowest possible value of $E[\rho(r)]$ gives the real solution for the ground state energy, and therefore $\rho(r)$. 
It is not possible to verify that the found $\rho(r)$ giving the lowest value of $E[\rho(r)]$ corresponds to a wave function obeying the Pauli exclusion principle requirement for antisymmetry. This problem and the unknown identity of $F[\rho(r)]$ were addressed by the Kohn-Sham equations. 
%\ac{wb97xd} is a range separated hybrid with D (self consistent exchange). 
%It is more computationally demanding than just post HF style perturbation theory exchange (as in B3LYP, using MP2 exchange for example  - fact check that). And according to the paper by Najibi and Goerigk even the D3 correction isn't better than the perturbation theory style exchange, for a massive amount of datasets that they tested. SO why even use this in the first place? Look to the origianl paper where it was published and see how they tout this method. Use those reasons to justify having chosen it to use on some of my systems. 

%remember to talk about the underestimation of the reaction barriers
%-------------------------------------------------------------------------------------------------------------------------------------------------------------------------------------------------%
\subsection{Kohn-Sham \ac{DFT}}
%https://pdfs.semanticscholar.org/b0de/c00cd81492b681c897249672ee8f42c79d34.pdf
The Kohn-Sham scheme establishes a system with $N$ non-interacting electrons, in a similar manner to \ac{HF} \cite{Kohn1965}. 
The wave function is described by a single Slater determinant of one-electron orbitals, and the electron density is set to be identical to that of the exact ground state wave function. 
Using this approximation, the energy of the system can again be divided up into its component contributions: 
\begin{equation}
E[\rho]=E_{KE}[\rho(r)]+E_{H}[\rho(r)]+E_{V}[\rho(r)]+E_{XC}[\rho(r)]
\label{equ:KS}
\end{equation}
%Leach pg 128
where $E_{KE}$ is the kinetic energy of the non-interacting electrons, $E_{H}$ the Hartree electrostatic energy corresponding to the electron-electron repulsion between electrons, $E_{V}$ the interaction between the electrons and the external potential due to the nuclei, and $E_{XC}$ the exchange-correlation energy, encapsulating non-classical exchange and correlation contributions not accounted for by the other terms. 
%and correction for the self-interaction included in the Coulomb term and the portion of the kinetic energy which corresponds to the differences between the non-interacting and the real system.
%
%NOTE: for all the other terms and all the other times, the evaluation of the observable (eg kinetic energy  / electron-nuclear interaction) was achieved by the getting the expectation value of the operator. for $E_{KE}$ this is NOT the case. See Bowlers notes Lecture 5
%
%Using equation \ref{equ:DFT} , we can also define $F[\rho(r)]$ as:
%\begin{equation}
%F[\rho(r)]=E_{KE}[\rho(r)]+E_{H}[\rho(r)]+E_{XC}[\rho(r)]
%\end{equation}
Referring back to the Born interpretation (equation \ref{equ:born}), the density can be obtained from the sum of the square moduli of the wave function:
%of a set of one-electron othernormal orbitals
\begin{equation}
\rho(r)=\sum_{i=1}^{N}|\psi_{i}(r)|^{2} = |\Psi|^2
\end{equation}
%http://theses.gla.ac.uk/925/1/2009davin1phd.pdf
Aside from the kinetic energy term in equation \ref{equ:KS}, the remaining terms can be summarised into an effective potential $\nu_{eff}$: 
\begin{equation}
\nu_{eff}=\nu_{H}(r)+\nu_{V}(r)+\nu_{XC}(r)
\label{equ:eff}
\end{equation}
where $\nu_{H}(r)$ is the electron repulsion potential and $\nu_{V}(r)$ the electron-ion potential. % and $\nu_{XC}(r)$ the exchange-correlation potential. Also, why not just nucleus instead of ion potential?
The exchange-correlation potential $\nu_{XC}(r)$ is the functional derivative of $E_{XC}$. If $E_{XC}$ is known then $\nu_{XC}(r)$ can be computed from the following equation: 
\begin{equation}
\nu_{XC}(r)=\frac{\delta E_{XC}}{\delta\rho}
\label{equ:xc}
\end{equation} 

%$\nu_{eff}$ is chosen such that the ground state energy 
The kinetic energy term in equation \ref{equ:KS} %of the non-interacting system ?
can be expressed in terms of the one-electron wave function:
\begin{equation}
E_{KE}[\rho(r)]=-\frac{1}{2}\nabla^{2}\sum^{N}_{i=1}\braket{\psi_{i}|\nabla^{2}|\psi_{i}}
%-\frac{\hbar}{2m}
\label{equ:KE}
\end{equation}
Combining equations \ref{equ:eff} and \ref{equ:KE}, a new Hamiltonian can be written, only considering the non-interacting system:
\begin{equation}
\mathbf{H}=%-\frac{\hbar^{2}}{2m}\nabla^{2}+\nu_{eff}
%-\frac{1}}{2}\sum^{N}_{i}\nabla^{2}_{i} + \nu_{eff}
-\frac{\nabla^{2}}{2}+\nu_{eff}
\end{equation}
%Is the -ve appropriate - yes it cancels out with the Veff I think? or just go with Emilia's one, that is not in A.u.? See page 129 in Leach , also
%http://theses.gla.ac.uk/925/1/2009davin1phd.pdf Pg 32 Says it should be -ve
Using the Kohn-Sham formulation of the  Schr\"{o}dinger equation, the one-electron orbitals $\psi_{i}(r)$ have the form:
\begin{equation}
\Big(-\frac{\nabla^{2}}{2}+\nu_{eff}(r)\Big)\psi_{i}(r)=\varepsilon_{i}\psi_{i}(r)
\label{equ:KS_schr}
\end{equation}
where $\nu_{eff}(r)$ is the effective potential, $\psi_{i}(r)$ the Kohn-Sham orbitals and $\varepsilon_{i}$ the Kohn-Sham orbital energies. %aka eigenvalues

The Kohn-Sham equations are solved self-consistently. Evaluation of equation \ref{equ:KS} gives the total electronic energy. A guess density is supplied for the initial evaluation of equation \ref{equ:KS_schr}, to generate a set of orbitals. This in turn informs the next iteration with an improved density value, until convergence is reached. 
%-------------------------------------------------------------------------------------------------------------------------------------------------------------------------------------------------
\subsubsection{Exchange-correlation functionals}
\subsubsection*{Local density approximation}
%There exchange potential term $\nu_{XC}(r)$ 
In practice, the exact solution for $E_{XC}$ in equation \ref{equ:eff} is not known, so an approximation is used. The simplest method is the \ac{LDA}, based upon a uniform electron gas. $\varepsilon_{XC}$ is calculated per electron as function of the density, and integration over all space gives the $E_{XC}$ for the whole system:
\begin{equation}
E_{XC}^{LDA}[\rho(r)]=\int\rho(r)\varepsilon_{XC}[\rho(r)]\,dr	
\end{equation}
%the exchange part is from uniform gas, the correlation part is from quantum monte carlo
This method has demonstrated good results for structural properties such as bond lengths and lattice constants, often improving on \ac{HF} results. %REF
The model falls down when considering systems with many molecules, overestimating binding energies and atomisation energies, and performing worse than \ac{HF} for open shell atoms. %van der Waals forces. %REF
%https://www.diva-portal.org/smash/get/diva2:164957/FULLTEXT01.pdf

\subsubsection*{Generalised gradient approximation}
To account for inhomogenous electron density present in real systems, the local gradient of the density $\nabla\rho(r)$ can be taken into account at each coordinate, on top of the existing dependency on the density $\rho(r)	$. 
This gradient corrected approach is called the \ac{GGA}: 
\begin{equation}
E_{XC}^{GGA}[\rho(r)]=\int\rho(r)\nabla[\rho(r)]\,dr	
\end{equation}
The results obtained using \ac{GGA} greatly improves upon \ac{LDA}, such as in the calculation of bond dissociation energies. 
The \ac{GGA} is split into its exchange and correlation contributions, which can be solved individually:
\begin{equation}
E_{XC}^{GGA}=E_{X}^{GGA} + E_{C}^{GGA}
\end{equation}
The function of $\nabla\rho(r)$ is not uniquely defined and no true form is known; further developments saw the proposal of numerous gradient correction schemes, often fitting to experimental parameters. Functionals that have been well established in literature include:
\begin{itemize}
\item B88: the exchange functional developed by Becke, and contains an empirical parameter $\beta$=0.0042 fitted to give the best agreement with the \ac{HF} energies of the noble gases \cite{Becke1988}. %($\beta$=0.0042). 
\item LYP: the most widely used \ac{GGA} correlation functional, developed by Lee, Yang and Parr, with empirical fitting to the helium atom. %, only based on \textit{ab initio} calculations. 
Combination with the B88 exchange functional gives the BLYP method \cite{Lee1988}. %REF
\item PBE: the exchange-correlation functional developed by Perdew, Burke and Ernzerhof derived purely from \textit{ab initio} calculations \cite{Perdew1996}. The PBE family of functionals have parameterisations optimised for different materials and interfaces, such as solids and surfaces \cite{Hammer1999,Perdew2008,Wu2006}. %ref PBE sol, RPBE , PBE WC
\end{itemize}
Meta-\ac{GGA} employs further parameterisation of the exchange-correlation functional; the laplacian (second derivative) of the electron density, $\nabla^{2}$, is included as an additional degree of freedom. %Or loacl kinetic energy density. 
These have made improvements on \ac{LDA} and \ac{GGA} performance, but are still considered relatively new. Whilst they have not been as extensively benchmarked as hybrid \ac{DFT} functionals such as \ac{B3LYP}, the existing results are promising \cite{Zhao2005,Zhao2005a,Zhao2006}.
%seen as popular use as the efficient hybrid \ac{DFT} functionals
%Make sure you know how the above functionals differ from the hybrids %-------------------------------------------------------------------------------------------------------------------------------------------------------------------------------------------------
\subsection{Hybrid functionals} 
%Self interaction error!
In \ac{DFT}, the influence of the exchange contribution to $E_{XC}$ is significantly larger than that of correlation, which only adds minor corrections \cite{Grafenstein2009}. 
In hybrid \ac{DFT}, a portion of exact exchange is introduced into the \ac{DFT} exchange energy \textit{via} linear combinations of \ac{HF} and \ac{GGA} exchange. 
This assists to counteract the self-interaction problem, a significant source of error in most approximate exchange-correlation functionals used for \ac{KS-DFT} calculations. This spurious interaction of an electron with itself does not appear in \ac{HF} where the exchange is defined exactly, and a cancelling of the coulomb and exchange-correlation energies occurs \cite{Bao2018}. The description of the exchange-correlation is imperfect in \ac{DFT}; the exchange and coulomb energy terms do not completely cancel out, and an energy contribution is experienced even in a one electron system. An observable manifestation of this in larger systems is the underestimation of reaction barriers \cite{Zhang2010}. 
%So does wb97xd circumvent this in any way?

Perhaps the most widely used functional of this category is \ac{B3LYP} \cite{Becke1993,Becke1993a}. It is employed in the current study to examine the degradation reactions of \ac{NC}. The formalism is as follows: 
\begin{equation}
_{XC}^{B3LYP}=E_{X}^{LDA}+a_{0}(E_{X}^{HF}-E_{X}^{LDA})+a_{X}(E_{X}^{B88}-E_{X}^{LDA})+E_{c}^{LDA}-a_{C}(E_{C}^{LYP}-E_{C}^{LDA})
\end{equation}
$a_{0}$,c$a_{X}$ and $a_{C}$ are parameters fitted to experimental atomisation energies, ionisation potentials, proton affinities and atomic energies. The semi-empirical coefficients have values of 0.20, 0.72 and 0.81, respectively, indicating that the \ac{B3LYP} hybrid functional contains 20\% \ac{HF} exchange, 72\% B88 exchange and 81\% electron correlation contribution from LYP. 
\ac{B3LYP} is of particular interest in the study of \ac{NC} as it has been applied in comparable computational investigations on glucose and its analogues, which forms the base structure of the \ac{NC} monomer \cite{Schnupf2007,Momany2005}. 

There are documented cases where \ac{B3LYP} performs extremely poorly, such as in the description of $\pi\rightarrow\sigma$ structural transformations \cite{Pieniazek2008}, and medium-long range dispersion interactions \cite{Schreiner2007}. In the first case, no similar transitions are expected for the reactions studied in this work. When considering the medium-range electron correlation reactions that may occur for the investigated species, the distance at which the intermolecular interactions take place and at which \ac{B3LYP} performance suffers (1.5-3.5 $\AA$), is of relevance. 
The calculations falling within this range will therefore be performed with both \ac{B3LYP} and the \ac{wb97xd}, in order to validate and compare results. 
%Self interaction error causes underestimation of reaction barrier heights
%https://www.diva-portal.org/smash/get/diva2:548763/FULLTEXT01.pdf
%\subsubsection{Dispersion correction}
%Grimme
%Paragraph on wb97xd

Stefan Grimme proposed an empirical \ac{DFT} method with dispersion correction to include van der Waals interactions, described as \ac{DFT-D} \cite{Grimme2004,Grimme2006}. 
The general form of the DFT-D scheme  is used to calculate the total energy of the system, with the addition of an empirical atomic-pairwise dispersion correction:
%what is that
\begin{equation}
E_{DFT-D}=E_{KS=DFT} + E_{disp}
\end{equation}
The \ac{wb97xd} dispersion corrected functional was presented by Chai \textit{et. al},  based on the $\omega$B97X, which is used to calculate the \ac{KS-DFT} energy, $E_{KS-DFT}$   \cite{Chai2008,Chai2008a}. %for the calculation of the \ac{KS-DFT} energy, $E_{KS-DFT}$ . %The $\omega$B97X followd Grimme's method for computing the 
The approach is founded on Becke's B97-D GGA functional \cite{Becke1997,Grimme2006}. For the B97-D functional, expansion coefficients were added to the original B97 and optimised to reduce the density functional at small separations where lower electron correlation was expected. At medium-long ranges, the density functional description was dictated by the semi-empirical dispersion correction term. 
%(B97D was obtained from the nonhybrid generalized gradient B97)
%https://research.library.mun.ca/8460/1/thesis.pdf
In the case of \ac{wb97xd}, the unscaled dispersion correction is defined as follows:
\begin{equation}
E_{disp} = - \sum_{i=1}^{N-1}\sum_{j=i+1}^{N}\frac{C^{ij}_{6}}{R^{6}_{ij}}f_{damp}(R_{ij})
%unscaled dispersion
\end{equation}
where $N$ is the number of atoms in the system, $C^{ij}_{6}$ is the dispersion coefficient for atom pair $ij$, and $R_{ij}$ is the interatomic distance. 
At short interatomic distances, dispersion is zero. The asymptotic, pairwise \ac{vdw} potentials are maintained by the damping function:
\begin{equation}
f_{damp}(R_{ij})=\frac{1}{1+a(R_{ij}/R_{r})^{-12}}
\label{equ:damp}
\end{equation}
The function decreases to 1 as $R_{ij}$ increases, but also diminishes quickly at small values of $R_{ij}$ to prevent the divergence observed for the undamped case (figure \ref{fig:damp}). $R_{r}$ is the sum of the \ac{vdw} radii of the atoms $ij$, whilst $a$ determines the strength of the dispersion corrections. % a is the only on linear paramter. 
%https://onlinelibrary.wiley.com/doi/epdf/10.1002/jcc.20495
%After two years of carefultesting and validation, however, knowledge of a few shortcomingsof the original DFT-D method accumulated, which will be alsoaddressed in the present work:1. Consistent atomic parameters (C6coefficients) are only availablefor elements H, C–Ne, but studies of supramolecular structuresor problems in material science require parameters for elementsfrom the whole periodic table.2. Test calculations for molecules with third-row elements showedsystematic errors.3. Adding the dispersion energy to the KS-DFT energy leads toinconsistencies for “normal” thermochemistry (e.g. atomizationenergies: the dispersion correction is zero for the free atom andalways nonzero (and large) for the molecule)
\begin{figure}[h]
\centering
\includegraphics[width=0.65\linewidth]{Chai}
\caption[Energy of \ce{Ne2} with damping function.]{Dispersion energy of \ce{Ne2} with and without the damping function (equation \ref{equ:damp}). The function of Wu and Yang ($f_{damp}^{WY}(R)=1/(1+e^{-\frac{dR}{R_{r}-1}}))$ %, where $R_{r}$ is the sum of the \ac{vdw} radii, 
is also compared \cite{Wu2002}.  From the work of Chai \textit{et. al} \cite{Chai2008,Chai2008a}.} 
\label{fig:damp}
\end{figure}
The \ac{wb97xd} functional incorporates 100\% long-range exchange, 22\% short-range exchange, a modified B97 short-range exchange and B97 correlation, with empirical dispersion corrections. When the constraint $\omega$ is reduced to 0, \ac{wb97xd} is equivalent to the B97 functional with the addition of the dispersion correction. 
The performance of \ac{wb97xd} will be compared to that of \ac{B3LYP} with particular attention paid to the reaction energies.  
%fig of the damping function from Chai's paper, if you can be bothered 
%-------------------------------------------------------------------------------------------------------------------------------------------------------------------------------------------------%
%\ac{B3LYP}
%In the context of this report, the \ac{B3LYP} functional will be used where the exchange is described as follows:
%\begin{equation}
%\centering
%E_{B3LYP^{XC}} = E_{LDA^{X}} + c_{0} (E_{HF^{X}}-E_{LDA^{X}} ) + c_{X} \Delta E_{B88^{X}} + E_{VWN3^{C}} + c_{C} (E_{LYP^{C}} -E_{VWN3^{C}} )
%\end{equation}
%
%Where c$_{0}$ = 0.2, c$_{X}$ = 0.72 and c$_{C}$ = 0.81.
%The coefficient c$_{0}$ allows mixing of E$_{HF^{X}}$ (Hartree-Fock) and E$_{LDA^{X}}$ (Local Density Approximation). E$_{VWN3^{C}}$ (VWN3 local correlation) is mixed with E$_{LYP^{C}}$ (Lee, Yang, Parr correlation function) via c$_{C}$. E$_{B88^{X}}$ is Becke’s gradient-corrected exchange functional.
%-------------------------------------------------------------------------------------------------------------------------------------------------------------------------------------------------%
%\ac{wb97xd}
%-------------------------------------------------------------------------------------------------------------------------------------------------------------------------------------------------%
\subsection{Basis set approximation}
%Slater (difficult) -> Gaussian (easier) -> Combination (more accurate) -> Diffuse and polarisation
%make sure you know the caveats - ins and outs of the bases you used in combination with the methods you used. 
%You know pople's one's aren't as stable as the aug- ones. State why you chose them. Also, MP2 will react differently with these bases as opposed to the aug- ones. 
A basis set is the collection of mathematical basis functions, used in linear combination, to construct the \ac{MO}. An individual \ac{MO} can be defined as: %The individual function are derived from the hydrogen atomic orbitals, as it is 
\begin{equation}
\psi_{i}=\sum^{N}_{\mu=1} c_{\mu i} \chi_{\mu}
\end{equation}
where $c_{\mu i}$ is the molecular orbital expansion coefficient and $\chi_{\mu}$ corresponds to the one-electron occupied orbitals, often atomic orbitals, also called \textit{basis functions}. %I don't see why we don't just call it \chi_{i} again. So it's not 
%$\Nu$
%{\mathrm{N}}
$N$ is the total number of basis functions, which are all chosen to be normalised. 
%The convention is that Roman subscripts are used for molecular orbital functions, but Greek used on the basis functions. 
%Roman subscripts on MO functions and Greek on basis functions
The smallest possible basis set is a single basis per occupied orbital on each atom in a molecular system, termed the ``minimal basis''. The \ac{HF} limit is when the addition of further bases does not lower the energy of the system any further. 

\ac{STO} with exponential dependence $Ae^{-\alpha r}$, are the intuitive choice for atomic functions; $\alpha$  controls how quickly the function decays (radial extent), with higher values for higher effective nuclear charge. They are extremely similar in their mathematical expression to the real atomic orbital, however, mathematically challenging to implement in molecular orbital calculations. 
% http://theses.gla.ac.uk/925/1/2009davin1phd.pdf
% (size - radial extent) - higher values for higher effective nuclear charge (Jamie Platts)
%Challenging intergrals when atomic orbitals are centered on different nuclei
\ac{GTO} of dependence $Ae^{-\alpha r^{2}}$, offer a more practically viable approximation. Linear combinations of primitive gaussians are used to form the actual basis functions. The constructed basis functions are then called contracted gaussians, and have the form:
\begin{equation}
\chi_{\mu}=\sum_{p}d_{\mu p}g_{p}
\end{equation}
where $d_{\mu p}$ are fixed constants. 
The expansion to the \ac{MO} therefore takes the form:
\begin{equation}
\psi_{i}=\sum_{\mu} c_{\mu i} \Big (\sum_{p}d_{\mu p}g_{p}\Big)
\end{equation}
%Diagram
In this study, Pople style basis sets will be employed, with general format \textbf{X-YZg} \cite{Ditchfield1971}. Split valence basis sets such as these, describe the core electrons using fewer basis functions than the interacting valence electrons; the valence electrons play a much more significant part in bonding and intermolecular interactions.  %REF % (Table 1. Examples of split valence basis sets.).40,41
An example is \textbf{6-31+G*}, where the core electrons are described using a single contracted gaussian consisting of six primitive functions. The valence orbital is split into two contracted gaussians, where one is described by three primitive gaussians, and the other, one primitive gaussian. 
The \textbf{+} sign corresponds to diffuse functions, the extension of the basis function as it tails away from the nucleus. This particularly important when describing the behaviour of ions. The \textbf{*} indicates the addition of polarisation functions, in this case, d-functions to all non-H atoms. Higher order modifications also add p-functions to H atoms, additional d-functions to non-H atoms, d-functions to H atoms, with increasing levels of polarisation. This mixing of orbitals facilitates a better description for the correct shape of the \ac{MO}. 
Increasing the number of basis functions on each occupied orbital allows for expansion or contraction, increasing flexibility in response to the environment. 

Basis set superposition error \acs{BSSE} is a false lowering of the energy that can occur when two species in a system approach one another to form a complex. Particle A borrows the extra basis functions belonging to particle B and an artificial stabilisation is observed. The error arises from the inconsistency in treatment between the individual particles at large separations, and the complex at short distances. This is particularly pertinent when considering reaction complexes, and performing \ac{PES} scans where bond breaking or formation is expected. 
The effect is particularly pronounced for smaller basis sets. Counterpoise correction can be used to circumvent \ac{BSSE}, at the expense of higher computational cost. Preliminary investigations into the species examined in this study showed that the use of counterpoise correction offered minimal improvements to the energy, and the additional CPU time was not proportionate to the marginal improvements. Calculations were performed at sufficiently high basis that the effects of \ac{BSSE} were insignificant, and counterpoise correction was not used for further investigations. 
%-------------------------------------------------------------------------------------------------------------------------------------------------------------------------------------------------%
%\subsection{Thermochemistry}
%(And vibrational frequencies)
%-------------------------------------------------------------------------------------------------------------------------------------------------------------------------------------------------%
%\subsection{(Transition State Theory \& Free Energy Calculations)}
%-------------------------------------------------------------------------------------------------------------------------------------------------------------------------------------------------%
%\subsection{Solvent models}
%%Their effect on free energies - use a free energy cycle to justify the lack of inclusion of solvation energy - though this might be best discussed in Chap 1
%%How they are modelled.
%%PCM vs SMD, why is SMD better for free energies. Are they similar in terms of compuational demand
%-------------------------------------------------------------------------------------------------------------------------------------------------------------------------------------------------%
%%\subsection{Optimisation algorithms}

%-------------------------------------------------------------------------------------------------------------------------------------------------------------------------------------------------%
\section{Implementation \& Analysis}
The \ac{QM} calculations employed in this study typically involved a geometry optimisation, followed by a frequency calculation conducted in \ac{G09} \cite{Frisch2013}. 
This section will provide an generic overview of the methods and techniques used to analyse the optimised geometry output from a \ac{QM} calculation. 
Subsequent investigations into transition state structures, reaction coordinates and analysis of critical bonding points will be explained.  
The details of individual calculation schemes are included in the methodology section at the the start of each chapter. This includes any details of the \ac{QM} methods and basis sets chosen, individual optimisation procedures, types of calculation or special keywords used, and any non-standard software or technical details of importance. 
%
%\subsecttion{Geometry optimisation, single point energy calculation and thermochemistry.}
%All energy minimsations emplyed the Berny algorith, amd 
%
%Unless specified otherwies, all calcs run in G09 D.01 though perhaps make mention of materials studio, orca - no, Gview and Avogadro for quick cleanup. PyRed, what did I use to make the PBDs?
%Visualisation done in avogadro, molden, Gview
%
\subsection{Potential Energy Surface scans}
%[Maybe this goes into a smaller methodology section?]
Geometry scans or \ac{PES} scans were used to probe the local energy landscape corresponding to a specific change in geometry. During the course of a scan, a structural property - such as a selected bond length, angle or dihedral is adjusted in incremental steps, as specified by the given scan parameters. In the case of a relaxed scan, at each step, the adjusted parameter is frozen and a geometry optimisation is performed, allowing the rest of the system to relax around the modified bond.  Each scan yields a \acs{PES} of the explored pathway. In the case of bond breaking or formation, the presented energy profile is a reaction co-ordinate diagram. % (see Figure 17, section 2.4.1).
In these cases, an energy maximum followed by a trough indicates a transition state and intermediate reaction product, respectively. The structural co-ordinates at the peak of the curve can be extracted and used for subsequent frequency calculations or transition state search, to validate the mechanistic pathway. % using intrinsic reaction co-ordinate methods. 
Intrinsic reaction coordinate calculation also test the reactant and product along the reaction pathway, to ensure that they lie on the same \ac{PES}. 
To explore the predicted degradation mechanisms of \ac{NC}, the scanning parameters were assigned to the bonds undergoing the most significant transformation during a particular step of the mechanism. In the case that more than one significant bond was altered, multiple scans with different bond specifications were compared. In this way, multi-dimensional scans could be performed. Two-dimensional scans were used to probe simultaneous processes in the system, however, these proved computationally intensive due to the high number of coordinate points to be calculated. 
Relaxed geometry scans were performed on the optimised reactant geometry using the Opt=ModRedundant keyword.
A rigid scan consisted of a single point energy calculation of the structure at each step, rather than full relaxation of the wider system, as in the relaxed scans. 
%These are specified in \ac{G09} by selecting two internal coordinates to be scanned, and stating the number of steps.

\subsection{Transition state searches}
%[Maybe this goes into a smaller metholodogy section?] % critical mechanistic detail 
Transition state searches are called through the Opt=TS, QST or QST3 keywords in \ac{G09}. The Opt=TS method attempts to optimise the input ``guess'' geometry to a maximum. The guess geometry can be obtained \textit{via} a \ac{PES} scan, from coordinates in the literature, or generated using the QST2 function. 
A \ac{TS} can be identified by the single imaginary frequency corresponding to the vibration transitioning from product to reactant. 
\ac{TS} searching is known to be difficult.
In many cases, a \ac{TS} calculation alone will not be able to optimise to the correct transition state and is usually used in conjunction the QST2 or QST3 methods. 
The QST2 option is able to generate a transition state geometry using the Synchronous Transit Quasi-Newton (STQN) method \cite{Govind2003}. %42 REF. 
Here, the transition geometry is assumed to be around midway between a given reactant and product; the calculation interpolates between the starting and end point, probing the energy profile. Thus, the corresponding atom labels must match in both the reactants and products. QST3 performs a similar function, but also considers in input guess transition state structure. % (Figure 10). 
It is widely acknowledged that transition state searching is challenging. In addition to the techniques above, the task requires a certain measure of chemical intuition when constructing guess structures, as the calculation is sensitive to small changes in geometry.
%“Scan” means that the reaction coordinates are changed gradually and “relaxed” means that the reaction coordinates are fixed, while other coordinates that are orthogonal to the reaction coordinates are relaxed during the energy minimisation \cite{Yao2018}.

%Imaginary frequencies and the Hessian
%If the second derivative is positive at a point, the graph is concave up. If the second derivative is positive at a critical point, then the critical point is a local minimum.
%If the second derivative is negative at a point, the graph is concave down. If the second derivative is negative at a critical point, then the critical point is a local maximum.
%An inflection point marks the transition from concave up and concave down. The second derivative will be zero at an inflection point.

%%(within Gaussian? See if you end up using any)
%%Frequency calculations
%\subsection{Intrinsic Reaction Coordinate}
%%[Maybe this goes into a smaller metholodogy section?]
%IRC calculations begin at the saddle point and descend the PES in both the forwards and backwards direction of the reaction co-ordinate dictated by the normal mode of the imaginary frequency. In a similar manner to a geometry scan, geometry optimisations are performed at each step point. Its purpose is to connect the two minima leading to the found transition state, thereby confirming whether the found transition state corresponds to your reactants and products of interest. 
%IRC calculations are called via the IRC keyword, with specifications of whether the forward or backward reaction is to be scanned, step size and maximum number of steps allowed.
%
%%not sure whether to talk about the transistion state searching methods / algorithms here or elsewhere, or whether to go into detail at all.
%-------------------------------------------------------------------------------------------------------------------------------------------------------------------------------------------------
\subsection{Topology analysis using the quantum theory of atoms in molecules}
%Some real space functions in Multiwfn are available for topology analysis, such as electron density, its Laplacian, orbital wave function, ELF, LOL ...
Topology analysis is a method of obtaining useful properties from the 3D representation of the electron density obtained from a \ac{QM} calculation. 
A quantitative way to obtain information on the topology of the electron density, is by taking the firstd derivative of the gradient ($\nabla(\rho)$). 
In 1991, Bader proposed a technique to analyse the electron density using the \ac{QTAIM} \cite{Bader1991}.
 %, which also extended to other real space functions, such as the \ac{LOL} and \ac{ELF}. - Only include these if you can describe them, also. 
 % Real-space calculation...when all operators (T^,V^,J^,K^) are evaluated on a grid.
 %
%The topology analysis technique proposed by Bader was firstly used for analysing electron density in "atoms in molecules" (AIM) theory, which is also known as "the quantum theory of atoms in molecules" (QTAIM), this technique has also been extended to other real space functions, e.g. the first topology analysis research of ELF for small molecules is given by Silvi and Savin.%, see Nature, 371, 683. 
The points in the topological landscape at which $\nabla(\rho)$ is zero (excluding points at infinity) %does infinity happen? gradient norm 
% wiki: a norm is a function that assigns a strictly positive length or size to each vector in a vector space—except for the zero vector, which is assigned a length of zero.
% A bit like modulus but referers to specific scenarios: https://www.physicsforums.com/threads/what-is-the-difference-between-norm-and-modulus.181812/
signify a stationary point. With respect to intermolecular interactions, these correspond to %areas of favourable 
interaction centres, and are deemed Critical Points (\acs{CP}). The matrix of partial second derivatives of the gradient is referred to as the \textit{Hessian}. 
%http://www.chem.gla.ac.uk/~louis/xdworkshop/workshop/documentation/jyvaskla_4.pdf
The Hessian is a (3$\times$3) symmetric matrix; diagonalisation sets the off-diagonal terms to zero and generates the principal axes of curvature. 
The sum of the diagonal terms returns the laplacian of the electron density ($\nabla^{2}(\rho)$); evaluation of $\nabla^{2}(\rho)$ identifies the characteristic of the \ac{CP}. 
% the second derivative of the gradient (
In the context of electron density, the \ac{CP} can be classified into four types based on the number of negative eigenvalues of the Hessian (table \ref{tab:cps}):
%CPs can be classified into four types according to how many eigenvalues of Hessian matrix of real space function are negative.
%At this point you shoudl have already explained what the hessian matrix is
%please explaint the (3,X) labelling
\begin{table}[htp]
\caption{Features of different types of critical point from \ac{QTAIM} topological analysis.}
\begin{center}
%\begin{tabular}{l c p{2cm} p{3cm} p{3cm}} 
\begin{tabular}{l l p{2cm} p{3cm} p{4cm}} 
\toprule
%derivation
Critical Point & Label  & Negative eigenvalues & Attribute & Representation \\
\midrule
Nuclear (\acs{NCP}) & (3,-3)  & 3  & Local maximum & Atomic nuclei \\
\hline
Bonding (\acs{BCP}) & (3,-1) & 2 & 2\textsuperscript{nd} order saddle point & Bonding site \\
\hline
Ring (\acs{RCP}) & (3,+1) & 1  & 1\textsuperscript{st} order saddle point & Steric point or centre of ring system \\
\hline
Cage (\acs{CCP}) & (3,+3) & 0& Local minimum & Centre of cage system \\
\bottomrule
\end{tabular}
\label{tab:cps}
\end{center}
\end{table}

Nuclear critical points (\acs{NCP}) are so called, as they are generally located at atomic nuclei positions. All three eigenvalues of the Hessian matrix of the function are negative, corresponding to a local maximum on the electron density landscape. The number of \acs{NCP} is usually equal to the number of atoms, though there are exceptions, such as in \ce{Li2} which has a greater number of \acs{NCP} than atoms, and for \ce{KrH+}, with a lower number. %Why???%, 
%For electron density analysis and for heavy atoms, the position of (3,-3) are nearly identical to nuclear positions, hence (3,-3) is also called nuclear critical point (NCP). Generally the number of (3,-3) is equal to the number of atoms, only in rarely cases the former can be more than (e.g. Li2) or less than (e.g. KrH+) the latter.
%
%Btw real space.... just means space. As in actual 3D space. Not virtual space. 
%In a electronic structure (specifically DFT) code, a "real space code" means using atom orbitals as basis functions rather than plane waves.
% https://chemistry.stackexchange.com/questions/84098/what-are-the-differences-between-the-hilbert-space-and-real-space-representation
When two eigenvalues of the Hessian matrix of a function are negative, a second-order saddle point is present. These sites are usually located between attractive atom pairs, and so are referred to as a \ac{BCP}. %The value of real space functions at BCP have great significance. %, for example the value of XXX and 
As the electron density at the \ac{BCP} is closely related to bond strength, the magnitude of $\nabla^{2}(\rho)$ will give an indication of the bonding type \cite{Matta2007}. % (The Quantum Theory of Atoms in Molecules-From Solid State to DNA and Drug Design, p11); 
% potential energy density at \ac{BCP} has been shown to be highly correlated with hydrogen bond energies% (Chem. Phys. Lett., 285, 170)
%; local information entropy at BCP is a good indicator of aromaticity.% (Phys. Chem. Chem. Phys., 12, 4742).
If one eigenvalue of the Hessian is negative, there exists a first-order saddle point.  %transition state in potential energy surface). For electron density analysis, 
This generally appears in the center of ring systems and highlights a local steric effect, hence it is known as a \ac{RCP}. 
When none of the eigenvalues are negative, it corresponds to the local minimum. For electron density analysis, these points generally appear in the center of cage systems, such as in pyramidal structure. These are referred to as a \ac{CCP}. 

The positions of \ac{CP}s are searched by the Newton method, %REF?
an initial guess point must be assigned; the minimisation will always converge to the CP that is closest to the guess point. 
By assigning different guesses and iterating over each of them, all \ac{CP}s can eventually be found. 
Once searches of \ac{CP}s are finished, the Poincaré-Hopf relationship \cite{Balanarayan2003} can be used to verify that the obtained topology is self-consistent, and that all \ac{CP}s are found. For an isolated system, the relationship states that:
\begin{equation}
\centering
n(3,-3) – n(3,-1) + n(3,+1) – n(3,+3) = 1
\end{equation}
%If the relationship is unsatisfied, then some of CPs must be missing, you may need to try to search those CPs by different guesses. However even if the relationship is satisfied, it does not necessarily mean that all CPs have been found. Notice that the function spaces of ELF/LOL and Laplacian of XX are much more complex than XXX, it is very difficult to locate all CPs for these functions, especially for middle and large system, so, you can stop trying for searching CPs once all CPs that you are interested in have been found.

The gradient path linking \ac{BCP}, or two local maxima, can be referred to as a ``bond path''. Presence of a bond path indicates an atomic interaction path %avenue of interaction?
available for a variety %any?
bonding interactions; these will lie along covalent bonds, as well as non-covalent interactions such as intra molecular bonding or H-bonding. A network of bond paths is known as molecular graph. Scrutiny of the graph returns a reliable image of the geometry of the molecule. %And highlights non-bonding interaction centre in the molecule.
% . Bond path can be straight line or curve, obviously for the latter case the length of bond path is longer than the sum of the distances between BCP and associated two (3,-3) CPs.
%Let us see an example. In the complex shown below, the imidazole plane is vertical to magnesium porphyrin plane, the nitrogen in imidazole coordinated to magnesium. Magenta, orange and yellow spheres correspond to (3,-3), (3,-1) and (3,+1) critical points, brown lines denote bond paths.
%-------------------------------------------------------------------------------------------------------------------------------------------------------------------------------------------------
%\section{Spectroscopy}
%
%\subsection{Nuclear Magnetic Resonance Spectra}
%Refer them to an actual resource, so you don't have to explain the theory of \ac{NMR}, but give an overview of the experimental, then give the theoretical detail about how Gaussian calculates it, and any discrepancies between experimental and calculated. 
%
%Comment on the different methods used to calculate NMR parameters - a tiny literature review, if you will, and say which you'll be using, why - even if it's becuase of the fact that it's built into Gaussian (obviously big it up, if that is the case) - and state any caveats. 
%
%Maybe have a look at: Accurate Calculation of NMR Chemical Shifts, Jurgen Gauss, 1995 as a basis, but you'll also need something more current. Any review papers out there?
%-------------------------------------------------------------------------------------------------------------------------------------------------------------------------------------------------%
%\subsection{Infra-Red spectra}
% Already covered in the thermochemistry section? Maybe just refer to that, but the include the scaling factors in here
%\subsection{Nuclear Magnetic Resonance spectra}
%-------------------------------------------------------------------------------------------------------------------------------------------------------------------------------------------------
% Partial charges (?)
%
% \subsubsection{Solvent models}
% \subsection{Optimisation algorithms}
%-------------------------------------------------------------------------------------------------------------------------------------------------------------------------------------------------%
%\subsubsection{Computational implementation}
%Caveats and scaling factors. 
% Computational chemistry now forms a fundamental part of chemical research either in conjunction with laboratory experimentation or independently. 
%Big clusters
%
%Solvent
%Summary paragraph?