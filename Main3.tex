% UCL Thesis LaTeX Template
%  (c) Ian Kirker, 2014
% 
% This is a template/skeleton for PhD/MPhil/MRes theses.
%
% It uses a rather split-up file structure because this tends to
%  work well for large, complex documents.
% We suggest using one file per chapter, but you may wish to use more
%  or fewer separate files than that.
% We've also separated out various bits of configuration into their
%  own files, to keep everything neat.
% Note that the \input command just streams in whatever file you give
%  it, while the \include command adds a page break, and does some
%  extra organisation to make compilation faster. Note that you can't
%  use \include inside an \include-d file.
% We suggest using \input for settings and configuration files that
%  you always want to use, and \include for each section of content.
% If you do that, it also means you can use the \includeonly statement
%  to only compile up the section you're currently interested in.
% You might also want to put figures into their own files to be \input.

% For more information on \input and \include, see:
%  http://tex.stackexchange.com/questions/246/when-should-i-use-input-vs-include


% Formatting rules for theses are here: 
%  http://www.ucl.ac.uk/current-students/research_degrees/thesis_formatting
% Binding and submitting guidelines are here:
%  http://www.ucl.ac.uk/current-students/research_degrees/thesis_binding_submission

% This package goes first and foremost, because it checks all 
%  your syntax for mistakes and some old-fashioned LaTeX commands.
% Note that normally you should load your documentclass before 
%  packages, because some packages change behaviour based on
%  your document settings.
% Also, for those confused by the RequirePackage here vs usepackage
%  elsewhere, usepackage cannot be used before the documentclass
%  command, while RequirePackage can. That's the only functional
%  difference as far as I'm aware.
\RequirePackage[l2tabu, orthodox]{nag}


% ------ Main document class specification ------
% The draft option here prevents images being inserted,
%  and adds chunky black bars to boxes that are exceeding 
%  the page width (to show that they are).
% The oneside option can optionally be replaced by twoside if
%  you intend to print double-sided. Note that this is
%  *specifically permitted* by the UCL thesis formatting
%  guidelines.
%
% Valid options in terms of type are:
%  phd
%  mres
%  mphil
%\documentclass[12pt,phd,draft,a4paper,oneside]{ucl_thesis}
\documentclass[11pt,phd,a4paper,twoside]{ucl_thesis}


% Package configuration:
%  LaTeX uses "packages" to add extra commands and features.
%  There are quite a few useful ones, so we've put them in a 
%   separate file.
% -------- Packages --------
% Fix the unicode symbols (arrows) issue, causing it to not compile on local machine.
% Seems to be fine on Overleaf
%\usepackage[utf8]{inputenc}
%\DeclareUnicodeCharacter{27f6}{\ensuremath{⟶}}
\DeclareUnicodeCharacter{2212}{-}
%\usepackage[utf8]{inputenc}

% This package just gives you a quick way to dump in some sample text.
% You can remove it -- it's just here for the examples.
\usepackage{blindtext}

% This package means empty pages (pages with no text) won't get stuff
%  like chapter names at the top of the page. It's mostly cosmetic.
\usepackage{emptypage}

% The graphicx package adds the \includegraphics command,
%  which is your basic command for adding a picture.
\usepackage{graphicx}

% A: For inserting multiple pics under one caption.
% See here: https://www.latex-tutorial.com/tutorials/figures/
\usepackage{subcaption}
%\captionsetup[subfigure]{font={bf,small}, skip=1pt, singlelinecheck=false}
% Get subfigure labels to the left of the image
\captionsetup[subfigure]{justification=raggedright,singlelinecheck=false,subrefformat=parens}

% The float package improves LaTeX's handling of floats,
%  and also adds the option to *force* LaTeX to put the float
%  HERE, with the [H] option to the float environment.
\usepackage{float}
% Just for the scheme thing, tbh.
\usepackage{newfloat}
\DeclareFloatingEnvironment[
  fileext = los ,
  listname = {List of Schemes} ,
  name = Scheme
]{scheme}

% The amsmath package enhances the various ways of including
%  maths, including adding the align environment for aligned
%  equations.
\usepackage{amsmath}

\usepackage{braket}

% Use these two packages together -- they define symbols
%  for e.g. units that you can use in both text and math mode.
\usepackage{gensymb}
\usepackage{textcomp}
% You may also want the units package for making little
%  fractions for unit specifications.
%\usepackage{units}

% Just for the text symbols
\usepackage{wasysym}
% Just to fix lack of space after the TM symbol in introduction tbh!
\usepackage{xspace}

% SI units - mostly angstrom
%\usepackage[round-mode=places,round-precision=4]{siunitx}
\usepackage[round-mode=places,round-precision=2]{siunitx}
% Don't know whether the following actually works - adding calories to siunitx
\DeclareSIUnit{\calorie}{cal}
\DeclareSIUnit{\Calorie}{\kilo\calorie}

\usepackage{xcolor}


% The setspace package lets you use 1.5-sized or double line spacing.
\usepackage{setspace}
\setstretch{1.5}

% That just does body text -- if you want to expand *everything*,
%  including footnotes and tables, use this instead:
%\renewcommand{\baselinestretch}{1.5}

% The command \renewcommand{\thefootnote}{\roman{footnote}} sets the number styles to lowercase roman. Other possible styles are:
%
% \arabic Arabic numerals. (The default)
% \Roman Upper case Roman numerals.
% \alph Alphabetic lower case.
% \Alph Alphabetic upper case.
% \fnsymbol A set of 9 special symbols.
\renewcommand{\thefootnote}{\fnsymbol{footnote}}


% PGFPlots is either a really clunky or really good way to add graphs
%  into your document, depending on your point of view.
% There's waaaaay too much information on using this to cover here,
%  so, you might want to start here:
%   http://pgfplots.sourceforge.net/
%  or here:
%   http://pgfplots.sourceforge.net/pgfplots.pdf
%\usepackage{pgfplots}
%\pgfplotsset{compat=1.3} % <- this fixed axis labels in the version I was using

% PGFPlotsTable can help you make tables a little more easily than
%  usual in LaTeX.
% If you're going to have to paste data in a lot, I'd suggest using it.
%  You might want to start with the manual, here:
%  http://pgfplots.sourceforge.net/pgfplotstable.pdf
%\usepackage{pgfplotstable}

% These settings are also recommended for using with pgfplotstable.
%\pgfplotstableset{
%	% these columns/<colname>/.style={<options>} things define a style
%	% which applies to <colname> only.
%	empty cells with={--}, % replace empty cells with '--'
%	every head row/.style={before row=\toprule,after row=\midrule},
%	every last row/.style={after row=\bottomrule}
%}

% Again, prettier tables
\usepackage{booktabs}
\usepackage{makecell}
% The mhchem package provides chemistry formula typesetting commands
%  e.g. \ce{H2O}
\usepackage[version=3]{mhchem}

% And the chemfig package gives a weird command for adding Lewis 
%  diagrams, for e.g. organic molecules
\usepackage{chemfig}

% The linenumbers command from the lineno package adds line numbers
%  alongside your text that can be useful for discussing edits 
%  in drafts.
% Remove or comment out the command for proper versions.
%\usepackage[modulo]{lineno}
% \linenumbers 


% Alternatively, you can use the ifdraft package to let you add
%  commands that will only be used in draft versions
%\usepackage{ifdraft}

% For example, the following adds a watermark if the draft mode is on.
%\ifdraft{
%  \usepackage{draftwatermark}
%  \SetWatermarkText{\shortstack{\textsc{Draft Mode}\\ \strut \\ \strut \\ \strut}}
%  \SetWatermarkScale{0.5}
%  \SetWatermarkAngle{90}
%}


% The multirow package adds the option to make cells span 
%  rows in tables.
\usepackage{multirow}


% Subfig allows you to create figures within figures, to, for example,
%  make a single figure with 4 individually labeled and referenceable
%  sub-figures.
% It's quite fiddly to use, so check the documentation.
%\usepackage{subfig}

% The natbib package allows book-type citations commonly used in
%  longer works, and less commonly in science articles (IME).
% e.g. (Saucer et al., 1993) rather than [1]
% More details are here: http://merkel.zoneo.net/Latex/natbib.php
%\usepackage{natbib}

% The bibentry package (along with the \nobibliography* command)
%  allows putting full reference lines inline.
%  See: 
%   http://tex.stackexchange.com/questions/2905/how-can-i-list-references-from-bibtex-file-in-line-with-commentary
\usepackage{bibentry} 

% The isorot package allows you to put things sideways 
%  (or indeed, at any angle) on a page.
% This can be useful for wide graphs or other figures.
%\usepackage{isorot}

% The caption package adds more options for caption formatting.
% This set-up makes hanging labels, makes the caption text smaller
%  than the body text, and makes the label bold.
% Highly recommended.
\usepackage[format=hang,font=small,labelfont=bf]{caption}

% If you're getting into defining your own commands, you might want
%  to check out the etoolbox package -- it defines a few commands
%  that can make it easier to make commands robust.
\usepackage{etoolbox}

% FIX CAPITILISATION AT BEGINNING OF SENTENCES
%abbreviations from this example: https://www.overleaf.com/latex/examples/automatic-acronym-list-in-latex/dzvxfzpsjrmm#.W08mNNUzphE
\usepackage{acronym}
%\usepackage{nomencl}


%greek in text
\usepackage{textgreek}

%font, doesn't work though. But UCL suggests sans-serif fonts
%\renewcommand{\sfdefault}{arial}

% List of words that should not be split across lines:
%\hyphenation{-OCH\textsubscript{3}}
%\hyphenation{-OH} 	

% Sets up links within your document, for e.g. contents page entries
%  and references, and also PDF metadata.
% You should edit this!
\input{LinksAndMetadata}
%\renewcommand{\textrightarrow}{$\rigtharrow$}

% And then some settings in separate files.
\input{FloatSettings} % For things like figures and tables
\bibliographystyle{unsrt}
% Change it to a different - perhaps more informative style, later   % For bibliographies

% These control how many number sections your subsections will take
%    e.g. Section 2.3.1.5.6.3
%  and how many of those will get put into the contents pages.
\setcounter{secnumdepth}{5}
\setcounter{tocdepth}{2}

\newenvironment{block_indent}[1]%
  {\begin{list}{}%
          {\setlength{\leftmargin}{#1}}%
          \item[]%
  }
  {\end{list}}

\begin{document}

\nobibliography*
% ^-- This is a dumb trick that works with the bibentry package to let
%  you put bibliography entries whereever you like.
% I used this to put references to papers a chapter's work was 
%  published in at the end of that chapter.
% For more information, see: http://stefaanlippens.net/bibentry

% If you haven't finished making your full BibTex file yet, you
%  might find this useful -- it'll just replace all your
%  citations with little superscript notes.
% Uncomment to use.
%Can't seem to use this with local compiler (Miktex) Make sure it's commented with local compile, otherwise will get error message and crash. 
%\renewcommand{\cite}[1]{\emph{\textsuperscript{[#1]}}}

% At last, content! Remember filenames are case-sensitive and 
%  *must not* include spaces.
%\cleardoublepage\phantomsection
\label{List-of-all}
\setcounter{page}{0}
\newenvironment{interlude}{
	\clearpage
	\thispagestyle{empty}
	\pagestyle{empty}
}

{\Large List of all activities / micro-topics explored during PhD:}
\newenvironment{changemargin}[2]{%
}

%\end{changemargin}
%\end{interlude}

%%% I may change the way this is done in a future version, 
%%  but given that some people needed it, if you need a different degree title 
%%  (e.g. Master of Science, Master in Science, Master of Arts, etc)
%%  uncomment the following 3 lines and set as appropriate (this *has* to be before \maketitle)
%% \makeatletter
%% \renewcommand {\@degree@string} {Master of Things}
%% \makeatother

\title{Computational Study of the Ageing Processes in Nitrocellulose}
\author{Amy J. Lai}
\department{Department of Chemistry}

\maketitle
\makedeclaration

%\cleardoublepage\phantomsection
\addcontentsline{toc}{chapter}{Abstract}
\begin{abstract} % 300 word limit
%My research is about stuff.
%It begins with a study of some stuff, and then some other stuff and things.
The dominant degradation pathways of \ac{NC} were investigated, using \ac{QM} methods to probe the primary denitration routes, followed by key secondary reactions.  %by looking at the explored degradation routes in alternative nitrate esters, and tested for \ac{NC} a truncated model for the wider polymer
The polymer structure was truncated in order to facilitate \ac{DFT} studies into the mechanistic details of denitration at individuate nitrate sites. 
Comparison of monomer, dimer and trimer units of the polymer using \ac{QTAIM} topology analysis of interaction sites, analysis of the \ac{ESP} and charges showed that the most suitable model for study of the decomposition reactions was \textbeta-glucopyranose monomer, bi-capped with methoxy groups. 
%This was determined via analysis of the \ac{QTAIM} \ac{BCP} between the capping groups and the wider monomer, when changing from hydroxyl and methoxy homogeneous or mixed groups at the C1 and C4 sites on the ring. 
The model was nitrated at the C2 position, to mimic the most stable nitration site \cite{Shukla2012a}.
%first site of nitration or last site of denitration. %REF
The primary thermolytic and hydrolytic denitration routes were explored using \ac{TS} searches and \ac{PES} scans. It was found that the thermolytic behaviour of the \ac{NC} denitration step matched the energy profile of other nitrate esters \cite{Tsyshevsky2013}. 
Protonation at the bridging oxygen site of the nitrate was found to be the most likely to lead to denitration. It was not possible to isolate a \ac{TS} for the hydrolytic reaction, though a number of coordination schemes were tested. 
Secondary processes following initial denitration were examined. Ethyl nitrate was used as a test system before extension to the monomer. Different reaction pathways for decomposition, with forward propagation of the evolved species to the reaction step, revealed that \ce{^{.}NO2} was the most likely cause for the the experimentally observed autocatalytic rate of degradation. 
\end{abstract}

%\addcontentsline{toc}{chapter}{Impact Statement}%%%%%%%%%%%%%%%%%%
%\begin{impact}
%This study examines the degradation properties of \ac{NC}, shedding light on the mechanistic details of decomposition that are yet currently either unknown, or are unclear with contradictory views in existing literature.  
%\ac{NC} is used in a massive variety of products, with utilisation in household, industrial, military and medicinal applications. 
%
%Improved understanding of the fundamental chemistry in the degradation of \ac{NC} could benefit in the following key impact areas: \\%Despite its long history, the majority of information on the subject is only based on ag
%
%\noindent\textbf{Knowledge \& collective benefit} %(for the common good) and expertise and education
%\begin{itemize}
%\item Broaden understanding and depth of knowledge in the area of \ac{NC} degradation, with the view to validate or reject conflicting schemes in literature. 
%%eludcidate for reserarch and commmon knowedlge
%%\noindent 
%\item Improve conservation procedures for existing and legacy \ac{NC} products, such as cinematographic film, artworks and historical munitions with better understanding of the handling procedures for these aged products. % \cite{Service1999}
%\end{itemize}
%\noindent\textbf{Environmental}
%%With increasing oil prices and forecasts of a future lack of availability, renewable non-petrochemical-based alternatives to materials synthesis could become more important. 
%
%%\item 
%\noindent Design next generation \ac{NC} products more environmentally friendly, in terms of durability and recycle-ability.
%
%%\item 
%\noindent Improve and clean up industrial processes for more environmentally benign production process.
%%It could mean that \ac{NC} 	\textendash which is not derived from petrochemicals \textendash\  can help fight the plastic / waste disposal problem. %It's non-toxic to humans (CHECK) and so in its degradation, could also help fight the micro-plastics ingestion problem, in the long term. (basically, polysaccharides, like PLA etc, could assist in the long term transition from petrochemical plastics to biodegradability. Though \ac{NC} isn't necessary biodegradable without specific microbes found in the lab yet - better understanding of its breakdown could facility an easier process for this.)
%
%%\item 
%\noindent Design appropriate \ac{NC} disposal methods, other than the existing harsh chemical or incineration treatment, with opportunity to feed back into other processes, such as the production of fertiliser. %insert into/  modify industrial process to re-formulate into other products, such as fertilisers etc. with leftover \ac{NC} etc with opportunity to streamline to more environmentally friendly methods,
%
%
%\noindent\textbf{Safety}
%
%%\item 
%\noindent Guidelines on safety and usage could be refined based on detailed knowledge of the degradation reactions, instead of current guidelines based on more crude, aggregate experimental observations.
%%with better understanding of the actual reactions that take place, instead of a broad-brush style guidelines based on historical \/ less precise experimental observations.
%%Current decomposition mechanisms are speculated based on aggregate experimental measurements giving a crude understanding of the degradation patterns in the material.
%
%\noindent\textbf{Industrial \& Commercial}
%
%%\item 
%\noindent More adaptability in the production of \ac{NC} and its formulation into products, in the case that the variety of source cellulose feedstock changes due to supply chain issues. 
%
%%\item 
%\noindent Streamline industrial processes to specific reactivity requirements, based on detailed mechanistic considerations, leading to cost saving.%  if we know what can be trimmed - commerical benefits by cutting waste, optimising work flows and tailoring processes to exact instead of wide-margin .
%
%\noindent\textbf{Innovation}
%
%%\item 
%\noindent Create new products, not limited by previous assumptions about the shelf life and reactivity. % etc (obviously this is a big generalisation. IT might be that the current industrial standards are ok / not too far off. But at least this might mean better/ more tailored / cheaper stabiliser / plasticiser formulations can be used, or things that are more readily available and more options, in case certain things become unavailable.) 
%
%%\item 
%\noindent Facilitate the design of products for applications that were previously not known that the material was suitable for.% we didn't know it was suitable for, before
%
%%\end{itemize}
%\end{impact}

%%\cleardoublepage\phantomsection
%\addcontentsline{toc}{chapter}{Acknowledgements}
%\begin{acknowledgements}
%The completion of this body of work %the hours behind it, emotional rollercoasters etc
% could not have happened if it were not for the people who dedicated time guiding it to completion. \\
%Firstly, thank you to Professor Nora de Leeuw for the opportunity to undertake this study, and for granting me the chance to visit conferences and research exchanges.
%
%%, and for taking time to check on me periodically, even with your busy schedule.
%A huge thanks to Professor Graham Worth for taking on a precarious t all his time and dedication spent %on guiding a lost PhD student
%Dr David Santos Carballal %for driving the study on the phase diagram stuff
%Professor David Scanlon for his assistance in navigating a difficult % taking me on in group meetings. Always asking how I am when we bump in the corridor. Always being real. Looking out for PhD students, and actually caring.
%Professor Ivan Parkin %again, for just caring 
%Professor Jim Anderson for being interested, taking the time to meet with me and always responding to my %pleas for help
%questions.
%Dr Zhimei Du
%The office crew - for keeping me alive with boba and love. 
% Special shoutout to the people who read chapters - Emilia, James, Jordan for reading
% Gabiee for always bein there and looking out for me, all these years. Been with me the whole way. 
% And Filipe too, for letting me have Gabie time, and opening the doors to their home to me too. 
%Both for looking after me so well!
%Dani - for doing what he could, for the encouragement, and being patient and understanding. 
%Lisa Patrick - being there during my insanity, offering to read
%Paul and Hafsa too
%Sarah
%Jordan, for housing me, encouraging me, being there for me
% Jona
% Olivia and Fred for letting me in their home. Fred for the banging playlists
% \newpage 
% Now that everyone else has given up on reading, this is to acknowledge the real MVP's of this PhD, without whom my days would have greyer and dryer (and by that I mean dry of bubble tea). \\
% The crew at 205, of whom I will be the last to graduate:\\
% Dr Abdul Rashidi + open mic\\
% Dr Lisa Richards + south africa trip\\
% Dr Qamreen Parker + boba crew\\
% Dr Emilia Olsson + reading\\
% Dr Simon Austin + cheese\\
% %Dr James Pegg + dinosaur memes\\
% Kit & Ceridwen
%Stephen in student records - for turning my PhD extension application around, so quickly. 
%
%We acknowledge the support of the Supercomputing Wales project, which is part-funded by the European Regional Development Fund (ERDF) via Welsh Government.
%\end{acknowledgements}


\setcounter{tocdepth}{3} 
% Setting this higher means you get contents entries for more minor section headers.

\tableofcontents

%\cleardoublepage\phantomsection
\addcontentsline{toc}{chapter}{List of Figures}
\listoffigures

%\cleardoublepage\phantomsection
\addcontentsline{toc}{chapter}{List of Tables}
\listoftables


%% In order to set an abbreviation: \ac{ }
% To force the shorthand version: \acs
% See here for more info and options: http://ramibaddour.com/2017/01/18/latex-working-with-acronyms/

\cleardoublepage\phantomsection\addcontentsline{toc}{chapter}{Abbreviations}
\label{abbreviations}
\abbreviations

\begin{acronym}[CAM-B3LYP] % Give the longest label here so that the list is nicely aligned

%Supress first
\acro{B3LYP}{Becke, 3-parameter, Lee-Yang-Parr hybrid functional}
\acro{CP}{critical point}
\acro{DFT}{density functional theory}
\acro{DOS}{degree of substitution}
\acro{EM}{energetic materials}
\acro{ESP}{electrostatic potential}
\acro{G09}{Gaussian 09 revision D.01}
\acro{GM} genetically modified
\acro{GView}{Gauss View 5.0.8}
\acro{HF}{Hartree Fock theory}
\acro{MM}{molecular mechanics}
\acro{MW}{molecular weight}
\acro{NC}{nitrocellulose}
\acro{NG}{nitroglycerine}
\acro{NMR}{nuclear magnetic resonance spectroscopy}
\acro{QM}{quantum mechanics}
\acro{QTAIM}{quantum theory of atoms in molecules}
% Quick fix: Supress the first expansion of this by using acs{} for SN2 specific ones. 
\acro{sn2}[S\textsubscript{N}2]{bi-molecular nucleophilic substitution reaction}


\end{acronym}




 %NOTE: abbreviations come out funny unless you compile both Abbreviations and Preamble
%\chapter{Introduction}
\label{chapterlabel1}
\graphicspath{ {./Intro_pics/} }

\section{The modern history of Nitrocellulose}

%Re-write all this better. And make it fun please. \\
\ac{NC}, cellulose nitrate, or "guncotton" is a nitrated cellulose derivative that has been widely utilised in the manufacture of plastics, inks, propellant formulations and thin films
%"micro-electronics as a resist material" 
%a low-order explosive 
since it's discovery in 1833. [PICTURE!] %(REF R Shorts' thesis, see references 1-5). 

Cellulose is the primary component of plant cell walls (REF). It was found (by who?) the when cellulose fibres, in the form of sawdust, cotton (linen) and paper were treated with nitric acid, (it was discovered that) the product burned rapidly and in the absence of the thick black smoke that was characteristic of gunpowder of the era.%REF E. C. Wordon and E. C. Worden, Nitrocellulose Industry, 1911
%OR
In his Berlin laboratory in 1845, Dr. Schönbein serendipitously discovered that acid treated cotton wool burned violently without the accompanying black smoke that was typical of gunpowders of the time. %REF A. S. Greenberg and S. Broder, J. Pat. Off. Soc., 1925
He was granted a U.S. patent the following year. 

Today \ac{NC} is present in adhesives, paints and lacquer coatings. Varieties of high percentage nitrogen by weight (\%NMax) are used for its explosive properties, such as in rocket and gun propellants.  Since  Schönbein’s day \ac{NC} has seen applications in dynamite, artificial silks, printing inks and reels of old film, which were famously flammable. %REF M. Edge, N. S. Allen, 1990 
Collodin, a solution of \ac{NC} in either ether or ethanol, is used both in surgical dressings and in theatrical make-up. The non-flexible variety is applied to skin, which puckers as the solvent evaporates, effectively creating the appearance of scarred tissue. [PICTURE!]


Cellulose is abundant in nature, however its biological variation has implications on the quality and properties of the \ac{NC}it produces. %[REF of miscanthus cellulose SEM - caption "SEM of (Italic) Miscanthus (un_    Italic) cellulose fibres (a) and following nitration (b). The synthesized NC   
During synthesis the backbone of cellulose is largely preserved (Figure conversion of C to NC).%REF A. S. Greenberg and S. Broder, J. Pat. Off. Soc., 1925 and H. J. Prask, C. S. Choi, R. Strecker and E. Turngren, Woodpulp Crystal Structure 
Wood pulp and cotton linters are two major sources of starting material for \ac{NC} production; though more expensive, cotton linters are preferred for their uniformity. 


\ac{NC} of above 12.5 \%NMax is considered explosive. The theoretical maximum level of nitration is 14.14 \%NMax, corresponding to a \ac{DOS} of three, where every free hydroxyl group has been replaced by a nitrate ester. In practice only 13.6 \%NMax has be achieved. %REF Dow Wolff Cellulosics, Walsroder® Nitrocellulose Essential 1998
When \ac{NC} is the only energetic component in an explosive mixture, it is termed a single-base propellant. \ac{NC} mixtures with one or more energetic materials such as \ac{NG}, where \ac{NG} also acts as a plasticiser, are classed as double or triple-base propellants. Due to the degradation of \ac{NC} over time, stabilisers are necessary to neutralise the decomposition products that facilitate further reactions.


% Background story and industrial applications and properties of \ac{NC}.
% Briefly explain why we need to know more about its degradation processes. 

Some stuff about things.\cite{example-citation} Some more things. 
%Inline citation: \bibentry{example-citation}

\begin{figure}[htp]
\centering
%\includegraphics[width=\textwidth]{NC}
\includegraphics[scale=0.5]{NC}
\caption{The structure of Nitrocellulose.}
\label{fig:NCstructure}
\end{figure}

% Taken from Rigas, Sebos, Doulia, 1997:
% "The nitration of organic compounds is a potentially
% dangerous process, since nitrations are exothermic
% reactions producing under controlled conditions explosive
% substances. Many accidental explosions of nitration
% industrial plants have been reported in the literature
% (Evans et al., 1977; Automatic Process of Handling
% Spent Acids in the Manufacture of EGDN, 1987; Urbanski,
% 1965; Van Dolah, 1963; Fritz, 1969). Nitric
% esters (e.g., nitroglycerine and nitroglycol), unlike other
% common explosives such as TNT, are prone to turn very
% unstable on prolonged contact with acids. Thus, many
% accidents have occurred during the manufacture of
% nitroglycerine and nitroglycol, particularly while handling
% or storing spent acid (residual acid after the
% nitration process) (Automatic Process of Handling Spent
% Acids in the Manufacture of EGDN, 1987; Urbanski,
% 1965).
% The separation of nitric esters from spent acid and
% their accumulation in unexpected places (e.g., transfer
% lines, drains, and collecting areas), building up dangerous
% quantities over a period of time, are the main
% reasons of accidents taking place at nitric ester production
% units. Since acidic nitric esters are chemically
% unstable and sensitive mixtures, all possibilities of
% unintentional accumulation must be avoided. The accumulation
% of nitroglycerine (NG) or nitroglycol (EGDN)
% may be attributed to a variety of reasons (Federation
% of European Explosives Manufacturers, 1988), namely,
% the following...."

%--------------------------------------------------------------------------------------------------------------------------------------------------------------------------------------------%
\section{Applications}
*Talk about the renewed interest in NC - see:

https://medium.com/@pandevedashri/global-nitrocellulose-market-is-estimated-to-grow-at-a-cagr-of-6-5-during-the-period-2019-2025-800e2942f316 

and

https://medium.com/photography-and-film/thoughts-on-celluloid-c1a6744312bb

Draw a nice info graphic of stats all the uses of NC, and a paper interest graph demonstrating the increased mention of Nitrocellulose in publications, since the war. 

(thought the latter is more a niche thing)*
\ac{NC} has been used widely in manufacturing, as a feedstock for lacquers, coatings, propellants and ....., It continues to be used in industry today, due to it's low cost and versatility.  However, new 
\subsection{(Explosive,  non-explosive}
Cf. plane seat ejector propellant formulation - important to know whether it can withstand humidity and temperature cycling. 


%--------------------------------------------------------------------------------------------------------------------------------------------------------------------------------------------%
\section{Synthesis and structure}
Nitrocellulose is dervied from cellulose. Cellulose is a naturally occurring polysaccharide found primarily in plant cell walls. It is the main component of cotton (FIGX). %cotton plant
The polysaccharide chain is comprised of glucose units linked via \textbeta-1,4-glycosidic bonds, resulting in the alternating orientation of the individual monomers (FIGX). A single strand of cellulose is estimated to contain between 100 - 200 glucose units, representing a \ac{MW} of 20,000-40,000 daltons.
% REF A review of the synthesis, chemistry and analysis of nitrocellulose, Saunders, 1990
%Include spectroscopic studies here
%Include structural studies of nitrocellulose here.
%See https://www.sciencedirect.com/science/article/pii/0032386190900484, 
%https://www.sciencedirect.com/science/article/pii/0032386184900120 and \\
%https://www.sciencedirect.com/science/article/pii/0032386187902539\hspace{4pt}.

cf James Tucker - A whole life assessment of extruded double base rocket propellants. - NC chapter on synthesis

\section{Reactions of Nitrate Esters}
Reactions of all nitrate esters in general, not just NC.

\subsection{Thermolytic reactions}
Incorporate all computational, experimental studies, with mechanistic detail where possible and reference to other industrially methods at the end.

\subsection{Hydrolysis reactions}

\section{Nitrocellulose degradation}
Now, which of the above are relevant to NC, and what additional ones happen in NC / are unique to NC?

Think - secondary reactions, synergy between different nitrate groups and other properties linked to the degradation of sugar / organic polymers. 

Review of Degradation studies
Intro to experimental studies - When, why, what, where, who and how. 
In subsequent sections, make sure to go into the "so what?"

\ac{NC} is extremely flammable when dry but is largely insoluble in water, and is usually kept in solvent to prevent detonation when under storage.18,19 Mixed with at least 25 \% of water or alcohol, \ac{NC} is completely stabilised.%REF F.-J. Kim, Byung J. ; Hsieh, Hsin-Neng ; Tai, Anaerobic Digestion
Accidents [He He Liu 2017 has a reference to an accident at TianJin, but you can find the news article]
This has facilitated the use of aqueous solvents as dispersion and transport mediums in manufacturing processes, leading to small \ac{NC} fibres, or “fines”, in output streams. Demilitarisation activities have resulted in large volumes of \ac{NC} waste which regulatory action now prevents the disposal of via incineration.
%REF J. Wilkinson and D. Watt, 2006, J. E. Alleman, B. J. Kim, D. M. Quivey 1994, N. Auer, J. N. Hedger and C. S. Evans, Biodegradation 2005, J. Kim, J. K. Park and L. W. Clapp, Characterization of Nitrocellulose Fine
A high level of side group substitution relative to cellulose, in addition to its insolubility, contributes to resistance to microbial degradation.
%REF no idea what:3,939,068, 1975, 6–9, M. Duran, B. J. Kim and R. E. Speece, Waste Manag., 1994,
As a result, \ac{NC} exhibits poor mobility and a long lifetime in the environment. 
A detailed understanding of the degradation mechanisms is imperative in the design of efficient and economical processes for \ac{NC} disposal. Whilst many studies in the past century have shed light on the various decomposition schemes, a review of the literature (section 2.1) [CLEAN THIS UP] reveals conflicting conclusions on the general mechanisms of decomposition via alkaline hydrolysis, acidic hydrolysis, thermolysis and biodegradation treatments. Most notably, a distinct lack of mechanistic detail hinders optimisation of the practised treatments, or effective redirection of degradation products to more useful substances, such as plant fertiliser. %REF  J. Wilkinson and D. Watt, 2006,F. H. Bissett and L. A. Levasseur, Chemical Conversion of Nitrocellulose

\subsection{Industrial disposal methods}
This can be a tack on, at the end to highlight the relevance of the NC system, applications and current modes of usage.  

Include  the industrial decomposition methodologies here (biological etc.)
Include an overview of all, but keep the alkaline and acid 

Discuss Alkaline hydrolysis in the context of industrial process. However we don't want this as a big thing, as we want this lit review tailored to the study, with the equivalent emphasis.
More detail on the studies that look at this.
Include computational studies where relevant(maybe making reference to the computational section) .

[HAVE TO CLEAN UP]
The alkaline hydrolysis reactions of NC have undergone thorough study due to their central role in large-scale NC disposal from manufacturing waste streams. It was realised early on by Kenyon and Gray that the action of alkalis on NC did not just yield cellulose and the corresponding metal nitrate salt, but a whole mixture of highly variable organic and inorganic products.28 Miles observed that whilst acid and alkaline hydrolysis both produced similar products, the rate of the latter reaction was much greater.29 
Typically, NC is treated with concentrated sodium hydroxide and heating. Though other alkalis maybe used, such as. barium hydroxide, calcium hydroxide or sodium carbonate, these alternative alkalis require longer contact and heating times in addition to a larger measure of alkali to solubilise the NC.8 In 1976 Wendt and Kaplan reported that a sodium hydroxide solution of 3 \% by weight at 95 oC was able to effectively degrade NC of 12.6-13.4 \%NMax to completion, in only 30 minutes. Su and Christodoulatos observe a similar rate, with 90 \% of digestion occurring within 35 minutes, in 2 \% sodium hydroxide at 70 oC for NC of 12.2 \% NMax.30  
Multiple studies demonstrate that microbial biodegradation is an effective method of degrading NC.31 Wendt and Kaplan’s study presented a bench-scale continuous treatment process involving initial degradation by alkaline hydrolysis followed by biological digestion using naturally occurring microbes in raw wastewater, for which they acquired a patent a year earlier25
It was noted by Fan et al. that highly substituted cellulose derivatives are resistant to direct biodegradation and require pre-treatment. Work on the enzymatic hydrolysis of cellulose revealed that increasing the crystallinity of the substrate reduced its digestibility.32 This is reinforced in the study by Mittal et al. involving alkaline and liquid-ammonia treatments on crystalline cellulose.33 Substitution of hydroxyl groups to increase solubility of cellulose derivatives lead to increased enzymatic hydrolysis, up to complete solubilisation. After this point, susceptibility slowly decreased. As solubility was no longer the limiting property, further increase in substitution began to inhibit digestion. It may be that as the levels of substitution increase, polymer geometries become less suited to the active sites of enzymes involved in degradation. 
Kim et al. explored the possibility of combining acid hydrolysis with biodegradation on an industrial scale. Feasibility of the hydrolysis reaction using hydrochloric acid with heat, was thoroughly investigated.20 The study assumes that the nitrate group of C6 reacts the fastest, reasoning that groups on C2 and C3 experience more steric hindrance. A much later computational study by Shukla disagrees, finding that the C3 nitrate is the first to be liberated.3 Kim achieved denitration by treatment with acid more dilute than that initially used to synthesise the NC sample. A lower concentration of acid promoted shift of the equilibrium towards denitration. Though glucose was the major product of the monitored denitration process, it was stated that the rate of denitration was more rapid than the peeling-off reaction. In theory, it should be possible to regenerate strands of cellulose under the appropriate conditions. However, results only focussed on the retrieval of end-stage hydrolysis products such as glucose, nitrates, nitrites and ammonia. In addition, acid catalysed side reactions may introduce complications. 
Though less studied than alkaline hydrolysis, the acid hydrolysis mechanism is extremely important for the understanding of decomposition and ageing processes of NC, as a residual amount of acid always remains in the system. Products liberated via other breakdown routes also go on to form acidic species within the environment and contribute to further decomposition via secondary reactions and autocatalysis.34  

\subsection{(Computational Studies)}
A review of published works yields very few NC mechanistic degradation studies involving computational methods. Due to the nature of the topic, many of the referenced works above are linked with private or military organisations. Thus, their release may be restricted until declassified many years later, or completely withheld. The timeline of much of the material reviewed here ranges from mid to late 1900s, before the widespread use of computational methods in chemical research. Consequently, it is to be expected that more NC computational studies will appear in the near future.
Shukla et al. performed a series of mechanistic studies on 2,3,6-trinitro-β-glucopyranose (FIG. 3), exploring its validity as monomer model for the NC polymer during alkaline hydrolysis via the saponification route.3,4,35 If the hydrolysis reaction were to proceed via the saponification route, in practice it is unlikely that all nitro groups are liberated simultaneously. 
The work examined the possible sequences of nitrate group removal for a fully nitrated NC polymer in the hydrolysis reaction, in addition to whether denitration, depolymerisation or rupture of the ring would initiate. It was reported that nitrate groups would undergo an SN2 substitution reaction whereby each nitrate ester group is replaced by the incoming hydroxide nucleophile, liberating a nitrate ion. 
Calculated enthalpies and free energies indicated a C3 $\rightarrow$  C2 $\rightarrow$  C6 sequence of denitration for the monomeric unit. Their later work involving the dimer and trimer models (FIG. 4.) suggests a significant change in behaviour when scaling up from the monomer. Dimer and trimer activation energies proved comparable, but showed an inconsistency with the monomer, instead exhibiting a C3 $\rightarrow$  C6 $\rightarrow$  C2 sequence.
There is the problem that the latter sequence was derived only from the energies of the initial nitrate removal step, whereas the monomer sequence was derived from energies calculated at each of the three stages of denitration. As there is little disparity between the dimer and trimer energies, but a large difference to the monomer results, it was said that the dimer should be the smallest unit used to describe the alkaline hydrolysis behaviour of NC.
Though not a study involving NC, detailed work on the nitrate ester degradation routes in pentaerythritol tetranitrate (PETN) was conducted by Tsyshevsky, Sharia and Kuklja (Figure 4).10 PETN is an energetic material possessing four nitroester moieties. Similarly to NG, it is used as both an explosive and in medicine.

Existing experimental activation energies for decomposition were scattered in the range of 30 to 70 kJ mol-1. The group attributed the dispersion in data to inconsistent procedures and experimental conditions between studies. 
The most common degradation products recorded were CO, CO2, NO, N2O, CH2O, HCN, and HNCO. Echoing earlier NC and other nitroester thermolysis studies,36 the first degradation step was assumed to be nitrate ester homolytic fission. The seven mechanisms explored, corresponding to the labels in Figure 4, are detailed in equations (1) to (7):
 (1)  (2) (3) (4) (5) (6) 						 (7)
 (1) homolytic cleavage of the O-NO2 bond, (2) the elimination of nitrous acid (HONO) which is usually considered as a competing reaction to homolytic fission, (3) the nitro-nitrite rearrangement (OONO), (4) γ-attack, the encounter of a peripheral oxygen atom and the central C atom (5) the homolytic C-O bond cleavage and (6), (7) two variations of the homolytic C-C bond cleavage. 
Activation barriers calculated using PBE, PBE0 and wB97XD were compared, finding that reaction (1) proceeded without a barrier and was therefore most favourable. Reaction (2) possessed an activation barrier of only approximately 6 kcal mol-1. However, this second reaction was also exothermic, and through bulk calculations it was determined the secondary reaction would accelerate global processes via self-heating. 
Analyses of the functionals used found that PBE consistently underestimated barrier height by 12 to 14 kcal mol-1 on average, with respect to the remaining methods used. Transition states were optimised within GAUSSIAN, followed by an intrinsic reaction co-ordinate computation for each mechanism.


Ring fission vs denitration / post denitration reactions. 

\subsection{Acid hydrolysis}
"Acid catalysed decomposition of sugar molecules in aqueous medium is initiated by the protonation of the hydroxyl groups." - Assary, Kim 2012
All the detail in literature that you can find on this. 
Summary of all acid hydrolysis reactions, including kinetics.
Include computational studies where relevant.
Elaborate on how acid is always present in the system and may be the main contributor to accelerated ageing.
Why is this important to know about?

\subsubsection{Autocatalysis}

[REF] suggests it is NO2+ that is the catalytic species, which is reinforce by Knight and Barry [REF] who found that the elevated catalytic degradation rate was not observed for cellulose acetate. (Though it is postulated to undergo similar hydrolysis steps? [REF]) 


\subsection{Post-denitration reactions}
What do the hydrolysis products go off to do. 
Include computational studies where relevant.

\section{Motivation}
Despite its long history, \ac{NC} is still an essential ingredient in many propellant and lacquer formulations. 
%Its versatility means that we come into contact with it in some shape or form everyday, in printing ink we use, tupperware etc. 
%NOTE: it might be that they substituted \ac{NC}for other reasons too. Bohn didnt say, but find out. 
Efforts have been made to substitute it with other polymeric binders in attempt to reduce the manufacturing risk it poses due to its volatility, but this has only been partly successful. %(REF 11-dr-m-bohn-ict-stability-decomp-ageing.pdf (Bohn, 2007) Give example where possible? Maybe these should be saved for the text above.). 
Insufficient understanding of the internal processes within nitrocellulose has lead to accidents in the past, sometimes resulting in lives lost. %(REF Tianjin, and AWE fire report). 
It is therefore imperative that we seek to clarify our understanding of the ageing mechanisms to inform the reduction of associated risks, whilst more effectively preserving existing \ac{NC} stock.
%, strive to eliminate associated risks,  

Experimental studies over the past 100 years %REF%
have shed light on the macroscopic degradation behaviour of bulk \ac{NC}. However, the fine mechanistic details of degradation have only been alluded to, and are as yet unvalidated. 

In this study we will elucidate the dominant degradation schemes in \ac{NC} with scrutiny of previously proposed decomposition pathways, and present new mechanistic considerations.
% what about generating our own new mechanisms?
%In this study we will examine / scrutinise previously proposed decomposition pathways in order to elucidate the dominant degradation schemes in \ac{NC}.
This will be achieved via the application of computational techniques to give insight where it has been restricted by the limitations of laboratory experimentation in the past. %the resolution of


%Point out the areas that are lacking in our knowledge, why we hit the limitations of experiment and why computational studies are required. 


%\chapter{Theory and Implementation}
\label{chapterlabel2}
%The theoretical basis for methods used in this thesis and more widely in compuational chemistry is described in this chapter.

\section{Electronic structure methods}
%Intro to the history spiel. What it is, where did it come from. 
%Wave particle duality. 
% Dodgy first sentence, change it. 
Electronic structure methods apply the principles of quantum mechanics to the evaluation of electron position and movement, thereby allowing chemists to derive the properties and interactions of molecules. At the most fundamental level, the wavefunction ($\Psi$) holds the description of a quantum system. %Need a linking sentence here - What is the wavefunction, and why do we need to look at the probablility densiy instead?
In a non-relativistic system, the probability of a particle possessing a given momentum, or residing in a particular place, is given by the probability density. This can be obtained by  multiplication of $\Psi$ with its complex conjugate,  $|\Psi^{2}|$. %Born interpretation. Ref? 
Integration of $|\Psi^{2}|$ over a region of space returns the probability that a system will be found within.
%, such that the integral of over a region of space returns the probability that a system will be found within. %in that region.$|\Psi^{*}\Psi|$ 
%In Born's statistical interpretation in non-relativistic quantum mechanics,[8][9][10] the squared modulus of the wave function, |ψ|2, is a real number interpreted as the probability density of measuring a particle's being detected at a given place – or having a given momentum – at a given time, and possibly having definite values for discrete degrees of freedom. The integral of this quantity, over all the system's degrees of freedom, must be 1 in accordance with the probability interpretation. This general requirement that a wave function must satisfy is called the normalization condition. Since the wave function is complex valued, only its relative phase and relative magnitude can be measured—its value does not, in isolation, tell anything about the magnitudes or directions of measurable observables; one has to apply quantum operators, whose eigenvalues correspond to sets of possible results of measurements, to the wave function ψ and calculate the statistical distributions for measurable quantities.
%From Wikipedia: https://en.wikipedia.org/wiki/Wave_function
%Read the page, it's pretty good. 
Values of $\Psi$ are chosen to be othonormal; integrating $|\Psi^{2}|$ over all space gives the probability of 1:
\begin{equation}
\braket{\Psi_{i}|\Psi_{j}} = \delta_{ij}
\end{equation}
Where all electrons are represented by $i$ and $j$, and:
\begin{center}
$\delta_{ij}=0$ for $i\neq j$\\
$\delta_{ij}=1$ for $i=j$
\end{center}
%that has been normalised,
%Therefore, for $|\Psi^{2}|$ that has been normalised, the integral over all space is equal to 1, indicating that the probability of finding the system in the space is equal to 1.  %[REF] 
Operators acting on $\Psi$ yield the observable properties of the system. The operator returning the energy of the system is called the the Hamiltonian operator ($\mathbf{H}$).
Erwin Schr\"{o}dinger first proposed his equation for the description of the wavefunction of a quantum system in 1926 \cite{schrodinger1926}. The time-independent %Schr\"{o}dinger's 
equation is:
%Unless you know the long form... given in short-form: %(equation \ref{equ:schrodinger}). NB MAKE
%\subsection{The Schr\"{o}dinger Equation} time independent schrodingers
\begin{equation}
%\centering
\mathbf{H}\Psi=E\Psi
\label{equ:schrodinger}
\end{equation}
%where $\mathbf{H}$ is the Hamiltonian operator and $E$ is the energy ofm the system. $\mathbf{H}$ is called the eigenvalue, $\Psi$ the eigenfunction and $E$ is scalar. 
where the Hamiltonian operator $\mathbf{H}$ is an eigenvalue of %the eigenfunction, 
the wavefunction 
$\Psi$, and $E$ is a scalar denoting the energy of the system. 
%Perhaps straight away reduce it down to atomic units. ie. When working in \ac{au}, the general form of the Hamiltonian is given by: And delete the line later about it getting reduced to 1. 
The general form of the Hamiltonian is given by:
\begin{equation}
\mathbf{H}= - \sum \frac{\hbar^{2}}{2m_e}\nabla_i^{2} - \sum \frac{\hbar^{2}}{2m_k}\nabla_i^{2} - \sum_{i} \sum_{k} \frac{e^{2}Z_{k}}{r_{ik}} + \sum_{i<j} \frac{e^{2}}{r_{ij}} + \sum_{k<l} \frac{e^{2}Z_{k}Z_{l}}{r_{kl}}
\label{equ:hamiltonian}
\end{equation}
where all electrons are represented by $i$ and $j$, and all nuclei by $k$ and $l$. 
%$i \rightarrow j$ encompasses all electrons and $k \rightarrow l$ all nuclei. 
$\hbar$ %= \frac{h}{2\pi}$
is the reduced Planck's constant ($\hbar = \frac{h}{2\pi}$), $m_e$ is the mass of an electron, $m_k$ is the mass of the nucleus $k$, $e$ is the charge of an electron, $Z_k$ is the atomic number of $k$ and $r_{xy}$ is the distance between particles $x$ and $y$. When using \ac{au}, the electronic mass, charge and reduced Planck's constant are reduced to a value of 1. $\nabla^{2}$ refers to the Laplacian operator, which describes the divergence of the gradient of a field. % What does this really mean, in a practical sense? 
In Cartesian space, this is defined as the second derivative of the gradient in the three dimensions: % Does it refer to the change in electronic position, in this case? Do I even need to mention much about Laplacian?
\begin{equation}
\nabla_i^{2} = \frac{\partial^{2}}{\partial x_{i}^{2}} + \frac{\partial^{2}}{\partial y_{i}^{2}} + \frac{\partial^{2}}{\partial z_{i}^{2}}
\label{equ:laplace}
\end{equation}
The first and second terms of equation (\ref{equ:hamiltonian}) correspond to the kinetic energy of the electrons and the nuclei, respectively. Electron-nuclear attraction is described by the third term, followed by the interelectronic and internuclear repulsive terms. The three latter potential energy terms are described just as in classical mechanics. %appear the same as in classical mechanics. %analogous with classical mechanics.
% attractive and repulsive terms are% potential energy terms that are
% described as in classical mechanics. 
%coloumbic interactions
The kinetic energy terms are expressed as the eigenvalue of the kinetic energy operator ($\mathbf{T}$):
\begin{equation}
\mathbf{T}=-\frac{\hbar^{2}}{2m}\nabla^{2}
\label{equ:KE}
\end{equation}
The Hamiltonian %Equation \ref{equ:hamiltonian}
can therefore be written in terms of the kinetic energy and potential energy operators:
\begin{equation}
\mathbf{H} = \mathbf{T}_{e} + \mathbf{T}_{N} + \mathbf{V}_{e-N}+ \mathbf{V}_{e-e} + \mathbf{V}_{NN}
\label{equ:Htot}
\end{equation}
where the terms are as they were in equation \ref{equ:hamiltonian}. %, the first two terms denoting the kinetic energy of the electrons and nuclei, and the latter three terms detailing the electron-nuclear interactions.
%Was there really any point in rewriting it?
%The Hamiltonian for the full system is represented in terms of the kinetic energy and potential energy operators. 
%$\hat{H}$
%
%Dirac’s equations [REF] present the most complete description of a whole N-electron system by accounting fully for special relativity in the context of quantum mechanics, however many layers of approximation must be adopted in order to arrive at a model that is pliable within computational limitations. In the context of quantum electronic structure calculations it is usually appropriate to exclude the contribution of relativistic effects and nuclear motion, thus applying the Born-Oppenheimer approximation. 
%This presents the time independent Schrödinger’s equation \ref{equ:schrodinger}, which the Hartree-Fock approximation and many other electronic structure wavefunction based methods aim to solve. 
%
%The Hamiltonian operator (Eq. 3) acting on the wavefunction produces the electronic energy of a many electron system, where each of the terms in the Hamiltonian correspond to kinetic energies of the electrons and nuclei, the attraction of the electrons to the nuclei, and the inter-electronic and inter-nuclear repulsions.7
%

\subsection{Born-Oppenheimer approximation}
In a real system, the motion of elections and nuclei are coupled. Electron density flows dynamically in response to the change in nuclear position and repulsion from other electrons. The correlated motion of particles is described by the pairwise attractive and repulsive terms of the Schr\"{o}dinger’s equation. However, this interdependency results in a wavefuction that is difficult to define. 
%of such a system 
%this representation is unworkable owing to the complexity of considering the interdependent movement of all particles in the system. 
Relative to electronic motion, nuclei move far more slowly, owing to their mass. %The relaxation time of electrons may be treated as instantaneous, when comparing against that of the nuclei. 
%This difference in electronic and nuclear motion leads
The difference is such, that the nuclear positions appear essentially stationary compared to the electrons. Exploiting this property, the 
%Owing to this difference in relative  
Born-Oppenheimer approximation fixes the nuclear positions. In this way, the motion of electrons and nuclei can be decoupled, and the electronic properties of the system may be calculated for the given nuclear coordinates. The nuclear kinetic energy term ($\mathbf{T}_{N}$) of the Hamiltonian is removed, as the nuclei are frozen. % becomes then independent of electronic motion; 
%The electron-nuclear interaction term ($\mathbf{V}_{e-N}$) disappears, and 
The nuclear-nuclear repulsive term ($\mathbf{V}_{NN}$) becomes a constant.  
The equation is reduced to its electronic components and nuclear constants as:
%
%\begin{equation}
%\mathbf{H} = \mathbf{T}_{e} + \mathbf{V}_{e-N}+ \mathbf{V}_{e-e} + \mathbf{V}_{NN}
%\label{equ:BO}
%\end{equation}
% With AU:
\begin{equation}
\mathbf{H}= - \frac{1}{2} \sum \nabla_i^{2} - \sum_{i} \sum_{k} \frac{Z_{k}}{r_{ik}} + \sum_{i<j} \frac{1}{r_{ij}} + \mathbf{V}_{NN}
\label{equ:hamiltonian}
\end{equation}
%
The Schr\"{o}dinger’s equation only dependent on the electronic coordinates is as follows: 
%
\begin{equation}
(\mathbf{H}_{el} + \mathbf{V}_{NN})
\Psi_{el}(\mathbf{q}_{i};\mathbf{q}_{k})=E_{el}(\mathbf{q}_{i};\mathbf{q}_{k})
\label{equ:elec_schro}
\end{equation}
where $\mathbf{H}_{el}$ is the pure electronic Hamiltonian, that is, only including the electronic kinetic energy , electron-nucelar attractive term, and electron-electron repulsion term from the total Hamiltonian given in \ref{equ:hamiltonian}. $\mathbf{V}_{NN}$ is the nuclear-nuclear repulsion, constant for the fixed positions. %And so can be an additive term at the end. 
The electronic coordinates are given by $\mathbf{q}_{i}$, and the nuclear positions, $\mathbf{q}_{k}$.
$E_{el}$ is the electronic energy of the system, which is now only dependent on the electronic coordinates.
%
%\subsection{Slater determinant}
%In order to 
%
\subsection{Variational principle}
The variational theorem states that the calculated energy of any guess wavefunction can only be greater than or equal to the real ground-state energy of the system. % The concept of \''lower bound'' is a property of wavefunctions? How is this derived, and where does it stem from?
This provides a criterion for selection of the best guess wavefunction.


\subsection{Hartree Fock}
Hartree Fock \ac{HF} 
%Hartree Fock \nomenclature{HF}
%this might need to be shifted


\subsection{Open shell systems}
[PLEASE REVAMP ALL OF THIS AND FILL IN DETAIL]
The forced pairing of electrons of opposing spin into a shared orbital is referred to as the Restricted (RB3LYP) scheme. For systems without unpaired electrons, or “closed shell”, this treatment is sufficient. For radicals and other species with unpaired electron spin such as transition metal complexes, an alternative model allowing singly occupied orbitals must be adopted. The Restricted-Open (ROB3LYP) scheme maintains electron pairing within orbitals except in the case of the highest occupied molecular orbital (HOMO). This is singly occupied. An alternative model is Unrestricted (UB3LYP), where all electrons are unpaired and reside in their own orbitals (Figure 9). A caveat of the unrestricted model is its susceptibility to spin contamination, which has consequences at large bond separations where the bond has not completely broken.

\subsection{Electron correlation}

\subsection{Post-Hartree Fock methods}

\subsubsection{MP2}

\subsection{Density functional theory}
Solutions to the Schr\''{o}dinger's equation using reference

See here for a nice sum of the weaknesses: Some Fundamental Issues in Ground-State Density, Perdew 2009 (https://pubs.acs.org/doi/10.1021/ct800531s)
Functional Theory: A Guide for the Perplexed
[PLEASE REVAMP ALL OF THIS AND FILL IN DETAIL]\\
Density functional theory derives from the Thomas-Femi-Dirac model, whereby the electron correlation is modelled via functionals of the electron density.38,39 
The total energy is defined by functionals split into the following terms:
\[E=E^T+E^V+ E^J+ E^XC\] %if you remove this comment for some reason it goes italic

Where E$^{T}$ %ok, its the ^ that makes it go italic. 
is the kinetic energy term, arising from electron motion; E$^{V}$
is the potential energy term arising from nuclear-electron attraction and nuclear-nuclear repulsion; E$^{J}$ 
is the columbic repulsion term arising from electron-electron repulsion and E$^{XC}$  
is the exchange correlation term containing the remainder of the electron-electron interactions. The first three terms are purely classical, and correspond to the classical energy of the charge distribution ($\rho$).
The exchange term is non-classical encompasses the exchange energy due to the antisymmetry of the wavefunction, and dynamic correlation of electron motion. It can be further divided into the exchange and correlation components:
%$E\^XC (\rho)= E\^X (\rho)+ E\^C (\rhos)$
In the context of this report, the B3LYP functional will be used where the exchange is described as follows:
%\[E_B3LYP^XC= E_LDA^X+ c_0 (E_HF^X-E_LDA^X )+ c_X ∆E_B88^X+E_VWN3^C+c_C (E_LYP^C-E_VWN3^C )\]

Here c$_{0}$ = 0.2, c$_{X}$ = 0.72 and c$_{C}$ = 0.81.
The coefficient c$_{0}$ allows mixing of E$_{HF^X}$ (Hartree Fock) and E$_{LDA^X}$ (Local Density Approximation). E$_{VWN3^C}$ (VWN3 local correlation) is mixed with E$_{LYP^C}$ (Lee, Yang, Parr correlation function) via c$_{C}$. E$_{B88^X}$ is Becke’s gradient-corrected exchange functional.

{\textomega}B97x-d is a range separated hybrid with D (self consistent exchange). It is more computationally demanding than just post HF style perturbation theory exchange (as in B3LYP, using MP2 exchange for example  - fact check that). And according to the paper by Najibi and Goerigk even the D3 correction isn't better than the perturbation theory style exchange, for a massive amount of datasets that they tested. SO why even use this in the first place? Look to the origianl paper where it was published and see how they tout this method. Use those reasons to justify having chosen it to use on some of my systems. 

%remember to talk about the underestimation of the reaction barriers

\subsubsection{Hohenburg-Kohn formalism}
\subsubsection{Kohn-Sham Equations}
\subsubsection{Exchange-correlation functionals}
\ac{B3LYP}
\ac{wb97xd}
%Think this is the same as the above
%\subsection{Hybrid functionals}
\subsubsection{Dispersion correction}
Grimme
\subsection{Basis sets}
%make sure you know the caveats - ins and outs of the bases you used in combination with the methods you used. 
%You know pople's one's aren't as stable as the aug- ones. State why you chose them. Also, MP2 will react differently with these bases as opposed to the aug- ones. 
[FIX ME]
A basis set is the collection of mathematical basis functions used in linear combination to construct the molecular orbitals. Split valence basis sets describe the core electrons with fewer basis functions than the interacting valence electrons, as they are not as significant in bonding or intermolecular interactions. In this study, the widely used Pople basis sets will be applied (Table 1. Examples of split valence basis sets.).40,41 

Basis set superposition error (BSSE) is a false lowering of the energy that can occur when two species in a system approach one another to form a complex. Particle A borrows the extra basis functions belonging to particle B and an artificial stabilisation is observed. The error arises from the inconsistency in treatment between the individual particles at long separations and the complex at short distances. The effect is particularly pronounced for smaller basis sets. Counterpoise correction I used to circumvent BSSE, at the expense of higher computational resources required for the calculation.

\subsection{Thermochemistry}
(And vibrational frequencies)

\subsection{(Transition State Theory \& Free Energy Calculations)}

\subsection{Solvent models}
%Their effect on free energies - use a free energy cycle to justify the lack of inclusion of solvation energy - though this might be best discussed in Chap 1
%How they are modelled.
%PCM vs SMD, why is SMD better for free energies. Are they similar in terms of compuational demand

%\subsection{Optimisation algorithms}


\section{Analysis}

\subsection{Topology analysis using \ac{QTAIM}}

*paraphrase this:
Topology analysis Theory

Some real space functions in Multiwfn are available for topology analysis, such as electron density, its Laplacian, orbital wavefunction, ELF, LOL ...

The topology analysis technique proposed by Bader was firstly used for analyzing electron density in "atoms in molecules" (AIM) theory, which is also known as "the quantum theory of atoms in molecules" (QTAIM), this technique has also been extended to other real space functions, e.g. the first topology analysis research of ELF for small molecules is given by Silvi and Savin, see Nature, 371, 683. In topology analysis language, the points at where gradient norm of function value is zero (except at infinity) are called as critical points (CPs), CPs can be classified into four types according to how many eigenvalues of Hessian matrix of real space function are negative.

(3,-3): All three eigenvalues of Hessian matrix of function are negative, namely the local maximum. For electron density analysis and for heavy atoms, the position of (3,-3) are nearly identical to nuclear positions, hence (3,-3) is also called nuclear critical point (NCP). Generally the number of (3,-3) is equal to the number of atoms, only in rarely cases the former can be more than (e.g. Li2) or less than (e.g. KrH+) the latter.

(3,-1): Two eigenvalues of Hessian matrix of function are negative, namely the second-order saddle point. For electron density analysis, (3,-1) generally appears between attractive atom pairs and hence commonly called as bond critical point (BCP). The value of real space functions at BCP have great significance, for example the value of XXX and the sign of  at BCP are closely related to bonding strength and bonding type respectively in analogous bonding type (The Quantum Theory of Atoms in Molecules-From Solid State to DNA and Drug Design, p11); the potential energy density at BCP has been shown to be highly correlated with hydrogen bond energies (Chem. Phys. Lett., 285, 170); local information entropy at BCP is a good indicator of aromaticity (Phys. Chem. Chem. Phys., 12, 4742).

(3,+1): Only one eigenvalue of Hessian matrix of function is negative, namely first-order saddle point (like transition state in potential energy surface). For electron density analysis, (3,+1) generally appears in the center of ring system and displays steric effect, hence (3,+1) is often named as ring critical point (RCP).

(3,+3): None of eigenvalues of Hessian matrix of function are negative, namely the local minimum. For electron density analysis, (3,+3) generally appears in the center of cage system (e.g. pyramid P4 molecule), hence is often referred to as cage critical point (CCP).


(table \ref{tab:cps})

\begin{table}[htp]
\caption{Features of different types of critical point from \ac{QTAIM} topological analysis.}
\begin{center}
\begin{tabular}{l c p{4cm} p{3cm} p{3cm}} 
\toprule
Critical Point & Label  & Derivation & Attribute & Representation \\
\midrule
Nuclear (\acs{NCP}) & (3,-3)  & All 3 eivalues of the Hessian matrix are negative & Local maximum & Atomic nuclei \\
\hline
Bonding (\acs{BCP}) & (3,-1) & 2 negative eigenvalues of Hessian matrix  & 2\textsuperscript{nd} order saddle point & Bonding site \\
\hline
Ring (\acs{RCP}) & (3,+1) & 1 negative eigenvalue of Hessian matrix & 1\textsuperscript{st} order saddle point & Steric point or centre of ring system \\
\hline
Cage (\acs{CCP}) & (3,+3) & No negative eigenvalue of Hessian matrix& Local minimum & Centre of cage system \\
\bottomrule
\end{tabular}
\label{tab:cps}
\end{center}
\end{table}


The positions of CPs are searched by Newton method, one need to assign an initial guess point, then the Newton iteration always converge to the CP that is closest to the guess point. By assigning different guesses and doing iteration for each of them, all CPs could be found. Once searches of CPs are finished, one should use Poincaré-Hopf relationship to verify if all CPs may have been found, the relationship states that (for isolated system)

n(3,-3) – n(3,-1) + n(3,+1) – n(3,+3) = 1

If the relationship is unsatisfied, then some of CPs must be missing, you may need to try to search those CPs by different guesses. However even if the relationship is satisfied, it does not necessarily mean that all CPs have been found. Notice that the function spaces of ELF/LOL and Laplacian of XX are much more complex than XXX, it is very difficult to locate all CPs for these functions, especially for middle and large system, so, you can stop trying for searching CPs once all CPs that you are interested in have been found.

 

The maximal gradient path linking BCP and associated two local maxima of density is termed as “bond path”, which reveals atomic interaction path for all kinds of bonding. The collection of bond paths is known as molecular graph, which provides an unambiguous definition of molecular structure. Bond path can be straight line or curve, obviously for the latter case the length of bond path is longer than the sum of the distances between BCP and associated two (3,-3) CPs.

Let us see an example. In the complex shown below, the imidazole plane is vertical to magnesium porphyrin plane, the nitrogen in imidazole coordinated to magnesium. Magenta, orange and yellow spheres correspond to (3,-3), (3,-1) and (3,+1) critical points, brown lines denote bond paths.



\subsection{(Transition state searches)}
[Maybe this goes into a smaller metholodogy section?]
Transition state searches are called through the Opt=TS, QST or QST3 keywords. The Opt=TS method in GAUSSIAN attempts to optimise the given “guess” geometry to a transition state. The guess structure can be obtained from a geometry scan, manually constructed or generated using the QST2 function. In many cases, a TS alone will not be able to isolate the suitable transition state and is usually used in conjunction the QST2 or QST3 methods, and combined with other techniques such as frequency calculations. 
The QST2 option is able to generate a transition state geometry using the Synchronous Transit Quasi-Newton (STQN) method42. Here the transition geometry is midway between a given reactant and product. The corresponding atoms labels must match in both the starting and end products.  QST3 performs similarly, but also considers a guess transition state (Figure 10). It is widely acknowledged that transition state searching is challenging; in addition to the techniques above, the task requires perhaps a certain measure of chemical intuition.

`“Scan” means that the reaction coordinates are changed gradually and “relaxed” means that the reaction coordinates are fixed, while other coordinates that are orthogonal to the reaction coordinates are relaxed during the energy minimiza- tion` - Yao2018

%(within Gaussian? See if you end up using any)
Frequency calculations
\subsection{(Intrinsic Reaction Coordinate)}
[Maybe this goes into a smaller metholodogy section?]
IRC calculations begin at the saddle point and descend the PES in both the forwards and backwards direction of the reaction co-ordinate dictated by the normal mode of the imaginary frequency. In a similar manner to a geometry scan, geometry optimisations are performed at each step point. Its purpose is to connect the two minima leading to the found transition state, thereby confirming whether the found transition state corresponds to your reactants and products of interest. 
IRC calculations are called via the IRC keyword, with specifications of whether the forward or backward reaction is to be scanned, step size and maximum number of steps allowed.



%not sure whether to talk about the transistion state searching methods / algorithms here or elsewhere, or whether to go into detail at all.
\subsection{(PES Scans)}
[Maybe this goes into a smaller methodology section?]
Relaxed potential energy surface (PES) scans, or geometry scans are used to probe the local energy landscape corresponding to specific change in geometry. During the course of a scan, a selected bond length, angle or dihedral is adjusted in incremental steps, as specified by the given scan parameters. At each step, the adjusted parameter is frozen and a geometry optimisation is performed, allowing the rest of the system to relax around the modified bond.  Each scan yields a PES of the explored pathway, presented in a reaction co-ordinate diagram (see Figure 17, section 2.4.1).
An energy maximum followed by a trough indicates a transition state and intermediate reaction product, respectively. The structural co-ordinates at the points of interest are extracted and used for subsequent frequency calculations, transition state searches and validated using intrinsic reaction co-ordinate methods. 
To explore the predicted degradation mechanisms, the scanning parameter was assigned to the bond undergoing the most significant transformation during a particular step of the mechanism. In the case that more than one significant bond was altered, multiple scans with different bond specifications were compared. 
Geometry scans were performed on the optimised reactant geometry using the Opt=ModRedundant keyword.

A rigid scan consists of a single point energy calculation of the structure at each of step the scan. % change in coo-rdinates.
A relaxed scans calls for a geometry optmisation at each of these points. 
Two-dimensional scans may be used to probe simultaenous processes in thte system. These are specified in \ac{G09} by selecting two internal coordinates to be scanned, and stating the number of steps.




\section{Spectroscopy}

\subsection{Nuclear Magnetic Resonance Spectra}
Refer them to an actual resource, so you don't have to explain the theory of \ac{NMR}, but give an overview of the experimental, then give the theoretical detail about how Gaussian calculates it, and any discrepancies between experimental and calculated. 

Comment on the different methods used to calculate NMR parameters - a tiny literature review, if you will, and say which you'll be using, why - even if it's becuase of the fact that it's built into Gaussian (obviously big it up, if that is the case) - and state any caveats. 

Maybe have a look at: Accurate Calculation of NMR Chemical Shifts, Jurgen Gauss, 1995 as a basis, but you'll also need something more current. Any review papers out there?

\subsubsection{Computational implementation}

\subsection{Infra-Red Spectra}

%Update this later, looking at Tom's thesis, when you get round to actually doing some of the stuff
%\section{Molecular dynamics simulation methods}
Intro to the history spiel.

%Save this for the specific chapter?: \subsection{AMBER Forcefield}
% \section{Molecular mechanics}
% What are MM methods, e.g monte carlo

% \subsection{Interaction potentials}
% \subsubsection{Bonded terms}
% \subsubsection{Non-bonded terms}
% \subsubsection{Amber Force-field}
% Partial charges (?)
% Brief parameterisation spiel

% \subsection{Molecular Dynamics}
% verlet and leapfrog algorithms, thermostats and barostats
% \subsubsection{Ensembles}
% \subsubsection{Time integration algorithm}
% \subsubsection{Periodic boundary conditions}
% \subsubsection{Solvent models}
% \subsection{Optimisation algorithms}
% used in minimisations - used in Monte carlo, DFT as well
% \subsubsection{Steepest descent algorithm}
% \subsubsection{Conjugate gradient algorithm}

\subsubsection{Computational implementation}
Caveats and scaling factors. 






%\chapter{Building the Model}
\label{chapterlabel3}

\section{Introduction}
The long, polysaccharide chain backbone of \ac{NC}, with its numerous reactive sites, leads to a complicated network of possible reaction routes towards degradation. This large system 

The NC polymer structure was truncated to a dimer by Shukla et al. for the purposes of investigating alkaline hydrolysis. [REF SHUKLA] 
Comparisons between the monomer, dimer and trimer found the dimeric structure to be the minimum model required to completely encapsulate the chemical behaviour of NC in the alkaline hydrolysis pathway. 

% This bit can go in the lit review section %
%This is also observed when considering the minimum unit encompassing all bonding interactions necessary for parameterisation for a forcefield, for implementation in molecular dynamics simulations. 
%The fully nitrated dimer structure used in this study consists of two non-planar β-D-glucopyranose rings joined by a glycosidic bond, with six nitrate groups attached at the 2,3,6 positions on each ring (Figure 11

The dimer ends are capped by methoxy groups rather than hydroxyl groups as were employed in Shukla’s study. From the perspective of partial charges, and to an extent steric considerations, methoxy groups are expected to provide a better approximation for the extended polysaccharide.

Shukla’s work identified the nitrate group attached to carbon three (C3) as the most susceptible to denitration and the first to be removed. This is supported by the distribution of partial charges in the molecule, with disregard of the capping groups. Thus, the nitrate group on C3 was used as the target site for degradation studies.

\section{Computational details}

\section{Results \& Discussion}

\section{Summary}
In this section 

%%% whilst you're still writing up
\setcounter{chapter}{1}

\chapter{Nitration and Denitration Sequence of Nitrocellulose}
\label{chapterlabel4}
\graphicspath{ {./R_chap_2_pics/} }

[Added as preamble for Chapter 4 below]\newline
[Review usage of NO\textsubscript{x} for consistency with the whole text. Maybe just stick to "nitrous oxides" for the whole document.]

% [ Fix decimal places etc https://tex.stackexchange.com/questions/111670/scientific-notation-only-for-large-numbers
% and table formatting https://tex.stackexchange.com/questions/425414/missing-number-treated-as-zero-using-booktabssiunits?noredirect=1&lq=1]

\section{Introduction}
\label{chapter4:intro}
% Read through an introduce more, the relevant peak intensities, etc. 
%This section utlises the numbering scheme introduced in section \ref{sect:labellingsystem}.
Moniruzzaman \textit{et al.} used the \acs{UV} absorption of an anthraquinone dye to determine the activation energies for the removal of the nitrate at C2, C3, C6 sites on \ac{NC} (figure \ref{fig:anthraquinone_SB59})\cite{Moniruzzaman2008,Moniruzzaman2014}. \acs{UVVis} was chosen as an efficient, non-destructive method of monitoring the decomposition process.
The reaction of the \ac{SB59} with NO\textsubscript{x} released by denitration, mimics the action of stabilisers such as \ac{DPA} and \ac{2NDPA} commonly used in \ac{NC} formulations. 
%https://www.islandpyrochemical.com/nitrocellulose-based-propellants/
The secondary amine groups of the dye consume any nitrates in the system, eliminating the possibility of successive reactions generating acidic species. The presence of acid has been linked to autocatalytic rates of degradation during later stages of \ac{NC} degradation\cite{Edge1990,FernandezdelaOssa2011,Baker1952,Binke1999}. %REF autocatalytic processes, and possibly evidence that stabilisers behave by absorbing NOx sepcies. Maybe even some examples of stabilisers that do this?

\begin{figure}[ht!]
  \centering
	\includegraphics[width=0.35\linewidth]{anthraquinone_SB59}
\caption{\acf{SB59} used to probe the release of nitrates from \ac{NC} using \ac{UVVis} and \textsuperscript{1}H NMR spectroscopy\cite{Moniruzzaman2014}. %The action of nitrate absorption by the dye imitates that of stabilisers commonly used with nitrate ester formulations. 
%Following reaction with NO\textsubscript{x}, the UV absorption peak of the dye is depleted.
}
\label{fig:anthraquinone_SB59}
\end{figure}

\begin{figure}[h]
  \centering
	\includegraphics[width=0.75\linewidth]{monirazzuman-UV}
\caption{\acs{UV} spectra of \ac{NC} thin films with and without accelerated ageing at 60 °C, over a total time of 37 days, from the work of Moniruzzaman \textit{et al.}\cite{Moniruzzaman2008}. The unaged peak demonstrates a strong absorbance in the region corresponding to the \ac{SB59} dye (. This decreases as the dye is consumed in the reaction of NO\textsubscript{x} released upon \ac{NC} denitration.}
\label{fig:monirazzuman-UV}
\end{figure}

Un-aged \ac{NC} thin films and films aged at 40\textdegree C, 50\textdegree C, 60\textdegree C and 70\textdegree C for timescales of up to 2000hrs for 40\textdegree C, were compared. Decreasing absorption peak intensity of the dye and appearance of new absorption regions gave insight into the extent of denitration and the presence of secondary reaction products (figure \ref{fig:monirazzuman-UV}).  
%include their graph?
%[CHECK THIS - they also mention the nitration of the dye - they may have used the nitration of the dye at the marker] 
The \ac{NC} starting material was 12.15\%N by mass,  with mean \ac{DOS}=2.307, indicating that individual glucopyranose rings were of mixed nitration level with non-uniform distribution of nitrate groups along the polymer. % they could all be uniformly nitrated at say, C6, an C2, but some would have C3 and some woudl not, etc.
%Initial conclusions from literature considering both experimental and QM studies
The study found that the nitrate at the C3 position would be most reactive, possessing the lowest activation barrier to removal. % \textit{via} thermolysis.  (Is it really only? And do we need to mention it?)
This was followed by C2 and C6. %, with the highest activation energy and deemed most stable to removal.  %,  based on what data/ conclusions? Gotta click through the literature rabbit holes 
The findings contrast with the computational work of Shukla \textit{et al.}, who determined that denitration via alkyaline hydrolysis followed the order of C3$\rightarrow$C6$\rightarrow$C2 \cite{Shukla2012,Shukla2012a}. 
In this case, the study only considered the fully nitrated system. 
%%%%%Shifted to Chap 3%%%%%%%%%%%%%%%%
There is evidence that nitration and denitration are influenced by the presence of nitrate groups at adjacent positions. Matveev \textit{et al.} demonstrated that for polynitro esters the rate of liquid-phase decomposition did not increase linearly with number of nitrate reaction centres, but was mainly dependent on individual structures {table \ref{tab:reactions}}\cite{Matveev2003}. 
%Explain the pattern of behaviour you see in the table, and what each of the columns mean - remove the ones that don't concern you! / you don't understand or know how to explain!
It was suggested that the trend in reactivity could be explained by the inductive effect of nitrate groups \cite{Oxley2003}.%, which influences reactivity based on the proximity of the groups. 
Hu \textit{et al.} found that the presence of adjacent OH groups hampered the rate of hydrolyis for aerosol dispersed organonitrates \cite{Hu2011}. It is therefore ambiguous whether the apparent rate increase due to adjacent nitrate groups arises as a result of the inductive effect of the nitrate, or whether it is solely due to the absence of the neighbouring hydroxyl. 

%%%%%%%%%%%%%%%%%%%%%%%%%%%%%
% Note, in Monirazz's paper, Bohn also did some modelling, to reach teh C6 > C2 > C3 conclusion (in addition to rate calcs.)
% Make sure that you compare the modelling they did to Shukla - method, structures,  and explain the differences in results.
%%%%%%%%%%%%%%%%%%%%%%%%%%%%%



% Check the units for log A
 %*{3}{S[table-format = 2.4]}
%T\textsubscript{C}, \degree C &
\begin{table}[htp]
\begin{center}
\caption{Comparison of rate constants of decomposition for various polynitrate esters at 140\degree C. Collated from literature sources by Matveev \textit{et al.}\cite{Matveev2003}.
% $\Delta$T is the aging temperature range (check this),  $E$ is the energy barrier to ...  log$A$ is the pre-expoenential factor (Check this), $k$\textsubscript{expt} is the rate constant for denitration.
}
%\begin{tabular}{ l c {S[table-format = 2.1]} {S[table-format = 2.1]} {S[table-format = 2.1]}} 
\begin{tabular}{p{15em} c c c c} 
\toprule
\multirow{2}{10em}{Compound	}	&	 $\Delta T$	&  $E$							&  log$A$ 				& $k$\textsubscript{expt} \\ %$k\subscript{exp}
 										&	/ \degree C & / kcal mol\textsuperscript{-1}	&[s\textsuperscript{-1}]	&  / \SI{e-6}s\textsuperscript{-1} \\
\midrule
\ce{O2NOCH2CH2CH2ONO2}					& 72–140 	& 39.1 	& 14.9 	& 1.7 \\
\ce{O2NOCH2CH2CH2CH2ONO2}			& 100–140 	& 39.0 	& 14.7 	& 1.1 \\
\ce{O2NOCH(CH3)CH(CH3)ONO2}			& 72–140 	& 40.3	& 14.9 	& 5.0 \\
%\ce{O2NOCH2CH2(NNO2)CH2CH2ONO2}		& 80–140 	& 41.5	& 16.5 	& 3.5 \\
\\
\ce{O2NOCH2CH2OCH2CH2ONO2}			& 80–140 	& 42.0 	& 16.5 	& 1.9 \\
\ce{O2NOCH2CH(OH)(CH2ONO2)}			& 80–140 	& 42.4 	& 16.8 	& 2.3 \\
\\
\ce{O2NOCH2CH(ONO2)(CH3)}				& 72–140 	& 40.3	& 15.8 	& 3.0 \\ 
\ce{[(O2NOCH2)CH(ONO2)CH(ONO2)]2}		& 80–140 	& 38.0	& 15.9 	& 63.0 \\
	 (hexanitromannite)							&			&		&		&	\\
%j)	& \ce{(O2NOCH2)4C (TEN)} 					& 145–171	& 39.0 & 15.6 	& 9.3 \\
%k)	& \ce{(O2NOCH2)2CHONO2} 					&		-	&	-	&	-	& 13.0 \\
%%%%%%%%%%%%%%%%%%%%%%%%%%%%%%%%%%%%%%%%%%%%%%%%%%%%%%%%%%%%%%%%%%%%%%%%
%%a)	& \ce{O2NOCH2CH2ONO2}						&	-	 	& 	-	& 	-	& 4.7 \\
%% 	& (nitroglycol) 									&			&		&		&	\\
%%b)
%	& \ce{O2NOCH2CH2CH2ONO2}					& 72–140 	& 39.1 	& 14.9 	& 1.7 \\
%%e)
%	& \ce{O2NOCH2CH2CH2CH2ONO2}			& 100–140 	& 39.0 	& 14.7 	& 1.1 \\
%%f)
%	& \ce{O2NOCH(CH3)CH(CH3)ONO2}			& 72–140 	& 40.3	& 14.9 	& 5.0 \\
%%h)	& \ce{O2NOCH2CH2(NNO2)CH2CH2ONO2}		& 80–140 	& 41.5	& 16.5 	& 3.5 \\
%\\
%%g)
%	& \ce{O2NOCH2CH2OCH2CH2ONO2}			& 80–140 	& 42.0 	& 16.5 	& 1.9 \\
%%d)
%	& \ce{O2NOCH2CH(OH)(CH2ONO2)}			& 80–140 	& 42.4 	& 16.8 	& 2.3 \\
%\\
%%c)
%	& \ce{O2NOCH2CH(ONO2)(CH3)}				& 72–140 	& 40.3	& 15.8 	& 3.0 \\ 
%%i)
%	& \ce{[(O2NOCH2)CH(ONO2)CH(ONO2)]2}		& 80–140 	& 38.0	& 15.9 	& 63.0 \\
% 	&	 (hexanitromannite)							&			&		&		&	\\
%%j)	& \ce{(O2NOCH2)4C (TEN)} 					& 145–171	& 39.0 & 15.6 	& 9.3 \\
%%k)	& \ce{(O2NOCH2)2CHONO2} 					&		-	&	-	&	-	& 13.0 \\
\bottomrule
\end{tabular}
\label{tab:reactions}
\end{center}
\end{table}

The inductive effect arises when a difference in the electronegativity between atoms connected by a $\sigma$ bond leads to a polarisation, or permanent dipole, in the bond. Electron donating groups % are of lower electronegativity than  
increase the $\delta$- partial charge on neighbouring atoms through the release of electrons, whilst electron withdrawing groups pull electron density away from neighbouring atoms generating a $\delta$+ charge on connected atoms. %pushing electron density towards the  %HOW DOES THIS LINK WITH DESHIELDLING - NMR shifts
%The characteristic step-wise removal of nitrate groups was also attributed to this property.   (Don't really understand why this is)
%Nitrates are electron donating? whilst O are withdrawing? Thus the presence of a second nitrate group in the vicinity would push electron density towards the C linking the nitrate to the ring (or perhaps the linking O), leading to stabilisation of charge in the form of a carbocation or at least increased nucleophilicity, for subsequent reactions / taking OH from  water, etc. 
From the studies above, it is seen that the denitration order of \ac{NC} can be influenced by the location, distribution and saturation of the nitrated sites along the polymer, as well as mechanistic differences in thermal and chemical degradation.
Crucially, a synergistic effect leading to facile removal of the C3 group when the C2 site is also nitrated may play an important part in the decomposition pathway. The \ac{DOS} value of the source \ac{NC} would therefore give an indication of the likelihood of adjacent nitrates.


%in this section... probe the energies of (de)nitration at different nitrate sites in order to garner the (magnitude of the) effect of adjacent nitrate groups on the ease of nitration or denitration at a specific location on a monomer ring. Results will be compared with experimental compositions of aged samples \textit{via} \ac{IR} and \ac{NMR} spectra. 

%% Note: nitration is via electrophilic subsitition of :OH by NO2+, and denitration via the reverse - hydrolysis / nucleophilic subsitition. We do not consider the other mechanisms of denitraiton here. 
%In this section the nitration and denitration order of a fully nitrated \ac{NC} monomer at thermal equilibrium is determined. The results are compared against experimental observations using simluated NMR and IR spectra, in order to deconflict contradictory views presented in literature on the order of subsitution. 

%\section{Methodology}
%\subsection{Computational details}
%%Programme, method (e.g. functional/ FF), settings(basis sets, thermostats, ensembles etc)
%
%%Include further calculation details - such as transition state theory etc, in another subsection (e.g. Transtion state theory) under this header.
%
%All chemical species underwent an initial \ac{QM} geometry optimisation to the level of $\omega$B97X-D / 6-31G(2df,p) and except where otherwise stated, were performed in Gaussian 09 D.01 (REF). %[REMEMBER TO INCLUDE ANY NMR AND FREQ OPTIONS I USED]
%
%All studies were performed in the gaseous phase and initial structures were geometry optimised with B3LYP/6-311+G(d,p) and tight convergence criteria. 
%Any incomplete or unconverged optimisations were restarted with generation of new internal co-ordinates via the geom=(newdefinition) keyword.  
%% //REJIGG THE BELOW
%The fully nitrated dimer structure was used for MECH 1-2. For the mechanism involving protonation (MECH3), a hydronium cation was independently optimised to the same level. The dimer+cation complex was then optimised with and without CP correction for comparison. For the starting geometry of the intramolecular SN2 reaction (MECH4), the first ring of original dimer was manually adjusted to a boat conformation. Substituents were adjusted to appropriate axial and equatorial positions. 
%All geometry scans were performed at 6-31+G(d) using either UB3LYP or ROB3LYP. Transition state searches were performed using UB3LYP/6-31+G(d). IRC calculations were performed using UB3LYP/6-31+G(d) and either the Hessian-based Predictor-Corrector (HPC), or the Euler integration predictor with the HPC corrector (EPC) algorithm.46
%
%Following each successful scan, a low-level frequency calculation was performed on the obtained transition state. If singular imaginary vibration matching the key bond transformation for the reaction step persisted, then a transition state search was performed using this geometry.
%Where possible, the intermediate “product” geometry obtained from the successful scan was also optimised to B3LYP/6-311+G(d,p) for use in transition state searching using QST2 and QST3 methods.
%
%\section{Results \& Discussion}
%
%\section{Summary}

\chapter{Mechanisms of denitration under ambient conditions}
\label{chapterlabel5}
\graphicspath{ {./R_chap_3_pics/} }

\section{Introduction}%%%%%%%%%%%%%%%%%%%%%%%%%%%%%%%
\label{chapter5:intro}
\subsection{Thermolytic reactions}

The first stage of thermolytic decomposition for nitrate esters is generally agreed to be homolytic fission of the O-N bond linking the nitrate to the alkyl chain, leading to the loss of \rad\ce{NO2} (equation \ref{equ:homofiss}) \cite{Oxley2003,Foltz2009,Shepodd1997}. %many refs
Though nitrate homolysis is an endothermic reaction, the weak O-N bond has a typical dissociation enthalpy of 42 kcal mol\textsuperscript{-1} and is easily cleaved when exposed to elevated temperatures, UV light or impact \cite{Zaki2017}. %REF UV 
Whilst the thermolytic degradation of energetic materials has been widely studied experimentally, the ambient, slow ageing mechanisms are less well documented. %REF
Low-temperature decomposition routes are influenced by many factors over a protracted lifetime, and in practical use, materials are usually subject to evolving environmental conditions. %Changes in pressure, humidity, stress and temperature cycles induce changes in the degradation patterns of energetic materials to a varying degree. %ref?
External changes in pressure, humidity, stress and temperature cycling introduce variation in the degradation patterns of energetic materials. %ref?
%The array of products produced as a result of slow ageing may therefore vary greatly from those obtained from thermolysis; a general observation is that the number of different species is far smaller. %REF
The presence of moisture has been observed to lower the activation energy and accelerate decomposition %of energetic materials 
\cite{Foltz2009}.  %PETN
Internal factors including impurities, residual solvent, and crystal growth within the bulk, also alter decomposition behaviour. 

\begin{scheme}[h]
\begin{equation}
\ce{R-ONO2 -> R-O}\rad+\rad\ce{NO2}
\label{equ:homofiss}
\end{equation}
%redraw this as a scheme with mechanism
\begin{equation}
\ce{R\textcolor{red}{\ce{H}}-ONO2 -> R=O + \ce{\textcolor{red}{\ce{H}}NO2 }}
\label{equ:hno2_elim}
\end{equation}
\end{scheme}
%
\begin{figure}[h]
%\begin{center}
\begin{tabular}{ p{0.5\textwidth} p{0.5\textwidth} }
\begin{enumerate}
\item \rad\ce{NO2} loss
\item \ce{HNO2} loss
\item \ce{OONO} rearrangement
\item $\gamma$-attack
\item \ce{ONO2}\rad loss
\item \ce{C-C} cleavage (\ce{CH2O + NO2})
\item \ce{C-C} cleavage \ce{(CO + HNO2)}
\end{enumerate}
&
\raisebox{-\totalheight}{\includegraphics[width=\linewidth]{PETN}}
\end{tabular}\\

\begin{center}
%{\textcolor{red}{\CIRCLE} oxygen \quad \textcolor{pink}{\CIRCLE} nitrogen \quad  \textcolor{yellow}{\CIRCLE} carbon \quad \textcolor{cyan}{\CIRCLE}hydrogen}
{\textcolor{red}{\CIRCLE} oxygen \quad \textcolor{blue}{\CIRCLE} nitrogen \quad  \textcolor{gray}{\CIRCLE} carbon \quad $\bigcirc$ hydrogen}
\end{center}
\caption[Intramolecular thermolytic reactions in \ac{PETN}.]{Intramolecular thermolytic reactions in \ac{PETN}, adapted from the work of Tsyshevsky \textit{et al.} with permission from the publisher \cite{Tsyshevsky2013}.}
\label{fig:PETN_react}
\end{figure}
%
The degradation of nitrate esters at temperatures over 100$\degree$C is primarily \textit{via} thermolytic processes, whilst under 100$\degree$C, decomposition is largely thought to be the result of hydrolysis \cite{Moniruzzaman2014}. 
Tsyshevsky \textit{et al.} studied the intramolecular reactions leading to denitration in \acf{PETN} in both the vacuum and the bulk crystal \cite{Tsyshevsky2013} (figure \ref{fig:PETN_react}). 
%
Seven mechanisms for the removal of \ce{NO2} were explored. Corresponding to the labels in figure \ref{fig:PETN_react}:
(1) homolytic cleavage of the \ce{O-NO2} bond, (2) the elimination of nitrous acid (\ce{HNO2}) which is usually considered a competing reaction to homolytic fission, (3) the nitro-peroxynitrite rearrangement (\ce{O-ONO}), (4) \textgamma-attack of the terminating nitrate oxygen atom and the bridging nitrate oxygen at their relative \textgamma-carbon sites, (5) the homolytic \ce{C-O} bond cleavage, (6) and (7) two variations of the homolytic \ce{C-C} bond cleavage. 
%
It was found that the two most significant decomposition reactions were homolysis of the nitrate ester \ce{O-NO2} bond (equation \ref{equ:homofiss}) and intramolecular elimination of \ce{HNO2} (equation \ref{equ:hno2_elim}). 
%\added{REF}. \added{[This was shown by XXX and XXX \textit{et al.} that whatever whatever process / spectroscopic technique / logic they used to show that this - did I also mentioned this in the intro?  reiterate here: Jacobs, Burke Elrod, ]}
Whilst elimination of \ce{HNO2} was found to be the most energetically favourable denitration pathway, homolytic fission dominated preliminary decomposition steps due to the lower activation barrier and faster rate of reaction. It was suggested that global decomposition processes were determined by the interplay between these two mechanisms. Initial homolysis facilitated wide-spread denitration, complemented by exothermic \ce{HNO2} elimination, promoting self-heating of the system and further bond dissociations. 
The presence of \rad\ce{NO2} and \ce{HNO2} have previously been linked to the autocatalytic rates observed for later-stage decomposition of nitrate esters \cite{Rodger1963,Lindblom2002,Volltrauer1981}, though some studies solely attribute it to the presence of acids \cite{Edge1990,FernandezdelaOssa2011,Baker1952,Binke1999}. %CHECK THESE ARE FACT
Other studies also implicate the action of \rad\ce{NO} and \ce{HNO3}, in addition to \rad\ce{NO2} and \ce{HNO2} \cite{Zayed2012,Trache2018}. 
%avanced stage decomp? Rodger actualy assumes NO2 is confirmed for PETN
% The presence of acid has been linked to autocatalytic rates of degradation during later stages of \ac{NC} degradation\cite{Edge1990,FernandezdelaOssa2011,Baker1952,Binke1999}. 
Inspection of these products generated from initial processes, with observation of the species permeating through to later stages, will shed light on the most likely contributors to  autocatalysis. 
%However, from these initial processes it is not possible to determine which is the species resposible % (this topic is further discussed in the next chapter.) %\ref{chapterlabel6}.) %REF
\subsection{Acid hydrolysis reactions}

Spent acids remain in the \ac{NC} matrix following synthesis, even with thorough washing procedures. %but how much acid? What concentrations? %REF
Additional acidic species are released via the subsequent reactions of \rad\ce{NO2} following homolysis. %, Any residual sulfuric acid from the nitration process catalyses the protonation of \ce{HNO3}, thereby the formation of nitronium (\ce{NO2+}) and hydronium (\ce{H3O}). 
These acids proceed to react with other moieties in the system, such as unsubstituted alcohol side chains on the polysaccharide, or small molecules free in the bulk. 

When exploring the interaction of nitroglycol and nitroglycerin in acid solution, Camera \textit{et al.} \cite{Camera1982} proposed a protonation-denitration scheme (scheme \ref{sch:cameraschema}) whereby initial protonation at the nitrate was rapid (equation \ref{equ:cam_1}), but the subsequent release of the nitronium ion was slow and rate determining (equation \ref{equ:cam_2}).
\begin{scheme}[htp!]
\centering
%\textbf{Hydrolysis scheme for ethyl nitrate}
%\begin{block_indent}{1cm}
%\ce{CH3CH2ONO2 + H+ <=>[\text{fast}] CH3CH2ONO2H+} \\
%\ce{CH3CH2ONO2H+ <=>[\text{slow}] CH3CH2H + NO2+} \\
%\ce{NO2+ + 2H2O <=>[\text{fast}] HNO3 + H3O+} \\
%\end{block_indent}
\begin{equation}
\ce{CH3CH2ONO2 + H+ <=>[\text{fast}] CH3CH2ONO2H+}% \\
\label{equ:cam_1}
\end{equation}
%
\begin{equation}
\ce{CH3CH2ONO2H+ <=>[\text{slow}] CH3CH2OH + NO2+} %\\
\label{equ:cam_2}
\end{equation}
%%Using generic R groups instead
%\begin{equation}
%\ce{RONO2 + H+ <=>[\text{fast}] RONO2H+}% \\
%\label{equ:cam_1}
%\end{equation}
%\begin{equation}
%\ce{RONO2H+ <=>[\text{slow}] ROH + NO2+} %\\
%\label{equ:cam_2}
%\end{equation}
%\begin{equation}
%\ce{NO2+ + 2H2O <=>[\text{fast}] HNO3 + H3O+} \\
%\end{equation}
\caption[The relative rate of stepwise protonation and denitration of nitrate esters.]{The relative rate of stepwise protonation and denitration of nitrate esters, using ethyl nitrate as an example. From the work of Camera \textit{et al.}  \cite{Camera1982}.}
\label{sch:cameraschema}
\end{scheme}

\begin{scheme}[b]
%\begin{figure}
\centering
\includegraphics[width=0.8\linewidth]{tertiary_nitrate_Hu2011}
%\end{figure}
\caption[Hydrolysis of a tertiary nitrate, originally derived from the reaction of isoprene in the aerosol phase.]{Hydrolysis of a tertiary nitrate derived from the reaction of isoprene in the aerosol phase, from the work of Hu \textit{et al.} \cite{Hu2011}.}
\end{scheme}

\ac{NC} in storage is kept wetted with solvents to prevent self-ignition \added{\cite{Rideal1948,Wei2017}}. Material with 12.6\acs{pN} or lower must be stored in 25\% water by mass, or in a controlled mixture of solvents, stabilisers and plasticisers. The material is therefore always exposed to water, with the fast-exchange of protons expected at inter-facial surfaces. % Something about the surfaces exposed to water? Faster NO'2 release or something?
%Fast exchange is expected for groups exposed to water within the bulk, with the dissociation of \ce{NO2+} to generate the alcohol as the rate determining step. %ref back to lit review, about having to store it in water for stability.
%more info on why the second step may be slow / what we expect to see?
In the study of organonitrates and organosulfates %in the acid phase 
generated from isoprene in the aerosol phase, % as secondary organic aerosol, %\ac{SOA}, 
Hu \textit{et al.} found that primary and secondary nitrates were resilient to hydrolysis for pH $>$ 0, whilst tertiary nitrates underwent hydrolytic nucleophilic substitution easily, reacting with water to form alcohols \cite{Hu2011}. %and sulfate to form alcohols and organosulfates
In tertiary nitrates, the carbon is fully substituted with no attached hydrogens. This group is usually sterically hindered and stabilising to carbocations, condition on the electron donation ability of the substituents remaining after nitrate removal. 
If formation of a carbocation intermediate is involved in the hydrolysis mechanism, this may explain why the tertiary nitrates exhibited highly efficient denitration, even under neutral conditions.

Though no specific mechanistic detail is given, the action of a protonated transition state during hydrolysis is alluded to %by Hu \textit{et al.}, 
through the contrast between the rate of acid-catalysed and neutral hydrolysis reactions. Neutral hydrolysis of the tertiary nitrates occured rapidly, but hydrolysis only occurred for primary and secondary nitrates under strongly acid catalysing conditions, and at a much slower rate.   % Adjacent OH groups did not slow down the acid cat hydrolysis of nitrates, but they did for the sulfates. OH groups slowed down hydrolysis for the primary and secondary nitrates.
% in Neutral hydrolysis for tertiary nitrates, more OH meant slowdown in rate of hydrolysis. This primary and secondary nitrates did not experinece this. 
%It was found that t
Additionally, the presence of adjacent OH groups hampered the rate of hydrolysis for some aerosol dispersed organonitrates. In the neutral hydrolysis of tertiary nitrates, increasing the number of adjacent \ce{OH} groups lead to protracted hydrolysis lifetimes.
Interestingly, the retardation effect of adjacent \ce{OH} groups was not observed for the the acid catalysed cases. Hu proposed that this could be due to the interaction of \ce{OH} with the transition state of the neutral hydrolysis system, compared to the protonated transition state of the acid catalysed system, impeding the reaction only in the former case. 
%Pretty weak sentence
The cause of this effect is unclear, and without understanding of the %transition states 
mechanisms involved, it is difficult to explain. % The neutral case happens much faster than the acid case - could it be that by having an OH next door, it is imparting some ''acid-like'' features to the reaction of the neutral process? %Need a bit more explanation here

There is evidence that nitration and denitration of nitrate esters is also influenced by the presence of nitrate groups at neighbouring positions. Matveev \textit{et al.} demonstrated that for poly-nitroesters the rate of liquid-phase decomposition did not increase linearly with number of nitrate reaction centres. It was found to mainly dependend on individual structures (table \ref{tab:reactions}) \cite{Matveev2003}.
% it really looks like the more stabilising the substituted group is - with respect to being able to  sustain a 
\begin{table}[t]
\begin{center}
\caption[Comparison of rate constants of decomposition for various polynitrate esters at 140$\degree$C.]{Comparison of rate constants of decomposition for various polynitrate esters at 140$\degree$C.
$\Delta$T is the decomposition temperature range,  $E$ is the experimental activation barrier for decomposition, log$A$ is the pre-exponential factor, $T_{c}$ is the combustion temperature, $k$\textsubscript{expt} is the rate constant for decomposition. 
Collated by Matveev \textit{et al.}\cite{Matveev2003} \added{from the work of Afanas'ev \textit{et al.} (*) \cite{Afanasev1967} and Lur'e \textit{et al.}(**) \cite{Lure1967}. }
}
%\begin{tabular}{ l c {S[table-format = 2.1]} {S[table-format = 2.1]} {S[table-format = 2.1]}} 
\begin{tabular}{p{15em} c c c c} 
\toprule
\multirow{2}{10em}{Compound	}	&	 $\Delta T$	&  $E$							&  log$A$ 				& $k$\textsubscript{expt} \\ %$k\subscript{exp}
 										&	/ $\degree$C & / kcal mol\textsuperscript{-1}	&[s\textsuperscript{-1}]	&  / 10\textsuperscript{-6}s\textsuperscript{-1} \\
\midrule
\ce{O2NOCH2CH2CH2ONO2}*					& 72–140 	& 39.1 	& 14.9 	& 1.7 \\
\ce{O2NOCH2CH2CH2CH2ONO2}*			& 100–140 	& 39.0 	& 14.7 	& 1.1 \\
\ce{O2NOCH(CH3)CH(CH3)ONO2}*			& 72–140 	& 40.3	& 14.9 	& 5.0 \\
%\ce{O2NOCH2CH2(NNO2)CH2CH2ONO2}		& 80–140 	& 41.5	& 16.5 	& 3.5 \\
\\
\ce{O2NOCH2CH2OCH2CH2ONO2}**			& 80–140 	& 42.0 	& 16.5 	& 1.9 \\
\ce{O2NOCH2CH(OH)(CH2ONO2)}*			& 80–140 	& 42.4 	& 16.8 	& 2.3 \\
\\
\ce{O2NOCH2CH(ONO2)(CH3)}*				& 72–140 	& 40.3	& 15.8 	& 3.0 \\ 
\ce{[(O2NOCH2)CH(ONO2)CH(ONO2)]2}		& 80–140 	& 38.0	& 15.9 	& 63.0 \\
	 (hexanitromannite)**							&			&		&		&	\\
\bottomrule
\end{tabular}
\label{tab:reactions}
\end{center}
\end{table}
%
%It was suggested that 
The trend in reactivity could be partially explained by the inductive effect of the nitro groups \cite{Oxley2003}. %actually he talks about nitro (NO2), not nitrate...
The inductive effect arises when a difference in the electronegativity between atoms connected by a $\sigma$ bond leads to a polarisation, or permanent dipole, in the bond. Electron donating groups % are of lower electronegativity than  
increase the $\delta^{-}$ partial charge on neighbouring atoms through the release of electrons, whilst electron withdrawing groups pull electron density away, %from neighbouring atoms
generating a $\delta^{+}$ charge on connected atoms. 
%However, the 
%%$\pi$ donation by lone pairs on oxygen and nitrogen also plays a significant role in increasing electron density at adjacent atoms %, known as 
%through the resonance effect. 
%Nitrate ester moieties
The \ce{NO3} presents a stronger electron withdrawing effect than \ce{OH}, which is a donating group. %%The \ce{NO3} presents a stronger electron donating effect via $\pi$ donation than \ce{OH}. In the case of the saturated polysaccharide ring, the \ce{OH} does not exhibit any $\pi$ donation properties, and instead acts as a % though both groups are activating. %, in the context of the polysaccharide ring.  %ref?  
%%OH is actually electron withdrawing for aliphatic alkyles? ``In organic chemistry, an alkyl substituent is an alkane missing one hydrogen.'' %ref? - SO maybe the OH is deactivating here. 
%%It would therefore be expected that both increase the rate of hydrolysis for nearby leaving groups. %(electron donating via \pi donation) 
The presence of an adjacent nitrate appears to facilitate denitration, whereas the presence of hydroxyl groups hinders this process, for neutral hydrolytic schemes. 
The resonance effect, arising from $\pi$ donation by lone pairs on oxygen and nitrogen is negligible %does not come into play 
between substituents at different sites on the polysaccharide ring, as the ring is saturated. 
%This suggests that the proposed interaction of the hydroxyl group with the neutral transition state supersedes its resonance effect. 
%OH - tertiary, neutral hydro: slow down, - primary, secondary, acid cat hydro: no effect
%As a result, it is ambiguous whether any apparent rate increase due to the presence of adjacent nitrate groups arises as a result of the resonance effect of the nitrate, or whether it is solely due to the absence of a neighbouring hydroxyl. 

The investigation by Hu \textit{et al.} exclusively focused on nitrates generated from an isoprene precursor, upon dispersion as an aerosol. The nitrate groups present in \ac{NC} are either of primary (C6) or secondary (C2, C3) structure, indicating that ambient hydrolysis is unlikely according to this scheme. However, solvent effects are expected to differ for condensed-phase reactions and aerosol phases. A greater build-up of acid concentration can be achieved in a closed, condensed system, and the lifetime of an aerosol is relatively short-lived when considering the timescale of slow ageing processes in \ac{NC}. Thus, the work of Hu \textit{et al.} does not provide a direct comparison for the \ac{NC} polymer but highlights the possible contribution from both neutral and acid-catalysed hydrolysis routes, and effect of increasing levels of substitution on the wider structure.%Though ordinarily it would be expected that reactions in aerosol would exhibit a rate increase due to a larger reaction surface, this may skew/increase the rate of specific reactions that do not rely on the diffusion of solvent or other species that move/migrate more slowly through the material. %Where is this even coming from. Some sort of literature proof please. 


In this section, the possible mechanisms for nitrate removal from the \ac{NC} backbone are explored. The homolytic fission and \ce{HNO2} elimination thermolytic processes suggested by Tsyshevsky will be compared to the acid hydrolysis scheme. %The energies of reactions will be compared, with derivation of the reaction rate where it is possible to isolate a transition state. 
Though the relative rates of reaction are not compared, the extended timescales involved in ambient ageing imply that the dominating reactions correspond to those most thermodynamically favourable.

\section{Methodology}%%%%%%%%%%%%%%%%%%%%%%%%%%%%%%%
%Thermolytic reactions
The energies of homolytic fission (equation \ref{equ:homofiss}) and elimination of \ce{HNO2} (equation \ref{equ:hno2_elim}) were calculated for PETN, as a test system before extension to the \ac{NC} monomer. 
The free energies of reaction ($\Delta$ G) were calculated according to equations \ref{equ:homofiss_calc} and \ref{equ:hno2_elim_calc} to reproduce the work of Tsyshevsky \textit{et al.}, using the published geometries of PETN and its derivatives obtained from the authors (R. Tsyshevsky, personal communication, 19 April 2017). 

% Some comement in blue to say I added the equation:
\begin{scheme}[h]
\begin{equation}
\added{\Delta G_{Homolysis} = (G_{\ce{RO}\rad} + G_{\rad\ce{NO2}}) - G_{\ce{RONO2}}}
\label{equ:homofiss_calc}
\end{equation}

\begin{equation}
\added{\Delta G_{\ce{HNO2}  elim.} = (G_{\ce{R=O}} + G_{\ce{HNO2}}) - G_{\ce{RONO2}}}
\label{equ:hno2_elim_calc}
\end{equation}
\quad
\added{R = \ac{PETRIN}, NC monomer}
%\caption{R = \ac{PETRIN}, NC monomer}
\end{scheme}
%}


In the case of the homolysis reaction, only the denitrated radical product geometry was provided. The product of \ce{HNO2} elimination was given as a complex of the \ce{HNO2} leaving group and the newly formed aldehyde. 
A single point energy and frequency calculation were performed on each of the species to determine the reaction energies; no geometry optimisation was performed on the given structures except for in the case the \rad\ce{NO2} molecule, where the geometry was not supplied. A separate \rad\ce{NO2} molecule was independently geometry optimised.

The intramolecular reactions of the \ac{NC} monomer were modelled according to scheme \ref{sch:intra_NC}. % \ref{fig:homofiss_NC} and \ref{fig:elim_NC}.
Rigid and relaxed \ac{PES} scans were attempted for both reactions for the \ac{NC} monomer to obtain an energy profile, and in the case of \ce{HNO2} elimination, identify a transition state. The homolysis reaction was treated as barrierless.  
Where the scans were unable to identify a valid transition state geometry, guess transition state geometries were constructed and optimised. 

\begin{scheme}[h]
\centering
%\begin{figure} Change to C2 instead of C3 in the diagrams
  	\begin{subfigure}[b]{0.75\linewidth}
  	\centering
	\includegraphics[width=\linewidth]{homofiss}
	\caption{Removal of a nitrate group \textit{via} homolytic fission of \ac{NC}}
	\label{fig:homofiss_NC}
	\end{subfigure}
	\par\bigskip
	\begin{subfigure}[b]{0.75\linewidth}
	\centering
	\includegraphics[width=\linewidth]{elimhono}
	\caption{Removal of a nitrate group \textit{via} elimination of \ce{HNO2}. }
	\label{fig:elim_NC}
	\end{subfigure}
%\end{figure}
\caption{The proposed intramolecular reactions for the initial denitration step during \ac{NC} degradation.}	
\label{sch:intra_NC}
\end{scheme}

The possible protonation sites for the \ac{NC} monomer were explored by placing a proton at each of the different oxygen sites surrounding the nitrate group. The structures were then geometry optimised and energies of protonation were compared. \ce{H3O+} was modelled as the donating species; as \ac{NC} is usually stored wetted in water, the hydronium ion is the most likely source of protons (equation \ref{equ:protonation}). %ref precentage water i storage? Pop it in lit review
%Pliego2001 It is well known that
%most esters undergo basic hydrolysis by nucleophilic attack of
%the hydroxide ion onto the carbonyl carbon to yield a
%tetrahedral intermediate, namely the so-called BAC2 mechanism. In rare cases, nucleophilic attack of the hydroxide ion on
%the saturated alkyl carbon leads to the final products in one
%step through an SN2 or BAL2 mechanism
\begin{equation}
\ce{RONO2 + H3O+ -> RONO2H+ + H2O}
\label{equ:protonation}
\end{equation}

It is also possible that the proton is donated by other acidic species in the system, particularly \ce{HNO2} or \ce{HNO3}. This is more likely at later stages of degradation when a higher concentration of acid has been generated by secondary reactions. %REF 
For investigations into the hydrolytic methods of nitrate removal following protonation, or a concerted mechanism involving a proton donor, an ethyl nitrate molecule was used as a truncation of the monomer. This facilitated a speedup of initial \ac{PES} scanning by reducing the degrees of freedom, whilst presenting the moieties necessary for preliminary \ac{TS} searches for the hydrolytic scheme. %CHECK long winded dodgy sentence
This was with the intention of later using the found \ac{TS} geometries to inform guesses for the \ac{NC} monomer structure.  
% Explain the caveats of this. 
%pKA H3O+ –1.74, HNO3 –1.3, HNO2 3.3. All weak acids, but H3O+ is most acidic. 
% SOME MORE INFO ON the later stage stuff you did on hydrolysis mechanism

\subsection{Computational details} %%%%%%%%%%%%%%%%%%%%%%%%
All geometry optimisations, thermochemistry calculations and \ac{PES} scans were performed in \ac{G09}. 
Geometry optimisation and thermal calculations were to the level of 6-31+G(2df,p) 
%, \ac{NC} monomer structures were optimised 
using \acs{wb97xd} and \acs{B3LYP}, with default convergence criteria \added{(max. force $4.5 \times 10^{-4}$ H/Bohr, \acs{RMS} force $3 \times 10^{-4}$ H/Bohr, max. displacement \mbox{$1.8 \times 10^{-4}$} H/Bohr and \acs{RMS} displacement $1.2 \times 10^{-4}$ H/Bohr)}. %and \ac{MP2}. 
$\Delta G$ values were obtained by the difference between the thermally corrected free energies of products and reactants. 
Zero-point corrected energies were determined by addition of individual zero point energies %(\acs{ZPE})( (obtained from the frequency calculation) 
to the free energy:
%
\begin{equation}
\centering
\Delta G^{ZPE} = \sum(G_{products}+ZPE_{products}) - \sum(G_{reactants}+ZPE_{reactants})
\label{equ:ZPE}
\end{equation}
%
\added{Whereby the \ac{ZPE} was obtained from the \acs{QM} frequency calculation by evaluation of the energy of the lowest vibrational level over all molecular vibrations: }% (equation \ref{equ:ZPE_dev}). }
%
\begin{equation}
\centering
\added{E_{ZPE} = E + \sum_{i}^{modes}\frac{1}{2}h\omega_{i}}
\label{equ:ZPE_dev}
\end{equation}
\added{Where $E$ is the energy of the molecule at the minimum (for geometry optimisations) or maximum (for transition state optimisations), $modes$ refers to the vibrational modes, $h$ is Planck's constant and $\omega$ is the vibrational frequency. %at the stationary point of the \ac{PES} *MEntion scaling factor? The diffuse doesn't make much difference in modelling BDE's - but I can't  :/ 
For reaction energy calculations, the \acs{ZPE} energies were scaled by the empirically derived factor of 0.9756 for \acs{wb97xd}/6-31G(2df,p) \cite{Kesharwani2015}. This was to correct for deviation %(from the true vibrational frequencies) 
arising from an incomplete description of electron correlation and neglect of anharmonicity during the calculation of vibrational frequencies \cite{Sinha2004,Cramer2004}. } %the use harmonic vibrational frequencies and
%deviation from the true anharmonicity
% ``(1) the overall neglect of anharmonicity, (2) an incomplete description of electron correlation due to the use of an incomplete basis set, and (3) an approximate method used to solve the Schrödinger equation. The second factor arises because the computational cost for methods including electron correlation increases rapidly as the number of basis functions increases, preventing an advanced theoretical treatment for all but the very smallest of molecules.''
%% Remember, these have not been scaled. Yao2018 : B3LYP/6-31+G(d,p) level and are scaled by a factor of 0.98, for the ZPE (And only certain components are scaled? Drama)


%IRC step ,maxpoints=256,maxcycle=256 , stepsize the default of  0.1 Bohr
\ac{PES} scans were performed to \acs{wb97xd}/6-31+G(d) or unrestricted \acs{wb97xd}, in the case of \ce{O-NO2} dissociation. \added{%high accuracy values for the energy were not required for each step of the \acs{PES} scans 
As high accuracy energy values for structures generated during \acs{PES} scans were not required, calculations were performed at a lower level to conserve time and computational effort. }
Rigid scans were carried out by fixing bond lengths, angles and dihedral values as constants. 
Only the variable of interest was allowed to change. %evolved
This was with the exception of relaxation of other specified coordinates required for accommodation of the new geometry, following each step of the scan.  %(semi-rigid scans). 
For example in the homolysis of the \ce{O-NO2} bond, as the \ce{NO2} group departed the internal \ce{O\textendash N\textendash O} angle was also allowed to relax, in addition to the angle of the departing \ce{NO2} relative to the remainder of the molecule.
In two-dimensional scans, two variables were scanned simultaneously. For the same reaction, the elongation of a the \ce{O-NO2} bond was scanned with simultaneous approach of a proton, to monitor the effect of protonation for the same reaction. %In this case, both the internal angle within  \ce{O-NO2} and the angle was allowed to relax. 
Relaxed scans were performed in Gaussian using the ``modredundant'' function, whereby the whole structure was geometry optimised after each step of the scan. 
Scans were performed with step size of 0.1 $\text{\AA}$. The number of steps varied with the property investigated, though the majority of the phenomena were observed within 20 steps (2 $\text{\AA}$).
Scans were attempted in vacuum, and for the protonation cases, implicit solvent using \ac{PCM} \added{with water ($\epsilon$=78.4) }\cite{Miertus1981}.
%Following each successful scan, a low-level frequency calculation was performed on the obtained transition state. If singular imaginary vibration matching the key bond transformation for the reaction step persisted, then a transition state search was performed using this geometry.
%Where possible, the intermediate “product” geometry obtained from the successful scan was also optimised to B3LYP/6-311+G(d,p) for use in transition state searching using QST2 and QST3 methods.

%The protonation studies conducted in solvent were calculated in \ac{PCM} ...% and \ac{SMD} intrinsic solvent models \cite{Marenich2009,Miertus1981}. 

\section{Results and discussion}%%%%%%%%%%%%%%%%%%%%%%%%%%
\subsection{Thermolytic decomposition mechanisms}
The energies of homolytic fission and intramolecular elimination of \ce{HNO2} from a \ac{PETN} nitrate group are shown in table \ref{tab:intramolec}. The energy values published by Tsyshevsky \textit{et al.} are denoted in parenthesis. As the reaction proceeds with no barrier, the $\Delta G_{r}$ and $E_{a}$ values are the same in the case of homolytic fission. 

\begin{table}[b]
\begin{center}
%\par\bigskip
\caption[Calculated free energies of reaction for the intramolecular reactions of PETN and the \ac{NC} monomer.]{Calculated free energies of reaction  at 298.15 K ($\Delta G_{r}$), reaction enthalpies ($\Delta H_{r}$), activation barriers ($E_{a}$)  with zero-point correction ($ZPE$) for the intramolecular reactions of PETN, and the \ac{NC} monomer calculated at \acs{wb97xd}/ 6-31+G(2df,p). Values expressed in kcal mol$^{-1}$. }

\begin{tabular}{ l *{8}{S[table-format = 2.1]}} 
%\begin{tabular}{ l l l l l l} 
\toprule
Reaction				& $\Delta G_{r}$	&$\Delta G_{r}^{ZPE}$ 	 & & $\Delta H_{r}$	& {\added{$\Delta H_{r}^{ZPE}$}}	& &	${E_{a}}$	& ${{E_{a}}^{ZPE}}$ \\
\toprule
PETN					& \multicolumn{5}{l}{}\\
\midrule
%% Orig: %21.50946; 16.55804
%\rad\ce{NO2} loss	&	{\added{41.05607}}	&	{\added{35.33918}}	&	35.61579		&	{\added{${41.21964}^{b}$}}	&{\added{35.09995}}\\
%						&	${(41.2)}^{a}$		&	${  (35.8)}$ 			&					& ${(41.2)}$ 						& ${(35.8)}$ 	\\
%%\rad\ce{NO2} loss	& 	21.50946		 &16.55804 		&	35.61579	& 	\textendash 	& \textendash 	 \\
%%						& ${(41.2)}^{a}$	 &	${  (35.8)}$ 	&					&	\textendash\  	& \textendash\  \\
%\rule{0pt}{4ex}   
%\ce{HNO2} loss		&  -23.62626 			&	-26.212190			& -20.39247		& 41.2902							& 36.28071  \\
%						& ${(-18.6)}$			&							&					&	${(47.3)}$						& ${(42.7)}$ \\
%%\rule{0pt}{4ex}   
%\toprule
%\ac{NC} monomer	& \multicolumn{5}{l}{} \\
%\midrule
%% Orig:23.25456; 18.68672	
%\rad\ce{NO2} loss	&  {\added{41.09805}}		& {\added{36.12741}} 		& 36.26343 	& {\added{41.09805}} 	& {\added{36.12741}}	\\
%\rule{0pt}{4ex}   
%\ce{HNO2} loss		& -36.04986 		& -39.42322		&-22.85892 	& 40.69863 	& 37.32527  \\
% Orig: %21.50946; 16.55804 [all plus 4]
\rad\ce{NO2} loss	&{\added{26.4}}	&	{\added{21.4}}& &	{\added{40.4}}& {\added{35.5}}		& &{{$\added{40.4}^{b}$}}	& 	{\added{35.5}}	\\
						&						&	 					& &	 					& ${  (35.8)}^{a}$ 	& & 	${(41.2)}$ 					& 	${(35.8)}$ 		\\
\rule{0pt}{4ex}   
\ce{HNO2} loss		&  -23.5 				&	-21.0				& &  	-20.3				& {\added{-17.7}}		& &	46.5							&	 41.5 				 \\
						& 						&						& &						&${(-18.6)}$			& &	${(47.3)}$					& ${(42.7)}$			 \\
%\rule{0pt}{4ex}   
\toprule
\ac{NC} monomer	& \multicolumn{5}{l}{} \\
\midrule
\rad\ce{NO2} loss	&  {\added{27.7}}& {\added{23.1}}	& & {\added{40.7}} & {\added{36.1}}		& &{\added{40.7}}				& {\added{36.1}}	\\
\rule{0pt}{4ex}   
\ce{HNO2} loss		& -32.4				& -35.8				& &-19.3 				&{\added{-22.7}}		& & 43.5  							& 39.5  				\\
\bottomrule
\end{tabular}
\label{tab:intramolec}
\end{center}
$^a$ values from the work of Tsyshevsky \textit{et al.} \cite{Tsyshevsky2013}.\\
$^b$ values for the activation energy and total energy of reaction are the same for bond dissociation via homolytic fission.
%when treating homolytic fission as a barrierless process. 
\end{table}
\acs{PETN} reactant, \acs{TS} and product geometries were supplied by the authors (R. Tsyshevsky, personal communication, 19 April 2017) except in the case of the \rad\ce{NO2} homolysis product, which was not provided. 
The \rad\ce{NO2} leaving group used in the homolysis reaction calculation was independently geometry optimised, with its energy contributing additively to the products of the reaction. \added{The \ac{PETN}\ce{-HNO2} \ac{TS} and product was provided as a complex. }
The same method and basis (\acs{wb97xd}/ 6-31+G(2df,p)) were used in order to recreate the published reaction energies, \added{however, the calculated values fall below the literature values by  0.3 - 1.2 kcal mol$^{-1}$. 
A possible explanation may be that the geometries provided do not exactly match those used in the study, particularly in the case of \rad\ce{NO2}.
A contribution may also arise from variation in calculation setup and the use of a different version and compilation of the \ac{G09} program, leading to fluctuations in the exact values obtained. 
These differences would be amplified %when deriving
when combining individual energies for the calculation of reaction energy values. Cumulatively these changes may explain the discrepancy between the observed values and those in the publsihed study. %Tsyshevsky \textit{et al.}.  
Nevertheless, the calculated value for the activation energy of homolytic fission of \acs{PETN} is comparable to the experimentally derived values of 45.9$\pm$1.2 - 47.3$\pm$0.3 kcal mol$^{-1}$ \cite{Ng1976,Oxley2012,Chambers2002}, and %, measured using \ac{DSC}, and 
$\Delta H _{r}$ falls within the experimental bounds of% \ce{O-NO2} 
nitrate homolysis in alkanes (35.6$\pm$0.3 - 42$\pm$0.3 kcal mol$^{-1}$) \cite{Luo2002}. 
}
%give a range for diff sizes of alkenes%%%%%%%%
%scaling factor for wb97xd CHECK
%How does inclusion of zero point energy affect the enthalpy values?
%
%Semi-empirical methods are also known to over-estimate reaction barriers (and underestimate reaction energies). CHECK THIS AND FIND REF
%, but are still not expected to account for the 20 kcal mol$^{-1}$ deficiency %discrepancy in the case of homolysis. 
%, it can be seen that the obtained PETN reaction energy in the case of homolytic fission (\rad\ce{NO2} loss) varies greatly from the value published by Tsyshevsky \textit{et al.}
%It was assumed that the supplied geometries were those used to generate the energy values quoted in their study. Inspection of the forces for the given structures showed that they were %in fact 
%not converged. 
%As the same structures were used to calculate values listed in table \ref{tab:intramolec}, the unconverged geometries do not explain the large discrepancy between the published energies and values obtained here.
%% or do they? Perhaps because forces are not converged, it is unclear what will happen when you just calculate the associated energy -though I'm sure its also fine. The calc uses whatever structure is given. And tbh I don't know the intricacies of its calculation/ integration loops/ whether it will really make a difference
%% , though a possible explanation may be that different geometries were used for the values presented in the study.  (maybe this is too obvious to say)
%A possible explanation is that the products of homolytic fission were treated as a complex, rather than individual molecules, though % remain complexed 
%following bond breaking, 
%this was not highlighted %discussed 
%in the original publication nor explicit in the supplied geometries. %from the authors. 
%Under this assumption, the product species would move apart immediately following fission. 
%If the energy of the product species were calculated separately in the original study, it could be that the departing \rad\ce{NO2} was not geometry optimised, but left in its complexed geometry.
%%In the quoted study, it , even if the leaving group and molecule were not calculated together. 
%This is relevant in the scenario where the \rad\ce{NO2} has not yet moved far enough from the remainder of the \ac{PETN} molecule to fully relax. 
%Whilst the calculated $\Delta G_{r}$ %free energy of reaction 
%deviates greatly from the value in Tsyshevsky's study, % by almost half, %poo sentence CHECK
%
%This was tested against the \ac{NC} monomer 
%However these details were omitted from both the literature and the supplied geometries. (Does it matter though - the following paragraphs confirms the mechanism as proof of concept that it actually happens - that way) so I can get away with it.)   
 %Additional explanations may be that %and the obvious suggestion that perhaps alternative geometries were used ,....  in addition to a different compilation of Gaussian
%Despite the difference in the calculated $E_{a}$ values, both fall into the range of PETN experimental activation energies for decomposition. Actually no, mine is too low. 
%NOTE: kuklja kept the HNO2 complexed with the 
%
%Add in the zero point correction for enthalpy
%Add in the optimised PETN geometries (that I was given)
%Add in the PETN geomtries that I did myself (from scratch - which are slightly off).
% These energies don't seem to reflect the excel sheet... update

The homolytic fission reaction was applied to a \ac{NC} monomer singly nitrated at C2 (figure \ref{fig:homol_scan}). 
% freenitr-NO2-C2-scan
% floppy_NO2-C2-scan ***
%get an react coord pic if you can (not v interesting though)
% one scan was complete freeze except with the O-NO2
% the other one alloed relaxation aroud the NO2 and its orientation
%multireference character, vs DFT
% Check Yao2018, and read up on multireference characters. 
%
%Scan of the PES
The \ce{O-NO2} bond was incrementally stretched using geometry scanning, to obtain an energy profile of the reaction. 
%The energy profile of the reaction was obtained via %\ac{PES} 
%geometry scans of the \ce{O-NO2} bond elongation, as \ce{NO2} moved away from the \ac{NC} monomer. 
For the rigid scan (figure \ref{fig:homofiss_graph}), only the internal angle of the departing %\rad\ce{NO2}
\ce{NO2} and coordinates referencing its orientation relative to the rest of the molecule were allowed to relax. %Include the details of this calc, which coords allowed to relax, in the appendicies?
As the scan progressed, the \ce{NO2} internal angle increased from 129.2$\degree$ to 134.6$\degree$, at maximum separation of 5 $\text{\AA}$ from the bridging oxygen (Ox). This corresponds to the literature value for the \ce{O-N-O} internal angle (134.3\degree) confirming the formation of a \rad\ce{NO2} radical. %\cite{Trostchansky2010}
%Nevertheless, it is known that the activation energy for bond dissociation is equivalent to the bond enthalpy. 
%SO what?
%This confirms the formation of the radical species via homolytic fission, and that at a separation of 4 $\text{\AA}$ the radical is fully formed and relaxed. 
%
%The energy of reaction (table \ref{tab:intramolec}) is %2kcal mol${-1} 
%higher than that of PETN, which may be expected

\begin{figure}[t]
\centering
%\begin{subfigure}[0.3\linewidth]
%\includegraphics[width=\linewidth]{EN_NO2_leave_prot_arrow}
%\label{fig:EN_homol_scan}
%\end{subfigure}
%
%\begin{subfigure}[0.3\linewidth]
\includegraphics[width=0.3\linewidth]{homolysis_arrow}
%\includegraphics[width=\linewidth]{homolysis_arrow}
%\caption{ }
%\label{fig:NC_homol_scan}
%\end{subfigure}
%\caption{The \ce{O-NO2} bond was elongated during rigid and relaxed \ac{PES} scans simulating homolytic fission for \ref{fig:EN_homol_scan} ethyl nitrate and the \ref{fig:NC_homol_scan} \ac{NC} monomer.}
\caption{The \ce{O-NO2} bond was elongated during rigid and relaxed \ac{PES} scan to simulate homolytic fission for the \ac{NC} monomer.}
\label{fig:homol_scan}
\end{figure}
%The weird little bump is step 14, ie. bond separation 2.69233 angle 137.286
%After this bump is where the angle starts to decrease (even though the energy is still rising)
\begin{figure}[ht]
\centering
\includegraphics[width=0.7\linewidth]{ONOangle_HOMOFISS_scan}
\caption[The change in \ce{O-N-O} internal angle with homolytic fission.]{The \added{change} in the \ce{O-N-O} internal angle as the \ce{NO2} group is pulled away from the \ac{NC} monomer during a rigid geometry scan of homolytic fission.}
\label{fig:homofiss_graph}
\end{figure}
%
%The values obtained for \ce{HNO2} elimination of PETN follow the results given by Tsyshevsky much more closely. 
%The energies fall within 5 kcal mol$^{-1}$ and 6 kcal mol$^{-1}$ for the Gibbs free energy of reaction and activation barrier, respectively. 
%This is within a reasonable margin of error for comparing with experimentally obtained values, though still larger than expected for those derived using the same method, basis and structure. %DO you have any experimental values for this reaction?

% Nevertheless, it still falls inside the range of experimentally measured PETN degradation activation barriers. %REF and check?
%I think the below actually goes into the methodology
In the case of the \ac{NC} monomer, both rigid and relaxed scans failed to capture the \ac{TS} for cleavage of the nitrate group via interaction with the \textalpha-hydrogen. A guess transition state was constructed based on the \ac{TS} of the analogous reaction for PETN (figure \ref{fig:elim_ts_NC}), and optimised %using opt=TS 
to produce the structure of the correct imaginary vibration. 
The pattern for the \ac{NC} monomer resembles that found for \ac{PETN}; homolysis is endothermic but with lower activation barrier, %put in the numbers
whilst \ce{HNO2} elimination is exothermic, with a much higher barrier. It is anticipated therefore, that the rate of homolytic fission will be faster, whilst \ce{HNO2} loss will happen more slowly, whilst contributing to system heating and increasing acid concentration. 
%Kinetics??? CHECK - should be a quick calc
%more details on what you found 

%Summary Sentence - why are the energies of interest, here?
%comment on the significance of the energies and descibe a bit more of what you see geometrically and via the discovered energies
%
%dashed lined heavier, please, and PETN is missing the C-H bond
\begin{figure}[h]
\centering
\begin{subfigure}[t]{0.4\linewidth}
\centering
\caption{}
\includegraphics[width=\linewidth]{elim_hono_ts_PETN_dash}
\label{fig:elim_ts_PETN}
\end{subfigure}
\begin{subfigure}[t]{0.4\linewidth}
\centering
\caption{}
\includegraphics[width=\linewidth]{elim_hono_ts_NC_dash}
\label{fig:elim_ts_NC}
\end{subfigure}\\

{\textcolor{red}{\CIRCLE} oxygen \quad \textcolor{blue}{\CIRCLE} nitrogen \quad  \textcolor{gray}{\CIRCLE} carbon \quad $\bigcirc$ hydrogen}
\caption[\ac{TS} for the elimination of \ce{HNO2} in \ref{fig:elim_ts_PETN} PETN and \ref{fig:elim_ts_NC} \ce{NC}.] {\ac{TS} for the elimination of \ce{HNO2} by removal of the \textalpha-hydrogen by the \ce{NO2} leaving group in a) PETN and b) \ce{NC}. Orange dashed lines indicate bonds breaking and blue dashed lines indicate bonds forming.}
\label{fig:elim_ts}
\end{figure}
%

\subsection{Acid hydrolysis mechanism}
\subsubsection{Protonation site}
\label{AH_Protonation}
%Why do we need to look at protonation?
In polar, protic solvents such as water, the fast-exchange of protons between the aqueous medium and  %acceptor groups 
the monomer is expected. %Acid species are also expected to protonate 
Computational studies by Jebber \textit{et al.} and Liu \textit{et al.} probed the protonation behaviour of $\beta$-glucose \cite{Jebber1996,Liu2010}. 
Key findings demonstrated that the most favoured protonation site in glucose was greatly influenced by the conformation of the C6 side branch. For the C6 hydroxymethyl chain orientated in the gauche position, protonation of the ring-oxygen produced the most stable structure.  %; the lowest energy site for protonation whilst the C6 hydroxymethyl chain was in the gauche position, was the oxygen of the C1 hydroxyl group. - doesn't lead to denitration though
%The analogous group in the \ac{NC} monomer, with the C6 nitrate in the gauche position being the lowest energy conformer %see chapt 1
%, is the C1 capping group oxygen.   

The protonated \ac{NC} monomers explored in this section are shown in figure \ref{fig:proton_site}. The bridging oxygen (Ox)% linking the nitrate to the remainder of the molecule
, the C1 capping group oxygen, and the terminal nitrate oxygen sites (Ot) were protonated and their relative energies compared in order to determine the site most likely to stabilise the proton at thermal equilibrium. Protonation also occurs on other sites in the molecule, such as at unsubstituted hydroxyl oxygen sites, the capping group oxygen on C4 and O1 in the glucose ring. Whilst it is possible that protonation at these further sites in the monomer would contribute to degradation, these processes would occur \textit{via} alternative mechanisms without the involvement of denitration. For the purposes of studying acid hydrolysis, only the sites peripheral to the nitrate leaving group were explored. %REf if there is any evidence - peeling off?

\begin{figure}[htp]
\centering
\begin{subfigure}[t]{0.3\linewidth}
\centering
\includegraphics[width=\linewidth]{H_bridging_b}
\caption{Bridging}
\label{fig:proton_site_bridge}
\end{subfigure}
\hfill
\begin{subfigure}[t]{0.3\linewidth}
\centering
\includegraphics[width=\linewidth]{H_terminal_b}
\caption{Terminal}
\label{fig:proton_site_terminal}
\end{subfigure}
\hfill
\begin{subfigure}[t]{0.3\linewidth}
\centering
\includegraphics[width=\linewidth]{H_cap_b}
\caption{Capping}
\label{fig:proton_site_cap}
\end{subfigure}
\caption{Protonation sites on the \ac{NC} monomer for hydrolysis of the nitrate at the C2 position.}
\label{fig:proton_site}
\end{figure}
%
%The mechanism of protonation was not explored in depth here; 
%It was assumed that protons in the system would be in fast exchange with the solvent. 
%Any effects of tunneling were included within this assumption. 
The energies for the optimised protonated monomer conformers are listed in table \ref{tab:reactions}. 
There is good agreement between \acs{wb97xd} and \acs{B3LYP} values in the vacuum, with \acs{B3LYP} predicting a \added{slightly} lower reaction energy in all cases. 
\added{This is likely due to the known underestimation of reaction energies by \acs{B3LYP}, particularly in systems involving multiple C-C bonds \cite{Check2005} and with weak, non-covalent interactions \cite{Bachrach2008,Grimme2010}. These issues are treated in \acs{wb97xd} with the inclusion of medium and long term dispersion, and has been shown to improve the accuracy of modelled C-C bond lengths \cite{Fokin2012}, in particular outperforming \acs{B3LYP} in calculations involving hydrogen bonding networks \cite{Bachrach2013}. } % WHY % and solvent, with \ac{B3LYP} . 
The large difference in the reaction energies between the gaseous and implicit solvent values are explained by the instability of \ce{H3O+} in vacuum, where it prefers to lose the proton and exist as water. Ionic species exhibit much greater reactivity in the gaseous phase compared with their neutral counterparts \cite{Pliego2001,Pliego2002,Chandrasekhar1985}. 
When in solution, the positive charge is solvent stabilised% and the reactivity of the  \ce{H3O+}ion is attenuated by the solvent
; the proton is less readily released. %OR rather, a solvent shell will form around it, allowing the charge to be stabilised by the polar water molecules. Proton exchange will occur, but essentially it is more stable. Reaction with anothes molecule requires disruption of the solvent shell. Overall more stabilised and less easy to react. 
% In the real system, rapid proton exchange 
\begin{table}
\begin{center}
\caption[Free energies of protonation for \ac{NC} monomer.]{Free energies of protonation, for each of the oxygen sites of interest on the \acs{CH3/CH3} monomer of \ac{NC}, nitrated at the C2 site.}
%Slot in MP2 if I have time (and a subsequent one-liner about it)
\begin{tabular}{ l *{4}{S[table-format = 2.1]}} 
\toprule
\multirow{2}{*}{Protonation site} & \multicolumn{4}{c}{$\Delta$G\textsubscript{r} /kcal mol\textsuperscript{-1}} %& \multicolumn{2}{c}{$\Delta$H\textsubscript{r} }
\\
\cline{2-5} 
  & \acs{wb97xd} & {PCM \added{(water)}} & \acs{B3LYP} & {PCM \added{(water)})} \\
\midrule
%% Note, using unconverged energies here 
% Bridging 				&  -26.04105 	& 4.02759   	& -28.67067 	& 11.29338  \\
% Terminal	 (syn)& -29.84814	&   24.28650	& -31.18500 	&  15.32916 \\
% Terminal (anti)	& -20.53548	&   10.44288	& -22.40595	& 11.66634  \\
% Capping 				& -29.84877	&   3.61683	& -31.18563 	&  -1.16235 \\    
% Non zpe corr, but structure corr (not for the PCM b3lyp though)
% Bridging 				&  -26.04105 	& 4.29786    	& -28.67067 	& 11.29338  \\
% Terminal	 (syn)& -29.84814 	& 13.41459	& -31.18500 	&  15.32916 \\
% Terminal (anti)	& -20.53548 	& 11.15415	& -22.40595	& 11.66634  \\
% Capping 				& -29.84877	& 1.01556		& -31.18563 	&  -1.16235 \\
%% Corrected with structures and zero point corr on 08/12/19  
% Bridging 				& -26.88687 	& 2.83903    	& -29.82831	& 2.68246  \\
% Terminal (syn)		& -30.06758 	& 12.7897		& -31.76046 	& 11.3996  \\
% Terminal (anti)		& -20.51871 	& 10.7196		& -22.78125	& 9.20866 	\\
% Capping 				& -30.06838	& 0.90806		& -31.76137 	& -1.08328	 \\ 
% Corrected with structures and zero point corr on 08/12/19  
 Bridging 				& -26.9 	& 2.8 	   	& -29.8	& 2.7 	\\
 Terminal \added{(syn)}		& -30.1 	& 12.8		& -31.8 	& 11.4	\\
 Terminal \added{(anti)}		& -20.6 	& 10.7		& -22.8	& 9.2 	\\
 Capping 				& -30.1	& 0.9			& -31.8 	& -1.1	\\ 
\bottomrule
\end{tabular}
\label{tab:reactions}
\end{center}
\end{table}
% Repeat these diagrams with the MP2 and B3LYP too, with a separate analysis (table) on the bonds and angles, in order to compare why the MP2 energies are a bit out there. 
%Redo this graphic to depict the new structures (4 pics)
% Pics in and out of solvent too, if poss

\begin{figure}[hp]
\centering
\begin{subfigure}[t]{0.35\linewidth}
\centering
\includegraphics[width=\linewidth]{corr_H_bridging_full}
\caption{Bridging}
\end{subfigure}
\hspace{5pt}
\begin{subfigure}[t]{0.35\linewidth}
\centering
\includegraphics[width=\linewidth]{corr_s_H_bridging}
\caption{Bridging [solvated]}
\end{subfigure}

\begin{subfigure}[t]{0.35\linewidth}
\centering
\includegraphics[width=\linewidth]{corr_H_terminal_up_full}
\caption{Terminal \added{(syn)}}
\label{fig:tu_vac}
\end{subfigure}
\hspace{5pt}
\begin{subfigure}[t]{0.35\linewidth}
\centering
\includegraphics[width=\linewidth]{corr_s_H_terminal_up}
\caption{Terminal \added{(syn)} [solvated]}
\label{fig:tu_solv}
\end{subfigure}

\begin{subfigure}[t]{0.35\linewidth}
\centering
\includegraphics[width=\linewidth]{corr_H_cap_full}
\caption{Capping}
\label{fig:cap_vac}
\end{subfigure}
\hspace{5pt}
\begin{subfigure}[t]{0.35\linewidth}
\centering
\includegraphics[width=\linewidth]{corr_s_H_cap}
\caption{Capping [solvated]}
\end{subfigure}

\begin{subfigure}[t]{0.35\linewidth}
\centering
\includegraphics[width=\linewidth]{corr_H_terminal_down_full}
\caption{Terminal \added{(anti)}}
\end{subfigure}
\hspace{5pt}
\begin{subfigure}[t]{0.35\linewidth}
\centering
\includegraphics[width=\linewidth]{corr_s_H_terminal_down}
\caption{Terminal \added{(anti)} [solvated]}
\end{subfigure}

\caption[Optimised protonated \ac{NC} monomer structures.]{Protonated \ac{NC} monomer structures after geometry optimisation to the level of \acs{wb97xd}/6-31+G(2df,p).
\textit{Left column:} In vacuum. 
\textit{Right column:} In implicit solvent \added{(\acs{PCM} = water).}
The interacting parts of the molecule are highlighted in 3D in the structures on the left (optimised in gas phase.)} %, showing interaction between the proton on the bridging site with the capping group oxygen.}
\label{fig:proton_site_full}
\end{figure}
%B3LYP ones in appendicies
%

In the gaseous phase, the most thermodynamically favoured protonation sites are at the terminal (syn) and capping positions. %The energy of each of the protonated species is identical. 
Inspection of the optimised geometries explains the similarity of their $\Delta G_{r}$ values. The protonated terminal (syn) and capping monomers (figures \ref{fig:tu_vac}) and \ref{fig:cap_vac})) present nearly identical geometries in vacuum. In the terminal (syn) case, the proton has moved towards the capping group oxygen, effectively undergoing a proton transfer and generating the same structure as the protonated capping monomer. 
This is in contrast to the solvent phase (figure \ref{fig:tu_solv})), where the protonated nitrate group is rotated perpendicularly to the ring and the proton remains on the terminal oxygen; the solvent presents stabilisation for the protonated Ot. 
This indicates that the terminal (syn) site is is only likely to remain protonated in solvent, and that protonation at the site in vacuum is highly unstable.
The remainder of the monomers exhibit only minor changes in geometry between vacuum and solvent. Under solvent conditions, the most stable site for protonation is the capping group (which is unlikely to lead to denitration), followed by the bridging oxygen. %explaining the energy difference between the terminal (syn) site protonation and capping group protonation between vacuum and solvent 

% Though $\Delta G_{r}$ values are positive in all cases apart from the capping group .... protonation is still going to happen, but it may tell us that the proton may not spend much time on the O
%
%Thus, the energy gained from losing the proton is negative, %less pronounced 	when in solution. 
%
%Appendicies, the  
%You wasted all this time because you did an excel cell error.
%FIXED
%For the solvated results, the two functionals perform similarly, except in the case of the Terminal (syn) structure. Inspection of the geometries and \ac{QTAIM} analysis did not uncover any significance differences in structure or critical points (see appendix \ref{AH_Protonation_QTAIM}). 

%Draw the two water one too
\begin{figure}[h!]
\centering
\includegraphics[width=0.25\linewidth]{neutral_hydro_1}
\caption{The attempted coordination geometry of a single water molecule to a \ac{NC} monomer.}
\label{fig:neutral_prot}
\end{figure}
Water as the protonating species was also attempted (figure \ref{fig:neutral_prot}), by optimisation of one, two and three water molecules in coordination with the nitrate site in the \ac{NC} monomer.  The same was repeated with the hydronium ion to probe likely coordination geometries. However in both cases, no stable complexes could be isolated. It is anticipated that a much larger network of waters around both the regions surrounding the nitrate moiety and the wider molecule, are required to achieve a stable water coordination structure. %ref that 5 water study  I jsut foudn 
This would be of interest for further investigation into the mechanism of neutral hydrolysis involving the autoionsation of water \cite{Silva2013}. %The optimisation of \ce{H3O} in coordination with the fully nitrated monomer was also tested, but was only possible up to the level HF/6-31g.  
%Dragrams of the water coord geometrties you tried, pls?
%to see whether it was possible to stabilise in coordination.  
%Thus, omitting entropy effects. 
 %also include the function of hydrolysis in the degradation of sugars, in the lit review
Evaluation of the energy of protonation at each site found that the bridging and capping
oxygens were the most likely sites; with only the bridging isomer likely to contribute to denitration. However, protonation at the terminal structures will also be explored as the starting point for the subsequent denitration stage.  \\
%Big table of all the scans I did (for hydrolysis TS) \\
%Columns:  \\
%Scanned co-ordinate. Distance scanned. Observation. (TS found? etc)\\
%add in B3LYP too, if you can be bothered
%
%SMD energies look ridiculous
%\begin{table}[htp]
%\begin{center}
%\caption[Free energies of protonation using \acs{PCM} and \acs{SMD} implicit solvents.]{Free energies of protonation on a monomer of \ac{NC} nitrated at the C2 site, under different implicit solvent conditions.}
%\begin{tabular}{ l *{3}{S[table-format = 2.4]}} 
%\toprule
%\multirow{2}{*}{Protonation site} & \multicolumn{2}{c}{$\Delta$G\textsubscript{r} /kcal mol\textsuperscript{-1}} %& \multicolumn{2}{c}{$\Delta$H\textsubscript{r} }
%\\\cline{2-3}
%  & PCM & SMD\\
%\midrule
% Bridging 				& 2.83903 \\
% Terminal (syn)	& 12.7897 \\
% Terminal (anti)	& 10.7196	\\
% Capping 				& 0.90806	 \\ 
%\bottomrule
%\end{tabular}
%\label{tab:smd}
%\end{center}
%\end{table}
%

\subsubsection{Denitration by hydrolysis}
Following the protonation step, possible transition states for the removal of the nitrate were investigated. Direct dissociation of \ce{NO2} from the protonated species was explored (figure \ref{fig:no2-leave})), along with the simultaneous approach of a proton and cleavage of the \ce{NO2}(figure \ref{fig:h-come-no2-leave})). %(figure \ref{fig:scan_coords}). 
The scan of the proton moving towards the bridging site was also conducted, to determine whether any elongation of the \ce{O-NO2} occurred as a result of the formation of the proton-oxygen bond (figure \ref{fig:h-approach})). 
 
%was also completed to gain insight to the energy profile of the process. %any scans to show? 
\begin{figure}[ht]
\centering
\begin{subfigure}[b]{0.25\linewidth}
\centering
\caption{}
\includegraphics[width=\linewidth]{NO2_leave-prot_arrow}
\label{fig:no2-leave}
\end{subfigure}
\hfill
\begin{subfigure}[b]{0.25\linewidth}
\centering
\caption{}
\includegraphics[width=\linewidth]{NO2_leave-prot_arrow_2d}
\label{fig:h-come-no2-leave}
\end{subfigure}
\hfill
\begin{subfigure}[b]{0.25\linewidth}
\centering
\caption{}
\includegraphics[width=\linewidth]{H_bridge-approach_arrow}
\label{fig:h-approach}
\end{subfigure}
\caption{The scanned coordinates of %\ref{fig:h-approach}
a) dissociation of \ce{NO2},  %\ref{fig:h-come-no2-leave}
b) concerted protonation and \ce{NO2} dissociation and
c) proton approach. 
\added{Green arrows indicate bond formation, red arrows indicate dissociation. }}%\ref{fig:no2-leave}
\label{fig:scan_coords}
\end{figure}

The relaxed \ac{PES} scan of \ce{NO2} removal from ethyl nitrate protonated at the bridging site was used as a preliminary test for the mechanism of denitration, following protonation (figure \ref{fig:PES_EN_scan}). 
Unrestricted \ac{wb97xd} was used in case of the formation of  \rad\ce{NO2} instead of the expected \ce{NO2+}, with 20 steps of 0.1 $\text{\AA}$. However, bond dissociation was not illustrated in the energy profile even when extending the scan distance to a maximum of 6.4 $\text{\AA}$. Instead, a steady increase in the energy was observed. %fig?
It was observed that as the nitro group distance increased, its internal angle increased to 180$\degree$, confirming the formation of \ce{NO2+}. It is anticipated that the departing \ce{NO2+} will further react with water in the system to produce acids conducive to further hydrolysis. 
As the scan proceeded, the molecule rotated and the \ce{NO2+} leaving group aligned with the hydroxyl in an orientation suitable for peroxy group formation (figure \ref{fig:EN_pero}).  %This was the expected outcome for hydrolysis, as 
This mechanism was previously considered in the degradation reactions of \ac{PETN}; %section ref
it was found that the energy of this process was higher than that of \ce{HNO2} elimination, where the nitrate was not initially protonated. The formation of the peroxy bond was not facilitated by the scan parameters due to forced increase of the \ce{O-N} distance. The same process was not observed when the scan was applied to the \ac{NC} monomer. As the formation of the peroxy geometry required re-orientation of the whole molecule, the energy of this rearrangement was not favourable for the bulky \ac{NC} unit. In the real polymeric system, it may induce distortion of the sugar ring, proving even more energetically demanding.  %This was also done n both B3LYP and wb97xd. (MP2 too, but lets leave that)

%2D scan + pics
%H approach scans

%did I apply this to my monomer?
%I didnt get a TS out of this, regardless....
%This may be due to the specification of the spin  /charge?
\begin{figure}[hb]
\centering
\begin{subfigure}[b]{0.3\linewidth}
\centering
\includegraphics[width=\linewidth]{EN_NO2_leave_prot_step1-BL}
\caption{}
\end{subfigure}
\hspace{2em}
\begin{subfigure}[b]{0.3\linewidth}
\centering
\includegraphics[width=\linewidth]{EN_NO2_leave_prot_step7-BL}
\caption{}
\end{subfigure} \\
\begin{subfigure}[b]{0.3\linewidth}
\centering
\includegraphics[width=\linewidth]{EN_NO2_leave_prot_step11-BL}
\caption{}
\end{subfigure}
\hspace{2em}
\begin{subfigure}[b]{0.32\linewidth}
\centering
\includegraphics[width=\linewidth]{EN_NO2_leave_prot_step_cont5-BL}
\caption{}
\label{fig:EN_pero}
\end{subfigure}
\caption{Geometries from steps 1, 7, 11 and 26 of the geometry scan of Ox protonated \ac{EN} denitration.}
\label{fig:PES_EN_scan}
\end{figure}
%%%% POSSIBLY EXPLORE PEROXY deg routes in the next chapter, OR whether the H is snatched back for HNO2?
%
% Actually, you only allowed internal angles and one orienation one to relax, in the case of the monomer - so the below is bogus - you weren't able to gauge anything from these very restricted scans, because the no2 wasnt' allowed to fully turn to planar. 
%Similar scans were also performed on the \ac{NC} monomer protonated at the bridging site, where the \ce{O-NO2} bond was elongated in order to determine nature of the lost \ce{NO2}. Initially, the whole monomer was held rigid, as only the \ce{O-NO2} was incrimentally increased to produce an energy profile of the bond separation. Subsequently, the internation angle within \ce{NO2} and those pertaining to it orientation with respect to the wider monomer structure were allowed to relax. Despite increasing the scanning distance to 4 $\text{\AA}$. , with only the nitrate and proton, with internal angles allowed to relax.  The same was repeated in implicit solvent. 
% leaving group is lost as \ce{NO2}, and the nature of the monomer following denitration. 
%probe the generated denitration products.
% to simulate the removal of \ce{NO2}. 
Proposed 4-membered ring and 6-membered ring \ac{TS} were also investigated in order to determine whether they formed energetically and geometrically reasonable structures facilitating nitrate removal, with reformation of the alcohol group on the sugar ring (figure \ref{fig:all_da_TS}).  
%
\begin{figure}[ht]
\centering
\begin{subfigure}[t]{0.2\linewidth}
\caption{}
\centering
%The bonds in the 4 mem ring TS are longer than those in the 6-rings. Sort this if you have time. 
\includegraphics[width=\linewidth]{4mem-terminal}
%\caption{4-membered ring transition state with protonation at the terminal nitrate oxygen.}
\end{subfigure}
\hspace{3em}
\begin{subfigure}[t]{0.2\linewidth}
\caption{}
\centering
\includegraphics[width=\linewidth]{4mem-bridge}
%\caption{Protonation at the bridging nitrate oxygen site.}
\label{fig:4mem_b}
\end{subfigure}\\
%
\begin{subfigure}[t]{0.2\linewidth}
\caption{}
\centering
\includegraphics[width=\linewidth]{6mem-terminal}
%\caption{Protonation at the terminal nitrate oxygen.}
\end{subfigure}
\hspace{3em}
\begin{subfigure}[t]{0.2\linewidth}
\caption{}
\centering
\includegraphics[width=\linewidth]{6mem-bridge}
%\caption{Protonation at the bridging nitrate oxygen site.}
\end{subfigure}\\
%
\begin{subfigure}[t]{0.2\linewidth}
\caption{}
\centering
\includegraphics[width=\linewidth]{6mem-neutral}
%\caption{Neutral hydrolysis conditions with not prior protonation.}
\end{subfigure}
\hspace{3em}
\begin{subfigure}[t]{0.2\linewidth}
\caption{}
\centering
\includegraphics[width=\linewidth]{6mem-h3o}
%\caption{Concerted protonation-denitration, under acid hydrolysis conditions.}
\end{subfigure}
\caption[Proposed 4-member and 6-member ring transition states for the denitration of a nitrate ester.]{Proposed 4-member and 6-member ring transition states for the denitration of a nitrate ester, under various hydrolytic conditions. R = \ce{CH3} in the case of methyl nitrate, R = \ce{CH2CH3} in the case of ethyl nitrate and R = \ce{(H3CO)2C6H9O3} for the monomer.}
\label{fig:all_da_TS}
\end{figure}
%
Optimisations were attempted with both full geometry relaxation, and various frozen coordiate schemes for each proposed \ac{TS}. R groups were %simplified 
truncated even further to methyl nitrate ( R = \ce{CH3}), in effort to limit degrees of freedom during %the challenging 
optimisation of the \ac{TS} structures. Despite this, no structures were able to achieve convergence \textit{via} full relaxation. %\added{suggesting that the denitration pathway through these \acs{TS} were unviable.} 
Freezing the bulk of the molecule with relaxation only around the nitrate group and its' coordinating species, followed by relaxation of the wider molecule with fixed coordinates around the nitrate allowed sequential optimisation of different moieties, increasing chances of global energy minimisation.  % with expansion to ethyl nitrate and the \ca{NC} monomer, following 
It was possible to optimise the 4-membered ring bridging \ac{TS} on the \ac{NC} monomer with frozen \ac{TS} ring geometry \textit{via} preliminary optimisation of the ring structure with methyl nitrate. The optimised ring geometry was then placed on the monomer, with fixing of thee coordinates, allowing the remainder of the molecule to relax. 
% and relaxation of the remaining molecule (figure \ref{fig:4mem_b}). %actually we opt the ring with methyl nitrate first then reattached it to ring
A rigid scan was then performed of the 4-membered ring transition state, starting from the bridging site protonated monomer (figure \ref{fig:all_da_TS}(d)). 
It was revealed that as the nitrate moved away from the system, the proton moved to the capping group site rather than remain on the bridging oxygen as a hydroxyl, %(even when solvated)
as was initially expected. Instead, a ketone group was formed between the bridging oxygen and the ring. At subsequent steps, the ketone group caused the C2 - C3 bond to elongate and break. The scan eventually showed the \ce{NO2} leaving group reclaiming the proton from the capping group oxygen, leading to ring fission. The activation and kinetic barrier involved in ring fission ($\delta$G = 56.9 kcal mol$^{-1}$) is much higher than that of denitration ($\delta$G = 10.1 - 24.8 kcal mol$^{-1}$) \cite{Shukla2012b}; a study on the acid hydrolysis of glucose and xylose demonstrated that ring-opening intermediates were either extremely short lived, or not observed at all \cite{Qian2011}. 
The open-chain product of the scan is likely due to the geometric constraints placed on the geometry of the departing \ce{NO2} group, rather than a likely physical %energetic 
process. However, it sheds light on the scheme by which ring fission may occur under conditions of elevated temperature or pressure, which has been implied in previous work involving the formation of a ketone at earlier stages of the reaction. %REf that oldschool paper with the diagrams
%include distances as before, later
%shorten the captions. they silly. Put them in the body of text / big caption at the bottom

\begin{figure}[ht]
\centering
\begin{subfigure}[b]{0.5\linewidth}
\centering
\includegraphics[width=\linewidth]{C2-OH-opt-ring-freeze-NOscan_2_step13}
\caption{In the initial stages of increasing \ce{O-NO2} distace, the proton moves to the capping group.} 
\end{subfigure}
%
\begin{subfigure}[b]{0.4\linewidth}
\centering
\includegraphics[width=\linewidth]{C2-OH-opt-ring-freeze-NOscan_2_step15}
\caption{At separation of over 3.3 $\text{\AA}$, the C2 \textendash C3 bond breaks leading to ring fission. The proton then moves back onto \ce{NO2} .}
\end{subfigure}
\caption{Relaxed scan of \ce{NO2} departure, starting with the 4-membered ring structure. }
\end{figure}
%
Attempts to isolate the other \ac{TS} structures were unsuccessful, even when simplifying the side chain to methyl nitrate and under implicit solvent conditions. %in the case it stabilised the charges on the strained structures. 
%Dualscan - compare different methods. This was the final nail in the head that we woudln't be able to isolate an actual TS

%It was found that none of the above structures were able to be optimised for the monomer and was attempted with a reduced methyl nitrate test system. 
%The transition states were also not obtainable for methyl nitrate, both in vacuum and solvent. 
%
%Things I did:\\
%4 membered ring\\
%\textendash	Scan using ethyl nitrate \\
%\textendash		Opt Ts using ethyl nitrate\\
%\textendash		considerations - sterics, and what energy barrier would be required to overcome the twist needed to obtain this state. Orbital overlaps?\\
%6 membered ring\\
%\textendash		considerations - sterics, and what energy barrier would be required to overcome the twist needed to obtain this state. Orbital overlaps?]\\
%\textendash		Would energetics allow you to skip the protonation step? Is it more favourable?\\
%
%C2 and water
%
%\textcolor{red}{Still to mention:\\
%- Compare the results from different methods - which were the best for describing the reactions?
%- Theoretical aspects to the above lines of arguement.
%}

\section{Summary}

%Homolysis is fastest, for the monomer too, with HNO2 coming in second due to slower rate / energetics. (Check if I did anything to actually find this out - may just have to compare energetics.)
\added{In this section, three different denitration reactions were explored as the first stage in ambient decomposition of \ac{NC}. The reactions were firstly tested on \ac{PETN}, where prior studies by Tsyshevsky \textit{et al.} had confirmed homolytic fission and elimination of \ce{HNO2} reactions to be the dominant mechanisms in thermolytic ambient ageing \cite{Tsyshevsky2013}. A singly nitrated monomer was used with methoxy capping groups, to simulate a unit in the \ac{NC} polymer chain. 

Thermolytic cleavage of the nitrate group was modelled \textit{via} homolysis and elimination of \ce{HNO2}. In the case of \ac{PETN} it was found that the reaction energies closely matched those in the published study and in the case of homolysis, fell within range of experimental measurements for enthalpy of reaction and activation energy. }
%were lower than expected when comparing with literature values. For homolytic fission, this may be  %due to treatment of the modelled reaction products as separate molecules instead of a complex, in the case of homolytic fission, or 
%due to the separate evaluation of the \ac{PETRIN} radical and \rad\ce{NO2} energies, where they should have remained in complex following the reaction. %Contributions may also have arisen as a result of differing software compilations. 
The same process was repeated for the \ac{NC} monomer, singly nitrated at the C2 site. The energy of homolytic fission was in good agreement with the expected value based on the outcome of the \ac{PETN} product. 
\ac{PES} scans of homolysis confirmed the loss of \rad\ce{NO2} for %both the case of \ac{PETN} and 
the \ac{NC} monomer. % \added{It is therefore expected that both reactions will occur in the system, with environmental }

The elimination of \ce{HNO2} \textit{via} intramolecular \textalpha-H transfer was compared with homolysis. The calculated energies of reaction and activation energy values gave good agreement with the published values in the case of \ac{PETN} \cite{Tsyshevsky2013}. 
Calculated \ac{NC} values were also within the anticipated range, based on the reaction for \ac{PETN}. \ac{PES} scans were unable to locate a \acs{TS} for the \ac{NC} monomer, however, a successful guess geometry was generated based on the analogous \acs{TS} structure in the reaction for \ac{PETN}. %In general, the energies of activation are higher for \ac{NC} than for \ac{PETN}, though this is reasonable, and neighbouring \ce{-ONO2} groups are said to destabilise ...
Enthalpies of reaction showed that this process was more exothermic in the case of \ac{NC} than for \ac{PETN}, and that the elimination of \ce{HNO2} was more thermodynamically favourable in \ac{NC} compared to homolysis. However, homolysis may occur more rapidly, as is the case for \ac{PETN}. 

%Scans did confirm that \rad\ce{NO2} left as a radical.
%Scans showed the energy profiles involved (again, was HNO2 ever seen to be formed? )
The protonation sites on the \ac{NC} monomer were probed for the most favourable position. It was found that in the gas phase, capping and bridging site protonation lead to the same protonated final structure. 
In the solvent phase, the capping site was energetically preferred seconded by the bridging site. %, though inspection of the optimised geometry showed that it was very close to that of protonation at the capping site. 
Protonation at the capping group site and subsequent reaction would more likely lead to chain scission; this avenue was disregarded in further studies focussing on the acid hydrolysis pathway, in favour of protonation at the bridging oxygen site where the geometry allowed for easy nitrate removal.
 
Optimisation of water - monomer and hydronium - monomer complexes were attempted, in order to obtain information on the nature and orientation of the protonation complex. It was not possible to isolate any stable structures, implying that a larger stabilising network of waters is likely required. 
%summarising lien here on the two diff reactions, and the protonaion results?

Dissociation of \ce{NO2} from the protonated analogues of ethyl nitrate and the \ac{NC} monomer were scanned using a variety of rigid and relaxed \ac{PES} schemes. In the removal of \ce{NO2} from protonated ethyl nitrate, the release of \ce{NO2+} was indicated by the change of geometry around the nitrate from bent to linear, as the \ce{O-NO2} bond elongated. 
Rotation of the remaining ethanol and complexed \ce{NO2+} showed orientation suitable for formation of a peroxide. This rotation was not observed in the case of the monomer, however the leaving group still presented as \ce{NO2+}.
4 and 6 membered ring \ac{TS} were also tested for the denitration reaction. 
%https://www.ch.imperial.ac.uk/rzepa/blog/?p=10015 see here for some rings that are supposed to work...
Unexpectedly, it was found that none of the 6-membered ring structures could be isolated, regardless of truncation of the system to a protonated methyl nitrate model, or using the un-protonated monomer to simulate concerted protonation-denitration. 
%prior protonation using methyl nitrate geometries. 
In the case of the bridge-protonated \ac{NC} with formation of the 4-membered ring \ac{TS} at the C2 nitrate, it was possible to relax the \ac{NC} monomer structure around the ring so long as the ring geometry itself was frozen. As the leaving group moved further from the remainder of the molecule, the hydroxyl group located at C2 formed a ketone, losing the proton to the departing \ce{NO2+}, to form \ce{HNO2} in later stages of the scan. Eventually ring fission occurred, as the \ce{HNO2} move sufficient distance away, and the formation of the ketone forced the adjacent \ce{C-C} bond in the ring to stretch, and then break. It is known that the energy required for this process is much higher than that of denitration, so is unlikely to contribute to the initial stages of ambient ageing.
%However, as the energy barrier associated with this are much higher than that of  - dont' expect it, should solely be denitration then peeling off first

\added{In summary, this section explored the degradation reactions responsible for denitration, the primary step leading to more extensive decomposition. With respect to thermolytic mechanisms, it was found that homolysis and \ce{HNO2} elimination were both feasible initial denitration routes. \ce{HNO2} elimination was more energetically favourable and likely to dominate in ambient, purely thermodynamic conditions. Investigation into acid hydrolys pathways found that protonation at the \acs{NC} bridging site was most likely, and whilst a \acs{TS} could not be obtained for the reaction, it was confirmed that hydrolytic denitration occurred with the release of \ce{NO2+}. %limitations?
This work should be extended with further exploration of water complexation around the \acs{NC} monomer through the use of explicit solvent.

In the next chapter the reactions of the species generated as a result of the primary denitration steps will be studied to elucidate the formation pathways of the final experimentally observed degradation products. }
% 
%AH rate was not able to be compared as a TS was not found.
%Protonation occurs on both terminal and bridging sites of the monomer, with location at the bridging site conducive to the removal of \ce{NO2+}.
%
%TS were not able to be isolated for the denitration step, even with coordination with water in different orientation and both 4 and 6 mem ring TS. 2D scans did reveal a possible TS but it did not lead to the desired denitration pathway. 
%See water clusters around NC by \cite{Gunko2014}.
%
%alternate story if given more time to correct geometies - do the scans and TS with the reactified structure, plus the additional of on the interaction with the capping site. 
%
%Determine whether 
%Overall, the formation of \rad\ce{NO2} for the homolytic reaction, and \ce{NO2} for the elimination of ce{HNO2} and in the initial stages of acid hydrolysis reactions was confirmed. But we have no idea of any rates, etc

%% whilst you're still writing up
\setcounter{chapter}{3}

\chapter{Post-Denitration Reactions}
\label{chapterlabel6}
\graphicspath{ {./R_chap_4_pics/} }

% In this chapter:
% Which reactions happen post denitration, and which don't (based on pure thermodynamics)
% Which ones contribute to the end products seen in literature, and which much be consumed in subsequent reactions
% {Mention that this list is not likely to be exhaustive, but cover some of the more ''obvious'' ones}

%[Fix the arrows of the reaction schemes? So that eluting products leave with a u shaped arrow?]

\section{Introduction}%%%%%%%%%%%%%%%%%%%%%%%%%%%%%%%%%%%%%
% Talk about the prevalence and importance of secondary + products. 

% Statement - it is these secondary reactions that do the majority of the decomposition, and breakdown of the polymer such that the experimental observables are produced. 
 
%Following the preliminary denitration step, products of intramolecular or hydrolytic nitrate elmination are evolved as gases or remain trapped in the \ac{NC} matrix. 
Products of the preliminary denitration step of \ac{NC} can be evolved as gases or remain trapped in the polymer matrix. 
% IS there any implicit evidence of species being trapped though? Or do they either escape, or continue reacting? Perhaps there is no state of just being “trapped” and latent in the NC matrix.
%
%Reactive species include %the acidic products from the elimination of \ce{HNO2} and 
Reactive nitrous oxide radicals generated from homolysis of the O-N bond %not evolved as \ce{NO2} gas, 
are likely to migrate within the bulk and attack other sites on the polysaccharide. %REF
% and free species
Nitrous and nitric acids released directly from denitration, or via transformation of released NO\textsubscript{x} species, contribute to the acidity of the overall system, lowering the pH and stimulating further hydrolysis processes \cite{Hu2011}. 

%As described in section \ref{chapter4:intro}, Moniruzzaman \textit{et al.} used the reaction of nitrates with an anthraquinone dye (\acs{SB59}) to probe the reactivity at each of the C2, C3 and C6 sites on \ac{NC}, using \acs{UVVis} and \textsuperscript{1}H NMR spectroscopy (figure \ref{fig:SB59_NOx}).
%
% Insert blank line between top and bottom rows, if you can - LATER
\begin{figure}[htp!]
\centering
  \begin{subfigure}[b]{0.42\linewidth}
	\caption{\ac{NC} film aged at 40\degree C.}
  \includegraphics[width=\linewidth]{40-monirazzuman-UV-2014}
  \end{subfigure}
    \begin{subfigure}[b]{0.4\linewidth}
	\caption{\ac{NC} film aged at 50\degree C.}
  \includegraphics[width=\linewidth]{50-monirazzuman-UV-2014}
  \end{subfigure}
  \begin{subfigure}[b]{0.4\linewidth}
    \caption{\ac{NC} film aged at 60\degree C.}
    \includegraphics[width=\linewidth]{60-monirazzuman-UV-2014}
  \end{subfigure}
  \begin{subfigure}[b]{0.4\linewidth}
    \caption{\ac{NC} film aged at 70\degree C.}
    \includegraphics[width=\linewidth]{70-monirazzuman-UV-2014}
    \label{fig:boldUV}
  \end{subfigure}
\caption{\acs{UVVis} spectra of aged \ac{NC}-based film, from the work of Moniruzzaman \textit{et al.}\cite{Moniruzzaman2014}. The peaks at 600 nm and 650 nm are attributed to the $\pi$ - $\pi$* transitions in the anthraquinone dye (\ac{SB59}). Spectral lines with highest absorbance peaks in this region correspond to the sample prior to heat treatment. Peaks below 400 nm indicate the formation of \acs{SB59} derivatives due to secondary reactions.
%The spectral lines with the highest absoprtions at the double peak region (with crests at 600 nm and 650 nm) correspond to the sample before heat treatment, with the highest concentration of \acs{SB59} dye.
}
\label{fig:UV_all}
\end{figure}


When studying the ageing of \ac{NC} using  \acs{UVVis} spectroscopy, Moniruzzaman \textit{et al.} observed %identified 
increasing concentrations of secondary reaction products following heat treatment over extended timescales%\footnote{First introduced in section \ref{chapter4:intro}} (figure \ref{fig:UV_all})
 \cite{Moniruzzaman2008,Moniruzzaman2014}. 
Samples exposed to higher ageing temperatures presented spectra dominated by consecutive products (figure \ref{fig:UV_all}). 
\acs{UV} absorbances at 600 nm and 650 nm were characteristic of the \ac{SB59} dye used to indicate the presence of NO\textsubscript{x}, released by the denitration of \ac{NC}. The isosbestic point identified at 552 nm showed that as the concentration of \acs{SB59} decreased, the concentration of the [\acs{SB59} + \ac{NC}] product increased. % accordingly. %though not necessarily proportionallys
%the relative concentrations of unreacted \acs{SB59} and \acs{SB59} following reaction with \ac{NC} were proportional.
For sample aged at temperatures $>$40\degree C, the isosbestic point demonstrated a downwards shift. %, until it was eventually lost. 
%This was most clearly illustrated by 
In the case of the 70\degree C treated run, the final measurement (indicated by the royal-blue line in bold, figure \ref{fig:boldUV}) deviated from the isosbestic point entirely, and showed more than 81\% consumption of the original dye concentration. The drift from the isosbestic point, in addition to the appearance of new absorbance peaks below 400 nm, alludes to the presence of new species in the reaction mixture not generated by the primary reaction of \acs{SB59} and \ac{NC}. It is likely that these arise from the continued reaction of \acs{SB59} derivatives with \ac{NC} degradation products, or further derivatives thereof, as suggested in scheme \ref{sch:SB59_NOx}. 

%The \ac{NC} films with higher ageing temperatures demonstrated a greater loss of \ac{SB59} dye absorbances, and more pronounced peaks corresponding to secondary reaction products below 400 nm. 
%
% were changing, but that the idenity of the complexes in the system remained the same. 
%Though figure \ref{fig:SB59_NOx} illustrates the reaction of the dye with NO groups, the study makes no indication of the source of NO\textsubscript{x}, except that they are products of thermolysis of \ac{NC}. 


\begin{scheme}[hbp]
  \centering
	\includegraphics[width=0.55\linewidth]{SB59_NOx_reaction}
\caption{Proposed pathway for the reaction of \acs{SB59} dye with $^{.}$NO released as a result of denitration of \ac{NC} \cite{Moniruzzaman2014}. }
\label{sch:SB59_NOx}
\end{scheme}

% Here is where we start talking about the NATURE and IDENTITY of the secondary reactions. 
%Following the possible denitration routes outlined in section \ref{chapter5:intro}Chapter 3, 
Following cleavage of the nitrate ester via homolytic fission, elimination of nitrous acid, or hydrolysis, the resulting residues are available for further reaction with the polymer or other free molecules in the system. Chin \textit{et al.} proposed schemes for the propagation of such reactions initated by both the thermolysis and hydrolysis of nitrate esters \cite{Chin2007}: \\



% Fix alignment left, please
%%Chin's equations \\
\begin{scheme}[h]
\noindent\textbf{Thermolytic initiation}
%\textit{Initiation:}
\begin{equation}
%\textit{Initiation}
%\begin{block_indent}{1cm}
\ce{R-ONO2 -> R-O^{.} + ^{.}NO2 }
%\end{block_indent}
\label{equ:chinT1}
\end{equation}

\textit\textbf{Propagation}
%\textit{Propagation}
%\begin{block_indent}{1cm}
\begin{equation}
\begin{split}
%\begin{multline}
%\ce{RO^{.} + ^{.}NO2 ->[\ce{RONO2}] RONO2 + NO + N2O4 + ^{.}NO2}\\
%								\ce{ + H2O  + N2O + CO2 + CO } \\ 
%								\ce{+ C2H2O +} \text{other organic fragments}
\ce{R-O^{.} + ^{.}NO2 ->[\ce{R-ONO2}]} & \ce{ R-ONO2 + ^{.}NO + ^{.}NO2} + \ce{N2O4}\\
								& \ce{ + H2O  + N2O + CO2 + CO } \\
								& \ce{+ C2H2O +} \text{other organic fragments}
%\end{multline}
\end{split}
\label{equ:chinT2}
\end{equation}


\begin{equation}
\ce{2^{.}NO + O2 -> 2^{.}NO2} 
\label{equ:chinT3}
\end{equation}
\begin{equation}
\ce{2^{.}NO2 <=> N2O4}
\label{equ:chinT4}
%\end{block_indent}
\end{equation}

\noindent\textbf{Hydrolytic initiation}
%\subsubsection*{Hydrolysis}
%\textit{Initial}
%\textit{Initiation:}
%\begin{block_indent}{1cm}
\begin{equation}
\ce{R-ONO2 + H2O -> R-OH + HNO3} 
\label{equ:chinH1}
\end{equation}
\begin{equation}
\ce{R-OH + HNO3 -> R=O + HNO2 + H2O}
\label{equ:chinH2}
\end{equation}
%\end{block_indent}

\textit\textbf{Propagation}
%\textit{Propagation}
%\begin{block_indent}{1cm}
\begin{equation}
\ce{HNO3 + HNO2 <=> N2O4 + H2O} 
\label{equ:chinH3}
\end{equation}
\begin{equation}
\ce{N2O4 <=> 2^{.}NO2}
\label{equ:chinH4}
\end{equation}
\begin{equation}
\ce{R-OH + ^{.}NO2 -> ^{.}R-OH + HNO2}
\label{equ:chinH5}
\end{equation}
\begin{equation}
\ce{^{.}R-OH + HNO3 -> R=O + H2O + ^{.}NO2}
\label{equ:chinH6}
\end{equation}
\label{sch:chinschema}
\end{scheme}
%\end{block_indent}

Termination reactions were not emphasised in the schemes for either of these cases. The hydrolysis scheme was adapted from an earlier work by Camera \textit{et al.} involving the nitrate ester decomposition and subsequent reactions of ethyl nitrate (where R = \ce{CH3CH2} for the scheme above) \cite{Camera1982}. The original study included an expansion of the hydrolysis step (equation \ce{PhCH2ONO2}, ), where the involvement of \ce{NO2+} is illustrated: \\ %pertaining to ethyl nitrate:



%\subsection*{Camera's equations}
\begin{scheme}[h!]
\textbf{Hydrolysis scheme for ethyl nitrate}
%\begin{block_indent}{1cm}
%\ce{CH3CH2ONO2 + H+ <=>[\text{fast}] CH3CH2ONO2H+} \\
%\ce{CH3CH2ONO2H+ <=>[\text{slow}] CH3CH2H + NO2+} \\
%\ce{NO2+ + 2H2O <=>[\text{fast}] HNO3 + H3O+} \\
%\end{block_indent}
\begin{equation}
\ce{CH3CH2ONO2 + H+ <=>[\text{fast}] CH3CH2ONO2H+} \\
\end{equation}
\begin{equation}
\ce{CH3CH2ONO2H+ <=>[\text{slow}] CH3CH2OH + NO2+} \\
\end{equation}
\begin{equation}
\ce{NO2+ + 2H2O <=>[\text{fast}] HNO3 + H3O+} \\
\end{equation}
\label{sch:cameraschema}
\end{scheme}

It was highlighted by Camera, that the oxidation of alcohol by nitric acid (equation \ref{equ:chinH2}) is slow and thus rate-limiting. The mechanism is likely to occur \textit{via} a series of intermediate reactions of which the details are not known.  Following the generation of nitrous acid, subsequent oxidations occur rapidly. %why and how? 
According to Rigas \textit{et al.}, alcohols are more susceptible to wet oxidation than esters \cite{Rigas1997}. A higher concentration of unsubstituted hydroxyl groups in the system, and therefore a fewer nitrate ester groups (or a lower \ac{DOS} value), decreases overall stabililty. % What is the mechanism, and why? This would suggest that below a certain threshold \ac{DOS} value, the oxidation of hydroxyl groups (equation \ref{equ:chinH2}) would dominate over the hydrolytic denitration reaction (equation \ref{equ:chinH2}). therefore indicates that a higher concentration of -OH groups present in the system decreases stabililty. 
% Check that this doesn’t contradict the point made in Chap 2, about the inductive effect of adjacent nitrates on the reactivity/ predisposition to denitrate easier, or if it does, ref the two contrasting papers. 

Equations \ref{equ:chinH3} - \ref{equ:chinH6} describe a possible branched radical chain mechanism, fed by the nitrous and nitric acids produced during the hydrolysis and alchohol oxidation reactions during the initiation stage. By contrast, the propagation reactions in the branched radical chain mechanism for thermolysis are poorly characterised (equation \ref{equ:chinT2}), defined only by the observable products. This is likely due to their rapid and varied nature, rendering it difficult to follow spectroscopically. %REF

%but eventually, everything feeds back into that equation of just radical deg, esp for the species contianing carbon 
%Who actually did the measureing? 
%measured by \ac{IR}, liquid \cite{Bluhm1977}, ion \cite{lopezlopez2011} and gas chromatography \cite{Huwei1988},  ignition loss determination \cite{Bluhm1977}. %and others

Aellig \textit{et al.} presented an alternative scheme for the decomposition of benzyl nitrate (R = \ce{PhCH2}), involving more interaction with the solvent  \cite{Aellig2011}:\\ %\ce{PhCH2ONO2}, 

%{Aellig's equations}
\begin{scheme}[h]
\noindent\textbf{\ce{HNO3} decomposition initiated}
%\textit{Initiation}
\begin{equation}
\ce{4HNO3 <=> 4^{.}NO2 + 2H2O + O2}
\end{equation}
\begin{equation}
\ce{2^{.}NO2 + H2O <=> N2O4 + H2O}
\end{equation}
\begin{equation}
\ce{N2O4 + H2O -> HNO3 + HNO2}
\end{equation}

\textit{Propagation}
\begin{equation}
\ce{R-OH + HNO2 <=> R-ONO + H2O}
\end{equation}
\begin{equation}
\ce{R-ONO -> R=O + HNO}
\end{equation}
\begin{equation}
\ce{^{.}NO2 + HNO -> HNO2 + ^{.}NO}
\end{equation}
\begin{equation}
\ce{2 ^{.}NO + O2 -> 2 ^{.}NO2}
\end{equation}

\textit{Termination}
\begin{equation}
\ce{2HNO -> HON=NOH}
\end{equation}
\begin{equation}
\ce{HON=NOH -> N2O + H2O}
\end{equation}
\label{sch:aelligschema}
\end{scheme}


Both the Camera/Chin and Aellig schemes above produce final end products observed in the decomposition of \ac{NC}. In particular, Aellig’s scheme accounts for the production of \ce{N2O}, which forms a significant part of the decomposition eluent \cite{Buelow2002}. %Whislt not a full-resolution, exhaustive depiction the full spectrum of reactions that take place in the \ac{NC} matrix during it’s slow ageing, the presented reactions encapsulate the early major reactions of the most prevalent and active species in the system.
Whilst the schemes do not a propose an exhaustive description of the full spectrum of reactions that take place in the \ac{NC} matrix during its slow ageing, the early stage reactions of the key species responsible for decomposition are  encapsulated.

%autocatalytic rate 
%%%%%%%%%%%%%%%% FIX / CHECK THIS SECTION %%%%%%%%%%%%%%%%%%%%%

% confused rates of thermolytic degradation: \cite{brill1997}
	% Despite good understanding about the initial step of the process, the kinetics of thermolysis of NC are complicated by the existence of rapid secondary reactions, autocatalysis, and self-heating. As a result, further kinetic descriptions of the process are fragmented and somewhat contradictory. To organize this subject, the past studies can be grouped according to the temperature range of study.''
% info on rates of hyrolysis

% what I actually want to focus on here, are the reactions AFTER intial hydrolysis.
% So knowledge on the rates quoted in literature are good, but I need to know what they correspond to:
	% All reactions, with just an increase in rate as the system gets hotter / more reactions splinter off? (ie more reactions happening, so more exothermic heat release - self heating
	% Does the sudden change to autocatalytic arise from a global change in the mechanism? / All the of the “first order” reaction stuff is done / it’s much less important, and that the 

It is widely agreed that first-stage decomposition follows a first-order process (or pseudo-first order, with respect to hydrolysis reactions). %REF for first & psuedo first order rate for NC
A number of studies observe catalytic rate of decay for the longer-term aging processes. %REF the studies in DAUERMAN's who quoted an autocatalytic rate. 
Dauerman \cite{Dauerman1968} observed that when \ac{NC} was treated with \ce{NO2} gas before heating, the time required for sample ignition halved. He suggested that the \ce{NO2} adsorbed onto the surface acted as a catalysing agent. 

% Now, how do the above rate observations relate to the post-denitration phases of reaction?

Neutral and alkaline hydrolysis reactions follow a pseudo-first order process, however it has been suggested that the presence of acid facilitates a catalytic rate of degradation after an initial incubation period. %Incubation is not correct - what I want to say is that the reaction starts off first order, but at an obeservable inflection point, becomes catalyitc. 
%Check whether i am jsut repeating myself from chap 2?
%The degradation of cellulose also follows a pseudo-first order rate\cite{Calvini2008}.
%
% Introducing the decondary reaction products %%%%%%%%%%%%%%%%%%%%
%Discuss the work of Camera, Aellig and Chin 2007
Multiple studies have addressed the decomposition reactions of nitrate esters following the initial scission of the nitrate group \cite{Baker1952,Camera1982,Camera1983,Matveev2003,Hu2011} %ETC refine this list later

%Baker1950


%%%%%%%%%%%%%%%%%%%%%%%%%%%%%%%%%%%%%%%%%%%%%%%%%%

In this section, secondary and extended reaction schemes for the low temperature ageing of \ac{NC} are explored. Decompostition pathways defined by Chin, Camera and Aellig \textit{et al.} are probed to determine the reactions responsible for the experimentally observed degradation products. 
The reactions found to be energetically feasible from the proposed routes will be scrutinised to determine whether an autocatalytic pathway can be formed from the thermodynamically validated reaction schemes. 

\section{Methodology}%%%%%%%%%%%%%%%%%%%%%%%%%%%%%%%%%%%%%
The reactions proposed by Chin, Camera and Aellig \textit{et al.} were used to construct degradation routes for \ac{NC}. %Proceeding on from the initial denitration step, 
The products of homolytic fission, elimination of \ce{HNO2} and acid hydrolysis of \ac{NC} were used as starting points.%, generating a for each proposed degradation pathway. 
Schemes were constructed based on the propagation of the given reactions in a step-wise fashion; subsequent reactions were dependent on the products generated in prior steps, in addition to the assumed availability of other reactants in the system. 
An abundance of water and oxygen were assumed present in the system, attributed to air exposure or the wetted storage conditions of \ac{NC}. Unsubsitutued alcohol moieties (R-OH) were also presumed available, due to incomplete nitration during the synthesis of \ac{NC} \cite{Wolf1997}, or re-generation following denitration \textit{via} hydrolysis. 
The schemes were modelled with both ethyl nitrate and the \ac{NC} monomer. Free energies of reaction ($\Delta$ G) were used to determine the feasibility of a reaction. %Where the choice of method lead to a variation in the result of a reaction, the geometries around the reaction centres were further scrutinised in order to ensure no spurious behaviour due to artefacts from functional choice. [Bit of a dodgy sentence, how would you know what to say was right / wrong?]

%The species reactions  were geometry optimised using \ac{wb97xd}, and \ac{B3LYP} functionals, in both vacuum and solvent. The reactions were modelled using ethyl nitrate as a test system before expansion to the full C2 monomeric model. 
%The \ac{DG} were used to determine . 
% Free energy paper with equations: Effect of nitrate content on thermal decomposition of nitrocellulose. Pourmortazavi,2009
%Where the choice of method lead to a variation in the result 
%Energetically feasible reactions were added to the decomposition pathway 

\subsection{Computational details}%%%%%%%%%%%%%%%%%%%%%%%%%%%%%%%
All geometry optimisations were performed in \ac{G09}, using the \acs{wb97xd} and \acs{B3LYP} functionals. Optimisations were to the level of 6-31+G(2df,p) with tight convergence criteria (table \ref{tab:convergence}). Chemical species were constructed using \ac{GView} and for molecules of more than 3 atoms, the “Clean” function was used to re-order atoms to a preliminary reasonable geometry. Optimisations were performed in both vacuum and with \ac{PCM} to introduce implicit solvent effects. Energies of optimised structures were checked against values listed on NIST Computational Chemistry Comparison and Benchmark Database \cite{JohnsonIII2018} where available.

\section{Results and Discussion}%%%%%%%%%%%%%%%%%%%%%%%%%%%%%%%%%%
%Collation of the above schemes to fit the starting products from denitration:
%The reaction energies for the proposed schemes 

Simplified schemes for the ageing reactions of \ac{NC} beginning from homolytic fission, elimination of \ce{HNO2} or acid hydrolysis are illustrated in schemes \ref{sch:homolytic} - \ref{sch:hydrolysis}. For the reactions starting from the products of homolytic fission, the propagation reactions are dominated by radical interactions. \ce{^{.}NO2} and \ce{HNO2} are consumed and regenerated, supporting the theory that these may be species contributing to the observed autocatalytic rate of decomposition, following a first-order rate induction period \cite{Rodger1963,Lindblom2002,Volltrauer1981}.  \ce{R=O} and \ce{N2O} are terminating species, which may go on to participate in wider reactions outside the scope of the proposed reactions. %why are they terminating species? - they do not go on to react and regenerate HNO2 and .NO2?

\textcolor{red}{Still to mention:\\
 - Describe the other two schemes\\
 - Describe the energies in the table\\
 - Discuss why some of the values may be positive. \\
 - Include enthalpies of reaction, zero point energies, and any experimental proxies I can find for the reaction enthalpies too. \\
 NOTE: ZPE energy correction means that you REMOVE the ZPE, so that you only compare the actual energy available for the reaction. 
 -  ?}

\begin{scheme}[htp!]
\centering
\schemestart
\textbf{\ce{R-ONO2}}\arrow(xx--aa){->[\textbf{homolysis}]}\textbf{\ce{R - O^{.} + ^{.}NO2}}
\arrow(aa--bb)[-25,2]Thermolyic fragmentation (equation \ref{equ:chinT2})
\arrow(@aa--cc)[-90,2]\ce{2^{.}NO2 + H2O <=> N2O4 + H2O}
\arrow(cc--dd){0}[-90,0.2]\ce{^{.}NO2 + R-OH -> ^{.}R-OH + HNO2}
\arrow(@dd--ee)[-90,1.5]\ce{N2O4 + H2O -> HNO3 + HNO2}
\merge>(dd)(ee)--(ff)\ce{^{.}R-OH + HNO3}
\arrow(@ff--nn)[90,1]\ce{R=O + H2O + ^{.}NO2}
%\arrow(@dd--ff)[-10,2.5]\ce{^{.}R-OH + HNO3 -> R=O + H2O + ^{.}NO2}
%\arrow(@ee--@ff)
\arrow(@ee--gg)[-90,1]\ce{HNO2 + R-OH <=> R-ONO + H2O}
\arrow(@ee--hh)[-25,3.5]\ce{HNO3 + R-OH -> R=O + HNO2 + H2O}
%\arrow(hh--ii){0}[-90,0.2]\ce{HNO3 + HNO2 <=> N2O4 + H2O} 
\arrow(@gg--jj)[-90,1.5]\ce{R-ONO -> R=O + HNO}
\arrow(jj--kk)[-90,1]\ce{2HNO -> HON=NOH}
\arrow(kk--ll){0}[-90,0.2]\ce{HNO + ^{.}NO2 -> HNO2 + ^{.}NO}
\arrow(ll--oo)[-90,1]\ce{2 ^{.}NO + O2 -> 2 ^{.}NO2}
%\arrow(@kk--mm)\ce{HON=NOH ->N2O + H2O}
\arrow(@kk--mm)\ce{N2O + H2O}
\schemestop 
\caption{Proposed degradation pathway starting from the homolysis products of a nitrate ester, derived from the schemes presented by Camera \cite{Camera1982} and Aellig\cite{Aellig2011}.}
\label{sch:homolytic}
\end{scheme}

%\subsection{Thermodynamics of Ethyl Nitrate reactions}%%%%%%%%%%%%%%%%%%%%%%%%%
%
%For an initial comparison of the methods you used, you could do a diagram like the one Kuklja did (kuja2014.pdf, page 89, fig 3.7).
%She also makes mention of the overestimation of activation barriers for pure DFT methods. Make sure you know the background surrounding this - why does this occur, and what is done to remedy it?
 
%\begin{figure}[h]
%  \centering
%	\includegraphics[width=0.8\linewidth]{name}
%\caption{\cite{}.}
%\end{figure}
%$\Delta$-G\textsubscript{r}
%$\textDelta$H\textsubscript{r}
 
%note, terminal up is right, terminal down is left
%\begin{scheme}[htp]
%\begin{table}
\begin{table}[htp]
\begin{center}
%\centering
\caption{Free energies of protonation for each oxygen site on ethyl nitrate.}
\begin{tabular}{ l l *{4}{S[table-format = 2.4]}} 
\toprule
\multicolumn{2}{l}{\multirow{2}{*}{Protonated site}} & \multicolumn{4}{c}{$\Delta$G\textsubscript{r} /kcal mol\textsuperscript{-1}} %& \multicolumn{2}{c}{$\Delta$H\textsubscript{r} }
\\\cline{3-6}
  & & \acs{wb97xd} & PCM & \acs{B3LYP} & PCM\\
\midrule
% Right is up, left is down, with ethanol in an “m” shape. 
 Terminal (upper) & \ce{CH3CH3ONO2H+} & -12.276810	& 8.821890 & -13.782510 & 5.625270\\
 Terminal (lower) & \ce{CH3CH3ONO2H+}  &-9.475200 &	9.459450	&-11.132100	&5.646060 \\ 
 Bridging & \ce{CH3CH3O(H+)NO2} &-9.322740	& 9.058140	& -15.309630 &	6.673590 \\
\bottomrule
\end{tabular}
\label{tab:reactions}
\end{center}
\end{table}

%replace pics with the atom labelled ones
%\begin{figure}
\begin{figure}[htp]
\centering
\begin{subfigure}[b]{0.3\linewidth}
\centering
\includegraphics[width=\linewidth]{terminal_r_up}
 \caption{Upper terminal oxgen.\\
 Bond(O--\ce{NO2}): \num[round-mode=places,round-precision=2]{1.27787} \AA} 
 \label{fig:t_l_up}
\end{subfigure}
\begin{subfigure}[b]{0.3\linewidth}
\centering
\includegraphics[width=\linewidth]{terminal_l_down}
\caption{Lower terminal oxgen.\\
 Bond(O--\ce{NO2}): \num[round-mode=places,round-precision=2]{1.27863} \AA} 
 \label{fig:t_r_down}
\end{subfigure}
\begin{subfigure}[b]{0.3\linewidth}
\centering
\includegraphics[width=\linewidth]{bridging}
\caption{Bridging oxgen.\\
 Bond(O--\ce{NO2}): \num[round-mode=places,round-precision=2]{1.97967} \AA} 
 \label{fig:bridge}
\end{subfigure}
 \caption{Optimised geometries of the possible protonation sites on ethyl nitrate.}
 \label{fig:en_protonation}
\end{figure}
%\end{scheme}

Due to the availability of oxygen sites on the ethyl nitrate molecule, the optimal site for protonation was determined for inclusion in the reaction scheme for the first stage of hydrolysis. Table \ref{tab:reactions} shows the protonation energies for the three different oxygen sites on ethyl nitrate. Despite the upper terminal oxygen possessing the most thermodynamically favourable energy of protonation, inspection of the reaction geometries shows that the bridging structure most resembles that expected for the liberation of the \ce{NO2+} group at the next step. Though appearing less thermodynamically favourable when compared to protonation at the terminal upper oxygen site, the higher energy of reaction likely arises from the instability of the protonated complex. The elongation of the O--\ce{NO2} bond allows to stabilisation of the proton at the bridging site, such that the departure of \ce{NO2+} is easily facilitated. Subsequent calculation involving the energy of the protonated ethyl nitrate will employ the values associated with the protonated bridging site. %The \ce{CO---NO2} bond lengths are (a) \num[round-mode=places,round-precision=4]{1.27787} \AA (b) \num[round-mode=places,round-precision=4]{1.27863} \AA (c) \num[round-mode=places,round-precision=4]{1.97967} \AA
%Comment on the diff between wb97xd and B3LYP.%

%something about why the energy of the HNO3 reaction looks so rubbish (but that it stilll degrades at room temp, appaz. But check the literature)
For the decomposition of \ce{HNO3} to \ce{{.}NO2},  \ce{2H2O} and \ce{O2}, Aellig prescribes the use of an amberlyst catalyst (amberlyst-15).
%\cite{Ellis2007,Robertson1955}
%The propagation reactions are acid catalysed by \ce{HNO2}. 
%
%\subsubsection{Radical mechanistic route}
%
%\subsubsection{Ionic mechanistic route}
%
%\subsection{Reactions of Nitrocellulose Monomer}
%\section{ \textit{(Kinetics of Ethyl Nitrate)}}
%\subsection{Radical mechanistic route}
%\subsection{Ionic mechanistic route}
%
%\section{ \textit{(Kinetics of Nitrocellulose Monomer)}}

%Insights into the reaction energies
%“HNO cannot be stored or concentrated and is typically studied using donor species that release HNO as a decomposition product.” (paper since retracted)

\begin{scheme}[ht!]
\centering
\schemestart
\textbf{\ce{R-ONO2}}\arrow(xx--aa){->[\textbf{Elimination}]}[,1.5]\textbf{\ce{R=O + HNO2}}
\arrow(@aa--bb)[-140,2]\ce{HNO2 + R-OH <=> R-ONO + H2O}
\arrow(bb--cc)[-90]\ce{R-ONO -> R=O + HNO}
\arrow(cc--dd)[-90]\ce{2HNO -> HON=NOH}
%\arrow(dd--gg){0}[-90,0.2]\ce{HNO + ^{.}NO2 -> HNO2 + ^{.}NO}
\arrow(dd--ee)[-90]\ce{HON=NOH -> N2O + H2O}
%\arrow(@aa--ff)[-35,2]\ce{R=O}
%\arrow(@gg--hh)\ce{2 ^{.}NO + O2 -> 2 ^{.}NO2}
\schemestop 
\caption{Proposed degradation pathway starting from the elimination of \ce{HNO2} from a nitrate ester, derived from the schemes presented by Camera \cite{Camera1982} and Aellig\cite{Aellig2011}.}
\label{sch:elimination}
\end{scheme}

\begin{scheme}[ht!]
\centering
\schemestart
\textbf{\ce{R-ONO2 + H+}}\arrow(xx--aa){->[\textbf{Hydrolysis}]}[,1.5]\textbf{\ce{R-OH + NO2+}}
\arrow(@aa--bb)[-150,2]\ce{NO2+ + 2H2O <=> HNO3 + H3O+}
%\merge>(@aa--@bb)
%\arrow(@aa--cc)[-30,2]\ce{R-OH + HNO3 -> R=O + HNO2 + H2O}
\arrow(@bb--cc)[-30,2]\ce{HNO3 + R-OH -> R=O + HNO2 + H2O}
\arrow(@aa--@cc)[-90]
\arrow(@cc--dd)[-90]\ce{HNO2 + HNO3 <=> N2O4 + H2O} 
\arrow(@dd--ee){0}[-90,0.2]\ce{HNO2 + R-OH <=> R-ONO + H2O}
%\arrow(@dd--ff)[180]\ce{N2O4 + H2O -> HNO3 + HNO2}
%\arrow(ff--gg){0}[-90,0.2]\ce{N2O4 <=> 2^{.}NO2}
\arrow(@dd--gg)[180]\ce{N2O4 <=> 2^{.}NO2}
\arrow(@ee--hh)[-90]\ce{R-ONO -> R=O + HNO}
\arrow(hh--ii)[-90]\ce{HNO + ^{.}NO2 -> HNO2 + ^{.}NO}
\arrow(ii--kk){0}[-90,0.2]\ce{2HNO -> HON=NOH}
\arrow(@gg--ii)[-50,3.3]
\arrow(@kk--jj)[-90]\ce{HON=NOH -> N2O + H2O}
\arrow(@ii--ll)[180,1]\ce{2 ^{.}NO + O2}
\arrow(@ll--mm)[180,0.5]\ce{2^{.}NO2}
\schemestop 
\caption{Proposed degradation pathway starting from the acid hydrolysis of a nitrate ester, derived from the schemes presented by Camera \cite{Camera1982} and Aellig\cite{Aellig2011}.}
\label{sch:hydrolysis}
\end{scheme}

\begin{table}[h]
\begin{center}
\caption{Energies of nitrate ester decomposition reactions proposed by Camera \cite{Camera1982}, Chin \cite{Chin2007} and Aellig \cite{Aellig2011}. R = \ce{CH3CH2} for ethyl nitrate, and R = \ce{(H3CO)2C6H9O3} (bi-methoxy capped glucopyraonse monomer unit).}
\begin{tabular} { l *{4}{S[table-format = 2.4]}} 
\toprule
\multirow{2}{*}{Reaction}	& \multicolumn{4}{c}{$\Delta$G\textsubscript{r} /kcal mol\textsuperscript{-1}}
\\\cline{2-5}
			& \acs{wb97xd} & PCM & \acs{B3LYP} & PCM\\
\midrule
%\ce{CH3CH2ONO2 + H+ <=>[\text{fast}] CH3CH2ONO2H+} &
%\ce{CH3CH2ONO2H+ <=>[\text{slow}] CH3CH2H + NO2+} & 14.930370&-2.247210&16.012710&-4.038930\\
%[\text{fast}]
\ce{NO2+ + 2H2O <=> HNO3 + H3O+}& -0.896490 & -1.338750 & 1.770300 & 2.464560\\
%\ce{R-ONO2 -> R-O^{.} + ^{.}NO2 } \\ Homolysis
\ce{2^{.}NO + O2 -> 2^{.}NO2} &-20.77047&	-21.97314&-21.16044&-22.15899
 \\
\ce{2^{.}NO2 <=> N2O4} &-0.122220&	-1.310400	&0.541800	&0.155610
\\
\ce{HNO3 + HNO2 <=> N2O4 + H2O} 	&	-2.251620	&	-1.854090	&	-5.131350	&	-4.180050 \\
\ce{N2O4 <=> 2^{.}NO2}	&	0.123480&	1.461600&	-0.539280&	-0.155610
 \\
\ce{4HNO3 <=> 4NO2 + 2H2O + O2}&53.35029&58.36446&42.60942&46.93563\\
\ce{2^{.}NO2 + H2O <=> N2O4 + H2O}&-0.12222&-1.4616&0.53928&0.15561\\
\ce{N2O4 + H2O -> HNO3 + HNO2}&2.25162&1.85409&5.13135&4.18005\\
\ce{^{.}NO2 + HNO -> HNO2 + ^{.}NO}&-28.21644&-28.66815&-27.32688&-27.6255\\
\ce{2 ^{.}NO + O2 -> 2 ^{.}NO2}&-59.89473&-60.47244&-60.46866&-60.99597\\
\ce{2HNO -> HON=NOH}&-38.96928&-39.71583&-36.62757&-37.40814\\
\ce{HON=NOH -> N2O + H2O}&-48.08286&-48.18429&-50.55309&-50.74902\\
\midrule
Ethyl nitrate ( R = \ce{CH3CH2} )\\
\hline
%\ce{R-ONO2 + H2O -> R-OH + HNO3}	&	4.556160	&	5.235930	&	4.000500	&	4.860450 \\ Hydrolysis
\ce{R-OH + HNO3 -> R=O + HNO2 + H2O}	&	-34.062210	&	-38.427480	&	-37.593990	&	-41.770260 \\
\ce{R-OH + ^{.}NO2 -> ^{.}R-OH + HNO2}	&	16.376220	&	13.923000	&	15.887340	&	13.699350 \\
\ce{^{.}R-OH + HNO3 -> R=O + H2O + ^{.}NO2}&-50.438430&-52.35048&-53.48133&-55.4715\\
\ce{R-OH + HNO2 <=> R-ONO + H2O}&-3.20544&-3.276&-2.64096&-2.94903\\
\ce{R-ONO -> R=O + HNO}&-1.49625&-5.82183&-4.36716&-8.50122\\
\midrule
\ac{NC} monomer ( R = \ce{(H3CO)2C6H9O3} )\\
\hline
\ce{R-ONO2 + H2O -> R-OH + HNO3}	&	0.67536	&	5.63094	&	0.61236	&	-0.70119 \\
%RONO2 &+&H3O+& <->&RONO2H+&+&H2O&(see protonation section)&&&-30.87756&3.5464&-31.98636&-0.24507
%\ce{R-OH + HNO3 -> R=O + HNO2 + H2O} &	-36.72522&	-38.34306	&	-41.71419	&	-41.70411 \\ Hydrolysis
\ce{R-OH + ^{.}NO2 -> ^{.}R-OH + HNO2}	&	14.71302	&	11.15163	&	13.03407	&	23.20983 \\
\ce{^{.}R-OH + HNO3 -> R=O + H2O + ^{.}NO2} &
-51.438240	&-49.494690	&-54.748260 &	-56.369250 \\
\ce{R-OH + HNO2 <=> R-ONO + H2O}&-4.43142&-7.30233&-4.30605&-0.17829\\
\ce{R-ONO -> R=O + HNO}&-2.93328&	-1.71108	&-6.82227&	-11.20581
\\
\bottomrule
\end{tabular}
\end{center}
\end{table}

\section{Summary}
%
%A disadvantage is that \ce{N2O} is a significant greenhouse gas, and cannot be re-converted back to \ce{HNO3}, so much be considered win experimental design when observing industrial and environmental impact. 
%\chapter{Conclusion and future work}
\label{chapterlabel7}
\graphicspath{ {./Conclusion_pics/} }

\section{Conclusion}
In this thesis, the degradation processes in \ac{NC} were explored using computational methods to elucidate the dominant processes and key reactants involved in ambient ageing. 
In the first section, the polymeric structure of \ac{NC} was introduced. Different sized truncations were tested as approximations for the polysaccharide. 
This was achieved by inspection of the partial charges, \ac{ESP} and critical interaction points for monomeric, dimeric and trimeric \textbeta-glucopyranose structures. 
The dimer was found to be the minimum structure required to reproduce the full properties of \ac{NC} within a repeat unit. 
%However, 
As a result of the high degrees of freedom and flexibility of the larger structures, the monomeric structure was instead chosen for subsequent mechanistic studies, to speed up geometry optimisation and simplify already challenging \ac{TS} searches. 

Methoxy and hydroxyl capping groups were compared; the methoxy groups provided a more sterically and chemically similar proxy for the extended polymer, following examination of charges %\ac{ESP}
and %the geometry profiles.
geometry dependent interactions.  
Comparison of the charge densities and intramolecular interactions around the monomer and dimer revealed that the former exhibited an acceptable level of 
%discrepancy 
deviation from the dimer behaviour %properties
, particularly with reference to further investigations concentrating only on localised reaction interactions. 
The bi-methoxy monomer was implemented as the model for \ac{NC} in later studies. %syn
%Hydroxyl and methoxyl capping group ends were also compared, finding that the methoxyl groups provided a more similar charge and geometry profile to the extended polymer than the small hydroxyl groups. 

Using the monomer model, the primary steps of decomposition were explored in Chapter \ref{chapterlabel5}. Thermolytic denitration reactions were investigated; homolytic fission of the nitrate \ce{O-NO2} bond, and elimination of \ce{HNO2} were tested for both the PETN test case, and the \ac{NC} monomer model. 
%Whilst the reaction energy for homolysis did not agree with the computational values in the 
Good agreement with literature values was found for the reaction energies and activation energies, in case of \ce{HNO2} elimination in both \ac{PETN} and \ac{NC}. The loss of \ce{^{.}NO2} \textit{via} homolysis was confirmed. %And in the HNO2 case for the monomer?
For the acid hydrolysis pathway %facilitated denitration
, possible protonation sites in the monomer were analysed. It was found that the proton 
%sat most favourably at the bridging oxygen site of the nitrate. % for the monomer.
site most amenable to denitration was the bridging oxygen position of the nitrate. 
%Inspection of the optimised geometry showed that it very closely resembled the same geometry of the capping group site. 
Further investigations considered denitration routes beginning from isomers protonated at both the terminal (upper) and bridging sites. %why - energy diff? means it will spend less time there, but would still happen
The denitration step was then explored \textit{via} a series of \ac{PES} scans. %probing 
The stability of different possible \ac{TS} ring structures involving both pre-protonation and concerted protonation-denitration was examined, in addition to the 
%identity
nature of the \ce{NO2} leaving group. 
No stable \ac{TS} structures presenting the correct vibration for denitration were isolated, however scans confirmed that the \ce{NO2} was released as \ce{NO2+}, with possible formation of \ce{HNO2} at greater separations. 
%It is likely that the formation of \ce{HNO2} and stabilisation of the \ac{TS} requires an explicit solvent shell around the monomer. 

Proposed decomposition routes originating from the primary denitration step were collated from nitrate ester reactions in literature. Using \ac{EN} as an initial test case, the energies of each reaction were evaluated to determine whether it were a viable %step in the extended network of secondary reactions 
secondary reaction step following liberation of the \ce{^{.}NO2}, \ce{NO2+} or \ce{HNO2} following first stage decomposition. Possible decomposition schemes were constructed, mapping from the point of \ce{NO2} liberation to the oxidation of the alcohol group. 
% on the sugar ring to a ketone. 
The reaction energies were determined for the \ac{NC} monomer. It was found that the energies were largely favourable from a thermodynamic equilibrium perspective. 
%Except in the case of Hno2 formation - 
The fate of the released nitrogen species was in the accumulation of \ce{N2O} or regeneration of \ce{^{.}NO2}, suggesting \ce{^{.}NO2} as the species responsible for autocatalytic processes in the system. Consumption of \ce{^{.}NO2} in the formation of acids proved to be thermodynamically unfavourable.%, as was \ce{HNO2} formation. 
\ce{HNO2} routes lead to the formation of \ce{N2O} without self-regeneration and \ce{HNO3} routes lead primarily to formation of \ce{^{.}NO2}. This indicates that \ce{HNO2} was unlikely to be a direct contributor to catalysis, and that \ce{HNO3} was the precursor to the \ce{^{.}NO2} catalytic species, acknowledging experimental observations that \ce{HNO3} appared to facilitate autocatalysis \cite{Baker1952}. 

Whilst this work has not exhaustively explored the myriad reactions that may occur in the complex ageing procedures of \ac{NC}, it has established the key reactions %shed light 
of the early stages of degradation, with presentation of an effective %approximation 
truncation of the polymeric structure %suitable 
applicable for further study in the topic. Key competing reactions for the denitration step, the identity of nitrogen species released and their role in the longer-range decomposition process has been presented. 
% brought to attention. %This project takes pause / set the stage for 
The conclusion of this project sets the scope for subsequent investigations into the later-stage 
%secondary
reaction processes that lead to deeper degradation of the \ac{NC} backbone.

\section{Further Work}
For the refinement the existing \ac{NC} model, a more rigorous examination of the partial charges and the subtle variation 
%modifications 
in geometry, may be conducted. 
This includes explicit calculation of charges, perhaps using \acs{NBO} methods \cite{Xie2012,Santos-Carballal2013} or \acs{RESP} charges \cite{Wang2004,Woods2000}, as has been applied to other saccharide and hetercyclic organic compounds. 
%dodgy
Conformational scans, in particular for the C6 chain, and for the orientation of the %floppy side chains and branches of 
units within the trimer, would be beneficial for identification of other low energy structures likely to be present in the natural polymer.
Here, only the denitration schemes for the singly nitrated \ac{NC} monomer were documented. 
The differing stabilities of \ac{NC} at varying levels of nitration will undoubtedly affect the reactivity at each site. 
Propagation of different nitration level and conformational structures through the 
%prescribed
denitration and secondary decomposition schemes may reveal alternative reactions, or alter the balance of products obtained. 

Classical \ac{MD} techniques would also provide further insight into the diffusion of the released products, and their interaction with the wider polymer. Studies involving the interaction of \ac{NC} with plasticisers has effectively probed the diffusion rates of plasticiser migration, which is of key interest in the preservation of stable \ac{NC} product formulations \cite{Richards2018}. 
%but I didn't look at the ratio. I mean product interactions?

Another avenue of interest is in the exploration of other transition structures for the denitration stage, and for further degradation following formation of the ketone. 
The inclusion of additional explicit water molecules or water clusters may stabilise \ac{TS} that were previously not viable, starting with reproduction of those for  \cite{Momany2005}. %cite water cluster glucose work 
\acs{AIMD} techniques may be effective for the elucidation of water and acid interactions with \ac{NC}, probing protonation behaviour and water clustering around monomer, dimer and trimer structures at different \ac{DOS} \cite{Ardura2009}. 
%A cyclic transition state involving the proton located on the capping group should be probed in the possibility that it leads to a facile route to the peeling off individual glucopyraonse units, perhaps even preserving the nitrate group. 
%
%- Rates of each reaction (homolysis, vs hno2 vs acid)
%- explore the effect of water coordination around NC
%- explore mechanism of OH retardation of hydrolysis (?) rate, for teritary nitrates. And why this doesn't happen for primary / secondary (does the ph need to really be that low for reactions?)

A natural extension to the study of the secondary reactions driving decomposition, is the expansion to a wider range of possible reaction pathways. These may include the widely documented mechanisms studied for glucose (figure \ref{fig:glucose_hmf}) \cite{Qian2009,Qian2010,Qian2011}.   
\begin{figure}[h]
\centering
\begin{subfigure}[b]{0.8\linewidth}
\centering
\caption{Conversion of glucose A to \ac{HMF} C \textit{via} a furan aldehyde intermediate B.}
\includegraphics[width=\linewidth]{conv_glucos_furan}
\label{fig:hmf}
\end{subfigure}\\
%\hfill
%
%\hfill
\begin{subfigure}[b]{0.8\linewidth}
\centering
\caption{CV3: Protonation of C2–OH on \textbeta-d-glucose, CV2: subsequent breakage of the \ce{C2-O2} Bond, CV1: the formation of the C2–O5 bond during glucose conversion to \ac{HMF}.}
\includegraphics[width=0.4\linewidth]{glucos_mech}
%\caption{Protonation of C2–OH on \textbeta-d-glucose (CV3), the subsequent breakage of the \ce{C2–O2} Bond (CV2), and the formation of the C2–O5 bond during glucose conversion to \acs{HMF} (CV1).}
\label{fig:hmf_mech}
\end{subfigure}
\caption{The conversion of glucose to \ac{HMF} with \ref{fig:hmf}) showing the proposed reaction  scheme, and  \ref{fig:hmf_mech}) displaying a possible mechanistic pathway, from the \ac{AIMD} study by Qian \cite{Qian2011}.}
\label{fig:glucose_hmf}
\end{figure}
%
This is in addition to further studies of possible ring opening mechanisms and chain scission reactions, in order to fully account for the broad spectrum of experimentally observed degradation products in \ac{IR} and \ac{NMR} measurements \cite{Gismatulina2018,Kovalenko1994,Huwei1988,Dauerman1968,Clark1982,Wu1980}. 
%\addcontentsline{toc}{chapter}{Appendices}
\graphicspath{ {./Appendicies_pics/} }
% The \appendix command resets the chapter counter, and changes the chapter numbering scheme to capital letters.

%Split appendicies by theme. E.g. One section is extra data, one section is working scripts, final one is notes and details about the working details of the calculations I did. 

\appendix

%
%\chapter{Physical constants}
%\begin{table}[htp]
%\centering

\chapter{Supplementary Data}
\label{appendixlabel1}
\section{Energies of \ac{NC} species with different capping groups}
Extra data tables etc.

\section{Chapter 4}
\label{AH_Protonation_QTAIM}
\begin{figure}[htp]
\centering
\begin{subfigure}[t]{0.4\linewidth}
\centering
\includegraphics[width=\linewidth]{corr_s_H_terminal_up}
%\caption{Terminal(Upper) optimised with \ac{wb97xd}/6-31+G(2df,p) with implicit solvent.  }
%\label{fig:proton_site_terminal}
\end{subfigure}
\begin{subfigure}[t]{0.4\linewidth}
\centering
\includegraphics[width=\linewidth]{B3LYP_s_C2_NC3-CH3_corr-OH_up-hi}
%\caption{Terminal(Upper) optimised with \ac{B3LYP}/6-31+G(2df,p) with implicit solvent.}
%\label{fig:proton_site_cap}
\caption{\ac{CP} analysis of \ac{NC} monomers protonated at the terminal oxygen (upper) site, with \ac{PCM} implicit solvent.}
\label{fig:terminal_upper_same}
\end{subfigure}
\begin{subfigure}[t]{0.4\linewidth}
\centering
\includegraphics[width=\linewidth]{s_C2_NC3-CH3_corr-OH_up-hi-CP}
\caption{Terminal(Upper) optimised with \ac{wb97xd}/6-31+G(2df,p) with implicit solvent.  }
%\label{fig:proton_site_terminal}
\end{subfigure}
\begin{subfigure}[t]{0.4\linewidth}
\centering
\includegraphics[width=\linewidth]{B3LYP_s_C2_NC3-CH3_corr-OH_up-hi-CP}
\caption{Terminal(Upper) optimised with \ac{B3LYP}/6-31+G(2df,p) with implicit solvent.}
%\label{fig:proton_site_cap}
\end{subfigure}
\caption{\{CP} analysis of \ac{NC} monomers protonated at the terminal oxygen (upper) site, with \ac{PCM} implicit solvent.}
\label{fig:terminal_upper_same}
\end{figure}
Figure \ref{fig:terminal_upper_same} shows the \ac{QTAIM} \ac{CP} analysis for the \ac{NC} monomer nitrated at the C2 position, protonated at the terminal (upper) oxygen of the nitrate group. Though the two different functionals produced significantly different values for reaction energy (\ac{wb97xd} 12.79 kcal mol$^{-1}$ and 3.72 kcal mol$^{-1}$) 
the difference was not reflected in the geometry of the optimised structures or \ac{CP} analysis, which was extremely similar for both. 

\chapter{Working scripts}
%\chapter{Appendix}
\label{appendixlabel2}
Analysis \& working scripts.
(Optional?)
%Just literally copypasta.

\chapter{Computational architecture}
\label{appendixlabel3}
%\textit{This is a description of the tools you used to make your thesis. It helps people make future documents, reminds you, and looks good.}

(Tabulate some details about the architecture I used.)

Slater (now out of commission)
Legion (now out of commission)
Grace
Huygens 1 (now out of commission)
Hugyens 2
Myriad
Hawk

% \textit{(example)} This document was set in the Times Roman typeface using \LaTeX\ and Bib\TeX , composed with a text editor. 
 % description of document, e.g. type faces, TeX used, TeXmaker, packages and things used for figures. Like a computational details section.
% e.g. http://tex.stackexchange.com/questions/63468/what-is-best-way-to-mention-that-a-document-has-been-typeset-with-tex#63503

% Side note:
%http://tex.stackexchange.com/questions/1319/showcase-of-beautiful-typography-done-in-tex-friends

%INCLUDE any little tips and notes for the next person - the thing about formchk between diff versions of Gaussian.
% - The thing about tweaking the last line in an output to make it work and visalise. 
 
% You could separate these out into different files if you have
%  particularly large appendices.

% This line manually adds the Bibliography to the table of contents.
% The fact that \include is the last thing before this ensures that it
% is on a clear page, and adding it like this means that it doesn't
% get a chapter or appendix number.
\cleardoublepage\phantomsection\addcontentsline{toc}{chapter}{Bibliography}

% Actually generates your bibliography.
%\bibliography{Thesis_refs}

% All done. \o/
\end{document}

% Labelling convention
% Equation: 	mathematical equations
%				 	text-only chemical equations
%					any chemistry equations with drawn structures
%
% Figure: 		any sort of picture
%				 	any sort of graph or graphic
%	
% Scheme: 		more than one equation of any kind [AVOID THESE for image based things. and replace with figure label] - may not appear in list of figs otherwise, and makes for unsightly repeated numbers

			 	

%Colour scheme:
% RED #FC0000T
% BLUEBONNET #1F1FFF
% SPANISH GREY #9B9B9B
% lighter grey #d9d7d7
% WHITE #FFFFFF
% CEIL #9B9ECE (like a dusty violet)
% PICTON BLUE #30BCED
% DARK TANGERINE #FAA916 
% magenta #ff00f6
% Dark sea green #8ACB88
% green #47cb47
% To generate more
% https://coolors.co/fc0000-1f1fff-9b9b9b-ffffff-faa916