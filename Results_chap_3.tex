\chapter{Mechanisms of denitration}
\label{chapterlabel5}
\graphicspath{ {./R_chap_3_pics/} }

\section{Introduction}
\label{chapter5:intro}
The first stage of thermolytic decomposition for nitrate esters is widely considered to be homolytic fission at the O-N bond linking the nitrate to the alkyl chain, leading to the loss of \ce{NO2.} \cite{Oxley2003}. %many refs
%
\begin{equation}
\ce{R-ONO2 -> R-O^{.} + ^{.}NO2 }
\label{equ:homofiss}
\end{equation}
%
Though nitrate homolysis is an endothermic reaction the weak O-N bond, with typical bond enthalpy of 42 kcal mol\textsuperscript{-1}%\si{\kilo\cal\per\mol}. %ref
 is easily cleaved when exposed to elevated temperatures, UV light or impact. %REF
Thermolytic degradation of energetic materials has been widely studied experimentally%REF
, however, the ambient, slow ageing routes are influenced by multiple factors over protracted lifetimes and subject to the influence of evolving environmental conditions.%ref?
%experimental studies
%computational studies
At temperatures over 100\degree C, decomposition is said to be dominated by thermolytic processes, whilst under 100\degree C, nitrate ester decomposition is largely due to hydrolysis. %ref

Spent acids remain in the \ac{NC} matrix following synthesis even with thorough washing procedures. %but how much acid? What concentrations? %REF
Acids are also generated via the reactions of \ce{NO2.}, following homolysis. 
These acidic species proceed to react with other moieties in the system, such as unsubstituted alcohol side chains on polysaccharide rings, or free molecules in the matrix. 
Hu \textit{et al.} found that whilst primary and secondary nitrates were resilient to hydrolysis for pH > 0, tertiary nitrates underwent hydrolytic nucleophilic substitution, reacting with water and sulfate to form alcohols and organosulfates \cite{hu2011}. %scheme
When exploring the interaction of nitroglycol and nitroglycerin in acid solution, Camera proposed a protonation-denitration scheme whereby initial protonation is rapid, but that the release of the nitronium ion was slow. %scheme 
Fast exchange is expected for  groups interacting with water, as in the case of \ac{NC} in wet storage. %ref back to lit review, about having to store it in water for stability.
Following denitration, the 
Tsyshevsky \textit{et al.} studied the intramolecular reactions in \ac{PETN} under vacuum and in the bulk crystal \cite{Tsyshevsky2013}. %diagram
It was found that the two dominating decomposition reactions were homolysis and intramolecular elimination of \ce{HNO2} \ref{equ:hno2_elim}.
%redraw this as a scheme with mechanism
\begin{equation}
\ce{R-ONO2 -> R=O + HNO2 }
\label{equ:hno2_elim}
\end{equation}
Whilst elimination of \ce{HNO2} was found to be the most energetically favourable denitration pathway, homolytic fission dominated preliminary decomposition steps due to the lower activation barrier and higher rate of reaction. It was suggested that global decomposition processes were determined by the interplay between the two mechanisms. Initial homolysis facilitated wide-spread denitration, complemented by exothermic \ce{HNO2} elimination promoting self-heating of the system and further bond dissociations. 
The presence of \ce{NO2^{.}} and \ce{HNO2} are linked to the observed autocatalytic rate of later-stage decomposition \cite{Rodger1963,Lindblom2002,Volltrauer1981}. From these initial processes it is not possible to determine which species is responsible; this topic is explored in chapter {chapterlabel6}.

In this section, the possible mechanisms for nitrate removal from the \ac{NC} backbone are explored. Homolytic fission and elimination of \ce{HNO2} thermolytic processes suggested by Tsyshevsky will be compared to the acid hydrolysis scheme. The energies of reactions will be compared, with derivation of the reaction rate where it is possible to isolate a transition state. 

\section{Methodology}
The energies of homolytic fission, elimination of \ce{HNO2} and acid hydrolysis of PETN were calculated according to equations \ref{equ:homofiss},\ref{equ:hno2_elim}.
Literature geometries were obtained for PETN from the authors of the

The intramolecular reactions of the \ac{NC} monomer were modelled according to schemes \ref{fig:homofiss_NC} and \ref{fig:elim_NC}.

\begin{scheme}[htp]
\centering
\includegraphics[width=0.75\linewidth]{homofiss}
\caption{Removal of nitrate group \textit{via} homolytic fission of \ac{NC}}
\label{fig:homofiss_NC}
\end{scheme}

\begin{scheme}[htp]
\centering
\includegraphics[width=0.75\linewidth]{elimhono}
\caption{Removal of nitrate group \textit{via} elimination of \ce{HNO2}. }
\label{fig:elim_NC}
\end{scheme}

\subsection{Computational details} 

%ZPE correction applied by addition of individual ZPE values back into their total energy
%IRC step ,maxpoints=256,maxcycle=256 , stepsize the default of  0.1 Bohr



\section{Results and discussion}
The energies for homolytic fission and intramolecular elimination of \ce{HNO2} from \ac{PETN} is shown in table \ref{tab:intramolec}.
%NOTE: kuklja kept the HNO2 complexed with the 

\begin{table}[hbp]
\begin{center}
\caption{Calulated free energies of reaction ($o\Delta G_{r}$) with zero-point energy correction ($ZPE_{corr}$), reaction enthalpies ($\Delta H_{r}$) and activation barriers ($E_{a}$) for the intramolecular reactions of PETN, and the \ac{NC} monomer.}
\begin{tabular}{ l *{4}{S[table-format = 2.4]}} 
%\begin{tabular}{ l l l l l } 
\toprule
Reaction			& $\Delta G_{r}$ 	&	$\Delta H_{r}$	&	${E_{a}}$		&  ${ZPE_{corr}}$ \\
\midrule
\multicolumn{5}{l}{PETN}\\
\hline
\ce{NO2^{.}} loss	& 	21.50946		&	35.61579		& \textendash	& 16.55804 \\
\ce{HNO2} loss		& -23.62626		& -20.39247		& 41.2902		& 36.28071 \\
\toprule
\multicolumn{5}{l}{\ac{NC} monomer} \\
\midrule
\ce{NO2^{.}} loss	& 23.25456			& 36.26343 	& \textendash	& 12.98415 \\
\ce{HNO2} loss		& -36.04986		& -22.85892 	& 40.69863 	& -39.42322 \\
\bottomrule
\end{tabular}
\label{tab:intramolec}
\end{center}
\end{table}



