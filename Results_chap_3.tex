\chapter{Mechanisms of denitration under ambient conditions}
\label{chapterlabel5}
\graphicspath{ {./R_chap_3_pics/} }

\section{Introduction}%%%%%%%%%%%%%%%%%%%%%%%%%%%%%%%
\label{chapter5:intro}
\subsection{Thermolytic reactions}

The first stage of thermolytic decomposition for nitrate esters is generally agreed to be homolytic fission of the O-N bond linking the nitrate to the alkyl chain, leading to the loss of \ce{^{.}NO2} (equation \ref{equ:homofiss}) \cite{Oxley2003,Foltz2009,Shepodd1997}. %many refs
Though nitrate homolysis is an endothermic reaction, the weak O-N bond has a typical dissociation enthalpy of 42 kcal mol\textsuperscript{-1} and is easily cleaved when exposed to elevated temperatures, UV light or impact. %REF UV \cite{Zaki2016}
Whilst the thermolytic degradation of energetic materials has been widely studied experimentally, the ambient, slow ageing mechanisms are less well documented. %REF
Low-temperature decomposition routes are influenced by many factors over a protracted lifetime, and in practical use, materials are usually subject to evolving environmental conditions. %Changes in pressure, humidity, stress and temperature cycles induce changes in the degradation patterns of energetic materials to a varying degree. %ref?
External changes in pressure, humidity, stress and temperature cycling introduce variation in the degradation patterns of energetic materials. %ref?
%The array of products produced as a result of slow ageing may therefore vary greatly from those obtained from thermolysis; a general observation is that the number of different species is far smaller. %REF
The presence of moisture has been observed to lower the activation energy and accelerate decomposition %of energetic materials 
\cite{Foltz2009}.  %PETN
Internal factors including impurities, residual solvent, and crystal growth within the bulk, also alter decomposition behaviour. 

\begin{scheme}[h]
\begin{equation}
\ce{R-ONO2 -> R-O^{.} + ^{.}NO2 }
\label{equ:homofiss}
\end{equation}
%redraw this as a scheme with mechanism
\begin{equation}
%\ce{\textcolor{red}{\ce{H}}}
\ce{R\textcolor{red}{\ce{H}}-ONO2 -> R=O + \ce{\textcolor{red}{\ce{H}}NO2 }}
\label{equ:hno2_elim}
\end{equation}
\end{scheme}
%
\begin{figure}[h]
%\begin{center}
\begin{tabular}{ p{0.5\textwidth} p{0.5\textwidth} }
\begin{enumerate}
\item \ce{^{.}NO2} loss
\item \ce{HNO2} loss
\item \ce{OONO} rearrangement
\item $\gamma$-attack
\item \ce{ONO2^{.}} loss
\item \ce{C-C} cleavage (\ce{CH2O + NO2})
\item \ce{C-C} cleavage \ce{(CO + HNO2)}
\end{enumerate}
&
%\centering
%\begin{minipage}{0.5\textwidth}
%\centering
\raisebox{-\totalheight}{\includegraphics[width=\linewidth]{PETN_orig}}
%\end{minipage} 
\end{tabular}\\
\begin{center}
%\centering
{\textcolor{red}{\CIRCLE} oxygen \quad \textcolor{pink}{\CIRCLE} nitrogen \quad  \textcolor{yellow}{\CIRCLE} carbon \quad \textcolor{cyan}{\CIRCLE}hydrogen}
\end{center}
\caption[Intramolecular thermolytic reactions in \ac{PETN}.]{Intramolecular thermolytic reactions in \ac{PETN}, from the work of Tsyshevsky \textit{et al.} \cite{Tsyshevsky2013}.}
\label{fig:PETN_react}
%\end{center}
\end{figure}
%
The degradation of nitrate esters at temperatures over 100$\degree$C is primarily \textit{via} thermolytic processes, whilst under 100$\degree$C, decomposition is largely thought to be the result of hydrolysis \cite{Moniruzzaman2014}. 
Tsyshevsky \textit{et al.} studied the intramolecular reactions leading to denitration in \acf{PETN} in both the vacuum and the bulk crystal \cite{Tsyshevsky2013} (figure \ref{fig:PETN_react}). 
%
Seven mechanisms for the removal of \ce{NO2} were explored. Corresponding to the labels in figure \ref{fig:PETN_react}:
% (1)  (2) (3) (4) (5) (6) 						 (7)
(1) homolytic cleavage of the \ce{O-NO2} bond, (2) the elimination of nitrous acid (\ce{HNO2}) which is usually considered a competing reaction to homolytic fission, (3) the nitro-peroxynitrite rearrangement (\ce{O-ONO}), (4) \textgamma-attack of the terminating nitrate oxygen atom and the bridging nitrate oxygen at their relative \textgamma-carbon sites, (5) the homolytic \ce{C-O} bond cleavage, (6) and (7) two variations of the homolytic \ce{C-C} bond cleavage. 

It was found that the two most significant decomposition reactions were homolysis of the nitrate ester \ce{O-NO2} bond (equation \ref{equ:homofiss}) and intramolecular elimination of \ce{HNO2} (equation \ref{equ:hno2_elim}).
Whilst elimination of \ce{HNO2} was found to be the most energetically favourable denitration pathway, homolytic fission dominated preliminary decomposition steps due to the lower activation barrier and faster rate of reaction. It was suggested that global decomposition processes were determined by the interplay between these two mechanisms. Initial homolysis facilitated wide-spread denitration, complemented by exothermic \ce{HNO2} elimination, promoting self-heating of the system and further bond dissociations. 
The presence of \ce{^{.}NO2} and \ce{HNO2} have previously been linked to the autocatalytic rates observed for later-stage decomposition of nitrate esters \cite{Rodger1963,Lindblom2002,Volltrauer1981}, though some studies solely attribute it to the presence of acids \cite{Edge1990,FernandezdelaOssa2011,Baker1952,Binke1999}. %CHECK THESE ARE FACT
Other studies also implicate the action of \ce{^{.}NO} and \ce{HNO3}, in addition to \ce{^{.}NO2} and \ce{HNO2} \cite{Zayed2012,Trache2018}. 
%avanced stage decomp? Rodger actualy assumes NO2 is confirmed for PETN
% The presence of acid has been linked to autocatalytic rates of degradation during later stages of \ac{NC} degradation\cite{Edge1990,FernandezdelaOssa2011,Baker1952,Binke1999}. 
Inspection of these products generated from initial processes, with observation of the species permeating through to later stages, will shed light on the most likely contributors to  autocatalysis. 
%However, from these initial processes it is not possible to determine which is the species resposible % (this topic is further discussed in the next chapter.) %\ref{chapterlabel6}.) %REF
\subsection{Acid hydrolysis reactions}

Spent acids remain in the \ac{NC} matrix following synthesis even with thorough washing procedures. %but how much acid? What concentrations? %REF
Additional acidic species are released via the subsequent reactions of \ce{^{.}NO2} following homolysis. %, Any residual sulfuric acid from the nitration process catalyses the protonation of \ce{HNO3}, thereby the formation of nitronium (\ce{NO2+}) and hydronium (\ce{H3O}). 
These acids proceed to react with other moieties in the system, such as unsubstituted alcohol side chains on the polysaccharide, or small molecules free in the bulk. 

When exploring the interaction of nitroglycol and nitroglycerin in acid solution, Camera proposed a protonation-denitration scheme (scheme \ref{sch:cameraschema}) whereby initial protonation at the nitrate is rapid (equation \ref{equ:cam_1}), but the subsequent release of the nitronium ion was slow and rate determining (equation \ref{equ:cam_2}).
\begin{scheme}[htp!]
\centering
%\textbf{Hydrolysis scheme for ethyl nitrate}
%\begin{block_indent}{1cm}
%\ce{CH3CH2ONO2 + H+ <=>[\text{fast}] CH3CH2ONO2H+} \\
%\ce{CH3CH2ONO2H+ <=>[\text{slow}] CH3CH2H + NO2+} \\
%\ce{NO2+ + 2H2O <=>[\text{fast}] HNO3 + H3O+} \\
%\end{block_indent}
\begin{equation}
\ce{CH3CH2ONO2 + H+ <=>[\text{fast}] CH3CH2ONO2H+}% \\
\label{equ:cam_1}
\end{equation}
%
\begin{equation}
\ce{CH3CH2ONO2H+ <=>[\text{slow}] CH3CH2OH + NO2+} %\\
\label{equ:cam_2}
\end{equation}
%%Using generic R groups instead
%\begin{equation}
%\ce{RONO2 + H+ <=>[\text{fast}] RONO2H+}% \\
%\label{equ:cam_1}
%\end{equation}
%\begin{equation}
%\ce{RONO2H+ <=>[\text{slow}] ROH + NO2+} %\\
%\label{equ:cam_2}
%\end{equation}
%\begin{equation}
%\ce{NO2+ + 2H2O <=>[\text{fast}] HNO3 + H3O+} \\
%\end{equation}
\caption[The relative rate of stepwise protonation and denitration of nitrate esters.]{The relative rate of stepwise protonation and denitration of nitrate esters, using ethyl nitrate as an example. From the work of Camera \textit{et al.} \cite{Camera1982}.}
\label{sch:cameraschema}
\end{scheme}

\begin{scheme}[hb]
%\begin{figure}
\centering
\includegraphics[width=0.8\linewidth]{tertiary_nitrate_Hu2011}
%\end{figure}
\caption[Hydrolysis of a tertiary nitrate, originally derived from the reaction of isoprene in the aerosol phase.]{Hydrolysis of a tertiary nitrate derived from the reaction of isoprene in the aerosol phase, from the work of Hu \textit{et al.} \cite{Hu2011}.}
\end{scheme}

\ac{NC} in storage is kept wetted with solvents to prevent self-ignition. Material with 12.6\acs{pN} or lower must be stored in 25\% water by mass, or in a controlled mixture of solvents, stabilisers and plasticisers. The material is therefore always exposed to water, with the fast-exchange of protons expected at inter-facial surfaces. % Something about the surfaces exposed to water? Faster NO'2 release or something?
%Fast exchange is expected for groups exposed to water within the bulk, with the dissociation of \ce{NO2+} to generate the alcohol as the rate determining step. %ref back to lit review, about having to store it in water for stability.
%more info on why the second step may be slow / what we expect to see?
In the study of organonitrates and organosulfates %in the acid phase 
generated from isoprene in the aerosol phase, % as secondary organic aerosol, %\ac{SOA}, 
Hu \textit{et al.} found that primary and secondary nitrates were resilient to hydrolysis for pH $>$ 0, whilst tertiary nitrates underwent hydrolytic nucleophilic substitution easily, reacting with water to form alcohols \cite{Hu2011}. %and sulfate to form alcohols and organosulfates

In tertiary nitrates, the carbon is fully substituted with no attached hydrogens. This group is usually sterically hindered and stabilising to carbocations, condition on the electron donation ability of the substituents remaining after nitrate removal. 
If formation of a carbocation intermediate is involved in the hydrolysis mechanism, this may explain why the tertiary nitrates exhibited highly efficient denitration, even under neutral conditions.

Though no specific mechanistic detail is given, the action of a protonated transition state during hydrolysis is alluded to %by Hu \textit{et al.}, 
through the contrast between the rate of acid-catalysed and neutral hydrolysis reactions. Neutral hydrolysis of the tertiary nitrates occured rapidly, but hydrolysis only occurred for primary and secondary nitrates under strongly acid catalysing conditions, and at a much slower rate.   % Adjacent OH groups did not slow down the acid cat hydrolysis of nitrates, but they did for the sulfates. OH groups slowed down hydrolysis for the primary and secondary nitrates.
% in Neutral hydrolysis for tertiary nitrates, more OH meant slowdown in rate of hydrolysis. This primary and secondary nitrates did not experinece this. 
%It was found that t
Additionally, the presence of adjacent OH groups hampered the rate of hydrolysis for some aerosol dispersed organonitrates. In the neutral hydrolysis of tertiary nitrates, increasing the number of adjacent \ce{OH} groups lead to protracted hydrolysis lifetimes.
Interestingly, the retardation effect of adjacent \ce{OH} groups was not observed for the the acid catalysed cases. Hu proposed that this could be due to the interaction of \ce{OH} with the transition state of the neutral hydrolysis system, compared to the protonated transition state of the acid catalysed system, impeding the reaction only in the former case. 
%Pretty weak sentence
The cause of this effect is unclear, and without understanding of the %transition states 
mechanisms involved, it is difficult to explain. % The neutral case happens much faster than the acid case - could it be that by having an OH next door, it is imparting some ''acid-like'' features to the reaction of the neutral process? %Need a bit more explanation here

There is evidence that nitration and denitration of nitrate esters is also influenced by the presence of nitrate groups at neighbouring positions. Matveev \textit{et al.} demonstrated that for poly-nitroesters the rate of liquid-phase decomposition did not increase linearly with number of nitrate reaction centres. It was found to mainly dependend on individual structures (table \ref{tab:reactions}) \cite{Matveev2003}.
% it really looks like the more stabilising the substituted group is - with respect to being able to  sustain a 
\begin{table}[t]
\begin{center}
\caption[Comparison of rate constants of decomposition for various polynitrate esters at 140$\degree$C.]{Comparison of rate constants of decomposition for various polynitrate esters at 140$\degree$C.
$\Delta$T is the decomposition temperature range,  $E$ is the experimental activation barrier for decomposition, log$A$ is the pre-exponential factor, $T_{c}$ is the combustion temperature, $k$\textsubscript{expt} is the rate constant for decomposition. 
Collated from literature sources by Matveev \textit{et al.}\cite{Matveev2003}.
}
%\begin{tabular}{ l c {S[table-format = 2.1]} {S[table-format = 2.1]} {S[table-format = 2.1]}} 
\begin{tabular}{p{15em} c c c c} 
\toprule
\multirow{2}{10em}{Compound	}	&	 $\Delta T$	&  $E$							&  log$A$ 				& $k$\textsubscript{expt} \\ %$k\subscript{exp}
 										&	/ $\degree$C & / kcal mol\textsuperscript{-1}	&[s\textsuperscript{-1}]	&  / \SI{e-6}s\textsuperscript{-1} \\
\midrule
\ce{O2NOCH2CH2CH2ONO2}					& 72–140 	& 39.1 	& 14.9 	& 1.7 \\
\ce{O2NOCH2CH2CH2CH2ONO2}			& 100–140 	& 39.0 	& 14.7 	& 1.1 \\
\ce{O2NOCH(CH3)CH(CH3)ONO2}			& 72–140 	& 40.3	& 14.9 	& 5.0 \\
%\ce{O2NOCH2CH2(NNO2)CH2CH2ONO2}		& 80–140 	& 41.5	& 16.5 	& 3.5 \\
\\
\ce{O2NOCH2CH2OCH2CH2ONO2}			& 80–140 	& 42.0 	& 16.5 	& 1.9 \\
\ce{O2NOCH2CH(OH)(CH2ONO2)}			& 80–140 	& 42.4 	& 16.8 	& 2.3 \\
\\
\ce{O2NOCH2CH(ONO2)(CH3)}				& 72–140 	& 40.3	& 15.8 	& 3.0 \\ 
\ce{[(O2NOCH2)CH(ONO2)CH(ONO2)]2}		& 80–140 	& 38.0	& 15.9 	& 63.0 \\
	 (hexanitromannite)							&			&		&		&	\\
\bottomrule
\end{tabular}
\label{tab:reactions}
\end{center}
\end{table}
%
%It was suggested that 
The trend in reactivity could be partially explained by the inductive effect of the nitro groups \cite{Oxley2003}. %actually he talks about nitro (NO2), not nitrate...
The inductive effect arises when a difference in the electronegativity between atoms connected by a $\sigma$ bond leads to a polarisation, or permanent dipole, in the bond. Electron donating groups % are of lower electronegativity than  
increase the $\delta^{-}$ partial charge on neighbouring atoms through the release of electrons, whilst electron withdrawing groups pull electron density away, %from neighbouring atoms
generating a $\delta^{+}$ charge on connected atoms. 
%However, the 
%%$\pi$ donation by lone pairs on oxygen and nitrogen also plays a significant role in increasing electron density at adjacent atoms %, known as 
%through the resonance effect. 
%Nitrate ester moieties
The \ce{NO3} presents a stronger electron withdrawing effect than \ce{OH}, which is a donating group. %%The \ce{NO3} presents a stronger electron donating effect via $\pi$ donation than \ce{OH}. In the case of the saturated polysaccharide ring, the \ce{OH} does not exhibit any $\pi$ donation properties, and instead acts as a % though both groups are activating. %, in the context of the polysaccharide ring.  %ref?  
%%OH is actually electron withdrawing for aliphatic alkyles? ``In organic chemistry, an alkyl substituent is an alkane missing one hydrogen.'' %ref? - SO maybe the OH is deactivating here. 
%%It would therefore be expected that both increase the rate of hydrolysis for nearby leaving groups. %(electron donating via \pi donation) 
The presence of an adjacent nitrate appears to facilitate denitration, whereas the presence of hydroxyl groups hinders this process, for neutral hydrolytic schemes. 
The resonance effect, arising from $\pi$ donation by lone pairs on oxygen and nitrogen is negligible %does not come into play 
between substituents at different sites on the polysaccharide ring, as the ring is saturated. 
%This suggests that the proposed interaction of the hydroxyl group with the neutral transition state supersedes its resonance effect. 
%OH - tertiary, neutral hydro: slow down, - primary, secondary, acid cat hydro: no effect
%As a result, it is ambiguous whether any apparent rate increase due to the presence of adjacent nitrate groups arises as a result of the resonance effect of the nitrate, or whether it is solely due to the absence of a neighbouring hydroxyl. 

The investigation by Hu \textit{et al.} exclusively focused on nitrates generated from an isoprene precursor, upon dispersion as an aerosol. The nitrate groups present in \ac{NC} are either of primary (C6) or secondary (C2, C3) structure, indicating that ambient hydrolysis is unlikely according to this scheme. However, solvent effects are expected to differ for condensed-phase reactions and aerosol phases. A greater build-up of acid concentration can be achieved in a closed, condensed system, and the lifetime of an aerosol is relatively short-lived when considering the timescale of slow ageing processes in \ac{NC}. Thus, the work of Hu \textit{et al.} does not provide a direct comparison for the \ac{NC} polymer but highlights the possible contribution from both neutral and acid-catalysed hydrolysis routes, and effect of increasing levels of substitution on the wider structure.%Though ordinarily it would be expected that reactions in aerosol would exhibit a rate increase due to a larger reaction surface, this may skew/increase the rate of specific reactions that do not rely on the diffusion of solvent or other species that move/migrate more slowly through the material. %Where is this even coming from. Some sort of literature proof please. 


In this section, the possible mechanisms for nitrate removal from the \ac{NC} backbone are explored. The homolytic fission and \ce{HNO2} elimination thermolytic processes suggested by Tsyshevsky will be compared to the acid hydrolysis scheme. %The energies of reactions will be compared, with derivation of the reaction rate where it is possible to isolate a transition state. 
Though the relative rates of reaction are not compared, the extended timescales involved in ambient ageing imply that the dominating reactions correspond to those most thermodynamically favourable.

\section{Methodology}%%%%%%%%%%%%%%%%%%%%%%%%%%%%%%%
%Thermolytic reactions
The energies of homolytic fission and elimination of \ce{HNO2} were calculated for PETN, as a test system before extension to the \ac{NC} monomer. 
The reaction energies were calculated according to equations \ref{equ:homofiss} and \ref{equ:hno2_elim} to reproduce the work of Tsyshevsky \textit{et al.}; the published geometries of PETN and its derivatives were obtained from the authors. 
Whilst only the denitrated radical product geometry was provided in the case of the homolysis reaction, the product of \ce{HNO2} elimination was given as a complex of the \ce{HNO2} leaving group and the newly formed aldehyde. 
A single point energy and frequency calculation were performed on each of the species to determine the reaction energies; no geometry optimisation was performed on the given structures except for in the case the \ce{^{.}NO2} molecule, where the geometry was not supplied. A separate \ce{^{.}NO2} molecule was independently geometry optimised.

The intramolecular reactions of the \ac{NC} monomer were modelled according to scheme \ref{sch:intra_NC}. % \ref{fig:homofiss_NC} and \ref{fig:elim_NC}.
Rigid and relaxed \ac{PES} scans were attempted for both reactions for the \ac{NC} monomer to obtain an energy profile, and in the case of \ce{HNO2} elimination, identify a transition state. The homolysis reaction was treated as barrierless.  
Where the scans were unable to identify a valid transition state geometry, guess transition state geometries were constructed and optimised. 

\begin{scheme}[h]
\centering
%\begin{figure} Change to C2 instead of C3 in the diagrams
  	\begin{subfigure}[b]{0.75\linewidth}
  	\centering
	\includegraphics[width=\linewidth]{homofiss}
	\caption{Removal of a nitrate group \textit{via} homolytic fission of \ac{NC}}
	\label{fig:homofiss_NC}
	\end{subfigure}
	\par\bigskip
	\begin{subfigure}[b]{0.75\linewidth}
	\centering
	\includegraphics[width=\linewidth]{elimhono}
	\caption{Removal of a nitrate group \textit{via} elimination of \ce{HNO2}. }
	\label{fig:elim_NC}
	\end{subfigure}
%\end{figure}
\caption{The proposed intramolecular reactions for the initial denitration step during \ac{NC} degradation.}	
\label{sch:intra_NC}
\end{scheme}

The possible protonation sites for the \ac{NC} monomer were explored by placing a proton at each of the different oxygen sites surrounding the nitrate group. The structures were then geometry optimised and energies of protonation were compared. \ce{H3O+} was modelled as the donating species; as \ac{NC} is usually stored wetted in water, the hydronium ion is the most likely source of protons (equation \ref{equ:protonation}). %ref precentage water i storage? Pop it in lit review
%Pliego2001 It is well known that
%most esters undergo basic hydrolysis by nucleophilic attack of
%the hydroxide ion onto the carbonyl carbon to yield a
%tetrahedral intermediate, namely the so-called BAC2 mechanism. In rare cases, nucleophilic attack of the hydroxide ion on
%the saturated alkyl carbon leads to the final products in one
%step through an SN2 or BAL2 mechanism
\begin{equation}
\ce{RONO2 + H3O+ -> RONO2H+ + H2O}
\label{equ:protonation}
\end{equation}

It is also possible that the proton is donated by other acidic species in the system, particularly \ce{HNO2} or \ce{HNO3}. This is more likely at later stages of degradation when a higher concentration of acid has been generated by secondary reactions. %REF 
For investigations into the hydrolytic methods of nitrate removal following protonation, or a concerted mechanism involving a proton donor, an ethyl nitrate molecule was used as a truncation of the monomer. This facilitated a speedup of initial \ac{PES} scanning by reducing the degrees of freedom, whilst presenting the moieties necessary for preliminary \ac{TS} searches for the hydrolytic scheme. %CHECK long winded dodgy sentence
This was with the intention of later using the found \ac{TS} geometries to inform guesses for the \ac{NC} monomer structure.  
% Explain the caveats of this. 
%pKA H3O+ –1.74, HNO3 –1.3, HNO2 3.3. All weak acids, but H3O+ is most acidic. 
% SOME MORE INFO ON the later stage stuff you did on hydrolysis mechanism

\subsection{Computational details} %%%%%%%%%%%%%%%%%%%%%%%%
All geometry optimisations, thermochemistry calculations and \ac{PES} scans were performed in \ac{G09}. 
Geometry optimisation and thermal calculations were to the level of 6-31+G(2df,p). \ac{NC} monomer structures were optimised using \ac{wb97xd} and \ac{B3LYP}, to default convergence criteria. %and \ac{MP2}. 
$\Delta G$ values were obtained by the difference between the thermally corrected free energies of products and reactants. 
Zero-point corrected energies \acs{ZPE} were determined by addition of individual \ac{ZPE} to the free energy:
\begin{equation}
\centering
\Delta G^{ZPE} = \sum(G_{products}+ZPE_{products}) - \sum(G_{reactants}+ZPE_{reactants})
\label{equ:ZPE}
\end{equation}
%IRC step ,maxpoints=256,maxcycle=256 , stepsize the default of  0.1 Bohr
\ac{PES} scans were performed to the level of \ac{wb97xd}/6-31+g(d), or using unrestricted \ac{wb97xd}, in the case of \ce{O-^{.}NO2} dissociation. 
Rigid scans were carried out by fixing bond lengths, angles and dihedral values as constants. 
Only the variable of interest was allowed to change. %evolved
This was with the exception of relaxation of other specified coordinates required for accommodation of the new geometry, following each step of the scan.  %(semi-rigid scans). 
For example, in the homolysis of the nitrate \ce{O-NO2} bond, as the \ce{NO2} group departed, the internal \ce{O\textendash N\textendash O} angle was also allowed to relax, in addition to the angle of the departing \ce{NO2} with respect to the remainder of the molecule.
In two-dimensional scans, two variables were scanned simultaneously. For the same reaction, the elongation of a the \ce{O-NO2} bond was scanned with simultaneous approach of a proton, to monitor the effect of protonation for the same reaction. %In this case, both the internal angle within  \ce{O-NO2} and the angle was allowed to relax. 
Relaxed scans were performed in Gaussian using the ``modredundant'' function, whereby the whole structure was geometry optimised after each step of the scan. 
Scans were performed with step size of 0.1 $\text{\AA}$. The number of steps varied with the property investigated, though the majority of the phenomena were observed within 20 steps (2 $\text{\AA}$).
Scans were attempted in vacuum, and for the protonation cases, implicit solvent using \ac{PCM} \cite{Miertus1981}.
%Following each successful scan, a low-level frequency calculation was performed on the obtained transition state. If singular imaginary vibration matching the key bond transformation for the reaction step persisted, then a transition state search was performed using this geometry.
%Where possible, the intermediate “product” geometry obtained from the successful scan was also optimised to B3LYP/6-311+G(d,p) for use in transition state searching using QST2 and QST3 methods.

%The protonation studies conducted in solvent were calculated in \ac{PCM} ...% and \ac{SMD} intrinsic solvent models \cite{Marenich2009,Miertus1981}. 

\section{Results and discussion}%%%%%%%%%%%%%%%%%%%%%%%%%%
\subsection{Thermolytic decomposition mechanisms}
The energies of homolytic fission and intramolecular elimination of \ce{HNO2} from a \ac{PETN} nitrate group are shown in table \ref{tab:intramolec}. The energy values published by Tsyshevsky \textit{et al.} are denoted in parenthesis. The $\Delta G_{r}$ and $E_{a}$ values are the same.

\begin{table}[b]
\begin{center}
%\par\bigskip
\caption[Calculated free energies of reaction for the intramolecular reactions of PETN and the \ac{NC} monomer.]{Calculated free energies of reaction ($\Delta G_{r}$), reaction enthalpies ($\Delta H_{r}$), activation barriers ($E_{a}$)  with zero-point correction ($^{ZPE}$) for the intramolecular reactions of PETN, and the \ac{NC} monomer. Values expressed in kCal mol$^{-1}$. }
% Remember, these have not been scaled. Yao2018 : B3LYP/6-31+G(d,p) level and are scaled by a factor of 0.98, for the ZPE (And only certain components are scaled? Drama)
%FIX THE FOOTNOTES
\begin{tabular}{ l *{5}{S[table-format = 2.1]}} 
%\begin{tabular}{ l l l l l l} 
\toprule
Reaction				& $\Delta G_{r}$	&$\Delta G_{r}^{ZPE}$ 	& $\Delta H_{r}$	&	${E_{a}}$	& ${{E_{a}}^{ZPE}}$ \\
\toprule
PETN					& \multicolumn{5}{l}{}\\
\midrule
% Do I double these up??
\ce{^{.}NO2} loss	& 	21.50946		 &16.55804 		&	35.61579	& 	21.50946  $ ^{b}$ 	&16.55804 	 \\
						& ${(41.2)}^{a}$	 &	${  (35.8)}$ 	&					&${(41.2)}$ 	& ${(35.8)}$ \\
%\ce{^{.}NO2} loss	& 	21.50946		 &16.55804 		&	35.61579	& 	\textendash 	& \textendash 	 \\
%						& ${(41.2)}^{a}$	 &	${  (35.8)}$ 	&					&	\textendash\  	& \textendash\  \\
\rule{0pt}{4ex}   
\ce{HNO2} loss		& -23.62626	 	 &	-26.212190		& -20.39247	& 41.2902		& 36.28071  \\
						& ${(-18.6)}$		 &						&					&	${(47.3)}$	& ${(42.7)}$ \\
%\rule{0pt}{4ex}   
\toprule
\ac{NC} monomer	& \multicolumn{5}{l}{} \\
\midrule
\ce{^{.}NO2} loss	& 23.25456 		& 18.68672 		& 36.26343 	& 23.25456 	&	18.68672 \\
\rule{0pt}{4ex}   
\ce{HNO2} loss		& -36.04986 		& -39.42322		&-22.85892 	& 40.69863 	& 37.32527  \\
\bottomrule
\end{tabular}
\label{tab:intramolec}
\end{center}
$^a$ values from the work of Tsyshevsky \textit{et al.} \cite{Tsyshevsky2013}.\\
$^b$ values for the activation energy and total energy of reaction are the same 
for bond dissociation via homolytic fission.
%when treating homolytic fission as a barrierless process. 
\end{table}
%
Despite using the author supplied geometries, same method and basis, it can be seen that the obtained PETN reaction energy in the case of homolytic fission (\ce{^{.}NO2} loss) varies greatly from the value published by Tsyshevsky \textit{et al.}
It was assumed that the supplied geometries were those used to generate the energy values quoted in their study. Inspection of the forces for the given structures showed that they were %in fact 
not converged. 
As the same structures were used to calculate values listed in table \ref{tab:intramolec}, the unconverged geometries do not explain the large discrepancy between the published energies and values obtained here.
% or do they? Perhaps because forces are not converged, it is unclear what will happen when you just calculate the associated energy -though I'm sure its also fine. The calc uses whatever structure is given. And tbh I don't know the intricacies of its calculation/ integration loops/ whether it will really make a difference
% , though a possible explanation may be that different geometries were used for the values presented in the study.  (maybe this is too obvious to say)
A contribution may arise from different compilations of the \ac{G09} program, leading to fluctuations in the exact values obtained. 
%small difference in the method of calculation 
%Semi-empirical methods are also known to over-estimate reaction barriers (and underestimate reaction energies). CHECK THIS AND FIND REF
These differences would be amplified %when deriving
when combining individual energies during the calculation of reaction energy values, but are still not expected to %do not 
account for the 20 kCal mol$^{-1}$ deficiency %discrepancy
in the case of homolysis. % reaction energy. 

A possible explanation is that the products of homolytic fission were treated as a complex, rather than individual molecules, though % remain complexed 
%following bond breaking, 
this was not highlighted %discussed 
in the original publication nor explicit in the supplied geometries. %from the authors. 
In the current study, %the homolysis %\ce{^{.}NO2} loss 
%calculations conducted, %listed in table \ref{tab:intramolec}, 
the \ce{^{.}NO2} leaving group was independently geometry optimised, with its energy contributing additively to the products %energies 
of the reaction. 
Under this assumption, the product species would move apart immediately following fission. 
If the energy of the product species were calculated separately in the original study, it could be that the departing \ce{^{.}NO2} was not geometry optimised, but left in its complexed geometry.
%In the quoted study, it , even if the leaving group and molecule were not calculated together. 
This is relevant in the scenario where the \ce{^{.}NO2} has not yet moved far enough from the remainder of the \ac{PETN} molecule to fully relax. 
Whilst the calculated $\Delta G_{r}$ %free energy of reaction 
deviates greatly from the value in Tsyshevsky's study, % by almost half, %shit sentence CHECK
the $\Delta H _{r}$ %enthalpy 
falls within 7kCal mol$^{-1}$ of the literature value for \ce{O-NO2} homolysis of alkanes (%35.62 kCal mol$^{-1}$ to
42$\pm$0.3 kCal mol$^{-1}$) \cite{Luo2004}. %scaling factor for wb97xd CHECK
%How does inclusion of zero point energy affect the enthalpy values?

%This was tested against the \ac{NC} monomer 
%However these details were omitted from both the literature and the supplied geometries. (Does it matter though - the following paragraphs confirms the mechanism as proof of concept that it actually happens - that way) so I can get away with it.)   
 %Additional explanations may be that %and the obvious suggestion that perhaps alternative geometries were used ,....  in addition to a different compilation of Gaussian
%Despite the difference in the calculated $E_{a}$ values, both fall into the range of PETN experimental activation energies for decomposition. Actually no, mine is too low. 
%NOTE: kuklja kept the HNO2 complexed with the 
%
%Add in the zero point correction for enthalpy
%Add in the optimised PETN geometries (that I was given)
%Add in the PETN geomtries that I did myself (from scratch - which are slightly off).
% These energies don't seem to reflect the excel sheet... update
%

The homolytic fission reaction was applied to a \ac{NC} monomer singly nitrated at C2 (figure \ref{fig:homol_scan}). 
% freenitr-NO2-C2-scan
% floppy_NO2-C2-scan ***
%get an react coord pic if you can (not v interesting though)
% one scan was complete freeze except with the O-NO2
% the other one alloed relaxation aroud the NO2 and its orientation
%multireference character, vs DFT
% Check Yao2018, and read up on multireference characters. 
%
%Scan of the PES
The \ce{O-NO2} bond was incrementally stretched using geometry scanning, to obtain an energy profile of the reaction. 
%The energy profile of the reaction was obtained via %\ac{PES} 
%geometry scans of the \ce{O-NO2} bond elongation, as \ce{NO2} moved away from the \ac{NC} monomer. 
For the rigid scan (figure \ref{fig:homofiss_graph}), only the internal angle of the departing %\ce{^{.}NO2}
\ce{NO2} and coordinates referencing its orientation relative to the rest of the molecule were allowed to relax. %Include the details of this calc, which coords allowed to relax, in the appendicies?
As the scan progressed, the \ce{NO2} internal angle increased from 129.2$\degree$ to 134.6$\degree$, at maximum separation of 5 $\text{\AA}$ from the bridging oxygen (Ox). This corresponds to the literature value for the \ce{O-N-O} internal angle (134.3\degree) confirming the formation of a \ce{^{.}NO2} radical. %\cite{Trostchansky2010}
%Nevertheless, it is known that the activation energy for bond dissociation is equivalent to the bond enthalpy. 
%SO what?
%This confirms the formation of the radical species via homolytic fission, and that at a separation of 4 $\text{\AA}$ the radical is fully formed and relaxed. 
%
%The energy of reaction (table \ref{tab:intramolec}) is %2kCal mol${-1} 
%higher than that of PETN, which may be expected

\begin{figure}[t]
\centering
%\begin{subfigure}[0.3\linewidth]
%\includegraphics[width=\linewidth]{EN_NO2_leave_prot_arrow}
%\label{fig:EN_homol_scan}
%\end{subfigure}
%
%\begin{subfigure}[0.3\linewidth]
\includegraphics[width=0.3\linewidth]{homolysis_arrow}
%\includegraphics[width=\linewidth]{homolysis_arrow}
%\caption{ }
%\label{fig:NC_homol_scan}
%\end{subfigure}
%\caption{The \ce{O-NO2} bond was elongated during rigid and relaxed \ac{PES} scans simulating homolytic fission for \ref{fig:EN_homol_scan} ethyl nitrate and the \ref{fig:NC_homol_scan} \ac{NC} monomer.}
\caption{The \ce{O-NO2} bond was elongated during rigid and relaxed \ac{PES} scan to simulate homolytic fission for the \ac{NC} monomer.}
\label{fig:homol_scan}
\end{figure}
%The weird little bump is step 14, ie. bond separation 2.69233 angle 137.286
%After this bump is where the angle starts to decrease (even though the energy is still rising)
\begin{figure}[ht]
\centering
\includegraphics[width=0.65\linewidth]{ONOangle_HOMOFISS_scan}
\caption[The change in \ce{O-N-O} internal angle with homolytic fission.]{The relaxation of the \ce{O-N-O} internal angle as the \ce{NO2} group is pulled away from the \ac{NC} monomer during a rigid geometry scan of homolytic fission.}
\label{fig:homofiss_graph}
\end{figure}

The values obtained for \ce{HNO2} elimination of PETN follow the results given by Tsyshevsky much more closely. 
The energies fall within 5 kCal mol$^{-1}$ and 6 kCal mol$^{-1}$ for the Gibbs free energy of reaction and activation barrier, respectively. 
This is within a reasonable margin of error for comparing with experimentally obtained values, though still larger than expected for those derived using the same method, basis and structure. %DO you have any experimental values for this reaction?

% Nevertheless, it still falls inside the range of experimentally measured PETN degradation activation barriers. %REF and check?
%I think the below actually goes into the methodology
In the case of the \ac{NC} monomer, both rigid and relaxed scans failed to capture the \ac{TS} for cleavage of the nitrate group via interaction with the \textalpha-hydrogen. A guess transition state was constructed based on the \ac{TS} of the analogous reaction for PETN (figure \ref{fig:elim_ts_NC}), and optimised %using opt=TS 
to produce the structure of the correct imaginary vibration. 

%dashed lined heavier, please, and PETN is missing the C-H bond
\begin{figure}[b]
\centering
\begin{subfigure}[t]{0.4\linewidth}
\centering
\caption{}
\includegraphics[width=\linewidth]{elim_hono_ts_PETN_dash}
\label{fig:elim_ts_PETN}
\end{subfigure}
\begin{subfigure}[t]{0.4\linewidth}
\centering
\caption{}
\includegraphics[width=\linewidth]{elim_hono_ts_NC_dash}
\label{fig:elim_ts_NC}
\end{subfigure}\\

{\textcolor{red}{\CIRCLE} oxygen \quad \textcolor{blue}{\CIRCLE} nitrogen \quad  \textcolor{gray}{\CIRCLE} carbon \quad $\bigcirc$ hydrogen}
\caption[\ac{TS} for the elimination of \ce{HNO2} in \ref{fig:elim_ts_PETN} PETN and \ref{fig:elim_ts_NC} \ce{NC}.] {\ac{TS} for the elimination of \ce{HNO2} by removal of the \textalpha-hydrogen by the \ce{NO2} leaving group in a) PETN and b) \ce{NC}. Orange dashed lines indicate bonds breaking and blue dashed lines indicate bonds forming.}
\label{fig:elim_ts}
\end{figure}

The pattern for the \ac{NC} monomer resembles that found for \ac{PETN}; homolysis is endothermic but with lower activation barrier, %put in the numbers
whilst \ce{HNO2} elimination is exothermic, but with a much higher barrier. It is anticipated therefore, that the rate of homolytic fission will be faster, whilst \ce{HNO2} loss will happen more slowly, whilst contributing to system heating and increasing acid concentration. 
%Kinetics??? CHECK - should be a quick calc
%more details on what you found 

%Summary Sentence - why are the energies of interest, here?
%comment on the significance of the energies and descibe a bit more of what you see geometrically and via the discovered energies

\subsection{Acid hydrolysis mechanism}
\subsubsection{Protonation site}
\label{AH_Protonation}
%Why do we need to look at protonation?
In polar, protic solvents such as water, the fast-exchange of protons between the aqueous medium and  %acceptor groups 
the monomer is expected. %Acid species are also expected to protonate 
Computational studies by Jebber and Liu \textit{et al.} probed the protonation behaviour of $\beta$-glucose \cite{Jebber1996,Liu2010}. 
Key findings demonstrated that the most favoured protonation site in glucose was greatly influenced by the conformation of the C6 side branch. For the C6 hydroxymethyl chain orientated in the gauche position, protonation of the ring-oxygen produced the most stable structure.  %; the lowest energy site for protonation whilst the C6 hydroxymethyl chain was in the gauche position, was the oxygen of the C1 hydroxy group. 
%The analogous group in the \ac{NC} monomer, with the C6 nitrate in the gauche position being the lowest energy conformer %see chapt 1
%, is the C1 capping group oxygen.   

The protonated \ac{NC} monomers explored in this section are shown in figure \ref{fig:proton_site}. The bridging oxygen (Ox)% linking the nitrate to the remainder of the molecule
, the C1 capping group oxygen, and the terminal nitrate oxygen sites (Ot) were protonated and their relative energies compared in order to determine the site most likely to stabilise the proton at thermal equilibrium. Protonation also occurs on other sites in the molecule, such as at unsubstituted hydroxyl oxygen sites, the capping group oxygen on C4 and O1 in the glucose ring. Whilst it is possible that protonation at further sites in the monomer would contribute to degradation, these processes would occur \textit{via} alternative mechanisms without the involvement of denitration. For the purposes of studying acid hydrolysis, only the sites peripheral to the nitrate leaving group were explored. %REf if there is any evidence - peeling off?

\begin{figure}[htp]
\centering
\begin{subfigure}[t]{0.3\linewidth}
\centering
\includegraphics[width=\linewidth]{H_bridging_b}
\caption{Bridging}
\label{fig:proton_site_bridge}
\end{subfigure}
\hfill
\begin{subfigure}[t]{0.3\linewidth}
\centering
\includegraphics[width=\linewidth]{H_terminal_b}
\caption{Terminal}
\label{fig:proton_site_terminal}
\end{subfigure}
\hfill
\begin{subfigure}[t]{0.3\linewidth}
\centering
\includegraphics[width=\linewidth]{H_cap_b}
\caption{Capping}
\label{fig:proton_site_cap}
\end{subfigure}
\caption{Protonation sites on the \ac{NC} monomer for hydrolysis of the nitrate at the C2 position.}
\label{fig:proton_site}
\end{figure}
%
\begin{table}[htp]
\begin{center}
\caption[Free energies of protonation for \ac{NC} monomer.]{Free energies of protonation, for each of the oxygen sites of interest on the \acs{CH3CH3} monomer of \ac{NC}, nitrated at the C2 site.}
%Slot in MP2 if I have time (and a subsequent one-liner about it)
\begin{tabular}{ l *{4}{S[table-format = 2.4]}} 
\toprule
\multirow{2}{*}{Protonation site} & \multicolumn{4}{c}{$\Delta$G\textsubscript{r} /kcal mol\textsuperscript{-1}} %& \multicolumn{2}{c}{$\Delta$H\textsubscript{r} }
\\\cline{2-5}
  & \acs{wb97xd} & PCM & \acs{B3LYP} & PCM\\
\midrule
%% Note, using unconverged energies here 
% Bridging 				&  -26.04105 	& 4.02759   	& -28.67067 	& 11.29338  \\
% Terminal	 (Upper)& -29.84814	&   24.28650	& -31.18500 	&  15.32916 \\
% Terminal (Lower)	& -20.53548	&   10.44288	& -22.40595	& 11.66634  \\
% Capping 				& -29.84877	&   3.61683	& -31.18563 	&  -1.16235 \\    
% Non zpe corr, but structure corr (not for the PCM b3lyp though)
% Bridging 				&  -26.04105 	& 4.29786    	& -28.67067 	& 11.29338  \\
% Terminal	 (Upper)& -29.84814 	& 13.41459	& -31.18500 	&  15.32916 \\
% Terminal (Lower)	& -20.53548 	& 11.15415	& -22.40595	& 11.66634  \\
% Capping 				& -29.84877	& 1.01556		& -31.18563 	&  -1.16235 \\
% Corrected with structures and zero point corr on 08/12/19  
 Bridging 				& -26.88687 	& 2.83903    	& -29.82831	& 2.68246  \\
 Terminal (Upper)		& -30.06758 	& 12.7897		& -31.76046 	& 11.3996  \\
 Terminal (Lower)		& -20.51871 	& 10.7196		& -22.78125	& 9.20866 	\\
 Capping 				& -30.06838	& 0.90806		& -31.76137 	& -1.08328	 \\ 
\bottomrule
\end{tabular}
\label{tab:reactions}
\end{center}
\end{table}
% Repeat these diagrams with the MP2 and B3LYP too, with a separate analysis (table) on the bonds and angles, in order to compare why the MP2 energies are a bit out there. 
%Redo this graphic to depict the new structures (4 pics)
% Pics in and out of solvent too, if poss

\begin{figure}[hp]
\centering
\begin{subfigure}[t]{0.35\linewidth}
\centering
\includegraphics[width=\linewidth]{corr_H_bridging_full}
\caption{Bridging}
\end{subfigure}
\hspace{5pt}
\begin{subfigure}[t]{0.35\linewidth}
\centering
\includegraphics[width=\linewidth]{corr_s_H_bridging}
\caption{Bridging [solvated]}
\end{subfigure}

\begin{subfigure}[t]{0.35\linewidth}
\centering
\includegraphics[width=\linewidth]{corr_H_terminal_up_full}
\caption{Terminal (Upper)}
\label{fig:tu_vac}
\end{subfigure}
\hspace{5pt}
\begin{subfigure}[t]{0.35\linewidth}
\centering
\includegraphics[width=\linewidth]{corr_s_H_terminal_up}
\caption{Terminal (Upper) [solvated]}
\label{fig:tu_solv}
\end{subfigure}

\begin{subfigure}[t]{0.35\linewidth}
\centering
\includegraphics[width=\linewidth]{corr_H_cap_full}
\caption{Capping}
\label{fig:cap_vac}
\end{subfigure}
\hspace{5pt}
\begin{subfigure}[t]{0.35\linewidth}
\centering
\includegraphics[width=\linewidth]{corr_s_H_cap}
\caption{Capping [solvated]}
\end{subfigure}

\begin{subfigure}[t]{0.35\linewidth}
\centering
\includegraphics[width=\linewidth]{corr_H_terminal_down_full}
\caption{Terminal (Lower)}
\end{subfigure}
\hspace{5pt}
\begin{subfigure}[t]{0.35\linewidth}
\centering
\includegraphics[width=\linewidth]{corr_s_H_terminal_down}
\caption{Terminal (Lower) [solvated]}
\end{subfigure}

\caption[Optimised protonated \ac{NC} monomer structures.]{Protonated \ac{NC} monomer structures after geometry optimisation to the level of \ac{wb97xd}/6-31+G(2df,p). Left column: In vacuum. Right column: With \ac{PCM} implicit solvent.} %, showing interaction between the proton on the bridging site with the capping group oxygen.}
\label{fig:proton_site_full}
\end{figure}
%B3LYP ones in appendicies
%
%The mechanism of protonation was not explored in depth here; 
%It was assumed that protons in the system would be in fast exchange with the solvent. 
%Any effects of tunneling were included within this assumption. 
The energies for the optimised protonated monomer conformers are listed in table \ref{tab:reactions}. 
There is good agreement between \ac{wb97xd} and \ac{B3LYP} values in the vacuum, with \ac{B3LYP} predicting a lower reaction energy in all cases. % WHY % and solvent, with \ac{B3LYP} . 
The large difference in the reaction energies between the gaseous and implicit solvent values are explained by the instability of \ce{H3O+} in vacuum, where it prefers to lose the proton and exist as water. Ionic species exhibit much greater reactivity in the gaseous phase compared with their neutral counterparts \cite{Pliego2001,Pliego2002,Chandrasekha1984}. 
When in solution, the positive charge is solvent stabilised% and the reactivity of the  \ce{H3O+}ion is attenuated by the solvent
; the proton is less readily released. %OR rather, a solvent shell will form around it, allowing the charge to be stabilised by the polar water molecules. Proton exchange will occur, but essentially it is more stable. Reaction with anothes molecule requires disruption of the solvent shell. Overall more stabilised and less easy to react. 
% In the real system, rapid proton exchange 

In the gaseous phase, the most thermodynamically favoured protonation sites are at the terminal (upper) and capping positions. %The energy of each of the protonated species is identical. 
Inspection of the optimised geometries explains the similarity of their $\Delta G_{r}$ values. The protonated terminal (upper) and capping monomers (figures \ref{fig:tu_vac}) and \ref{fig:cap_vac})) present nearly identical geometries in vacuum. In the terminal (upper) case, the proton has moved towards the capping group oxygen, effectively undergoing a proton transfer and generating the same structure as the protonated capping monomer. 
This is in contrast to the solvent phase (figure \ref{fig:tu_solv})), where the protonated nitrate group is rotated perpendicularly to the ring and the proton remains on the terminal oxygen; the solvent presents stabilisation for the protonated Ot. 
This indicates that the terminal (upper) site is is only likely to remain protonated in solvent, and that protonation at the site in vacuum is highly unstable.
The remainder of the monomers exhibit only minor changes in geometry between vacuum and solvent. Under solvent conditions, the most stable site for protonation is the capping group (which unlikely to lead to denitration), followed by the bridging oxygen. %explaining the energy difference between the terminal (upper) site protonation and capping group protonation between vacuum and solvent 

% Though $\Delta G_{r}$ values are positive in all cases apart from the capping group .... protonation is still going to happen, but it may tell us that the proton may not spend much time on the O
%
%Thus, the energy gained from losing the proton is negative, %less pronounced 	when in solution. 
%
%Appendicies, the  
%You wasted all this time because you did an excel cell error.
%FIXED
%For the solvated results, the two functionals perform similarly, except in the case of the Terminal (Upper) structure. Inspection of the geometries and \ac{QTAIM} analysis did not uncover any significance differences in structure or critical points (see appendix \ref{AH_Protonation_QTAIM}). 

%Draw the two water one too
\begin{figure}[h!]
\centering
\includegraphics[width=0.25\linewidth]{neutral_hydro_1}
\caption{The attempted geometry of a single water molecule in coordination with the \ac{NC} monomer.}
\label{fig:neutral_prot}
\end{figure}

Water as the protonating species was also attempted (figure \ref{fig:neutral_prot}), by optimisation of one, two and three water molecules in coordination with the nitrate site in the \ac{NC} monomer, however no stable complex could be isolated. It is anticipated that a much larger network of waters around both the regions surrounding the nitrate moiety, and the wider molecule, are be required to achieve a stable water coordination structure. This would be of interest for further investigation into the mechanism of neutral hydrolysis involving the autoionsation of water \cite{DaSilva2013}. %The optimisation of \ce{H3O} in coordination with the fully nitrated monomer was also tested, but was only possible up to the level HF/6-31g.  
%Dragrams of the water coord geometrties you tried, pls?
%to see whether it was possible to stabilise in coordination.  
%Thus, omitting entropy effects. 
 %also include the function of hydrolysis in the degradation of sugars, in the lit review
Evaluation of the energy of protonation at each site found that the bridging and capping
oxygens were the most likely sites; with only the bridging isomer likely to contribute to denitration. However, protonation at the terminal structures will also be explored as the starting point for the subsequent denitration stage.  \\
%Big table of all the scans I did (for hydrolysis TS) \\
%Columns:  \\
%Scanned co-ordinate. Distance scanned. Observation. (TS found? etc)\\
%add in B3LYP too, if you can be bothered
%
%SMD energies look ridiculous
%\begin{table}[htp]
%\begin{center}
%\caption[Free energies of protonation using \acs{PCM} and \acs{SMD} implicit solvents.]{Free energies of protonation on a monomer of \ac{NC} nitrated at the C2 site, under different implicit solvent conditions.}
%\begin{tabular}{ l *{3}{S[table-format = 2.4]}} 
%\toprule
%\multirow{2}{*}{Protonation site} & \multicolumn{2}{c}{$\Delta$G\textsubscript{r} /kcal mol\textsuperscript{-1}} %& \multicolumn{2}{c}{$\Delta$H\textsubscript{r} }
%\\\cline{2-3}
%  & PCM & SMD\\
%\midrule
% Bridging 				& 2.83903 \\
% Terminal (Upper)	& 12.7897 \\
% Terminal (Lower)	& 10.7196	\\
% Capping 				& 0.90806	 \\ 
%\bottomrule
%\end{tabular}
%\label{tab:smd}
%\end{center}
%\end{table}
%

\subsubsection{Denitration by hydrolysis}
Following the protonation step, possible transition states for the removal of the nitrate were investigated. Direct dissociation of \ce{NO2} from the protonated species was explored (figure \ref{fig:no2-leave})), along with the simultaneous approach of a proton and cleavage of the \ce{NO2}(figure \ref{fig:h-come-no2-leave})). %(figure \ref{fig:scan_coords}). 
The scan of the proton moving towards the bridging site was also conducted, to determine whether any elongation of the \ce{O-NO2} occurred as a result of the formation of the proton-oxygen bond (figure \ref{fig:h-approach})). 
 
%was also completed to gain insight to the energy profile of the process. %any scans to show? 
\begin{figure}[ht]
\centering
\begin{subfigure}[b]{0.25\linewidth}
\centering
\caption{}
\includegraphics[width=\linewidth]{NO2_leave-prot_arrow}
\label{fig:no2-leave}
\end{subfigure}
\hfill
\begin{subfigure}[b]{0.25\linewidth}
\centering
\caption{}
\includegraphics[width=\linewidth]{NO2_leave-prot_arrow_2d}
\label{fig:h-come-no2-leave}
\end{subfigure}
\hfill
\begin{subfigure}[b]{0.25\linewidth}
\centering
\caption{}
\includegraphics[width=\linewidth]{H_bridge-approach_arrow}
\label{fig:h-approach}
\end{subfigure}
\caption{The scanned coordinates of %\ref{fig:h-approach}
a) proton approach, %\ref{fig:no2-leave}
b) dissociation of \ce{NO2} and %\ref{fig:h-come-no2-leave}
c) concerted protonation and \ce{NO2} dissociation.}
\label{fig:scan_coords}
\end{figure}

The relaxed \ac{PES} scan of \ce{NO2} removal from ethyl nitrate protonated at the bridging site was used as a preliminary test for the mechanism of denitration, following protonation (figure \ref{fig:PES_EN_scan}). 
Unrestricted \ac{wb97xd} was used in case of the formation of  \ce{^{.}NO2} instead of the expected \ce{NO2+}, with 20 steps of 0.1 $\text{\AA}$. However, bond dissociation was not illustrated in the energy profile even when extending the scan distance to a maximum of 6.4 $\text{\AA}$. Instead, a steady increase in the energy was observed. %fig?
It was observed that as the nitro group distance increased, its internal angle increased to 180$\degree$, confirming the formation of \ce{NO2+}. It is anticipated that the departing \ce{NO2+} will further react with water in the system to produce acids conducive to further hydrolysis. 
As the scan proceeded, the molecule rotated and the \ce{NO2+} leaving group aligned with the hydroxyl in an orientation suitable for peroxy group formation (figure \ref{fig:EN_pero}).  %This was the expected outcome for hydrolysis, as 
This mechanism was previously considered in the degradation reactions of \ac{PETN}; %section ref
it was found that the energy of this process was higher than that of \ac{HNO2} elimination, where the nitrate was not initially protonated. The formation of the peroxy bond was not facilitated by the scan parameters due to forced increase of the \ce{O-N} distance. The same process was not observed when the scan was applied to the \ac{NC} monomer. As the formation of the peroxy geometry required re-orientation of the whole molecule, the energy of this rearrangement was not favourable for the bulky \ac{NC} unit. In the real polymeric system, it may induce distortion of the sugar ring, proving even more energetically demanding.  %This was also done n both B3LYP and wb97xd. (MP2 too, but lets leave that)

%2D scan + pics
%H approach scans

%did I apply this to my monomer?
%I didnt get a TS out of this, regardless....
%This may be due to the specification of the spin  /charge?
\begin{figure}[hb]
\centering
\begin{subfigure}[b]{0.3\linewidth}
\centering
\includegraphics[width=\linewidth]{EN_NO2_leave_prot_step1-BL}
\caption{}
\end{subfigure}
\hspace{2em}
\begin{subfigure}[b]{0.3\linewidth}
\centering
\includegraphics[width=\linewidth]{EN_NO2_leave_prot_step7-BL}
\caption{}
\end{subfigure} \\
\begin{subfigure}[b]{0.3\linewidth}
\centering
\includegraphics[width=\linewidth]{EN_NO2_leave_prot_step11-BL}
\caption{}
\end{subfigure}
\hspace{2em}
\begin{subfigure}[b]{0.32\linewidth}
\centering
\includegraphics[width=\linewidth]{EN_NO2_leave_prot_step_cont5-BL}
\caption{}
\label{fig:EN_pero}
\end{subfigure}
\caption{Geometries from steps 1, 7, 11 and 26 of the geometry scan of Ox protonated \ac{EN} denitration.}
\label{fig:PES_EN_scan}
\end{figure}
%%%% POSSIBLY EXPLORE PEROXY deg routes in the next chapter, OR whether the H is snatched back for HNO2?
%
% Actually, you only allowed internal angles and one orienation one to relax, in the case of the monomer - so the below is bogus - you weren't able to gauge anything from these very restricted scans, because the no2 wasnt' allowed to fully turn to planar. 
%Similar scans were also performed on the \ac{NC} monomer protonated at the bridging site, where the \ce{O-NO2} bond was elongated in order to determine nature of the lost \ce{NO2}. Initially, the whole monomer was held rigid, as only the \ce{O-NO2} was incrimentally increased to produce an energy profile of the bond separation. Subsequently, the internation angle within \ce{NO2} and those pertaining to it orientation with respect to the wider monomer structure were allowed to relax. Despite increasing the scanning distance to 4 $\text{\AA}$. , with only the nitrate and proton, with internal angles allowed to relax.  The same was repeated in implicit solvent. 
% leaving group is lost as \ce{NO2}, and the nature of the monomer following denitration. 
%probe the generated denitration products.
% to simulate the removal of \ce{NO2}. 
Proposed 4-membered ring and 6-membered ring \ac{TS} were also investigated in order to determine whether they formed energetically and geometrically reasonable structures facilitating nitrate removal, with reformation of the alcohol group on the sugar ring (figure \ref{fig:all_da_TS}).  
%
\begin{figure}[ht]
\centering
\begin{subfigure}[t]{0.25\linewidth}
\caption{}
\centering
%The bonds in the 4 mem ring TS are longer than those in the 6-rings. Sort this if you have time. 
\includegraphics[width=\linewidth]{4mem-terminal}
%\caption{4-membered ring transition state with protonation at the terminal nitrate oxygen.}
\end{subfigure}
\hspace{3em}
\begin{subfigure}[t]{0.25\linewidth}
\caption{}
\centering
\includegraphics[width=\linewidth]{4mem-bridge}
%\caption{Protonation at the bridging nitrate oxygen site.}
\label{fig:4mem_b}
\end{subfigure}\\
%
\begin{subfigure}[t]{0.2\linewidth}
\caption{}
\centering
\includegraphics[width=\linewidth]{6mem-terminal}
%\caption{Protonation at the terminal nitrate oxygen.}
\end{subfigure}
\hspace{3em}
\begin{subfigure}[t]{0.2\linewidth}
\caption{}
\centering
\includegraphics[width=\linewidth]{6mem-bridge}
%\caption{Protonation at the bridging nitrate oxygen site.}
\end{subfigure}\\
%
\begin{subfigure}[t]{0.2\linewidth}
\caption{}
\centering
\includegraphics[width=\linewidth]{6mem-neutral}
%\caption{Neutral hydrolysis conditions with not prior protonation.}
\end{subfigure}
\hspace{3em}
\begin{subfigure}[t]{0.2\linewidth}
\caption{}
\centering
\includegraphics[width=\linewidth]{6mem-h3o}
%\caption{Concerted protonation-denitration, under acid hydrolysis conditions.}
\end{subfigure}
\caption[Proposed 4-member and 6-member ring transition states for the denitration of a nitrate ester.]{Proposed 4-member and 6-member ring transition states for the denitration of a nitrate ester, under various hydrolytic conditions. R = \ce{CH3} in the case of methyl nitrate, R = \ce{CH2CH3} in the case of ethyl nitrate and R = \ce{(H3CO)2C6H9O3} for the monomer.}
\label{fig:all_da_TS}
\end{figure}
%
Optimisations were attempted with both full geometry relaxation, and various frozen coordiate schemes for each proposed \ac{TS}. R groups were %simplified 
truncated even further to methyl nitrate ( R = \ce{CH3}), in effort to limit degrees of freedom during challenging optimisation of the \ac{TS} structures, however no fully relaxed structures were able to achieve convergence. Fixing of the bulk molecule with relaxation only around the nitrate and coordinating species, or relaxation of the wider molecule with fixed coordinates around the nitrate allowed sequential optimisation of different moieties, increasing chances of global energy minimisation.  % with expansion to ethyl nitrate and the \ca{NC} monomer, following 
It was possible to optimise the 4-membered ring bridging \ac{TS} on the \ac{NC} monomer with frozen \ac{TS} ring geometry \textit{via} preliminary optimisation of the ring structure with methyl nitrate. The optimised ring geometry was then placed on the monomer, with fixing of thee coordinates, allowing the remainder of the molecule to relax. 
% and relaxation of the remaining molecule (figure \ref{fig:4mem_b}). %actually we opt the ring with methyl nitrate first then reattached it to ring
A rigid scan was then performed of the 4-membered ring transition starting from the bridging site protonated monomer. It was revealed that as the nitrate moved away from the system, the proton moved to the capping group site rather than remain on the bridging oxygen as a hydroxyl, %(even when solvated)
as was initially expected. Instead, a ketone group was formed between the bridging oxygen and the ring. At subsequent steps, the ketone group caused the C2 - C3 bond to elongate and break. The scan eventually showed the \ce{NO2} leaving group reclaiming the proton from the capping group oxygen, leading to ring fission. The activation and kinetic barrier involved in ring fission is much higher than that of denitration; a study on the acid hydrolysis of glucose and xylose demonstrated that ring-opening intermediates were either extremely short lived, or not observed at all \cite{Qian2005}. 
The open-chain product of the scan is likely due to the geometric constraints placed on the geometry of the departing \ce{NO2} group, rather than a likely physical %energetic 
process. However, it sheds light on the scheme by which ring fission may occur under conditions of elevated temperature or pressure, which has been implied in previous work involving the formation of a ketone at earlier stages of the reaction. %REf that oldschool paper with the diagrams
%include distances as before, later
%shorten the captions. they silly. Put them in the body of text / big caption at the bottom
\begin{figure}[ht]
\centering
\begin{subfigure}[b]{0.5\linewidth}
\centering
\includegraphics[width=\linewidth]{C2-OH-opt-ring-freeze-NOscan_2_step13}
\caption{In the initial stages of increasing \ce{O-NO2} distace, the proton moves to the capping group.} 
\end{subfigure}
%
\begin{subfigure}[b]{0.4\linewidth}
\centering
\includegraphics[width=\linewidth]{C2-OH-opt-ring-freeze-NOscan_2_step15}
\caption{At separation of over 3.3 $\text{\AA}$, the C2 \textendash C3 bond breaks leading to ring fission. The proton then moves back onto \ce{NO2} .}
\end{subfigure}
\caption{Relaxed scan of \ce{NO2} departure, starting with the 4-membered ring structure. }
\end{figure}
%
Attempts to isolate the other \ac{TS} structures were unsuccessful, even when simplifying the side chain to methyl nitrate and under implicit solvent conditions. %in the case it stabilised the charges on the strained structures. 
%Dualscan - compare different methods. This was the final nail in the head that we woudln't be able to isolate an actual TS

%It was found that none of the above structures were able to be optimised for the monomer and was attempted with a reduced methyl nitrate test system. 
%The transition states were also not obtainable for methyl nitrate, both in vacuum and solvent. 
%
%Things I did:\\
%4 membered ring\\
%\textendash	Scan using ethyl nitrate \\
%\textendash		Opt Ts using ethyl nitrate\\
%\textendash		considerations - sterics, and what energy barrier would be required to overcome the twist needed to obtain this state. Orbital overlaps?\\
%6 membered ring\\
%\textendash		considerations - sterics, and what energy barrier would be required to overcome the twist needed to obtain this state. Orbital overlaps?]\\
%\textendash		Would energetics allow you to skip the protonation step? Is it more favourable?\\
%
%C2 and water
%
%\textcolor{red}{Still to mention:\\
%- Compare the results from different methods - which were the best for describing the reactions?
%- Theoretical aspects to the above lines of arguement.
%}

\section{Summary}

%Homolysis is fastest, for the monomer too, with HNO2 coming in second due to slower rate / energetics. (Check if I did anything to actually find this out - may just have to compare energetics.)
Thermolytic cleavage of the nitrate was modelled \textit{via} homolysis and elimination of \ce{HNO2}. In the case of \ac{PETN} it was found that the reaction energies were lower than expected when comparing with literature values. For homolytic fission, this may be  %due to treatment of the modelled reaction products as separate molecules instead of a complex, in the case of homolytic fission, or 
due to the separate evaluation of the \ac{PETRIN} radical and \ce{^{.}NO2} energies, where they should have remained in complex following the reaction. %Contributions may also have arisen as a result of differing software compilations. 
The same process was repeated for the \ac{NC} monomer, singly nitrated at the C2 site. The energy of homolytic fission was in good agreement with the expected value based on the outcome of the \ac{PETN} product. 
\ac{PES} scans of homolysis confirmed the loss of \ce{^{.}NO2} for %both the case of \ac{PETN} and 
the \ac{NC} monomer. 

The elimination of \ce{HNO2} \textit{via} intramolecular \textalpha-H transfer was also explored. Compared with homolysis, the energy of reaction and activation energy values gave better agreement with literature in the case of \ac{PETN}. Calculated \ac{NC} values were also within anticipated ranges, based on the reaction for \ac{PETN}. \ac{PES} scans were unable to locate a \ac{TS} for the \ac{NC} monomer, however, a successful guess geometry was generated based on the analogous structure in the reaction for the \ac{PETN}. %In general, the energies of activation are higher for \ac{NC} than for \ac{PETN}, though this is reasonable, and neighbouring \ce{-ONO2} groups are said to destabilise ...
Enthalpies of reaction energies show that this processes was more exothermic in the case of \ac{NC}, than for \ac{PETN} and showed that the elimination of \ce{HNO2} was more thermodynamically favourable in \ac{NC}, though homolysis may occur more rapidly, as is the case for \ac{PETN}.

%Scans did confirm that \ce{^{.}NO2} left as a radical.
%Scans showed the energy profiles involved (again, was HNO2 ever seen to be formed? )
The protonation sites on the \ac{NC} monomer were probed for the most favourable position. It was found that in the gas phase, capping and bridging site protonation lead to the same protonated final structure. 
In the solvent phase the capping site was energetically preferred, seconded by the bridging site. %, though inspection of the optimised geometry showed that it was very close to that of protonation at the capping site. 
As protonation and subsequent reaction would more likely lead to chain scission in the case of capping protonation, this avenue was disregarded in further studies focussing on the acid hydrolysis pathway. 
Optimisation of [water - monomer] and [hydronium - monomer] complexes were attempted, in order to obtain information on the nature and orientation of the protonation complex. However, it was not possible to isolate any stable structures, implying that a larger stabilising network of waters is likely required.

Removal of \ce{NO2} from the protonated analogues of ethyl nitrate and the \ac{NC} monomer was scanned using a variety of rigid and relaxed \ac{PES} schemes. In the removal of \ce{NO2} from ethyl nitrate the released of \ce{NO2+} was indicated by the change of geometry around the nitrate from bent to linear, as the \ce{O-NO2} bond elongated. 
Rotation of the remaining ethanol and complexed \ce{NO2+} showed orientation suitable for formation of a peroxide. This rotation was not observed in the case of the monomer, however the leaving group still presented as \ce{NO2+}.
4 and 6 membered ring \ac{TS} were also tested for the denitration reaction. 
%https://www.ch.imperial.ac.uk/rzepa/blog/?p=10015 see here for some rings that are supposed to work...
Unexpectedly, it was found that none of the 6-membered ring structures could be isolated, regardless of truncation of the system to a protonated methyl nitrate model, or using the un-protonated monomer to simulate concerted protonation-denitration. 
%prior protonation using methyl nitrate geometries. 
In the case of the bridge-protonated \ac{NC} with formation of the 4-membered ring \ac{TS} at the C2 nitrate, it was possible to relax the \ac{NC} monomer structure around the ring so long as the ring geometry itself was frozen. As the leaving group moved further from the remainder of the molecule, the hydroxyl group located at C2 formed a ketone, losing the proton to the departing \ce{NO2+}, to form \ce{HNO2} in later stages of the scan. Eventually ring fission occurred, as the \ce{HNO2} move sufficient distance away, and the formation of the ketone forced the adjacent \ce{C-C} bond in the ring to stretch, and then break. It is known that the energy required for this process is much higher than that of denitration, so is unlikely to contribute to initial stages of ambient ageing.
%However, as the energy barrier associated with this are much higher than that of  - dont' expect it, should solely be denitration then peeling off first

% 
%AH rate was not able to be compared as a TS was not found.
%Protonation occurs on both terminal and bridging sites of the monomer, with location at the bridging site conducive to the removal of \ce{NO2+}.
%
%TS were not able to be isolated for the denitration step, even with coordination with water in different orientation and both 4 and 6 mem ring TS. 2D scans did reveal a possible TS but it did not lead to the desired denitration pathway. 
%See water clusters around NC by \cite{Gunko2014}.
%
%alternate story if given more time to correct geometies - do the scans and TS with the reactified structure, plus the additional of on the interaction with the capping site. 
%
%Determine whether 
%Overall, the formation of \ce{^{.}NO2} for the homolytic reaction, and \ce{NO2} for the elimination of ce{HNO2} and in the initial stages of acid hydrolysis reactions was confirmed. But we have no idea of any rates, etc
