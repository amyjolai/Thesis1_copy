\chapter{Mechanisms of denitration under ambient conditions}
\label{chapterlabel5}
\graphicspath{ {./R_chap_3_pics/} }

\section{Introduction}%%%%%%%%%%%%%%%%%%%%%%%%%%%%%%%
\label{chapter5:intro}
\subsection{Thermolytic reactions}

The first stage of thermolytic decomposition for nitrate esters is generally agreed to be homolytic fission of the O-N bond linking the nitrate to the alkyl chain, leading to the loss of \ce{^{.}NO2} (equation \ref{equ:homofiss}) \cite{Oxley2003,Foltz2009,Shepodd1997}. %many refs
Though nitrate homolysis is an endothermic reaction, the weak O-N bond has a typical dissociation enthalpy of 42 kcal mol\textsuperscript{-1} and is easily cleaved when exposed to elevated temperatures, UV light or impact. %REF UV \cite{Zaki2016}
Whilst the thermolytic degradation of energetic materials has been widely studied experimentally, the ambient, slow ageing mechanisms are less well documented. %REF
Low-temperature decomposition routes are influenced by many factors over a protracted lifetime, and in practical use, materials are usually subject to evolving environmental conditions. %Changes in pressure, humidity, stress and temperature cycles induce changes in the degradation patterns of energetic materials to a varying degree. %ref?
External changes in pressure, humidity, stress and temperature cycling introduce variation in the degradation patterns of energetic materials. %ref?
%The array of products produced as a result of slow ageing may therefore vary greatly from those obtained from thermolysis; a general observation is that the number of different species is far smaller. %REF
The presence of moisture has been observed to lower the activation energy and accelerate decomposition %of energetic materials 
\cite{Foltz2009}.  %PETN
Internal factors including impurities, residual solvent, and crystal growth within the bulk, also alter decomposition behaviour. 

\begin{scheme}[h]
\begin{equation}
\ce{R-ONO2 -> R-O^{.} + ^{.}NO2 }
\label{equ:homofiss}
\end{equation}
%redraw this as a scheme with mechanism
\begin{equation}
%\ce{\textcolor{red}{\ce{H}}}
\ce{R\textcolor{red}{\ce{H}}-ONO2 -> R=O + \ce{\textcolor{red}{\ce{H}}NO2 }}
\label{equ:hno2_elim}
\end{equation}
\end{scheme}
%
\begin{figure}[h]
%\begin{center}
\begin{tabular}{ p{0.5\textwidth} p{0.5\textwidth} }
\begin{enumerate}
\item \ce{^{.}NO2} loss
\item \ce{HNO2} loss
\item \ce{OONO} rearrangement
\item $\gamma$-attack
\item \ce{ONO2^{.}} loss
\item \ce{C-C} cleavage (\ce{CH2O + NO2})
\item \ce{C-C} cleavage \ce{(CO + HNO2)}
\end{enumerate}
&
%\centering
%\begin{minipage}{0.5\textwidth}
%\centering
\raisebox{-\totalheight}{\includegraphics[width=\linewidth]{PETN_orig}}
%\end{minipage} 
\end{tabular}\\
\begin{center}
%\centering
{\textcolor{red}{\CIRCLE} oxygen \quad \textcolor{pink}{\CIRCLE} nitrogen \quad  \textcolor{yellow}{\CIRCLE} carbon \quad \textcolor{cyan}{\CIRCLE}hydrogen}
\end{center}
\caption[Intramolecular thermolytic reactions in \ac{PETN}.]{Intramolecular thermolytic reactions in \ac{PETN}, from the work of Tsyshevsky \textit{et al.} \cite{Tsyshevsky2013}.}
\label{fig:PETN_react}
%\end{center}
\end{figure}
%
The degradation of nitrate esters at temperatures over 100$\degree$C is primarily \textit{via} thermolytic processes, whilst under 100$\degree$C, decomposition is largely thought to be the result of hydrolysis \cite{Moniruzzaman2014}. 
Tsyshevsky \textit{et al.} studied the intramolecular reactions leading to denitration in \acf{PETN} in both the vacuum and the bulk crystal \cite{Tsyshevsky2013} (figure \ref{fig:PETN_react}). 
%
Seven mechanisms for the removal of \ce{NO2} were explored. Corresponding to the labels in figure \ref{fig:PETN_react}:
% (1)  (2) (3) (4) (5) (6) 						 (7)
(1) homolytic cleavage of the \ce{O-NO2} bond, (2) the elimination of nitrous acid (\ce{HNO2}) which is usually considered a competing reaction to homolytic fission, (3) the nitro-peroxynitrite rearrangement (\ce{O-ONO}), (4) \textgamma-attack of the terminating nitrate oxygen atom and the bridging nitrate oxygen at their relative \textgamma-carbon sites, (5) the homolytic \ce{C-O} bond cleavage, (6) and (7) two variations of the homolytic \ce{C-C} bond cleavage. 

It was found that the two most significant decomposition reactions were homolysis of the nitrate ester \ce{O-NO2} bond (equation \ref{equ:homofiss}) and intramolecular elimination of \ce{HNO2} (equation \ref{equ:hno2_elim}).
Whilst elimination of \ce{HNO2} was found to be the most energetically favourable denitration pathway, homolytic fission dominated preliminary decomposition steps due to the lower activation barrier and faster rate of reaction. It was suggested that global decomposition processes were determined by the interplay between these two mechanisms. Initial homolysis facilitated wide-spread denitration, complemented by exothermic \ce{HNO2} elimination, promoting self-heating of the system and further bond dissociations. 
The presence of \ce{^{.}NO2} and \ce{HNO2} have previously been linked to the autocatalytic rates observed for later-stage decomposition of nitrate esters \cite{Rodger1963,Lindblom2002,Volltrauer1981}, though some studies solely attribute it to the presence of acids \cite{Edge1990,FernandezdelaOssa2011,Baker1952,Binke1999}. %CHECK THESE ARE FACT
Other studies also implicate the action of \ce{^{.}NO} and \ce{HNO3}, in addition to \ce{^{.}NO2} and \ce{HNO2} \cite{Zayed2012,Trache2018}. 
%avanced stage decomp? Rodger actualy assumes NO2 is confirmed for PETN
% The presence of acid has been linked to autocatalytic rates of degradation during later stages of \ac{NC} degradation\cite{Edge1990,FernandezdelaOssa2011,Baker1952,Binke1999}. 
Inspection of these products generated from initial processes, with observation of the species permeating through to later stages, will shed light on the most likely contributors to  autocatalysis. 
%However, from these initial processes it is not possible to determine which is the species resposible % (this topic is further discussed in the next chapter.) %\ref{chapterlabel6}.) %REF
\subsection{Acid hydrolysis reactions}

Spent acids remain in the \ac{NC} matrix following synthesis even with thorough washing procedures. %but how much acid? What concentrations? %REF
Additional acidic species are released via the subsequent reactions of \ce{^{.}NO2} following homolysis. %, Any residual sulfuric acid from the nitration process catalyses the protonation of \ce{HNO3}, thereby the formation of nitronium (\ce{NO2+}) and hydronium (\ce{H3O}). 
These acids proceed to react with other moieties in the system, such as unsubstituted alcohol side chains on the polysaccharide, or small molecules free in the bulk. 

When exploring the interaction of nitroglycol and nitroglycerin in acid solution, Camera proposed a protonation-denitration scheme (scheme \ref{sch:cameraschema}) whereby initial protonation at the nitrate is rapid (equation \ref{equ:cam_1}), but the subsequent release of the nitronium ion was slow and rate determining (equation \ref{equ:cam_2}).
\begin{scheme}[htp!]
\centering
%\textbf{Hydrolysis scheme for ethyl nitrate}
%\begin{block_indent}{1cm}
%\ce{CH3CH2ONO2 + H+ <=>[\text{fast}] CH3CH2ONO2H+} \\
%\ce{CH3CH2ONO2H+ <=>[\text{slow}] CH3CH2H + NO2+} \\
%\ce{NO2+ + 2H2O <=>[\text{fast}] HNO3 + H3O+} \\
%\end{block_indent}
\begin{equation}
\ce{CH3CH2ONO2 + H+ <=>[\text{fast}] CH3CH2ONO2H+}% \\
\label{equ:cam_1}
\end{equation}
%
\begin{equation}
\ce{CH3CH2ONO2H+ <=>[\text{slow}] CH3CH2OH + NO2+} %\\
\label{equ:cam_2}
\end{equation}
%%Using generic R groups instead
%\begin{equation}
%\ce{RONO2 + H+ <=>[\text{fast}] RONO2H+}% \\
%\label{equ:cam_1}
%\end{equation}
%\begin{equation}
%\ce{RONO2H+ <=>[\text{slow}] ROH + NO2+} %\\
%\label{equ:cam_2}
%\end{equation}
%\begin{equation}
%\ce{NO2+ + 2H2O <=>[\text{fast}] HNO3 + H3O+} \\
%\end{equation}
\caption[The relative rate of stepwise protonation and denitration of nitrate esters.]{The relative rate of stepwise protonation and denitration of nitrate esters, using ethyl nitrate as an example. From the work of Camera \textit{et al.} \cite{Camera1982}.}
\label{sch:cameraschema}
\end{scheme}

\ac{NC} in storage is kept wetted with solvents to prevent self-ignition. Material with 12.6\acs{pN} or lower must be stored in 25\% water by mass, or in a controlled mixture of solvents, stabilisers and plasticisers. The material is therefore always exposed to water, with the fast-exchange of protons expected at inter-facial surfaces. % Something about the surfaces exposed to water? Faster NO'2 release or something?
%Fast exchange is expected for groups exposed to water within the bulk, with the dissociation of \ce{NO2+} to generate the alcohol as the rate determining step. %ref back to lit review, about having to store it in water for stability.
%more info on why the second step may be slow / what we expect to see?
In the study of organonitrates and organosulfates %in the acid phase 
generated from isoprene in the aerosol phase, % as secondary organic aerosol, %\ac{SOA}, 
Hu \textit{et al.} found that primary and secondary nitrates were resilient to hydrolysis for pH $>$ 0, whilst tertiary nitrates underwent hydrolytic nucleophilic substitution easily, reacting with water to form alcohols \cite{Hu2011}. %and sulfate to form alcohols and organosulfates

\begin{scheme}[htp!]
%\begin{figure}
\centering
\includegraphics[width=0.8\linewidth]{tertiary_nitrate_Hu2011}
%\end{figure}
\caption[Hydrolysis of a tertiary nitrate, originally derived from the reaction of isoprene in the aerosol phase.]{Hydrolysis of a tertiary nitrate derived from the reaction of isoprene in the aerosol phase, from the work of Hu \textit{et al.} \cite{Hu2011}.}
\end{scheme}

In tertiary nitrates, the carbon is fully substituted with no attached hydrogens. This group is usually sterically hindered and stabilising to carbocations, condition on the electron donation ability of the substituents remaining after nitrate removal. 
If formation of a carbocation intermediate is involved in the hydrolysis mechanism, this may explain why the tertiary nitrates exhibited highly efficient denitration, even under neutral conditions.

Though no specific mechanistic detail is given, the action of a protonated transition state during hydrolysis is alluded to %by Hu \textit{et al.}, 
through the contrast between the rate of acid-catalysed and neutral hydrolysis reactions. Neutral hydrolysis of the tertiary nitrates occured rapidly, but hydrolysis only occurred for primary and secondary nitrates under strongly acid catalysing conditions, and at a much slower rate.   % Adjacent OH groups did not slow down the acid cat hydrolysis of nitrates, but they did for the sulfates. OH groups slowed down hydrolysis for the primary and secondary nitrates.
% in Neutral hydrolysis for tertiary nitrates, more OH meant slowdown in rate of hydrolysis. This primary and secondary nitrates did not experinece this. 
%It was found that t
Additionally, the presence of adjacent OH groups hampered the rate of hydrolysis for some aerosol dispersed organonitrates. In the neutral hydrolysis of tertiary nitrates, increasing the number of adjacent \ce{OH} groups lead to protracted hydrolysis lifetimes.
Interestingly, the retardation effect of adjacent \ce{OH} groups was not observed for the the acid catalysed cases. Hu proposed that this could be due to the interaction of \ce{OH} with the transition state of the neutral hydrolysis system, compared to the protonated transition state of the acid catalysed system, impeding the reaction only in the former case. 
%Pretty weak sentence
The cause of this effect is unclear, and without understanding of the %transition states 
mechanisms involved, it is difficult to explain. % The neutral case happens much faster than the acid case - could it be that by having an OH next door, it is imparting some ''acid-like'' features to the reaction of the neutral process? %Need a bit more explanation here

There is evidence that nitration and denitration of nitrate esters is also influenced by the presence of nitrate groups at neighbouring positions. Matveev \textit{et al.} demonstrated that for poly-nitroesters the rate of liquid-phase decomposition did not increase linearly with number of nitrate reaction centres. It was found to mainly dependend on individual structures (table \ref{tab:reactions}) \cite{Matveev2003}.
% it really looks like the more stabilising the substituted group is - with respect to being able to  sustain a 
\begin{table}[t]
\begin{center}
\caption[Comparison of rate constants of decomposition for various polynitrate esters at 140$\degree$C.]{Comparison of rate constants of decomposition for various polynitrate esters at 140$\degree$C.
$\Delta$T is the decomposition temperature range,  $E$ is the experimental activation barrier for decomposition, log$A$ is the pre-exponential factor, $T_{c}$ is the combustion temperature, $k$\textsubscript{expt} is the rate constant for decomposition. 
Collated from literature sources by Matveev \textit{et al.}\cite{Matveev2003}.
}
%\begin{tabular}{ l c {S[table-format = 2.1]} {S[table-format = 2.1]} {S[table-format = 2.1]}} 
\begin{tabular}{p{15em} c c c c} 
\toprule
\multirow{2}{10em}{Compound	}	&	 $\Delta T$	&  $E$							&  log$A$ 				& $k$\textsubscript{expt} \\ %$k\subscript{exp}
 										&	/ $\degree$C & / kcal mol\textsuperscript{-1}	&[s\textsuperscript{-1}]	&  / \SI{e-6}s\textsuperscript{-1} \\
\midrule
\ce{O2NOCH2CH2CH2ONO2}					& 72–140 	& 39.1 	& 14.9 	& 1.7 \\
\ce{O2NOCH2CH2CH2CH2ONO2}			& 100–140 	& 39.0 	& 14.7 	& 1.1 \\
\ce{O2NOCH(CH3)CH(CH3)ONO2}			& 72–140 	& 40.3	& 14.9 	& 5.0 \\
%\ce{O2NOCH2CH2(NNO2)CH2CH2ONO2}		& 80–140 	& 41.5	& 16.5 	& 3.5 \\
\\
\ce{O2NOCH2CH2OCH2CH2ONO2}			& 80–140 	& 42.0 	& 16.5 	& 1.9 \\
\ce{O2NOCH2CH(OH)(CH2ONO2)}			& 80–140 	& 42.4 	& 16.8 	& 2.3 \\
\\
\ce{O2NOCH2CH(ONO2)(CH3)}				& 72–140 	& 40.3	& 15.8 	& 3.0 \\ 
\ce{[(O2NOCH2)CH(ONO2)CH(ONO2)]2}		& 80–140 	& 38.0	& 15.9 	& 63.0 \\
	 (hexanitromannite)							&			&		&		&	\\
\bottomrule
\end{tabular}
\label{tab:reactions}
\end{center}
\end{table}
%
%It was suggested that 
The trend in reactivity could be partially explained by the inductive effect of the nitro groups \cite{Oxley2003}. %actually he talks about nitro (NO2), not nitrate...
The inductive effect arises when a difference in the electronegativity between atoms connected by a $\sigma$ bond leads to a polarisation, or permanent dipole, in the bond. Electron donating groups % are of lower electronegativity than  
increase the $\delta^{-}$ partial charge on neighbouring atoms through the release of electrons, whilst electron withdrawing groups pull electron density away, %from neighbouring atoms
generating a $\delta^{+}$ charge on connected atoms. 
%However, the 
%%$\pi$ donation by lone pairs on oxygen and nitrogen also plays a significant role in increasing electron density at adjacent atoms %, known as 
%through the resonance effect. 
%Nitrate ester moieties
The \ce{NO3} presents a stronger electron withdrawing effect than \ce{OH}, which is a donating group. %%The \ce{NO3} presents a stronger electron donating effect via $\pi$ donation than \ce{OH}. In the case of the saturated polysaccharide ring, the \ce{OH} does not exhibit any $\pi$ donation properties, and instead acts as a % though both groups are activating. %, in the context of the polysaccharide ring.  %ref?  
%%OH is actually electron withdrawing for aliphatic alkyles? ``In organic chemistry, an alkyl substituent is an alkane missing one hydrogen.'' %ref? - SO maybe the OH is deactivating here. 
%%It would therefore be expected that both increase the rate of hydrolysis for nearby leaving groups. %(electron donating via \pi donation) 
The presence of an adjacent nitrate appears to facilitate denitration, whereas the presence of hydroxyl groups hinders this process, for neutral hydrolytic schemes. 
The resonance effect, arising from $\pi$ donation by lone pairs on oxygen and nitrogen is negligible %does not come into play 
between substituents at different sites on the polysaccharide ring, as the ring is saturated. 
%This suggests that the proposed interaction of the hydroxyl group with the neutral transition state supersedes its resonance effect. 
%OH - tertiary, neutral hydro: slow down, - primary, secondary, acid cat hydro: no effect
%As a result, it is ambiguous whether any apparent rate increase due to the presence of adjacent nitrate groups arises as a result of the resonance effect of the nitrate, or whether it is solely due to the absence of a neighbouring hydroxyl. 

The investigation by Hu \textit{et al.} exclusively focused on nitrates generated from an isoprene precursor, upon dispersion as an aerosol. The nitrate groups present in \ac{NC} are either of primary (C6) or secondary (C2, C3) structure, indicating that ambient hydrolysis is unlikely according to this scheme. However, solvent effects are expected to differ for condensed-phase reactions and aerosol phases. A greater build-up of acid concentration can be achieved in a closed, condensed system, and the lifetime of an aerosol is relatively short-lived when considering the timescale of slow ageing processes in \ac{NC}. Thus, the work of Hu \textit{et al.} does not provide a direct comparison for the \ac{NC} polymer but highlights the possible contribution from both neutral and acid-catalysed hydrolysis routes, and effect of increasing levels of substitution on the wider structure.%Though ordinarily it would be expected that reactions in aerosol would exhibit a rate increase due to a larger reaction surface, this may skew/increase the rate of specific reactions that do not rely on the diffusion of solvent or other species that move/migrate more slowly through the material. %Where is this even coming from. Some sort of literature proof please. 


In this section, the possible mechanisms for nitrate removal from the \ac{NC} backbone are explored. The homolytic fission and \ce{HNO2} elimination thermolytic processes suggested by Tsyshevsky will be compared to the acid hydrolysis scheme. %The energies of reactions will be compared, with derivation of the reaction rate where it is possible to isolate a transition state. 
Though the relative rates of reaction are not compared, the extended timescales involved in ambient ageing imply that the dominating reactions correspond to those most thermodynamically favourable.

\section{Methodology}%%%%%%%%%%%%%%%%%%%%%%%%%%%%%%%
%Thermolytic reactions
The energies of homolytic fission and elimination of \ce{HNO2} were calculated for PETN, as a test system before extension to the \ac{NC} monomer. The reaction energies were calculated according to equations \ref{equ:homofiss} and \ref{equ:hno2_elim} to reproduce the work of Tsyshevsky \textit{et al.}; the published geometries of PETN and its derivatives were obtained from the authors. A single point energy and frequency calculation were performed on each of the species to determine the reaction energies; no geometry optimisation was performed on the given structures except for in the case the \ce{^{.}NO2} molecule, where the geometry was not supplied. A separate \ce{^{.}NO2} molecule was independently geometry optimised.

The intramolecular reactions of the \ac{NC} monomer were modelled according to scheme \ref{sch:intra_NC}. % \ref{fig:homofiss_NC} and \ref{fig:elim_NC}.
Rigid and relaxed \ac{PES} scans were attempted in order to locate transition states for both reactions for the \ac{NC} monomer. 
Where the scans were unable to identify a valid transition state geometry, guess transition state geometries were constructed and optimised. 

\begin{scheme}[h]
\centering
%\begin{figure} Change to C2 instead of C3 in the diagrams
  	\begin{subfigure}[b]{0.75\linewidth}
  	\centering
	\includegraphics[width=\linewidth]{homofiss}
	\caption{Removal of a nitrate group \textit{via} homolytic fission of \ac{NC}}
	\label{fig:homofiss_NC}
	\end{subfigure}
	\par\bigskip
	\begin{subfigure}[b]{0.75\linewidth}
	\centering
	\includegraphics[width=\linewidth]{elimhono}
	\caption{Removal of a nitrate group \textit{via} elimination of \ce{HNO2}. }
	\label{fig:elim_NC}
	\end{subfigure}
%\end{figure}
\caption{The proposed intramolecular reactions for the initial denitration step during \ac{NC} degradation.}	
\label{sch:intra_NC}
\end{scheme}

The possible protonation sites for the \ac{NC} monomer were explored by placing a proton at each of the different oxygen sites surrounding the nitrate group. The structures were then geometry optimised and energies of protonation were compared. \ce{H3O+} was modelled as the donating species; as \ac{NC} is usually stored wetted in water, the hydronium ion is the most likely source of protons. %ref precentage water i storage? Pop it in lit review
It is also possible that the proton is donated by other acidic species in the system, particularly \ce{HNO2} or \ce{HNO3}. This is more likely at later stages of degradation when a higher concentration of acid has been generated by secondary reactions. %REF 
% Explain the caveats of this. 
%pKA H3O+ –1.74, HNO3 –1.3, HNO2 3.3. All weak acids, but H3O+ is most acidic. 

\subsection{Computational details} %%%%%%%%%%%%%%%%%%%%%%%%
All geometry optimisations, thermochemistry calculations and \ac{PES} scans were performed in \ac{G09}. 
Geometry optimisation and thermal calculations were to the level of 6-31+G(2df,p). \ac{NC} monomer structures were optimised using \ac{wb97xd}, \ac{B3LYP} and \ac{MP2}. 
$\Delta G$ values were obtained by the difference between the thermally corrected free energies of products and reactants. 
Zero-point corrected energies \acs{ZPE} were determined by addition of individual \ac{ZPE} to the free energy:
\begin{equation}
\centering
\Delta G^{ZPE} = \sum(G_{products}+ZPE_{products}) - \sum(G_{reactants}+ZPE_{reactants})
\label{equ:ZPE}
\end{equation}
%IRC step ,maxpoints=256,maxcycle=256 , stepsize the default of  0.1 Bohr
\ac{PES} scans were performed to the level of \ac{wb97xd}/6-31+g(d), or using unrestricted \ac{wb97xd}, in the case of \ce{O-^{.}NO2} dissociation. 
Rigid scans were carried out by fixing bond lengths, angles and dihedral values as constants. 
Only the variable of interest was allowed to change. %evolved
This was with the exception of relaxation of other specified coordinates required for accommodation of the new geometry, following each step of the scan.  %(semi-rigid scans). 
For example, in the homolysis of the nitrate \ce{O-NO2} bond, as the \ce{NO2} group departed, the internal \ce{O\textendash N\textendash O} angle was also allowed to relax, in addition to the angle of the departing \ce{NO2} with respect to the remainder of the molecule.
In two-dimensional scans, two variables were scanned simultaneously. For the same reaction, the elongation of a the \ce{O-NO2} bond was scanned with simultaneous approach of a proton, to monitor the effect of protonation for the same reaction. %In this case, both the internal angle within  \ce{O-NO2} and the angle was allowed to relax. 
Relaxed scans were performed in Gaussian using the ``modredundant'' function, whereby the whole structure was geometry optimised after each step of the scan. 
Scans were performed with step size of 0.1 $\text{\AA}$. The number of steps varied with the property investigated, though the majority of the phenomena were observed within 20 steps (2 $\text{\AA}$).
Scans were attempted in vacuum, and for the protonation cases, implicit solvent using \ac{PCM} \cite{Miertus1981}.
%Following each successful scan, a low-level frequency calculation was performed on the obtained transition state. If singular imaginary vibration matching the key bond transformation for the reaction step persisted, then a transition state search was performed using this geometry.
%Where possible, the intermediate “product” geometry obtained from the successful scan was also optimised to B3LYP/6-311+G(d,p) for use in transition state searching using QST2 and QST3 methods.

%The protonation studies conducted in solvent were calculated in \ac{PCM} ...% and \ac{SMD} intrinsic solvent models \cite{Marenich2009,Miertus1981}. 

\section{Results and discussion}%%%%%%%%%%%%%%%%%%%%%%%%%%
\subsection{Thermolytic decomposition mechanisms}
The energies of homolytic fission and intramolecular elimination of \ce{HNO2} from a \ac{PETN} nitrate group are shown in table \ref{tab:intramolec}. The energy values published by Tsyshevsky \textit{et al.} are denoted in parenthesis.
\begin{table}[htp]
\begin{center}
%\par\bigskip
\caption[Calculated free energies of reaction for the intramolecular reactions of PETN and the \ac{NC} monomer.]{Calculated free energies of reaction ($\Delta G_{r}$), reaction enthalpies ($\Delta H_{r}$), activation barriers ($E_{a}$)  with zero-point correction ($^{ZPE}$) for the intramolecular reactions of PETN, and the \ac{NC} monomer. Values expressed in kCal mol$^{-1}$.}
% Remember, these have not been scaled. Yao2018 : B3LYP/6-31+G(d,p) level and are scaled by a factor of 0.98, for the ZPE (And only certain components are scaled? Drama)
%FIX THE FOOTNOTES
\begin{tabular}{ l *{5}{S[table-format = 2.1]}} 
%\begin{tabular}{ l l l l l l} 
\toprule
Reaction				& $\Delta G_{r}$	&$\Delta G_{r}^{ZPE}$ 	& $\Delta H_{r}$	&	${E_{a}}$	& ${{E_{a}}^{ZPE}}$ \\
\toprule
PETN					& \multicolumn{5}{l}{}\\
\midrule
% Do I double these up??
\ce{^{.}NO2} loss	& 	21.50946		 &16.55804 		&	35.61579	& 	21.50946  $^{b}$ 	&16.55804 	 \\
						& ${(41.2)}^{a}$	 &	${  (35.8)}$ 	&					&${(41.2)}$ 	& ${(35.8)}$ \\
%\ce{^{.}NO2} loss	& 	21.50946		 &16.55804 		&	35.61579	& 	\textendash 	& \textendash 	 \\
%						& ${(41.2)}^{a}$	 &	${  (35.8)}$ 	&					&	\textendash\  	& \textendash\  \\
\rule{0pt}{4ex}   
\ce{HNO2} loss		& -23.62626	 	 &	-26.212190		& -20.39247	& 41.2902		& 36.28071  \\
						& ${(-18.6)}$		 &						&					&	${(47.3)}$	& ${(42.7)}$ \\
%\rule{0pt}{4ex}   
\toprule
\ac{NC} monomer	& \multicolumn{5}{l}{} \\
\midrule
\ce{^{.}NO2} loss	& 23.25456 		& 18.68672 		& 36.26343 	& 23.25456 	&	18.68672 \\
\rule{0pt}{4ex}   
\ce{HNO2} loss		& -36.04986 		& -39.42322		&-22.85892 	& 40.69863 	& 37.32527  \\
\bottomrule
\end{tabular}
\label{tab:intramolec}
\end{center}
$^a$ values from the work of Tsyshevsky \textit{et al.} \cite{Tsyshevsky2013}.\\
$^b$ values for the activation energy and total energy of reaction are the same 
for bond dissociation via homolytic fission.
%when treating homolytic fission as a barrierless process. 
\end{table}
%
Despite using the author supplied geometries, same method and basis, it can be seen that the obtained PETN reaction energy in the case of homolytic fission (\ce{^{.}NO2} loss) varies greatly from the published value found by Tsyshevsky \textit{et al.}
It was expected that the supplied geometries were those used to generate the energy values quoted in their study. Inspection of the forces for the given structures showed that they were in fact not converged. 

As the same structures were used to calculate values listed in table \ref{tab:intramolec}, the unconverged geometries do not explain the large discrepancy between the published energies and values obtained here.
% , though a possible explanation may be that different geometries were used for the values presented in the study.  (maybe this is too obvious to say)
A contribution may arise from a different compilation of the \ac{G09} program, leading to fluctuations in the exact values obtained. These differences are amplified %when deriving
during the calculation of reaction energies, though these are not expected to account for the 20 kCal mol$^{-1}$ deficiency %discrepancy
in the homolysis reaction energy. 
A possible explanation is that the products of homolytic fission remain complexed following bond breaking, which was not discussed in the original publication nor explicit in the supplied geometries. %from the authors. 
In the calculations above, the \ce{^{.}NO2} leaving group was separately geometry optimised and its energy taken into consideration additively%reaction energy
; it was assumed the product and reactant species would move apart following fission. It may be that the \ce{^{.}NO2} was not further geometry optimised, but left in its complexed position. This is relevant in the scenario where the \ce{^{.}NO2} has not yet moved far enough from the remainder of the \ac{PETN} molecule to fully relax. 

The same reaction was  applied to \ac{NC} monomer
%This was tested against the \ac{NC} monomer 
%However these details were omitted from both the literature and the supplied geometries. (Does it matter though - the following paragraphs confirms the mechanism as proof of concept that it actually happens - that way) so I can get away with it.)   
 %Additional explanations may be that %and the obvious suggestion that perhaps alternative geometries were used ,....  in addition to a different compilation of Gaussian
%Despite the difference in the calculated $E_{a}$ values, both fall into the range of PETN experimental activation energies for decomposition. Actually no, mine is too low. 
%NOTE: kuklja kept the HNO2 complexed with the 
%
%Add in the zero point correction for enthalpy
%Add in the optimised PETN geometries (that I was given)
%Add in the PETN geomtries that I did myself (from scratch - which are slightly off).
% These energies don't seem to reflect the excel sheet... update
%
\begin{figure}[ht]
\centering
%\begin{subfigure}[0.3\linewidth]
%\includegraphics[width=\linewidth]{EN_NO2_leave_prot_arrow}
%\label{fig:EN_homol_scan}
%\end{subfigure}
%
%\begin{subfigure}[0.3\linewidth]
\includegraphics[width=0.3\linewidth]{homolysis_arrow}
%\includegraphics[width=\linewidth]{homolysis_arrow}
%\caption{ }
%\label{fig:NC_homol_scan}
%\end{subfigure}
%\caption{The \ce{O-NO2} bond was elongated during rigid and relaxed \ac{PES} scans simulating homolytic fission for \ref{fig:EN_homol_scan} ethyl nitrate and the \ref{fig:NC_homol_scan} \ac{NC} monomer.}
\caption{The \ce{O-NO2} bond was elongated during rigid and relaxed \ac{PES} scan to simulate homolytic fission for the \ac{NC} monomer.}
\label{fig:homol_scan}
\end{figure}
% freenitr-NO2-C2-scan
% floppy_NO2-C2-scan ***
%get an react coord pic if you can (not v interesting though)
% one scan was complete freeze except with the O-NO2
% the other one alloed relaxation aroud the NO2 and its orientation
%multireference character, vs DFT
% Check Yao2018, and read up on multireference characters. 

%Scan of the PES
The energy profile of homolytic fission was obtained via a %\ac{PES}
rigid geometry scan of \ce{^{.}NO2} leaving the \ac{NC} monomer (figure \ref{fig:homol_scan}). The internal angle of the departing \ce{^{.}NO2} was allowed to relax, in addition to the coordinates referencing its orientation relative to the rest of the molecule. As the scan progressed, the \ce{^{.}NO2} internal angle increased from 129.2$\degree$ to 134.6$\degree$ at a maximum separation of $\text{\AA}$ from the bridging oxygen (Ox). This corresponds to the literature value for the \ce{O-N-O} internal angle (134.3\degree) and confirms the formation of a \ce{^{.}NO2} radical. %Ref the bond angle later. Wherefrom???
%Nevertheless, it is known that the activation energy for bond dissociation is equivalent to the bond enthalpy. 
%This confirms the formation of the radical species via homolytic fission, and that at a separation of 4 $\text{\AA}$ the radical is fully formed and relaxed. 

\begin{figure}[b]
\centering
\includegraphics[width=0.65\linewidth]{ONOangle_HOMOFISS_scan}
\caption[The change in \ce{O-N-O} internal angle with homolytic fission.]{The relaxation of the \ce{O-N-O} internal angle as the \ce{NO2} group is pulled away from the \ac{NC} monomer during a geometry scan of homolytic fission.}
\lable{fig:homofiss_graph}
\end{figure}

The values obtained for \ce{HNO2} elimination of PETN match the results given by Tsyshevsky much more closely. The energies fall within 5 kCal mol$^{-1}$ and 6 kCal mol$^{-1}$ for the Gibbs free energy of reaction and activation barrier, respectively. This is within a reasonable margin of error for comparing with experimentally obtained values, though larger than expected for those derived using the same method, basis and structure.

% Nevertheless, it still falls inside the range of experimentally measured PETN degradation activation barriers. %REF and check?
%I think the below actually goes into the methodology
In the case of the \ac{NC} monomer, both rigid and relaxed scans failed to capture the \ac{TS} for cleavage of the nitrate group via interaction with the \textalpha-hydrogen. A guess transition state was constructed based on the \ac{TS} of the analogous reaction for PETN, and optimised to produce the structure of the correct imaginary vibration. 
%more details on what you found 
%dashed lined heavier, please
\begin{figure}[htp!]
\centering
\begin{subfigure}[t]{0.4\linewidth}
\centering
\caption{}
\includegraphics[width=\linewidth]{elim_hono_ts_PETN_dash}
\label{fig:elim_ts_PETN}
\end{subfigure}
\begin{subfigure}[t]{0.4\linewidth}
\centering
\caption{}
\includegraphics[width=\linewidth]{elim_hono_ts_NC_dash}
\label{fig:elim_ts_NC}
\end{subfigure}\\

{\textcolor{red}{\CIRCLE} oxygen \quad \textcolor{blue}{\CIRCLE} nitrogen \quad  \textcolor{gray}{\CIRCLE} carbon \quad $\bigcirc$ hydrogen}
\caption[\ac{TS} for the elimination of \ce{HNO2} in \ref{fig:elim_ts_PETN} PETN and \ref{fig:elim_ts_NC} \ce{NC}.] {\ac{TS} for the elimination of \ce{HNO2} by removal of the \textalpha hydrogen by the \ce{NO2} leaving group in \ref{fig:elim_ts_PETN} PETN and \ref{fig:elim_ts_NC} \ce{NC}. Orange dashed lines indicate bonds breaking and blue dashed lines indicate bonds forming.}
\label{fig:elim_ts}
\end{figure}
%comment on the significance of the energies and descibe a bit more of what you see geometrically and via the discovered energies

\subsection{Acid hydrolysis mechanism}
\subsubsection{Protonation site}
The protonated \ac{NC} monomer species are shown in figure \ref{fig:proton_site}. The bridging oxygen (Ox) linking the nitrate to the remainder of the molecule, the capping group oxygen, and the interchangeable terminal nitrate oxygen sites were protonated in order to compare their relative energies for determination the site most likely to stabilise the proton at thermal equilibrium. Protonation also occurs on other sites in the molecule, such as at unsubstituted hydroxyl oxygen sites, the capping group oxygen on C4 and O1 in the glucose ring. Though it is a possibility that protonation at further sites in the monomer would contribute to degradation, these processes would occur \textit{via} alternative mechanisms without the involvement of denitration. For the purposes of studying acid hydrolysis, only the sites peripheral to the nitrate leaving group were explored. %REf if there is any evidence - peeling off?

\begin{figure}[htp]
\centering
\begin{subfigure}[t]{0.3\linewidth}
\centering
\includegraphics[width=\linewidth]{H_bridging_b}
\caption{Bridging}
\label{fig:proton_site_bridge}
\end{subfigure}
\begin{subfigure}[t]{0.3\linewidth}
\centering
\includegraphics[width=\linewidth]{H_terminal_b}
\caption{Terminal}
\label{fig:proton_site_terminal}
\end{subfigure}
\begin{subfigure}[t]{0.3\linewidth}
\centering
\includegraphics[width=\linewidth]{H_cap_b}
\caption{Capping}
\label{fig:proton_site_cap}
\end{subfigure}
\caption{Protonation sites on the \ac{NC} monomer for hydrolysis of the nitrate at the C2 position.}
\label{fig:proton_site}
\end{figure}
%
\begin{table}[htp]
\begin{center}
\caption{Free energies of protonation at each of the oxygen sites of interest on \ac{CH3CH3} C2 monomer of \ac{NC}.}
\begin{tabular}{ l *{4}{S[table-format = 2.4]}} 
\toprule
\multirow{2}{*}{Protonation site} & \multicolumn{4}{c}{$\Delta$G\textsubscript{r} /kcal mol\textsuperscript{-1}} %& \multicolumn{2}{c}{$\Delta$H\textsubscript{r} }
\\\cline{2-5}
  & \acs{wb97xd} & PCM & \acs{B3LYP} & PCM\\
\midrule
%% Note, using unconverged energies here :( the B3LYP solvent bridging one
% Bridging 	&  -30.87756 	&  0.85365 & -31.98636 	& -0.24507 \\
% Terminal	& -23.12541 	& 9.99558 	& -24.06285 	& 10.97397 \\
% Capping 	& -30.43404 	& 0.85491 	& -31.98447 	& 0.64386 \\  
% 
 Bridging 	&  -26.04105 	& 4.02759    & -28.67067 	& 11.29338  \\
 Terminal	 (Upper)& -29.84814 	&   24.28650	& -31.18500 	&  15.32916 \\
 Terminal (Lower)	& -20.53548 &   10.44288	& -22.40595	& 11.66634  \\
 Capping 	& -29.84877	&   3.61683	& -31.18563 	&  -1.16235 \\  
\bottomrule
\end{tabular}
\label{tab:reactions}
\end{center}
\end{table}
% Repeat these diagrams with the MP2 and B3LYP too, with a separate analysis (table) on the bonds and angles, in order to compare why the MP2 energies are a bit out there. 
\begin{figure}[htp]
\centering
\begin{subfigure}[t]{0.3\linewidth}
\centering
\includegraphics[width=\linewidth]{H_bridging_full_cut}
\caption{Bridging}
\end{subfigure}
\begin{subfigure}[t]{0.3\linewidth}
\centering
\includegraphics[width=\linewidth]{H_terminal_full_cut}
\caption{Terminal}
\end{subfigure}
\begin{subfigure}[t]{0.3\linewidth}
\centering
\includegraphics[width=\linewidth]{H_cap_full_cut}
\caption{Capping}
\end{subfigure}
\caption[Optimised protonated \ac{NC} monomer structures.]{Optimised protonated \ac{NC} monomer structures, showing interaction between the proton on the bridging site with the capping group oxygen.}
\label{fig:proton_site_full}
\end{figure}
%
%The mechanism of protonation was not explored in depth here; 
It was assumed that protons in the system would be in fast exchange with the solvent. The process has been studied computationally by Jebber and Liu \textit{et al.} \cite{Jebber1996,Liu2010}. Any effects of tunneling were included within this assumption. It can be seen that the bridging and capping values are very similar using both \ac{wb97xd} and \ac{B3LYP}.  Inspection of the geometries reveal that the optimised bridging and capping structures are extremely similar (figures \ref{fig:proton_site_bridge} and \ref{fig:proton_site_cap}). The difference in energies between the gaseous and implicit solvent values can be explained as \ce{H3O+} is highly unstable in vacuum and prefers to lose the proton to exist as water, whereas when solvated, the positive charged is stabilised. Thus the energy gain from losing the proton is less pronounced when in solvent. 
\begin{figure}[htp]
\centering
\includegraphics[width=0.25\linewidth]{neutral_hydro_1}
\label{fig:neutral_prot}
\caption{The attempted geometry of a single water molecule in coordination with the \ac{NC} monomer.}
\end{figure}
Water as the protonating species was attempted, by optimisation of one, two and three water molecules in coordination with the nitrate site in the \ac{NC} monomer, however no stable complex could be isolated. It is anticipated that a much larger network of waters around both the region surrounding the nitrate and the wider molecule would be required to achieve a stable water coordination in order for further investigation into the nature of neutral hydrolysis (figure \ref{fig:neutral_prot}). %The optimisation of \ce{H3O} in coordination with the fully nitrated monomer was also tested, but was only possible up to the level HF/6-31g.  
%to see whether it was possible to stabilise in coordination.  
%Thus, omitting entropy effects. 
%Explain a bit about the study pls
 %also include the function of hydrolysis in the degradation of sugars, in the lit review
Evaluation of the energy of protonation at each site found that the bridging and capping sites most likely. However, all possible structures will be explored for the subsequent denitration stage.  \\
%Big table of all the scans I did (for hydrolysis TS) \\
%Columns:  \\
%Scanned co-ordinate. Distance scanned. Observation. (TS found? etc)\\

\subsection{Denitration by hydrolysis}
Following the protonation step, possible transition states for the removal of the nitrate were investigated. Direct dissociation of \ce{NO2} from the protonated species was explored, along with the simultaneous approach of a proton and cleavage of the \ce{NO2} (figure \ref{fig:scan_coords}). The scan of the proton moving towards the bridging site was also completed to gain insight to the energy profile of the process. 
\begin{figure}[htp]
\centering
\begin{subfigure}[b]{0.25\linewidth}
\centering
\includegraphics[width=\linewidth]{H_bridge-approach_arrow}
\label{fig:h-approach}
\end{subfigure}
\begin{subfigure}[b]{0.25\linewidth}
\centering
\includegraphics[width=\linewidth]{NO2_leave-prot_arrow}
\label{fig:no2-leave}
\end{subfigure}
\begin{subfigure}[b]{0.25\linewidth}
\centering
\includegraphics[width=\linewidth]{NO2_leave-prot_arrow_2d}
\label{fig:h-come-no2-leave}
\end{subfigure}
\caption{The scanned coordinates of %\ref{fig:h-approach}
a) proton approach, %\ref{fig:no2-leave}
b) dissociation of \ce{NO2} and %\ref{fig:h-come-no2-leave}
c) concerted protonation and \ce{NO2} dissociation.}
\label{fig:scan_coords}
\end{figure}
The relaxed \ac{PES} scan of \ce{NO2} removal from ethyl nitrate protonated at the bridging site was used as a preliminary test for the mechanism of denitration following protonation (figure \ref{fig:PES_EN_scan}). 
Unrestricted \ac{wb97xd} was used, with 20 steps of 0.1 $\text{\AA}$, however bond dissociation was not illustrated in the energy profile even when extending the scan distance to a maximum of 6.4 $\text{\AA}$. Instead, a steady increase in the energy was observed. It can be seen that as the nitrate departs, the whole molecule rotates and the \ce{NO2} leaving group aligns with the hydroxyl in an orientation suitable for formation of a peroxy group. The internal angle of the leaving group increases to 180$\degree$, confirming that \ce{NO2} leaves as \ce{NO2+}. This was the expected outcome for hydrolysis, as it is anticipated that the \ce{NO2+} will further react to produce acids conducive to further hydrolysis. 
%This may be due to the specification of the spin  /charge?
\begin{figure}[htp]
\centering
\begin{subfigure}[b]{0.3\linewidth}
\centering
\includegraphics[width=\linewidth]{EN_NO2_leave_prot_step1-BL}
\caption{}
\end{subfigure}
\hspace{2em}
\begin{subfigure}[b]{0.3\linewidth}
\centering
\includegraphics[width=\linewidth]{EN_NO2_leave_prot_step7-BL}
\caption{}
\end{subfigure} \\
\begin{subfigure}[b]{0.3\linewidth}
\centering
\includegraphics[width=\linewidth]{EN_NO2_leave_prot_step11-BL}
\caption{}
\end{subfigure}
\hspace{2em}
\begin{subfigure}[b]{0.32\linewidth}
\centering
\includegraphics[width=\linewidth]{EN_NO2_leave_prot_step_cont5-BL}
\caption{}
\end{subfigure}
\caption{Geometries from steps 1, 7, 11 and 26 of the \ac{PES} scan of \ac{EN}}
\label{fig:PES_EN_scan}
\end{figure}
%%%% POSSIBLY EXPLORE PEROXY deg routes in the next chapter, OR whether the H is snatched back for HNO2?
%
% Actually, you only allowed internal angles and one orienation one to relax, in the case of the monomer - so the below is bogus - you weren't able to gauge anything from these very restricted scans, because the no2 wasnt' allowed to fully turn to planar. 
%Similar scans were also performed on the \ac{NC} monomer protonated at the bridging site, where the \ce{O-NO2} bond was elongated in order to determine nature of the lost \ce{NO2}. Initially, the whole monomer was held rigid, as only the \ce{O-NO2} was incrimentally increased to produce an energy profile of the bond separation. Subsequently, the internation angle within \ce{NO2} and those pertaining to it orientation with respect to the wider monomer structure were allowed to relax. Despite increasing the scanning distance to 4 $\text{\AA}$. , with only the nitrate and proton, with internal angles allowed to relax.  The same was repeated in implicit solvent. 
% leaving group is lost as \ce{NO2}, and the nature of the monomer following denitration. 
%probe the generated denitration products.
% to simulate the removal of \ce{NO2}. 
Proposed 4-membered ring and 6-membered ring \ac{TS} were investigated in order to determine whether they energetically and geometrically reasonable structures.  
%
\begin{figure}[htp]
\centering
\begin{subfigure}[t]{0.25\linewidth}
\caption{}
\centering
%The bonds in the 4 mem ring TS are longer than those in the 6-rings. Sort this if you have time. 
\includegraphics[width=\linewidth]{4mem-terminal}
%\caption{4-membered ring transition state with protonation at the terminal nitrate oxygen.}
\end{subfigure}
\hspace{3em}
\begin{subfigure}[t]{0.25\linewidth}
\caption{}
\centering
\includegraphics[width=\linewidth]{4mem-bridge}
%\caption{Protonation at the bridging nitrate oxygen site.}
\label{fig:4mem_b}
\end{subfigure}\\
%
\begin{subfigure}[t]{0.2\linewidth}
\caption{}
\centering
\includegraphics[width=\linewidth]{6mem-terminal}
%\caption{Protonation at the terminal nitrate oxygen.}
\end{subfigure}
\hspace{3em}
\begin{subfigure}[t]{0.2\linewidth}
\caption{}
\centering
\includegraphics[width=\linewidth]{6mem-bridge}
%\caption{Protonation at the bridging nitrate oxygen site.}
\end{subfigure}\\
%
\begin{subfigure}[t]{0.2\linewidth}
\caption{}
\centering
\includegraphics[width=\linewidth]{6mem-neutral}
%\caption{Neutral hydrolysis conditions with not prior protonation.}
\end{subfigure}
\hspace{3em}
\begin{subfigure}[t]{0.2\linewidth}
\caption{}
\centering
\includegraphics[width=\linewidth]{6mem-h3o}
%\caption{Concerted protonation-denitration, under acid hydrolysis conditions.}
\end{subfigure}
\caption[Proposed 4-member and 6-member ring transition states for the denitration of a nitrate ester.]{Proposed 4-member and 6-member ring transition states for the denitration of a nitrate ester, under various hydrolytic conditions. R = \ce{CH3} in the case of methyl nitrate, R = \ce{CH2CH3} in the case of ethyl nitrate and R = \ce{(H3CO)2C6H9O3} for the monomer.}
\end{figure}
%
Optimisations were attempted with both full geometry relaxation, and various frozen coordiate schemes for each proposed \ac{TS}. The R groups were simplified to methyl nitrate ( R = \ce{CH3}) in effort to limit degrees of freedom during optimisation of the \ac{TS} structures, however no fully relaxed structures were able to achiever convergence. Fixing of the bulk molecule with relaxation only around the nitrate and coordinating species, or relaxation of the wider molecule with fixed coordinates around the nitrate, allowed sequential optimisation of different moieties, increasing chances of global energy minimisation.  % with expansion to ethyl nitrate and the \ca{NC} monomer, following 
It was possible to optimise the 4-membered ring bridging \ac{TS} on the \ac{NC} monomer with frozen \ac{TS} ring geometry \textit{via} preliminary optimisation of the ring structure with methyl nitrate. The optimised ring geometry was then placed on the monomer, with fixing of thee coordinates, allowing the remainder of the molecule to relax. 
% and relaxation of the remaining molecule (figure \ref{fig:4mem_b}). %actually we opt the ring with methyl nitrate first then reattached it to ring
A rigid scan was then performed of the 4-membered ring transition starting from the bridging site protonated monomer. It was revealed that as the nitrate moved away from the system, the proton moved to the capping group site rather than remain on the bridging oxygen as a hydroxyl, as was expected. Instead, a ketone group was formed between the bridging oxygen and the ring. At subsequent steps, the ketone group causes the C2 - C3 bond to elongate and break. The scan eventually revealed the \ce{NO2} leaving group reclaiming the proton from the capping group oxygen, leading to ring fission. The activation and kinetic barrier involved in ring fission is much higher than that of denitration, %REF
so the product of the scan is likely due to the geometric constraints placed on the geometry of the departing \ce{NO2} group, rather than a physically energetic process. However, it sheds light on the scheme by which ring fission occurs, with has been implied in previous work involving the prior formation of a ketone. %REf that oldschool paper with the diagrams
%include distances as before, later
\begin{figure}[htp]
\centering
\begin{subfigure}[b]{0.5\linewidth}
\centering
\includegraphics[width=\linewidth]{C2-OH-opt-ring-freeze-NOscan_2_step13}
\caption{In the initial stages of increasing \ce{O-NO2} distace, the proton moves to the capping group.} 
\end{subfigure}
%
\begin{subfigure}[b]{0.4\linewidth}
\centering
\includegraphics[width=\linewidth]{C2-OH-opt-ring-freeze-NOscan_2_step15}
\caption{At separation of over 3.3 $\text{\AA}$, the C2 \textendash C3 bond breaks leading to ring fission. The proton then moves back onto \ce{NO2} .}
\end{subfigure}
\caption{Relaxed scan of \ce{NO2} departure, starting with the 4-membered ring structure. }
\end{figure}
%
Attempts to isolate the other \ac{TS} structures were unsuccessful, even when simplifying the side chain to methyl nitrate and applying implicit solvent in the case it stabilised the charges on the strained structures. 
%Dualscan - compare different methods. This was the final nail in the head that we woudln't be able to isolate an actual TS

%It was found that none of the above structures were able to be optimised for the monomer and was attempted with a reduced methyl nitrate test system. 
%The transition states were also not obtainable for methyl nitrate, both in vacuum and solvent. 
%
%Things I did:\\
%4 membered ring\\
%\textendash	Scan using ethyl nitrate \\
%\textendash		Opt Ts using ethyl nitrate\\
%\textendash		considerations - sterics, and what energy barrier would be required to overcome the twist needed to obtain this state. Orbital overlaps?\\
%6 membered ring\\
%\textendash		considerations - sterics, and what energy barrier would be required to overcome the twist needed to obtain this state. Orbital overlaps?]\\
%\textendash		Would energetics allow you to skip the protonation step? Is it more favourable?\\
%
%C2 and water
%
%\textcolor{red}{Still to mention:\\
%- Compare the results from different methods - which were the best for describing the reactions?
%- Theoretical aspects to the above lines of arguement.
%}

\section{Summary}

%Homolysis is fastest, for the monomer too, with HNO2 coming in second due to slower rate / energetics. (Check if I did anything to actually find this out - may just have to compare energetics.)
Thermolytic cleavage of the nitrate was modelled \textit{via} homolysis and elimination of \ce{HNO2}. In the case of \ac{PETN} it was found that the reaction energies were lower than expected when comparing with literature values. This may be due to differing geometries of the modelled reaction products, or due to the separate evaluation of the \ac{PETRIN} radical and \ce{^{.}NO2} engergies, where they should have remained in complex following the reaction. The same process was repeated for the \ac{NC} monomer, singly nitrated at the C2 site. The energy of homolytic fission was in good agreement with the expected value based on the outcome of the \ac{PETN} product. 
\ac{PES} scans of homolysis confirmed the loss of \ce{^{.}NO2} for both the case of \ac{PETN} and the \ac{NC} monomer. 

The elimination of \ce{HNO2} \textit{via} intramolecular \textalpha-H attack was also explored. Compared to the homolysis reaction, the energy of reaction and activation energy values gave better agreement to literature in the case of \ac{PETN}. Calculated \ac{NC} values were also within anticipated values, based on the reaction for \ac{PETN}. \ac{PES} scans were unable to location a \ac{TS} for the \ac{NC} monomer, however, a successful guess geometry was generated based on the analogous structure in the reaction for the \ac{PETN}. %In general, the energies of activation are higher for \ac{NC} than for \ac{PETN}, though this is reasonable, and neighbouring \ce{-ONO2} groups are said to destabilise ...
Enthalpies of reaction energies show that this processes was more exothermic in the case of \ac{NC}, then for \ac{PETN}.

%Scans did confirm that \ce{^{.}NO2} left as a radical.
%Scans showed the energy profiles involved (again, was HNO2 ever seen to be formed? )
The protonation sites on the \ac{NC} monomer were probed for the most favourable position. It was found that the bridging site was energetically preferred, though inspection of the optimised geometry showed that it was very close to that of protonation at the capping site. As protonation and subsequent reaction would more likely lead to chain scission in the case of capping protonation, this avenue was disregarded in further studies focussing on the acid hydrolysis pathway. Optimisation of [water - monomer] and [hydronium - monomer] complexes were attempted, in order to obtain information on the nature and orientation of the protonation complex. However, it was not possible to isolate any stable structures, implying that a larger stabilising network of waters is likely required.

Removal of \ce{NO2} from the protonated analogues of ethyl nitrate and the \ac{NC} monomer was scanned using a variety of rigid and relaxed \ac{PES} schemes. In the removal of \ce{NO2} from ethyl nitrate the released of \ce{NO2+} was indicated by the change of geometry around the nitrate from bent to linear, as the \ce{O-NO2} bond elongated. Rotation of the remaining ethanol and complexed \ce{NO2+} showed orientation suitable for formation of a peroxide. This rotation was not observed in the case of the monomer, however the leaving group still presented as \ce{NO2+}.
4 and 6 membered \ac{TS} were also tested for the denitration reaction. Unexpectedly, it was found that none of the 6-membered ring structures could be isolated, regardless of prior protonation of concerted protonation-denitration. In the case of the bridging-protonated \ac{NC} with formation of the 4-membered ring \ac{TS} at the C2 nitrate, it was possible to relax the \ac{NC} monomer structure around the ring so long as the ring geometry itself was frozen. As the leaving group moved further from the remainder of the molecule, the hydroxyl group located at C2 formed a ketone, losing the proton to the departing \ce{NO2+}, to form \ce{HNO2} in later stages of the scan. Eventually ring fission occurred, as the \ce{HNO2} move sufficient distance away, and the formation of the ketone forced the adjacent \ce{C-C} bond in the ring to stretch, and then break. %However, as the energy barrier associated with this are much higher than that of  - dont' expect it, should solely be denitration then peeling off first

% 
%AH rate was not able to be compared as a TS was not found.
%Protonation occurs on both terminal and bridging sites of the monomer, with location at the bridging site conducive to the removal of \ce{NO2+}.
%
%TS were not able to be isolated for the denitration step, even with coordination with water in different orientation and both 4 and 6 mem ring TS. 2D scans did reveal a possible TS but it did not lead to the desired denitration pathway. 
%See water clusters around NC by \cite{Gunko2014}.



%alternate story if given more time to correct geometies - do the scans and TS with the reactified structure, plus the additional of on the interaction with the capping site. 

%Determine whether 

