% In order to set an abbreviation: \ac{ }
% To force the shorthand version: \acs
% See here for more info and options: http://ramibaddour.com/2017/01/18/latex-working-with-acronyms/

%ISSUE: expansion of acronyms in the list of figures - not what we want, obvs, so find a way to not expand in the list of figs, OR go back and remove all teh acros in the figure captions.

\cleardoublepage\phantomsection\addcontentsline{toc}{chapter}{Abbreviations}
\label{abbreviations}
\abbreviations

\begin{acronym}[CAM-B3LYP] % Give the longest label here so that the list is nicely aligned

%Supress first
\acro{au}[a.u.]{atomic units}
\acro{B3LYP}{Becke, 3-parameter, Lee-Yang-Parr hybrid functional} %REF
\acro{BCP}{bonding critical point}
%not sure whether having the subscript loks better or not
\acro{CH3CH3}[\ce{CH3CH3}] {NC repeat unit with two methoxy capping groups}
\acro{CH3OH} [\ce{CH3OH}]{NC repeat unit with a methoxy capping group on ring 1, hydroxy capping on ring 2}
\acro{CCP}{cage critical point}
\acro{CP}{critical point}
\acro{DFT}{density functional theory}
\acro{DOS}{degree of substitution}
\acro{EM}{energetic materials}
\acro{ESP}{electrostatic potential}
\acro{G09}{Gaussian 09 revision D.01} %REF
\acro{GM} {genetically modified}
\acro{GView}{Gauss View 5.0.8} %REF
\acro{HF}{Hartree Fock theory}
\acro{IR}{infra-red spectroscopy}
\acro{MEP}{minimum energy path}
\acro{MM}{molecular mechanics}
\acro{MMFF94}{Merck molecular force field 94}
\acro{MW}{molecular weight}
\acro{NC}{nitrocellulose}
\acro{NCP}{nuclear critical point}
\acro{NG}{nitroglycerine}
\acro{NMR}{nuclear magnetic resonance spectroscopy}
\acro{OHCH3}[\ce{OHCH3}]{NC repeat unit with hydroxy capping group on ring 1, methoxy capping group on ring 2}
\acro{PCM}{polarisable continuum model}
\acro{PES}{potential energy surface}
\acro{PETN}{pentaerythritol tetranitrate}
\acro{QM}{quantum mechanics}
\acro{QTAIM}{quantum theory of atoms in molecules}
\acro{RCP}{ring critical point}
\acro{SEM}{scanning electron microscopy}
%
% Quick fix: Supress the first expansion of this by using acs{} for SN2 specific ones. 
\acro{sn2}[S\textsubscript{N}2]{bi-molecular nucleophilic substitution reaction}
\acro{TS}{transition state}
\acro{UFF}{universal force field}
\acro{UV}{ultraviolet}
\acro{UVVIS}[UV-Vis]{ultraviolet–visible spectroscopy}
\acro{wb97xd}[$\omega$B97X-D]{$\omega$B97X-D long-range corrected hybrid functional} %REF


\end{acronym}




