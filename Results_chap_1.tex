\chapter{Building the model}
\label{chapterlabel3}
\graphicspath{ {./R_chap_1_pics/} }

\section{Introduction}%%%%%%%%%%%%%%%%%%%%%%% 4 pages%%%%%%%%%%%%%%%
%Why can't we model the whole polymer / why do we need to truncate?
%What is contained in this chapter? What question is posed?

A wide range of reaction products are experimentally observed during the low temperature decomposition and slow ageing of \ac{NC}. Small gaseous species such as \ce{CO2}, \ce{CO}, \ce{CH2O}, \ce{N2O}, \ce{NO2} are idenitified using IR spectroscopy, whilst the presence of larger eluants such as glycolic acid and oxalate are alluded to in the spectra of the complex reaction mixture. \ce{jin2015,bluhm1977}
The generation of these products can be attributed to the range of susceptible attack sites along the polysaccharide backbone and at side groups, as well as the multitude of possible secondary reactions following denitration. These include depolymerisation and ring-cleavage, and can involve contaminants or acids remaining in the \ac{NC} matrix following synthesis, due to imperfect washing procedures\cite{chin2007,edge1990}. %(REF something about the deg products or many secondary reactions) \cite{gun'ko2014}

%[Move this bit to the literature review if appropriate:%%%%%%%%%%%%%%%%%%%%%%%]
Storage conditions such as the choice of wetting solvent, temperature, pressure, humidity, as well as fluctuations in these conditions over time, also contribute to the spectrum of products evolved. A practical application is the propellant stored under a pilot's seat for ejection from the cockpit during emergencies\cite{Defence2001,SBIR2013}. During the lifetime of the propellant, it will endure a great number of flights, each involving variable temperature and pressure cycles. The propellant device will include an \ac{EM}:plasticiser:stabiliser mixture that adheres to established industry safety standards. This will incorporate a conservative estimation of the shelf-life and stability of the formulation under different environmental and storage conditions (figure \ref{fig:cartridge}). 
Understanding of the ageing reactions will shed light on the possible interactions between the \ac{EM} and the stabilising mixture, informing better industrial practices, safety standards and product formulations with improved estimates of service life. 

\begin{figure}[htp]
\centering
\includegraphics[width=0.65\linewidth]{HBEA-P15-0021-crop}
\caption{Markings on the side of a charge bag, with a cartridge containing $\acs{NC}$ propellant. The shelf-life of the product can be calculated from its manufacture date, with consideration of storage conditions.}
\label{fig:cartridge}
\end{figure}

%[*End*]%%%%%%%%%%%%%%%%%%%%%%%]

\begin{figure}[htp]
\centering
\includegraphics[width=0.75\linewidth]{Cellulose}
\caption{The primary, secondary, tertiary and quaternary structure of cellulose fibres\cite{online textile academy2016}.}
\label{fig:fibre}
\end{figure}

In addition to environmental factors and contaminants, the degradation properties of \ac{NC} depend on the primary, secondary and tertiary polymer chain structure.
These properties are largely defined by the cellulose feedstock (figure \ref{fig:fibre}). \acs{SEM} images of \textit{Miscanthus} cellulose following nitration show a bloating of the cellulose fibres, with almost full retention of the fibre structure integrity (figure \ref{fig:miscanthus} ).
%More detail about the following in the literature review. 
Hydrogen bonding networks determine the supramoleular arrangement of the cellulose polysaccharide chains, and variation leads to a mixture of crystalline and amorphous regions\cite{Nishiyama2002,nishiyama2003}. 
Regions of crystallinity are more difficult to penetrate by solvent, thus are more resistant to hydrolytic decomposition methods; amorphous regions are more porous, exhibit less hydrogen bonding, and are more prone to digestion by both microbes and chemicals\cite{chundawat2011}. %REF
%also, any info on which types of cellulose have shorter/ longer chains, and are more / less suscpetible to certain types of degradation?

\begin{figure}[hbp]
\centering
\includegraphics[width=0.75\linewidth]{miscanthus}
%(x400 magnification) but thats not what it says in the actual image. 
%REF Nitrocellulose Synthesis from Miscanthus Cellulose, Gismatulina, 2018
\caption{\ac{SEM} images of \textbf{a)} Miscanthus cellulose and \textbf{b)} Miscanthus nitrocellulose, after nitration. From the work of Gismatulina \textit{et al.}\cite{Gismatulina2018}.}
%\begin{subfigure}{0.66\linewidth}
%    \includegraphics[width=\linewidth]{miscanthus}
%	\caption{ }
%  \end{subfigure}
%    \begin{subfigure}[b]{0.33\linewidth}
%    \includegraphics[width=\linewidth]{Miscanthus_sinensis}
%   	\caption{ }
%  \end{subfigure}
%    \caption{\ac{SEM} images of \textbf{a)} Miscanthus cellulose and \textbf{b)} Miscanthus nitrocellulose  from the work of Gismatulina \textit{et al.}. \textbf{c)} Miscanthus Sinensis plant.}
\label{fig:miscanthus}
\end{figure}

%It follows that a strains of \ac{NC} with a higher degree of polymerisation will require longer timescales to attain the same level of decomposition as a variety with lower degree of polymerisation. %$\textalpha \textbeta$ celluose types, if not in the lit review earlier - which ones are more easier digested?

%%%%%%%%%%%%%%%%%%%%%%%%%%%%%%%%%%%%%%%%%%%%%%%%%%%%%%%%%%%%%%%%%%%%%%%%%%%%%


%FIGURE of different fibre orientations and structures of cellulose / NC if I can find it, and if also not already done to death in the lit reviews

%talk about polymer ends, and changes in properties that can arise from changing fibre lengths (in cellulose, which are then translated to nitrocellulose)

%Quantum mechanical methods are routinely used to ...
Wavefunction and density functional methods are routinely used to explore the energetics of small molecular reactions\cite{rezac2011,Parthiban2001,Bento2008}. %REF one of the S66 benchmark reaction papers, and another one for a slightly bigger system - maybe mine. And perhaps include one study on reactions of cellulose? 
%include a one liners to give exmaples / demonstrate the use in literature. 
% Include a bit about why we are using DFT and QM for looking at reaction eneriges in the first place. What are other ways of doing it, and why are we doing it this way?
When probing the details of individual chemical reactions using computational methods, the extended polymer structure becomes unwieldy due to the large number of atoms. %In practice, \textit{ab initio} quantum mechanical and semi-empirical
%density functional methods are limited to between XX - XX atoms. %Respectively. %(REF max atoms for DFT and max atoms for wavefunction methods) {either refer to the theory section to talk about which methods are limited in which ways, but make mention of why you have you cap your model to a size that fits the methods available to probe the chemistry.)

%Ways that other people have chopped down big polymer models. 
% incl cellulose
\begin{figure}[htp]
\centering
%\includegraphics[width=\linewidth]{NC}
%\includegraphics[width=\linewidth]{Shukla_11224_2012_9977_Fig1_HTML}
\includegraphics[scale=0.4]{Shukla_11224_2012_9977_Fig1_HTML}
\caption{\textbf{a)} Numbering scheme of 2,3,6-trinitro-$\beta$-D-glucopyranose, as monomer of \ac{NC}. \textbf{b)} Transition state showing S\textsubscript{N}2 (opposite side) attack by OH\textsuperscript{-} and \textbf{c)} angular attack (same side) by OH\textsuperscript{-} during the addition–elimination process at the C2 site of \ac{NC} from the work of Shukla \textit{et al.}}
%REF Theoretical investigation of reaction mechanisms of alkaline hydrolysis of 2,3,6-trinitro-β-d-glucopyranose as a monomer of nitrocellulose
\label{fig:C2_C3_C6}
\end{figure}

Shukla \textit{et al.} truncated the polymer structure to a single-ring monomer for the purposes of investigating the alkaline hydrolysis behaviour of \ac{NC}\cite{Shukla2012}.
The study analysed the \acs{sn2} nucleophilic attack at the nitrate-carbon site, leading to release of the nitrate ion in favour of a hydroxyl group (figure \ref{fig:C2_C3_C6}). Findings suggested that denitration via alkaline hydrolysis starting from a fully nitrated monomer followed a sequence of C3$\rightarrow$C2$\rightarrow$C6. 
Later studies applying the same reaction to the dimer and trimer found that the sequence instead followed C3$\rightarrow$C6$\rightarrow$C2\cite{shukl2012a} for the larger systems. %The sequence of nitration and denitration is explored in \ref{chapterlabel4}
Comparisons between the monomer, dimer and trimer found the dimeric structure to be the minimum repeat unit encompassing all bonding and non-bonding interactions, and the smallest acceptable model to accurately demonstrate the denitration behaviour of the polymer. This can be attributed to the 1$\rightarrow$4 glycosidic linkage between each of the β-D-glucopyranose rings. The angle of the glycosidic oxygen leads to an alternating, non-planar orientation for each additional ring added (figure \ref{fig:models}). The unique interactions between the alternate rings are represented in the dimer and trimer, but lacking in the monomer. 

%The addition of a second ring generates further interactions between the rings within a dimer, than if the chain consisted of solely monomer repeat units. 
%for implementation in molecular dynamics simulations, this is also observed. 

%FIG of dimer, with interactions highlighted, vs monomer

%*A line segway-ing into the reaction coordinate of reactions, etc. May be easiest to talk about Shukla's work further (or Kuklja's?).*
%
%%what else did he/ others find  / use to make the polymer model workable?
%
%*Whilst Shukla demonstrated that the dimer model was the minimum "complete" unit; 
%when considering degrees of freedom in the \ac{PES} in the search for transition state maxima, %\ac{TS}
%%of reaction co-ordinates
%finite computational resources restrict the ... it may be that the dimer is not the most appropriate for further studies.*


In this chapter, the electronic properties of the monomer, dimer and trimer truncations of the \ac{NC} polymer are compared. The most suitable model for subsequent investigations is determined with consideration for accurate representation of the chemistry of the full system, alongside conservation of computational effort% during the exploratory phase of the study
. Hydroxyl  (-OH) and methoxy  (-OCH\textsubscript{3}) capping groups are tested at chain ends in the C1, C4 positions, to explore their interaction with the glucopyranose rings and effect on the distribution of charge on the chosen model.

%change to scale=0.4
\begin{figure}[ht]
  \begin{subfigure}[b]{\linewidth}
  \centering
	\caption{Monomer with \acs{DOS}=3. }
    \includegraphics[scale=0.5]{NC_mono_H}
%    \caption{Monomer}
  \end{subfigure}
  
  
  
    \begin{subfigure}[b]{\linewidth}
    \centering
   	\caption{Dimer with \acs{DOS}=3.}
    \includegraphics[scale=0.5]{NC_di_H}
%    \caption{Dimer}
  \end{subfigure}
  
  
  
    \begin{subfigure}[b]{\linewidth}
    \centering
	\caption{Trimer with \acs{DOS}=3.}
    \includegraphics[scale=0.5]{NC_tri_H}
%    \caption{Trimer}
  \end{subfigure} 
    \caption{Truncated \acs{NC} %to \textbf{a)}  monomer, \textbf{b)} dimer and \textbf{c)} trimer 
    units, whereby each additional glucpyranose ring is added in the 1$\rightarrow$4 position, as in the structure of cellulose.}
  \label{fig:models}
\end{figure}
 
%Comparisons between the monomer, dimer and trimer found %desingated
%the dimeric structure to be the minimum model required to accurately represent the chemical behaviour of NC in the alkaline hydrolysis pathway. 

 
 %Maybe for the lit review:
 %The monomer model exhibited the C3 $\rightarrow$ C2 $\rightarrow$ C6 denitration order, contrary to the dimer and trimer order of C3 $\rightarrow$ C6 $\rightarrow$ C2. 
%
%
% This bit can go in the lit review section %
%This is also observed when considering the minimum unit encompassing all bonding interactions necessary for parameterisation for a forcefield, for implementation in molecular dynamics simulations. 
%The fully nitrated dimer structure used in this study consists of two non-planar β-D-glucopyranose rings joined by a glycosidic bond, with six nitrate groups attached at the 2,3,6 positions on each ring (Figure 11

\section{Methodology}%HOW YOU DID EACH CALCULATION%An account of the procedures and techniques used in your research%%%
% ie. how were your results generated? 
%The fully nitrated dimer structure used in this study consists of two non-planar β-D-glucopyranose rings joined by a glycosidic bond, with six nitrate groups attached at the 2,3,6 positions on each ring (FIG XX). 

\subsection{Geomtry optimisation of \ac{NC} units}

%The monomer starting structure was constructed from the geometry used by Shukla \textit{et. al.} (REF). 
Monomer, dimer and trimer structures were drawn and capped with either methoxy or hydroxyl groups, then \acs{QM} geometry optimised. Side chains were drawn as closely matching literature orientations as possible. For dimer and trimer structures, individual glucopyranose rings were added to the first ring in the 1$\rightarrow$4 position, with alternating planarity (figure \ref{fig:models}).
In some instances, it was found that the final optimised geometries varied according to the input starting geometry. In the cases of more than one possible converged geometry for the same species, the conformer with lowest absolute energy was chosen. Select dimer and trimer molecules were pre-optimised with a \ac{MM} geometry optimisation using UFF or MMFF94 forcefields to determine whether any improvements could be achieved with respect to identification of the global minimum structure or reduction of \ac{QM} optimisation time. 
%To see whether they helped i finding the global minimum easier / sped up optimisation.
Following \ac{MM} pre-optimisation, the dimer cases did not show any noteable speed-up or improvement in convergence success during optimisation, or any significant difference in the final optimised geometry. Further investigations did not implement \ac{MM} pre-optimisation for dimer geometries. 
%also, conversion of file formats was timely and labour intensive, with some loss of bonding in formation that had to be restored later on - had to re-draw in nitrate bonds at times.
Convergence of trimer geometries posed more challenging, oftentimes with convergence failure. Pre-optimisation on the ni-methoxyl capped trimer structure was used to generate a reasonable starting input geometry for \ac{QM} optimisation.  This enabled convergence of the final structure, and reduced the number of steps required.
However, it was found that the lowest energy trimer conformer was derived from the fully optimised dimer structure with duplication of the first ring. Thus, rather than construction of a new trimer geometry ''from scratch'', the trimer structures used in this study were cosntructed from extension of the fully optimised dimer structures.

\subsection{Labelling system}

The numbering scheme for structures is detailed in figure \ref{fig:labelling}. Counting from the oxygen of the glucopyranose ring as atom 1 (O1), the numbering of carbons proceeds in a clockwise direction following the Cahn-Ingold-Prelog rules\cite{cahn1966}. For simplicity, the nitrate groups are numbered 1 - 3, also moving clockwise around the ring. The oxygen linking the nitrate to the polysaccharide backbone is labelled (Ox). Oxygen on the terminal ends of the nitrate group are labelled (Ot). Identical oxygen labelling is applied across all nitrate groups. Labels are denominated from the largest structure to the smallest: [Ring; Moiety; Attached atoms]. For example, a terminal oxygen (Ot) on the nitrate at the C6 position (N3) of the second ring (R2), would be referred to as R2N3Ot. When referring to only the nitrogen of this nitrate, the label would be R2N3. For the carbon at the C6 position on the second ring, the label would be R2C6.

\begin{figure}[t]
	\centering
	\includegraphics[width=0.75\linewidth]{3_3_rot_label_caption}
	\caption{The numbering scheme for dimers used in this study. Ring 1 (R1) is the first ring, whereby oxygen (O1) is \textit{one bond} separated from the glycosidic oxygen linking the two rings. Ring 2 (R2) is where O1 is \textit{two bonds} separated from the glycosidic oxygen. 
\newline
	The nitrate oxygen on the terminating ends are referred to as (Ot), and the nitrate oxygen connected to the glucopyranose ring is referred to as (Ox). }
  \label{fig:labelling}
\end{figure}

\subsection{System size and chain ends}
In order to investigate the influence of different capping groups, the fully nitrated dimer structure was used as a reference, following the minimum complete structure approximation established by Shukla \textit{et al.}. %suitable for accurate reproduction of wider system behaviour  (as per Shukla, above). %REF if necessary
Chain ends were capped with methoxy groups, or a combination of a methoxy and a hydroxyl group (figure \ref{fig:capping-groups-all}). The differences in charge distribution and the nature of intra-molecular interactions were probed using \ac{QTAIM} and analysis of the \ac{ESP} around the molecule.
When considering the system size, the \ac{ESP} and topology analyses were compared across monomer, dimer and trimer models. Nitration was limited to C2 sites, to explore interactions at lower levels of nitration and to simplifiy the optimisation of the trimer structure. The monomer was nitrated only at C2, the dimer nitrated only at R1C2 and the trimer nitrated at R1C2 and R3C2.
%
%With increasing nitration level, hydroxyl groups were substituted by nitrate groups at the C2, C3 and C6 positions in reverse order of denitration according to the study by Shukla \textit{et al.}\cite{shukla2012a},  C2$\rightarrow$C6$\rightarrow$C3. The original conformation was conserved; equatorial in the case of C2 and C3.
%%talk about C6 brnach orientation / angles?

%\subsubsection{\ac{ESP} mapping}

%\subsubsection{Laplacian of electron density}


%A bit of extra detail practical implementation of the partial charges, ESP and lapacian

%Structures underwent geometry optimisation in vacuum and implicit solvent, using various methods detailed in section XX.

%Reasons why I woudn't put it in a perdioci system - 
% I am only looking local chemical reactions at this stage. Though these reactions happen all along the structure, in theory, as I am trying to probe individual reactions that do not yet affect the wider chain, it woudl be a waste of computational effort to model the rest of the chain, when I could see probe the reaction using a smaller segement.

% The NC system isn't periodic in the same way that solid state materials are - the chains twist and bend. Whilst duplicating the same periodic box would allow me to expand the interactions to wider parts of the system, it would still not capture all the behvaiours - sterics, alternate configurations. 
% I would still just be replicating the chemistry that I see with the smaller model. 
% I'd need molecular dynamics for the wider description of the chain. And if I wanted to look at wider chemical interactions - I'd need ab intio MD, which is expensive.
% This would be useful for later probing the wider impact of the products of first stage degradation, to see how the products that are more likely to migrate further into the NC matrix will interact. But whilst I'm still looking at initial denitration, the smaller model will suffice. 

\subsection{Computational details} %TECHNICAL DETAILS OF YOUR SETUP%%%%%%
The \ac{B3LYP} density functional was chosen for initial exploration of electronic properties of the system. It is an efficient and extensively benchmarked method %for calculating electronic properties 
for main group elements and appropraite for the current model system size, where the largest trimer extends to 76 atoms. 
%\subsubsection{Geometry optimisation}
All electronic structure calculations in this section, including geometry optimisation and thermodynamic calculations were performed in vacuum to the level of \acs{B3LYP}\/ 6-311+G(d,p) with tight convergence criteria. Minima were located using the Berny optimisation algorithm\cite{Schlegel1982}, implimented in the \ac{G09} quantum chemistry suite\cite{frisch2013}.

\begin{table}[hbp]
\begin{center}
\caption{Convergence criteria used in \ac{G09}. Units in Hartree/Bohr.}
\begin{tabular}{ l c c c } 
\toprule
Convergence criteria & Normal & Tight \\
\hline
 Maximum Force &  0.000450 & 0.000015 \\
 RMS     Force & 0.000300 & 0.000010 \\
 Maximum Displacement & 0.001800 & 0.000060 \\
 RMS     Displacement & 0.001200 & 0.000040 \\
\bottomrule
\end{tabular}
\label{tab:convergence}
\end{center}
\end{table}
%\multirow{3}{4em}{Multiple row} & cell2 & cell3 \\ 
%& cell5 & cell6 \\ 
%& cell8 & cell9 \\ 
%\hline
%\end{tabular}
%\end{center}
%
%6-31+g(2df,p)/ $\omega$B97X-D [This is only for the subsequent chapters. Here I used the TWORUN calculations. May have to explain basis set choice later, since I used less ]  
%here I used b3lyp/6-311+g**. 
% More basis functions on the valence e, but less polarisation. WHAT SIGNIFICANCE will this have on my results/ system?
% ACtually, I think this is an ok thing at this stage, because I'm not really looking at longer range interaction - more intramolecular ones. So missing out on some of the diffuse  / polarisation isn't a big as deal as when with looking at bimolecular+ reactions. 
 
Structures were built using %z-matrix notation or 
the \ac{GView} graphical user interface. Molden 5.0.2 and \ac{GView} packages were used for visualisation. Avogadro molecular editor was used for \ac{MM} pre-optimisation. UFF or MMFF94 forcefields were applied with Steepest Descent algorithm\cite{cauchy1857,meza2010}, 500 steps and convergence of 10e\textsuperscript{-7}.
\acs{ESP} was mapped to the electron density, extracted from the \ac{G09} formatted checkpoint file following optimisation. \acs{ESP} maps were visualised using \ac{GView}, using the CubeGen utility. Density and \ac{ESP} cubes were generated with 80\textsuperscript{3} point grids. 
\ac{QTAIM} topological analyses, including generation of Laplacian electron density maps and \ac{CP} analysis on the optimised structures were performed using MultiWFN 3.6.  %REF (also update your diagrams).


%Partial charges were obtained via PyRed (R.E.D. Server version 3.0). 


\section{Truncating the polymer}% RESULTS & DISCUSSION%%%%%%%%%%%%%%%%%%%%%
% Results: what you did and what you got %%%%%%%%%%%%%%%%%%%%%%%%%%
% "A presentation of the data obtained from your research."
% Discussion: what does it mean, and so what? %%%%%%%%%%%%%%%%%%%%%
% "An explanation of the significance of your findings and how they relate
% to the work of other scholars.
% A review of your findings and their importance as well as suggestions
% for further research in your chosen area."

%\subsection{Choice of functional}
%Discuss the validity and caveats of each methods, including solvent vs vacuum.
%(This is possibly for the next section, as you haven't done anything on this.)


\subsection{Comparison of capping groups}
% Graphics I need:
	% Laplacian of select dimer structures, with different cappign groups
	% ESP of one dimer of one dimer of each nitration level
	% Partial charges of one dimer of each nitration level

% Want to look at the changes in partial charges, ESP, Laplacian and CPs, and see how they differ the methyl and H capping sites.
% Look at partial charges with and without capping included. 
% Generate some of the Laplacian maps looking at the capping ends, as well as the next nearest nitrate group.
% Analyse with respect to different levels of nitration
% Justify the choice of methyl.

%- Variation in partial charges, with different capping groups
%explain why you chose the dimers you did - it was based on Shukla's denitration sequence.

Dimers with \acs{DOS}=3 were bi-capped with methoxy (\ac{CH3CH3}), methoxy-hydroxy (\ac{CH3OH}) or hydroxy-methoxy (\ac{OHCH3}) groups at C1 and C4 positions, (figure \ref{fig:capping-groups-all}).

\begin{figure}[htp]
  \centering
  \begin{subfigure}[b]{0.5\linewidth}
	\caption{ }
    \includegraphics[width=\linewidth]{CH3_CH3_cap}
%    \caption{Fully nitrated NC dimer with methoxy capping groups in the C1 and C4 positions.}
  \end{subfigure}
    \begin{subfigure}[b]{0.5\linewidth}
   	\caption{ }
    \includegraphics[width=\linewidth]{CH3_OH_cap}
%    \caption{Fully nitrated NC dimer with a methoxy capping group in the C1 and hydroxyl group in the C4 position.}
  \end{subfigure}
      \begin{subfigure}[b]{0.5\linewidth}
   	\caption{ }
    \includegraphics[width=\linewidth]{OH_CH3_cap}
%    \caption{Fully nitrated NC dimer with a hydroxyl capping group in the C1 and methoxy group in the C4 position.}
  \end{subfigure}
    \caption{Fully nitrated \acs{NC} dimer with \textbf{a)} methoxy groups capping chain ends on both ring 1 and ring 2 (\acs{CH3CH3}), \textbf{b)} a methoxy capping group on ring 1 and hydroxyl on ring 2 (\acs{CH3OH}), \textbf{c)} hydroxyl group on ring 1 and methoxy capping group on ring 2(\acs{OHCH3}).}
  \label{fig:capping-groups-all}
\end{figure}

The electronic and steric properties of both capping groups and the effect on the whole dimer molecule was compared. The methoxy group was expected to provide a more chemically and sterically accurate proxy for the extended polysaccharide, though the hydroxyl capping group has demonstrated wider usage in literature. This may be due to its smaller atomic radius which minimises distortion to the wider structure of the model and limits the introduction of artificial steric effects. The hydrogen of the hydroxyl remains relatively chemically benign and it can be argued that it corresponds to polysaccharide chain endings. % resf? and at slightly lower computational cost. [bit of a weak sentence] non-reducing end?
The topology analysis of all dimers reveals notable intermolecular interactions across the two rings, in particular where the C6 side chain is able to interact with the nearest nitrate group on the opposite ring. All three structures exhibited similar characteristics; nuclear critial points (\acs{NCP}) lay at all nuclear sites and each chemical bond was described by a \ac{BCP}. Ring critical points (\acs{RCP}) lay at ring centres and in points of steric interference between side chain moieties. A single \ac{CCP} was observed in the region between R1C5 - R1C6 - R2N2 where the arrangement of the side chains formed a pseudo cage-like environment  (table \ref{tab:table_cp}). 

%appendix - list of all CPs?
\begin{figure}[htp]
  \centering
  \begin{subfigure}[b]{0.65\linewidth}
	\caption{Critical points idenitified for the \acs{CH3CH3} dimer.}
  \includegraphics[width=\linewidth]{CH3_CH3-dislin_crop}
  \end{subfigure}
  \begin{subfigure}[b]{0.7\linewidth}
    \caption{Critical points for the \acs{CH3OH} dimer.}
    \includegraphics[width=\linewidth]{CH3_OH-dislin_crop}
  \end{subfigure}
      \begin{subfigure}[b]{0.65\linewidth}
	  \caption{Critical points for \acs{OHCH3} dimer.}
    \includegraphics[width=\linewidth]{OH_CH3-dislin_crop}
  \end{subfigure}
    \caption{All critical points identified by \ac{QTAIM} topology analysis. \ac{NCP} are located at atomic nuclear sites, \ac{BCP} (orange points) lie on chemical bonds, or intramolecular bonding paths shown in orange; \ac{RCP} denote centres of steric interaction (yellow points) and the \ac{CCP}(green point) show the centre of a cage-like system.}
  \label{fig:cp-all}
\end{figure}

\begin{table}[htp]
\begin{center}
\caption{Topology analysis of dimers with different capping groups. * Only present in CH3CH3 and CH3OH; ** Only present in CH3CH3; *** Intramolecular steric centres located between R1 and R2, and between nitrate groups and nearest &\alpha&-hydrogen on the glucopyranose ring.}
\begin{tabular}{ l c c c l} 
\toprule
Dimer & CP type & Number & Interacting atoms & Intramolecular bonding\\
\hline
\multirow{4}{3em}{CH3CH3} & NCP & 63 & All atomic nuclei & \\ \cline{2-5}
& BCP &75 & \makecell{64 chemical bonds\\11 intramolecular bonds} &  \makecell[l]{\textbf{R1 -R1}\\R1C2H - R1N1Ot\\\textbf{R2 - R2}\\R2C3H - R2N2Ot\\R2C2H - R2N1Ot\\\textbf{R1 - R2}\\R1C3H - R1N2Ot\\R1C6H - R2N2Ot\\R1C5H - R2H2Ot\\R1O1 - R2N2*\\R1O1 - R2N2Ox\\R1C1H - R2N2Ox\\R2C6H - R1N1Ot\\R1N1Ot - R2N3Ot**}\\ \cline{2-5}
& RCP & 14 & \makecell[l]{R1, R2 ring centres, \\ Steric centres***} 
%R1C6 - R2N2, R1N1 - RC6, R1O1 - 2Ox and steric centres N1Ot - C2H, N2Ot - C3H}
%\textsubscript{n}Ot and nearest H on the ring***}
\\ \cline{2-5}
& CCP & 1 & R1C5 - R1C6 - R2N2 \\ 
\bottomrule
\multirow{4}{4em}{CH3OH} & NCP & 60 & All atomic nuclei \\ \cline{2-5}
& BCP & 71 &  \makecell{61 chemical bonds \\10 intramolecular bonds} & \makecell[l]{\textbf{R1 -R1}\\R1C2H - R1N1Ot\\\textbf{R2 - R2}\\R2C3H - R2N2Ot\\R2C2H - R2N1Ot\\\textbf{R1 - R2}\\R1C3H - R1N2Ot\\R1C6H - R2N2Ot\\R1C5H - R2H2Ot\\R1O1 - R2N2*\\R1O1 - R2N2Ox\\R1C1H - R2N2Ox\\R2C6H - R1N1Ot}\\\cline{2-5} 
& RCP & 14 & R1, R2 ring centres,***\\ \cline{2-5}
& CCP & 1 & R1C5 - R1C6 - R2N2 \\
\bottomrule
\multirow{4}{4em}{OHCH3} & NCP & 60 & All atomic nuclei \\ \cline{2-5}
& BCP & 70 & \makecell{61 chemical bonds \\9 intramolecular bonds} &  \makecell[l]{\textbf{R1 -R1}\\R1C2H - R1N1Ot\\\textbf{R2 - R2}\\R2C3H - R2N2Ot\\R2C2H - R2N1Ot\\\textbf{R1 - R2}\\R1C3H - R1N2Ot\\R1C6H - R2N2Ot\\R1C5H - R2H2Ot\\R1O1 - R2N2Ox\\R1C1H - R2N2Ox\\R2C6H - R1N1Ot}\\ \cline{2-5}
& RCP & 13 & R1, R2 ring centres,*** \\ \cline{2-5}
& CCP & 1 & R1C5 - R1C6 - R2N2 \\ 
\bottomrule
\end{tabular}
\label{tab:table_cp}
\end{center}
\end{table}

Hydrogen bonding was observed between within the nitrate groups directly connected to the ring and their associated $\alpha$-hydrogens (figure \ref{fig:alpha}). N1 and N2 nitrates are orientated in the eclipsed conformation relative to the  $\alpha$-hydrogen, bringing the terminal nitrate oxygen within H-bonding distance and generating a steric critical point between the oxygen and $\alpha$-hydrogen (table \ref{tab:Hconformation}). 
This was not the case for C6 nitrates, which are not directly connected to the ring. The flexbility of the C6 alkyl chain allows for relaxation of the nitrate to the gauche orientation, relative to the \ce{CH2} group. The increased distance between the terminal oxygen of the nitrate and the $\alpha$-hydrogens exceeded H-bonding distance and redued steric crowding.% in the region.

\begin{figure}[H]
\centering
\includegraphics[width=0.3\linewidth]{alpha_a}
\caption{Alpha hydrogen relative to the nitrate group, highlighted in red.}
\label{fig:alpha}
\end{figure}

\begin{table}[hb]
\begin{center}
\caption{Distances between terminal oxygen of nitrates and nearest $\alpha$-hydrogen.}
\begin{tabular}{ l l l l l } 
\toprule
Nitrate oxygen & Nearest $\alpha$-H & Distance \textbackslash \si{\angstrom} & Dihedral & Conformation\\
\hline
 \textbf{R1 - R1} & & & \\
 R1N1Ot & R1C2H & 2.1667 & -3.4599 & Eclipsed \\
 R1N2Ot & R1C3H & 2.1410 & 2.5662 & Eclipsed\\
 R1N3Ot & R1C6H & 2.5332 & 25.3615 & Gauche\\
 \textbf{R2 - R2} & & & \\
 R2N1Ot & R2C2H & 2.1788 & 3.3742 & Eclipsed\\
 R2N2Ot & R2C3H & 2.1729 & 9.3912 & Eclipsed\\
 R2N3Ot & R2C6H & 2.5185 & -24.9355 & Gauche \\
%
\bottomrule
\end{tabular}
\label{tab:Hconformation}
\end{center}
\end{table}
Inter-ring H-bonding occurred between C6 hydrogens with terminal nitrate oxygens of adjacent nitrate groups of the other ring, in addition to N - O interactions and O - O interactions. 
Upon replacement of a hydroxyl capping group for a bulkier methoxy, the geometry of the nitrate groups adjusted to accommodate. It can be seen in the case of \ac{CH3CH3} in figure \ref{fig:cp-all}.a), that there is an additional \ac{BCP} and intramolecular bonding path between R1N1Ot and R2N3Ot that was not present in the structures with only one methoxy group, and an R1O1 - R2N2 interaction that is not present in \ac{OHCH3}. 
Inspection of nitrate-nitrate distances show that with substition to methoxy capping groups, the inter-moeity distances did not change significantly. Despite a lower number of intramolecular bonding interactions, \ac{OHCH3} did not exhibit larger bonding distances between interaction centres. 
%laplacian on the R1C5 - N2 region. esp atoms H48, C5, N35, O6  to explain this? Also 20 and 41
%
%(or put his with QTAM)
%table of all bonds and angles in the appendicies, expansion on the one below? Optional
%possible changes: relative ring planarity, forced closer interaction between nearby nitrate groups, twisting of the ring, and intramolecular interactions
\begin{table}[hbp]
\begin{center}
\caption{Distances between nitrate groups with changing capping groups.}
\begin{tabular}{ l c c c c} 
\toprule
Capping group & Nitrate-nitrate pair &  Distance \textbackslash \si{\angstrom} \\
\hline
 \multirow{4}{4em}{CH3CH3} & R1N1 - R1N2 & 3.9871 \\
 & R1N1 - R2N3 & 4.8328 \\
 & R1N3 - R2N2 & 5.2424\\
 & R2N2 - R2N1 & 4.1421 \\
\hline
 \multirow{4}{4em}{CH3OH} & R1N1 - R1N2 & 3.9887 \\
 & R1N1 - R2N3 & 4.8327 \\
 & R1N3 - R2N2 & 5.2371\\
 & R2N2 - R2N1 & 4.1325 \\
 \hline
 \multirow{4}{4em}{OHCH3} & R1N1 - R1N2 & 3.9359 \\
 & R1N1 - R2N3 & 4.8208 \\
 & R1N3 - R2N2 & 5.2868\\
 & R2N2 - R2N1 & 4.1297 \\
\bottomrule
\end{tabular}
\label{tab:geomety_cap}
\end{center}
\end{table}

%wait for NC2_0 to be done.
%Here I want to comment on the interaction with diff cappign groups - why do I continue with CH3?
%Does this segway into something about the model size in the next section?
Topology analysis on dimers with lower levels of nitration show interaction between hydroxyl moeties with the capping group on R1 (figure \ref{fig:2_0}). NC dimers nitrated at only the R1C2 position exhibit H-bonding and H-H interaction between the capping group hydrogens and those of C6, or hydroxyl oxygens at C3 and C6, as in the case of CH3CH3 and CH3OH dimers. 
It is observed that the variation of interaction site is dependent on the geometry. Changes in the capping group partially influence the change of geometry around the interacting sites, though the site of nitration is most crucial.  

\begin{table}[hbp]
\begin{center}
\caption{Ring tortion angles between R1C2 nitrated dimers.}
\begin{tabular}{ l c c c c} 
\toprule
Dimer & Ring tortion angle \\
CH3CH3 & -177.1370 \\
CH3OH & -177.1120 \\
OHCH3 & -176.4909 \\
\bottomrule
\end{tabular}
\label{tab:tortion}
\end{center}
\end{table}

\begin{figure}[htp]
\centering
  \begin{subfigure}[b]{0.4\linewidth}
	\caption{}
  \includegraphics[width=\linewidth]{NC2_0-dislin_crop}
  \end{subfigure}
    \begin{subfigure}[b]{0.4\linewidth}
	\caption{}
  \includegraphics[width=\linewidth]{NC2_0_CH3_OH-dislin_Crop}
  \end{subfigure}
  \begin{subfigure}[b]{0.4\linewidth}
    \caption{}
    \includegraphics[width=\linewidth]{adj_NC2_0-OH_CH3_dislin_crop}
  \end{subfigure}
\caption{Critical points identified by \ac{QTAIM} topology analysis of NC dimer nitrated at C2 position for \textbf{a)} CH3CH3 dimer, \textbf{b)} CH3OH dimer and \textbf{c)} OHCH3 dimer.}
\label{fig:2_0}
\end{figure}


%split into two columns and have the CPS on the right hand side. 
\begin{figure}[htp!]
  \centering
	\includegraphics[width=0.8\linewidth]{scale_bar}
    \begin{subfigure}[b]{0.6\linewidth}
   		\caption{ }
	    \includegraphics[width=\linewidth]{CH3_CH3-NC3_3_ESP_surface_w-5-287e-2}
	\end{subfigure}
    \begin{subfigure}[b]{0.7\linewidth}
   		\caption{ }
	    \includegraphics[width=\linewidth]{CH3_OH_NC3_3_ESP_surface_w-5-287e-2}
    \end{subfigure}
    \begin{subfigure}[b]{0.65\linewidth}
	   	\caption{ }
	    \includegraphics[width=\linewidth]{OH_CH3_NC3_3_ESP_surface_w-5-287e-2}
  \end{subfigure}
    \caption{Electrostatic potential maps of \textbf{a)} \ac{CH3CH3}, \textbf{b)} \ac{CH3OH}, \textbf{c)} \ac{CH3OH}. *UNITS?*}
  \label{fig:ESP}
\end{figure}
%
%On comparing surfaces - you need to compare ESPs directly, rather than visually (this is last resort) - you need a table. These aren't the same as partial charges. Can I get them from the cube files? 

The \ac{ESP} maps % for the bi-capped methoxy dimer (\ac{CH3CH3}), methoxy-hydroxy capped dimer (\ac{CH3OH}) and hydroxy-methoxy capped dimer (\ac{OHCH3}) 
represent the charge density around the molecule (figure \ref{fig:ESP}). A more negative value (increasingly red) indicates an area of higher electron density, whilst a more positive value (increasingly blue) indicates an area of lower electron density. It can be seen that the areas with lowest electron density lie around the centre of the glucopyranose rings and at the C6 position of the side branches. This can be explained by the number of adjacent oxygens drawing the electron density away from the less electronegative alkyl groups. The most electron rich areas are the outermost oxygens of the nitrate groups (Ot positions).
The presence of the hydroxyl capping group promotes a small decrease in electron density around the adjacent ring, illustrated by the slight deepening of the blue shading around the nearest ring bonded to the hydroxyl. The hydroxy group itself presents the lowest concentration of electron density, suggesting that the terminating hydrogen would be suceptible to nucleophilic attack. 
The oxygen of both methoxy and hydroxy capping groups exhibit a lower concentration of electron density than that of the bridging oxygen between the two rings. This deficiency is particularly pronouced in the case of the hydroxyl group, suggesting that the methoxy group provides a more representative approximation for the extended polymer with respect to charge density. 
 %reinforced by partial charges?
%The trimer model's ESPs should reinforce that CH3 is a more chemically similar approximation - no concentration of charge 

%Shukla’s work identified the nitrate group attached to carbon three (C3) as the most susceptible to denitration and the first to be removed. This is supported by the distribution of partial charges in the molecule, with disregard of the capping groups. %Thus, the nitrate group on C3 was used as the target site for degradation studies.

\subsection{Model size}%%%%%%%%%%%%%%%%%%%%%%%%%%%%%%%%%%%%%%%%%%%%%%%%%%%
% Graphics I need:
	% Laplacian of C2 nitrated monomer, dimer and trimer structures
	% ESP of each
	% Partial charges of each

% Want to say that partial charges / ESP / Laplacian / CP's do not look so different, across each of the model systems, but that the dimer is the best model for capturing all the interactions based on the payoff between computational efficiency and accuracy of resutls. The interaction between the rings is missing for the monomer, and this is a crucial aspect of the chemistry. 

The electrostatic potential and topology analysis of a mononer, dimer and trimer were compared in order to explore the limit of each truncation and determine which model to proceed with further investigations. During the optimisation of the dimer and trimer geometries, location of the global minimum proved challenging due to the number of flexible bonds and high degrees of freedom. Rotation barriers were low for side chains, resulting in smooth gradients in the \ac{PES}. \ac{MM} pre-optimisation using UFF and MMFF94 did not improve the convergence frequency of optimisation calculations and were not used for subsequent calculations. Whilst the monomer lacks many essential intramolecular interactions present in the polymer, when exploring the chemistry of a localised group of atoms, such as in the case of the denitration reaction, it is a suitable model for a primary investigation into the mechanism. The mechanisms can then be applied to the dimer and wider systems to re-introduce ring-ring interactions and higher steric effects. 


%Repeat for solvent and vacuum, if there is time (there isn't)

\begin{figure}[htp!]
\centering
	\includegraphics[width=0.6\linewidth]{scale_bar} 
    \begin{subfigure}[b]{0.4\linewidth}
	   	\caption{ }
	    \includegraphics[width=\linewidth]{C2_ESP_surface_w}
    \end{subfigure}
    \begin{subfigure}[b]{0.45\linewidth}
	   	\caption{ }
	    \includegraphics[width=\linewidth]{C2_dislin_3_crop}
    \end{subfigure}
    \caption{\textbf{a)} Electrostatic potential map and \textbf{b)} Critical point analysis of NC monomer nitrated at C2 position.}
  \label{fig:monomer_top}
\end{figure}

\begin{figure}[htp!]
\centering
	\includegraphics[width=0.7\linewidth]{scale_bar}
    \begin{subfigure}[b]{0.6\linewidth}
	   	\caption{ }
	    \includegraphics[width=\linewidth]{NC2_0_ESP_surface_w}
    \end{subfigure}
    \begin{subfigure}[b]{0.6\linewidth}
	   	\caption{ }
	    \includegraphics[width=\linewidth]{NC2_0-dislin_crop}
    \end{subfigure}
    \caption{\textbf{a)} Electrostatic potential map and \textbf{b} Critical point analysis of NC dimer nitrated at R1C2 position }
  \label{fig:dimer_top}
\end{figure}

\begin{figure}[htp!]
\centering 
	\includegraphics[width=0.8\linewidth]{scale_bar} 
    \begin{subfigure}[b]{0.7\linewidth}
   		\caption{ }
	    \includegraphics[width=\linewidth]{NC2_0_2_ESP_surface_w}
    \end{subfigure}
    \begin{subfigure}[b]{0.75\linewidth}
   		\caption{ }
	    \includegraphics[width=\linewidth]{NC2_0_2-dislin_crop}
	\end{subfigure}
    \caption{\textbf{a)} Electrostatic potential map and \textbf{b)} Critical point analysis of NC trimer nitrated at R1C2, R3C2}
  \label{fig:trimer_top}
\end{figure}

%If the other trimer I get has a similar energy but completely different geomtry, can tabulate it and comment on how many degrees of freedom - difficult to find local minimum, even with initial MM optimisation. This counts for dimer too


\section{Summary}
In this section the polymer structure of \ac{NC} was truncated to a system size suitable for application of density functional theory, for use in subsequent investigations into the degradation reactions. Capping group aproximations were compared. \ac{ESP} maps showed that the methoxy group was closer in charge density to that of polymer chain units, and topological analysis using \ac{QTAIM} highlighted non-bondng interactions between the methoxy chain endings and the side chains of the dimer. Findings showed that the methoxy capping group provided a more sterically and chemically accurate model and will be adopted for work in further chapters.

Topological analysis of monomer, dimer and trimer models found that dimer and trimer systems were more consistent with respect to uniformity of charge density, and intramolecular interactions. Whilst the dimer and trimer models are more representative of the wider polymer, a monomer model is sufficient for initial studies into the chemistry at specific reaction sites on the ring. The limitation to secondary iteractions from additional sites on larger models, and reduction in the degrees of freedom due to decrease in the model size, will simplify and thus speed-up structure searches. This is of particular note when considering transition state searches and geometry optimisations.

%When considering electroststic potential and to a lesser extent steric considerations, methoxy groups were found to provide a better approximation for the extended polysaccharide with respect to both electronic and geometric properties. 

