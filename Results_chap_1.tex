\chapter{Building the Model}
\label{chapterlabel3}

\section{Introduction}
The polysaccahride backbone of \ac{NC} presents numerous reactive sites. Myriad chemical processes contribute to a complex network of possible decomposition routes. This accounts for the large variation in experimentally observed products during low temperature degradation and ageing under ambient conditions. (REF degradation products paper).


When probing these individual chemical reactions%that can contribute to a decomposition pathway
, the extended polymer structure becomes unweildy due to the large number of atoms. \textit{Ab initio} and denisty fucntional methods used to determine the reaction energies are limited to between XXXX atoms, thus limiting the size of system. (REF max atoms for DFT and max atoms for wavefunction methods)


The \ac{NC} polymer structure was truncated to a single-ring monomer by Shukla \textit{et al.} for the purposes of investigating alkaline hydrolysis. [REF SHUKLA] The study analysed the \ac{S_{N}^{2}} nucleophlilic attack at the nitrate carbon, releasing the nitrate ion in favour of a hydroxyl group, at the C2, C3 and C6 nitrate sites. Comparisons between the monomer, dimer and trimer found %desingated
 the dimeric structure to be the minimum model required to accurately represent the chemical behaviour of NC in the alkaline hydrolysis pathway. 
 
 %Maybe for th lit review:
 %The monomer model exhibited the C3 \rightarrow C2 \rightarrow C6 denitration order, contrary to the dimer and trimer order of C3 \rightarrow C6 \rightarrow C2. 
 

% This bit can go in the lit review section %
%This is also observed when considering the minimum unit encompassing all bonding interactions necessary for parameterisation for a forcefield, for implementation in molecular dynamics simulations. 
%The fully nitrated dimer structure used in this study consists of two non-planar β-D-glucopyranose rings joined by a glycosidic bond, with six nitrate groups attached at the 2,3,6 positions on each ring (Figure 11



Chains ends are capped by methoxy groups rather than hydroxyl groups as were employed in Shukla’s study. From the perspective of partial charges, and to an extent steric considerations, methoxy groups are expected to provide a better approximation for the extended polysaccharide.



\section{Computational details}

\section{Results \& Discussion}

\section{Summary}
In this section 
