\chapter{Building the Model}
\label{chapterlabel3}

\section{Introduction}
The long, polysaccharide chain backbone of \ac{NC}, with its numerous reactive sites, leads to a complicated network of possible reaction routes towards degradation. This large system 

The NC polymer structure was truncated to a dimer by Shukla et al. for the purposes of investigating alkaline hydrolysis. [REF SHUKLA] 
Comparisons between the monomer, dimer and trimer found the dimeric structure to be the minimum model required to completely encapsulate the chemical behaviour of NC in the alkaline hydrolysis pathway. 

% This bit can go in the lit review section %
%This is also observed when considering the minimum unit encompassing all bonding interactions necessary for parameterisation for a forcefield, for implementation in molecular dynamics simulations. 
%The fully nitrated dimer structure used in this study consists of two non-planar β-D-glucopyranose rings joined by a glycosidic bond, with six nitrate groups attached at the 2,3,6 positions on each ring (Figure 11

The dimer ends are capped by methoxy groups rather than hydroxyl groups as were employed in Shukla’s study. From the perspective of partial charges, and to an extent steric considerations, methoxy groups are expected to provide a better approximation for the extended polysaccharide.

Shukla’s work identified the nitrate group attached to carbon three (C3) as the most susceptible to denitration and the first to be removed. This is supported by the distribution of partial charges in the molecule, with disregard of the capping groups. Thus, the nitrate group on C3 was used as the target site for degradation studies.

\section{Computational details}

\section{Results \& Discussion}

\section{Summary}
In this section 
