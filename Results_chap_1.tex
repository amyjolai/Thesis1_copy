\chapter{Model building and validation}
\label{chapterlabel3}
\graphicspath{ {./R_chap_1_pics/} }
%Emilias corrections implemented up to pg 6. 
%Stll yet to inorporate Grahams?
\section{Introduction}%%%%%%%%%%%%%%%%%%%%%%% 4 pages%%%%%%%%%%%%%%%
%Why can't we model the whole polymer / why do we need to truncate?
%What is contained in this chapter? What question is posed?

A wide range of reaction products are experimentally observed during the low temperature decomposition and slow ageing of \ac{NC}. Small gaseous species such as \ce{CO2}, \ce{CO}, \ce{CH2O}, \ce{N2O}, \ce{NO2} were identified using IR spectroscopy, whilst the presence of larger species such as glycolic acid and oxalate were revealed by analysis of the complex reaction mixture after laboratory simulated ageing%alluded to ..spectra, suggested by 
\cite{Jin2015,Bluhm1977}.
% Jin's paper also stated that \ac{NC} of different macroscopic structures - aerogel vs fibres, decomposed differently. TG/FTIR showed that the \ac{NC} aerogel decomposed \textit{via} initial splitting of the urethane bond (where was that supposed to be? No C=O present?), followed by denitration. In the aerogel, chain scission begins at the  C-O-C in the ring, whereas in the fibres, the glycosisid bond breaks first. 
% Ossa2012 gives high temp proof of these products by many techniques
%Good table of degradation studies - pyrolysis, mechanochemical, biological etc FernandezdelaOssa2011
Generation of these products is attributed to the range of possible%susceptible 
attack sites along the polysaccharide backbone and at side groups, as well as the numerous possible secondary reactions following denitration. 
These include depolymerisation and ring-cleavage, and can involve contaminants and acids residual in the \ac{NC} matrix following synthesis. 
These are left behind due to imperfect washing procedures, or produced by %as a result of
low level ambient degradation \cite{Chin2007,Edge1990}. %(REF something about the deg products or many secondary reactions) \cite{gun'ko2014}

%[Move this bit to the literature review if appropriate:%%%%%%%%%%%%%%%%%%%%%%%]
%\begin{figure}[b]
%\centering
%\includegraphics[width=0.5\linewidth]{HBEA-P15-0021-crop}
%\caption[A cartridge containing $\acs{NC}$ propellant.]{Markings on the side of a charge bag, with a cartridge containing $\acs{NC}$ propellant. The shelf-life of the product can be calculated from its manufacture date, with consideration of storage conditions. Image from a pamphlet  issued at the end of the Second World War, to enable safe identification of German ammunition \cite{WarOffice24May19452015}. }
%\label{fig:cartridge}
%\end{figure}

Storage conditions such as the choice of wetting solvent, temperature, pressure, humidity and fluctuations in these conditions over time, contribute to the spectrum of products evolved. 
This has implications for the shelf life of \ac{NC} formulations; a practical example is the propellant stored under a pilot's seat for emergency ejection from the cockpit \cite{MinistryofDefence2001,SBIR2013}. 
The propellant device will include a mixture of energetic materials, plasticisers and stabilisers in ratios adhering to established industry safety standards. %REF
During the lifetime of the propellant, it will endure a great number of flights, each involving variable temperature and pressure cycles. 
%%%%Put into Intro %%%%%%%%%%%%%
%This will incorporate a conservative estimation of the shelf-life and stability of the formulation under different environmental and storage conditions. % (figure \ref{fig:cartridge}). 
Understanding of the ageing reactions will shed light on the possible interactions between the energetic materials and the stabilising mixture, informing better industrial practices, safety standards and products with improved %estimates of 
service life. 

%[*End*]%%%%%%%%%%%%%%%%%%%%%%%]
In addition to environmental factors and contaminants, the degradation properties of \ac{NC} depend on the unique %\deleted{ primary, secondary and tertiary} 
polymer chain structure\added{,}
%{. These properties are}
 largely defined by the parent cellulose feedstock. 
 Scanning electron micrograph (\acs{SEM}) images of \textit{Miscanthus} cellulose following nitration show a bloating of the cellulose fibres, with almost full retention of the fibre structure integrity (figure \ref{fig:miscanthus} ).  
\added{Cellulose fibres found in nature are formed of macrofibres, which in turn are formed of microfibrils ranging from 3 - 50 nm in diameter (figure \ref{fig:fibre}) \cite{Dumanli2012}. 
Each microfibril is assembled from stacked polysaccharide chains held together by Van der Waals and hydrogen bonding interactions. Each linear chain is formed by a sequence of individual monomer units, in this case $\beta$-D-glucopyranose.} 
%The side groups of the monomeric units determine the local intermolecular interactions between polymer strand, that lead to the fibre structure and variation in crystalline and amorphous sections. }
%More detail about the following in the literature review. 
Hydrogen bonding networks determine the supramolecular arrangement of the cellulose polysaccharide chains, and variation leads to a mixture of crystalline and amorphous regions \cite{Nishiyama2002,Nishiyama2003}. 
Regions of high crystallinity are more difficult to penetrate by solvent, thus are more resistant to hydrolytic decomposition methods; amorphous regions are more porous, exhibit less hydrogen bonding, and are more prone to digestion by both microbes and chemicals \cite{Chundawat2011}. %REF
%also, any info on which types of cellulose have shorter/ longer chains, and are more / less suscpetible to certain types of degradation?
Hydrolysis in \ac{NC} is largely attributed to the presence of spent acids in the system. Small amounts of moisture present in the bulk or from the air, have also been suspected to accelerate hydrolytic ageing. %REF / Facilitating it? Because if its just in water it doesn't really break down. 
The interplay of reactions between small acid species, water and the polysaccharide, in particular at the nitrate sites, are of critical importance in interpreting the mechanistic pathways leading to widespread degradation throughout the material. 

\begin{figure}[h]
\centering
\includegraphics[width=0.9\linewidth]{miscanthus}%[width=0.85\linewidth]
%(x400 magnification) but thats not what it says in the actual image. 
%REF Nitrocellulose Synthesis from Miscanthus Cellulose, Gismatulina, 2018
\caption{Scanning electron micrograph (\acs{SEM}) images of \textbf{a)} Miscanthus cellulose and \textbf{b)} Miscanthus nitrocellulose, after nitration. From the work of Gismatulina \textit{et al.}\cite{Gismatulina2018}. \added{Reprodu ced with permission.}}% from the authors.} }
%\begin{subfigure}{0.66\linewidth}
%    \includegraphics[width=\linewidth]{miscanthus}
%	\caption{ }
%  \end{subfigure}
%    \begin{subfigure}[b]{0.33\linewidth}
%    \includegraphics[width=\linewidth]{Miscanthus_sinensis}
%   	\caption{ }
%  \end{subfigure}
%    \caption{\ac{SEM} images of \textbf{a)} Miscanthus cellulose and \textbf{b)} Miscanthus nitrocellulose  from the work of Gismatulina \textit{et al.}. \textbf{c)} Miscanthus Sinensis plant.}
\label{fig:miscanthus}
\end{figure}

\begin{figure}[h]
\centering
\includegraphics[width=0.9\linewidth]{Cellulose}
\caption{\added{Natural cellulose fibres are constructed from individual linear strands of cellulose polymer, inter-linked by hydrogen bonds to form microfibrils. The intermolecular bonding structure between multiple strands influence degradation properties by altering the crystallinity and solvent porosity of the bulk material. Adapted from the work of Dumanli \cite{Dumanli2017} with permission.}}
\label{fig:fibre}
\end{figure}
%
%It follows that a strains of \ac{NC} with a higher degree of polymerisation will require longer timescales to attain the same level of decomposition as a variety with lower degree of polymerisation. %$\textalpha \textbeta$ celluose types, if not in the lit review earlier - which ones are more easier digested?

%%%%%%%%%%%%%%%%%%%%%%%%%%%%%%%%%%%%%%%%%%%%%%%%%%%%%%%%%%%%%%%%%%%%%%%%%%%%%


%FIGURE of different fibre orientations and structures of cellulose / NC if I can find it, and if also not already done to death in the lit reviews

%talk about polymer ends, and changes in properties that can arise from changing fibre lengths (in cellulose, which are then translated to nitrocellulose)

%Quantum mechanical methods are routinely used to ...
%Wavefunction and 
Density functional methods are routinely used to explore the energetics of small molecular reactions \cite{rezac2011,Parthiban2001,Bento2008}. %REF one of the S66 benchmark reaction papers, and another one for a slightly bigger system - maybe mine. And perhaps include one study on reactions of cellulose? 
%include a one liners to give exmaples / demonstrate the use in literature. 
% Include a bit about why we are using DFT and QM for looking at reaction eneriges in the first place. What are other ways of doing it, and why are we doing it this way?
When probing the details of individual chemical reactions using computational methods, the extended polymer structure becomes unwieldy due to the large number of atoms. It is therefore necessary to reduce the system to a representative model within a manageable scale.
%In practice, \textit{ab initio} quantum mechanical and semi-empirical
%density functional methods are limited to between XX - XX atoms. %Respectively. %(REF max atoms for DFT and max atoms for wavefunction methods) {either refer to the theory section to talk about which methods are limited in which ways, but make mention of why you have you cap your model to a size that fits the methods available to probe the chemistry.)
%Ways that other people have chopped down big polymer models. 
% incl cellulose
\added{In 2012, }Shukla \textit{et al.} truncated the \acs{NC} polymer structure to a single-ring monomer for %purpose of investigating 
investigation into the alkaline hydrolysis behaviour using \added{\acs{B3LYP} }\ac{DFT} \cite{Shukla2012a}.
The study analysed the \acs{sn2} nucleophilic attack at the nitrate-carbon site, leading to release of the nitrate ion in favour of a hydroxyl group (figure \ref{fig:C2_C3_C6}). 
%%%%Put into Intro %%%%%%%%%%%%%
%Findings suggested that when starting from a fully nitrated monomer, denitration \textit{via} alkaline hydrolysis followed a sequence starting from carbon 3 (C3), followed by the carbon 2 (C2) and the carbon 6 (C6) site. 
%%C3$\rightarrow$C2$\rightarrow$C6. 
%Later studies applying the same reaction to the dimer and trimer found that the sequence instead followed C3$\rightarrow$C6$\rightarrow$C2 for the larger systems \cite{Shukla2012a}. %The sequence of nitration and denitration is explored in \ref{chapterlabel4}
Comparisons between the monomer, dimer and trimer structures found the dimeric structure to be the smallest repeat unit encompassing all bonding and non-bonding interactions, and thus the minimum model that %should be adopted in order to accurately
can fully describe the denitration behaviour of the polymer. This can be attributed to the 1$\rightarrow$4 glycosidic linkage between each of the $\beta$-D-glucopyranose rings. The angle of the glycosidic oxygen leads to an alternating, non-planar orientation for each additional ring added (figure \ref{fig:models})\added{ giving rise to inter-ring interactions and steric effects that cannot be observed in the monomer. }
%The unique interactions between the alternate rings are represented in the dimer and trimer, but lacking in the monomer. 

\begin{figure}[b]
\centering
%\includegraphics[width=\linewidth]{NC}
%\includegraphics[width=\linewidth]{Shukla_11224_2012_9977_Fig1_HTML}
\includegraphics[width=0.9\linewidth]{Shukla_11224_2012_9977_Fig1_HTML}\quad

  {\textcolor{red}{\CIRCLE} oxygen \quad \textcolor{blue}{\CIRCLE} nitrogen \quad  \textcolor{gray}{\CIRCLE} carbon \quad $\bigcirc$ hydrogen}
\caption{\textbf{a)} Numbering scheme of 2,3,6-trinitro-$\beta$-D-glucopyranose, as a monomer of \ac{NC}. 
			\textbf{b)} Transition state showing S\textsubscript{N}2 (opposite side) attack by OH\textsuperscript{-} and 
			\textbf{c)} angular attack (same side) by OH\textsuperscript{-} during the addition–elimination process at the C2 site of \ac{NC} from the work of Shukla \textit{et al.} The solid arrow indicates nucleophilic attack and the dashed arrow the leaving group. 
			\added{Reproduced with permission.}}% from the authors.}}
%REF Theoretical investigation of reaction mechanisms of alkaline hydrolysis of 2,3,6-trinitro-$\beta$-d-glucopyranose as a monomer of nitrocellulose
\label{fig:C2_C3_C6}
\end{figure}

%The addition of a second ring generates further interactions between the rings within a dimer, than if the chain consisted of solely monomer repeat units. 
%for implementation in molecular dynamics simulations, this is also observed. 
%
%FIG of dimer, with interactions highlighted, vs monomer
%
%*A line segway-ing into the reaction coordinate of reactions, etc. May be easiest to talk about Shukla's work further (or Kuklja's?).*
%
%%what else did he/ others find  / use to make the polymer model workable?
%
%*Whilst Shukla demonstrated that the dimer model was the minimum "complete" unit; 
%when considering degrees of freedom in the \ac{PES} in the search for transition state maxima, %\ac{TS}
%%of reaction co-ordinates
%finite computational resources restrict the ... it may be that the dimer is not the most appropriate for further studies.*
%Make this constructive wrt to how I built up the trimer from the monomer and dimer
\begin{figure}[b]
  \begin{subfigure}[b]{\linewidth}
  \centering
	\caption{Monomer with \acs{DOS}=3. }
    \includegraphics[scale=0.7]{NC_mono_H}
%    \caption{Monomer}
  \end{subfigure}
  
  
  
    \begin{subfigure}[b]{\linewidth}
    \centering
   	\caption{Dimer with \acs{DOS}=3.}
    \includegraphics[scale=0.7]{NC_di_H}
%    \caption{Dimer}
  \end{subfigure}
  
  
  
    \begin{subfigure}[b]{\linewidth}
    \centering
	\caption{Trimer with \acs{DOS}=3, constructed from the dimer. }
    \includegraphics[scale=0.7]{NC_tri_H}
%    \caption{Trimer}
  \end{subfigure} 
    \caption{Truncated \acs{NC} %to \textbf{a)}  monomer, \textbf{b)} dimer and \textbf{c)} trimer 
    units, whereby each additional glucopyranose ring is added in the 1$\rightarrow$4 position, as in the structure of cellulose. 
    The trimer was constructed from the optimised dimer, with duplication of the first ring \added{(shown with the blue arrow)}. The degree of substitution (\acs{DOS}) refers to the replacement of the cellulose hydroxyl side groups for nitrate groups.}
  \label{fig:models}
\end{figure}
%
 %Could probably use a linking line here
In this chapter, the electronic properties of the monomer, dimer and trimer truncations of the \ac{NC} polymer were compared. The most suitable model size for subsequent mechanistic investigations was determined with consideration for accurate representation of the chemistry of the full system, %alongside conservation of computational effort% during the exploratory phase of the study
within the limitations of computational resource. %feasibility. 
Hydroxyl  (-OH) and methoxy (-OCH\textsubscript{3}) capping groups were tested at chain ends in the C1, C4 positions, to explore their interaction with the glucopyranose rings and effect on the distribution of charge on the chosen model.
%Comparisons between the monomer, dimer and trimer found %desingated
%the dimeric structure to be the minimum model required to accurately represent the chemical behaviour of NC in the alkaline hydrolysis pathway. 
% 
 %Maybe for the lit review:
 %The monomer model exhibited the C3 $\rightarrow$ C2 $\rightarrow$ C6 denitration order, contrary to the dimer and trimer order of C3 $\rightarrow$ C6 $\rightarrow$ C2. 
%
%
% This bit can go in the lit review section %
%This is also observed when considering the minimum unit encompassing all bonding interactions necessary for parameterisation for a forcefield, for implementation in molecular dynamics simulations. 
%The fully nitrated dimer structure used in this study consists of two non-planar $\beta$-D-glucopyranose rings joined by a glycosidic bond, with six nitrate groups attached at the 2,3,6 positions on each ring (Figure 11

\section{Methodology}%HOW YOU DID EACH CALCULATION%An account of the procedures and techniques used in your research%%%
% ie. how were your results generated? 
%The fully nitrated dimer structure used in this study consists of two non-planar $\beta$-D-glucopyranose rings joined by a glycosidic bond, with six nitrate groups attached at the 2,3,6 positions on each ring (FIG XX). 
%
%NOTE%% ESP surface was calculated using the Gaussian charges, not AIM. I'm not sure whether AIM ones are better, or whether the ones I used from G09 were simply based off Mulliken. I don't think so , but Milliken is poor, esp with diffuse functions (I also don't know why this is). So this might all be rubbish.
%
\subsection{Geometry optimisation}
%\begin{scheme}
%\centering
%\schemestart
%\textbf{Draw polymer unit according to literature orientations}
%\arrow(@aa--bb)[-90,dashed]{Dimer and trimer structure: MM optimisation}
%\arrow(@aa--bb)[-90]{QM optimisation}
%%\merge>(@aa--@bb)
%%\arrow(@aa--cc)[-30,2]\ce{R-OH + HNO3 -> R=O + HNO2 + H2O}
%\arrow(@bb--cc)[-30,2]\ce{HNO3 + R-OH -> R=O + HNO2 + H2O}
%\arrow(@aa--@cc)[-90]
%\arrow(@cc--dd)[-90]\ce{HNO2 + HNO3 <=> N2O4 + H2O} 
%\arrow(@dd--ee){0}[-90,0.2]\ce{HNO2 + R-OH <=> R-ONO + H2O}
%%\arrow(@dd--ff)[180]\ce{N2O4 + H2O -> HNO3 + HNO2}
%%\arrow(ff--gg){0}[-90,0.2]\ce{N2O4 <=> 2^{.}NO2}
%\arrow(@dd--gg)[180]\ce{N2O4 <=> 2^{.}NO2}
%\arrow(@ee--hh)[-90]\ce{R-ONO -> R=O + HNO}
%\arrow(hh--ii)[-90]\ce{HNO + ^{.}NO2 -> HNO2 + ^{.}NO}
%\arrow(ii--kk){0}[-90,0.2]\ce{2HNO -> HON=NOH}
%\arrow(@gg--ii)[-50,3.3]
%\arrow(@kk--jj)[-90]\ce{HON=NOH -> N2O + H2O}
%\arrow(@ii--ll)[180,1]\ce{2 ^{.}NO + O2}
%\arrow(@ll--mm)[180,0.5]\ce{2^{.}NO2}
%\schemestop 
%\caption{Proposed degradation pathway starting from the acid hydrolysis of a nitrate ester, derived from the schemes presented by Camera \cite{Camera1982} and Aellig\cite{Aellig2011}.}
%\label{sch:hydrolysis}
%\end{scheme}
%The monomer starting structure was constructed from the geometry used by Shukla \textit{et. al.} (REF).

\added{The monomer starting structure was constructed using the geometry of $\beta$-D-glucopyranose obtained from literature \cite{NIST2020}, with the substitution of three nitrate groups at each of the carbon 2, 3 and 6 sites.} For dimer and trimer structures, additional glucopyranose rings were appended to the first ring in a 1$\rightarrow$4 position, with alternating planarity (figure \ref{fig:models}). 
%\deleted{Monomer, dimer and trimer structures were constructed with side chains as closely matching literature orientations as possible.}
Each unit was capped with either methoxy or hydroxyl groups and geometry optimised using quantum mechanical (\acs{QM}) methods. %\ac{QM}
 %Ref intro section of relevance?

Selected dimer and trimer molecules were pre-optimised with a \ac{MM} using \acs{UFF} \added{\cite{Rappe1992}} and \acs{MMFF94} \added{\cite{Halgren1996I,Halgren1996II,Halgren19961III,Halgren1996IV,Halgren1996V}} forcefields. 
%, before the \acs{QM} optimisation. 
\added{This was to determine whether an initial 'rough' optimisation using a less expensive \acs{MM} method could lead to a reduction in the \acs{QM} optimisation time for the larger structures, or facilitate easier identification of global minimum structures. 
The \acs{UFF} forcefield was chosen as a general all-element model; \acs{MMFF94} was parameterised for organic compounds and includes hydrogen bonding.}
% \deleted{to determine whether any improvements could be achieved with respect to identification of the global minimum structure or reduction of the \acs{QM} optimisation time. }
%To see whether they helped i finding the global minimum easier / sped up optimisation.

Following \ac{MM} pre-optimisation, the dimer cases did not show any notable speed-up in convergence on the minimised structure, or any significant difference in the final optimised geometry. Further investigations did not implement \ac{MM} pre-optimisation for dimer geometries. 
%also, conversion of file formats was timely and labour intensive, with some loss of bonding in formation that had to be restored later on - had to re-draw in nitrate bonds at times.

Minimisation of trimer geometries proved more challenging, oftentimes with convergence failure,  \added{due to the presence of many floppy bonds and high degrees of rotational freedom for hydrogen atoms and along side chains. This lead to a smooth, slow evolving potential energy surface that is difficult to explore using gradient optimisation methods}. \acs{MM} pre-optimisation of the bi-methoxy capped trimer structure generated a %\deleted{reasonable }
starting input geometry for \ac{QM} optimisation that enabled faster convergence to a final structure, reducing the number of steps required.
However, it was found that a trimer input geometry derived from an already-optimised dimer structure, with duplication of the first ring and appending it to the end of the chain (figure \ref{fig:models}(c)), provided a lower energy starting conformer and more stable optimisation procedure. Thus, rather than construction of a new trimer geometry ``from scratch'', the trimer structures used in the study were constructed from extension of the fully optimised dimer structures.

In some instances, it was found that the final optimised geometries varied according to the input starting geometry. In the cases of more than one possible converged geometry for the same species, the conformer with lowest absolute energy was chosen. 

\subsection{Labelling system}
\label{sect:labellingsystem}

The numbering scheme for structures is detailed in figure \ref{fig:labelling}. Counting from the oxygen of the first glucopyranose ring as atom 1 (O1), the numbering of carbons proceeds in a clockwise direction. %\deleted{ following the Cahn-Ingold-Prelog rules\cite{Cahn1966}, whereby heavier atoms have greater priority.}
For simplicity, the nitrate groups are numbered 1 - 3, also moving clockwise around the ring. The oxygen linking the nitrate to the ring 
%polysaccharide backbone 
is labelled (Ox). Oxygen on the terminal ends of the nitrate group are labelled (Ot). Identical oxygen labelling is applied across all nitrate groups. 

Labels are given from the largest structure to the smallest: [ring; substituent group; component atoms]. For example, a terminal oxygen (Ot) on the nitrate at the C6 position (N3) of the second ring (R2), would be referred to as R2N3Ot. When referring to only the nitrogen of this nitrate, the label would be R2N3. For the carbon at the C6 position on the second ring, the label would be R2C6.

% Change the colour from Magenta to green, or something?
\begin{figure}[t]
	\centering
	\includegraphics[width=0.8\linewidth]{3_3_rot_label_caption_stick}
	\includegraphics[width=0.8\linewidth]{3_3_rot_label_caption}\quad
  
  {\textcolor{red}{\CIRCLE} oxygen \quad \textcolor{blue}{\CIRCLE} nitrogen \quad  \textcolor{gray}{\CIRCLE} carbon \quad $\bigcirc$ hydrogen}
	\caption{The numbering scheme for dimers used in this study. Ring 1 (R1) is the first ring, whereby oxygen (O1) is \textit{one bond} separated from the glycosidic oxygen linking the two rings. Ring 2 (R2) is where O1 is \textit{two bonds} separated from the glycosidic oxygen. 
\newline
	The nitrate oxygen on the terminating ends are referred to as (Ot), and the nitrate oxygen connected to the glucopyranose ring is referred to as (Ox). }
  \label{fig:labelling}
\end{figure}

\subsection{System size and chain ends}
In order to investigate the influence of different capping groups, the fully nitrated dimer structure was used as a reference, following the minimum complete structure approximation established by Shukla \textit{et al.}. %suitable for accurate reproduction of wider system behaviour  (as per Shukla, above). %REF if necessary
Chain ends were capped with methoxy groups, or a combination of a methoxy and a hydroxyl groups (figure \ref{fig:capping-groups-all}). The differences in charge distribution and the nature of intra-molecular interactions were probed using \ac{QTAIM} and inspection of the \ac{ESP} around the molecule.
When considering the system size, the \ac{ESP} and \ac{QTAIM} topology analyses were compared across monomer, dimer and trimer models. \added{To simplify optimisation of the trimer structure and to explore interactions at lower degrees of substitution, nitration was limited to the C2 site. The C2 position was found by Shukla \textit{et al.} to be the first to be nitrated, and last to be denitrated \cite{Shukla2012a}.} %{Nitration was limited to C2 sites, to explore interactions at lower levels of nitration and to simplify the optimisation of the trimer structure.} 
The monomer was nitrated only at C2, the dimer nitrated only at R1C2 and the trimer nitrated at R1C2 and R3C2.
%
%With increasing nitration level, hydroxyl groups were substituted by nitrate groups at the C2, C3 and C6 positions in reverse order of denitration according to the study by Shukla \textit{et al.}\cite{Shukla2012a},  C2$\rightarrow$C6$\rightarrow$C3. The original conformation was conserved; equatorial in the case of C2 and C3.
%%talk about C6 brnach orientation / angles?
%
%\subsubsection{\ac{ESP} mapping}
%
%\subsubsection{Laplacian of electron density}
%
%
%A bit of extra detail practical implementation of the partial charges, ESP and lapacian
%
%Structures underwent geometry optimisation in vacuum and implicit solvent, using various methods detailed in section XX.
%
%Reasons why I woudn't put it in a perdioci system - 
% I am only looking local chemical reactions at this stage. Though these reactions happen all along the structure, in theory, as I am trying to probe individual reactions that do not yet affect the wider chain, it woudl be a waste of computational effort to model the rest of the chain, when I could see probe the reaction using a smaller segement.
%
% The NC system isn't periodic in the same way that solid state materials are - the chains twist and bend. Whilst duplicating the same periodic box would allow me to expand the interactions to wider parts of the system, it would still not capture all the behvaiours - sterics, alternate configurations. 
% I would still just be replicating the chemistry that I see with the smaller model. 
% I'd need molecular dynamics for the wider description of the chain. And if I wanted to look at wider chemical interactions - I'd need ab intio MD, which is expensive.
% This would be useful for later probing the wider impact of the products of first stage degradation, to see how the products that are more likely to migrate further into the NC matrix will interact. But whilst I'm still looking at initial denitration, the smaller model will suffice. 

\subsection{Computational details} %TECHNICAL DETAILS OF YOUR SETUP%%%%%%
The \acs{B3LYP} functional was chosen for initial exploration of the electronic properties of the system. It is an efficient and extensively benchmarked method %for calculating electronic properties 
for main group elements, suitable for the current model system size, where the largest trimer extends to 76 atoms\cite{Bento2005,Bento2008,TiradoRives2008}. 
%\subsubsection{Geometry optimisation}
All electronic structure calculations in this section, including geometry optimisation and thermodynamic calculations were performed in vacuum to the level of \acs{B3LYP}/ 6-311+G(d,p) with tight convergence criteria (table \ref{tab:convergence})%\deleted{Minima were located using the Berny optimisation algorithm\cite{Schlegel1982}
, implemented in the \ac{G09} quantum chemistry suite\cite{G09}.

\begin{table}[hbp]
\begin{center}
\caption{Convergence criteria used in \ac{G09}.  Units in Hartree/Bohr.}
\begin{tabular}{ l c c c } 
\toprule
Convergence criteria 	& Normal 		& Tight \\
\hline
% Maximum Force 		&  0.000450	& 0.000015 \\
% Root Mean Square     Force 			& 0.000300 	& 0.000010 \\
% Maximum Displacement & 0.001800 & 0.000060 \\
% Root Mean Square     Displacement & 0.001200 	& 0.000040 \\
 Maximum Force 		&  4.5$\times 10^{-4}$	& 1.5$\times 10^{-5}$ \\
 Root Mean Square     Force 			& 3.0$\times 10^{-4}$ 	& 1.0$\times 10^{-5}$ \\
 Maximum Displacement & 1.8$\times 10^{-3}$ & 6.0$\times 10^{-5}$ \\
 Root Mean Square     Displacement & 1.2$\times 10^{-3}$ 	& 4.0$\times 10^{-5}$  \\
\bottomrule
\end{tabular}
\label{tab:convergence}
\end{center}
\end{table}
%\multirow{3}{4em}{Multiple row} & cell2 & cell3 \\ 
%& cell5 & cell6 \\ 
%& cell8 & cell9 \\ 
%\hline
%\end{tabular}
%\end{center}
%
%6-31+g(2df,p)/ $\omega$B97X-D [This is only for the subsequent chapters. Here I used the TWORUN calculations. May have to explain basis set choice later, since I used less ]  
%here I used b3lyp/6-311+g**. 
% More basis functions on the valence e, but less polarisation. WHAT SIGNIFICANCE will this have on my results/ system?
% ACtually, I think this is an ok thing at this stage, because I'm not really looking at longer range interaction - more intramolecular ones. So missing out on some of the diffuse  / polarisation isn't a big as deal as when with looking at bimolecular+ reactions. 
 
Structures were built using %z-matrix notation or 
the \ac{GView} graphical user interface \cite{GV5}. Molden 5.0.2 \cite{Schaftenaar2000,Schaftenaar2017} and \ac{GView} packages were used for visualisation. %ref molden gview
Avogadro molecular editor 1.1.1 \cite{hanwell2012} was used for \ac{MM} pre-optimisation. \acs{UFF} and {MMFF94} forcefields were applied with Steepest Descent algorithm \cite{Cauchy1847,Meza2010} with 500 steps \added{and energy convergence of 1$\times$10\textsuperscript{-7} kJmol\textsuperscript{-1}. } %change in energy $\delta$E}
Electrostatic potential (\acs{ESP}) surfaces were mapped to the electron density, extracted from \ac{G09} formatted checkpoint files following optimisation. Gaussian's CubeGen utility was used to generate the 3D point grids for mapping; density and \ac{ESP} cubes were generated with 80\textsuperscript{3} points \added{(default grid point density)}. \acs{ESP} maps were visualised using \ac{GView}. 
\ac{QTAIM} topological analyses including %generation of Laplacian electron density maps and 
\ac{CP} analysis on the optimised structures were performed using MultiWFN 3.6 \cite{Lu2012}.  %REF (also update your diagrams).
%Partial charges were obtained via PyRed (R.E.D. Server version 3.0). 

\section{Truncating the polymer}% RESULTS & DISCUSSION%%%%%%%%%%%%%%%%%%%%%
% Results: what you did and what you got %%%%%%%%%%%%%%%%%%%%%%%%%%
% "A presentation of the data obtained from your research."
% Discussion: what does it mean, and so what? %%%%%%%%%%%%%%%%%%%%%
% "An explanation of the significance of your findings and how they relate
% to the work of other scholars.
% A review of your findings and their importance as well as suggestions
% for further research in your chosen area."
%
%\subsection{Choice of functional}
%Discuss the validity and caveats of each methods, including solvent vs vacuum.
%(This is possibly for the next section, as you haven't done anything on this.)
%
\subsection{Comparison of capping groups}
% Graphics I need:
	% Laplacian of select dimer structures, with different cappign groups
	% ESP of one dimer of one dimer of each nitration level
	% Partial charges of one dimer of each nitration level
%
% Want to look at the changes in partial charges, ESP, Laplacian and CPs, and see how they differ the methyl and H capping sites.
% Look at partial charges with and without capping included. 
% Generate some of the Laplacian maps looking at the capping ends, as well as the next nearest nitrate group.
% Analyse with respect to different levels of nitration
% Justify the choice of methyl.

%- Variation in partial charges, with different capping groups
%explain why you chose the dimers you did - it was based on Shukla's denitration sequence.
Fully nitrated dimers (\acs{DOS}=3) were bi-capped with methoxy (\acs{CH3/CH3}), methoxy-hydroxyl (\acs{CH3/OH}) or hydroxy-methoxy (\acs{OH/CH3}) groups at C1 and C4 positions (figure \ref{fig:capping-groups-all}).
The effect of each capping group combination on the dimer geometry, electronic properties and intramolecular non-bonding interactions within the dimer were observed. 

%%TBh don't need this:
%\added{
%In contrast to the larger methyl group, the smaller volume of the hydroxyl group was not expected to introduce any significant distortion or non-bonding interactions with nitrate groups on side branches. 
%The hydroxyl capping group can be considered as analogous to polysaccharide chain terminating ends. %FIG 
%Some steric and intramolecular interactions are expected between neighbouring dimeric units in the \ac{NC} chain, however these effects may be less pronounced with a smaller capping group. 
%The methoxy group was therefore anticipated to provide a more chemically and sterically accurate proxy for the extended polysaccharide. }
 
\begin{figure}[htp]
  \centering
  \begin{subfigure}[b]{0.5\linewidth}
	\caption{ }
    \includegraphics[width=\linewidth]{CH3_CH3_cap}
%    \caption{Fully nitrated NC dimer with methoxy capping groups in the C1 and C4 positions.}
  \end{subfigure}
    \begin{subfigure}[b]{0.5\linewidth}
   	\caption{ }
    \includegraphics[width=\linewidth]{CH3_OH_cap}
%    \caption{Fully nitrated NC dimer with a methoxy capping group in the C1 and hydroxyl group in the C4 position.}
  \end{subfigure}
      \begin{subfigure}[b]{0.5\linewidth}
   	\caption{ }
    \includegraphics[width=\linewidth]{OH_CH3_cap}
%    \caption{Fully nitrated NC dimer with a hydroxyl capping group in the C1 and methoxy group in the C4 position.}
  \end{subfigure}
    \caption{Fully nitrated \acs{NC} dimer with \textbf{a)} methoxy groups capping chain ends on both ring 1 and ring 2 (\acs{CH3/CH3}), \textbf{b)} a methoxy capping group on ring 1 and hydroxyl on ring 2 (\acs{CH3/OH}), \textbf{c)} hydroxyl group on ring 1 and methoxy capping group on ring 2 (\acs{OH/CH3}).}
  \label{fig:capping-groups-all}
\end{figure}


\acs{QTAIM} electron density topology analysis of all dimers revealed notable intermolecular interaction across the two rings, in particular where the C6 side chain was able to interact with the nearest nitrate group of the opposite ring. %(fig?)
All three capping group combinations presented similar characteristics; nuclear critical points (\acs{NCP}) lay at all atomic sites and each chemical bond was described by a \ac{BCP}. Ring critical points (\acs{RCP}) lay at ring centres and at points of concentrated steric interference within side chain moieties. A single \ac{CCP} was observed in the region between R1C5 - R1C6 - R2N2, %fig?
where the arrangement of the side chains formed a %pseudo
pyramidal cage-like environment (table \ref{tab:table_cp}).  %any significance? BEtween which atoms specifically? please highlight
%
%\begin{table}
%\begin{center}
%%\caption{Convergence criteria used in \ac{G09}.  Units in Hartree/Bohr.}
%\begin{tabular}{ l c c c } 
%\toprule
%%Citical point 	& Normal 		& Tight \\
%%\hline
%Nuclear critical points (\acs{NCP}) 		&  												& atomic nuclei 				\\
%Bonding critical points (\ac{BCP})			& \textcolor{orange}{\CIRCLE}			& bonding interactions 	\\
%Intramolecular bonding paths 				& \textcolor{orange}{\emph{---}} 	& bonding paths				\\
%Ring critical points (\acs{RCP})     		& \textcolor{yellow}{\CIRCLE} 			& steric centres				\\
%Cage critical points (\acs{RCP})     		& \textcolor{green}{\CIRCLE} 	& centre of cage systems	\\
%\bottomrule
%\end{tabular}

\begin{figure}
\caption[All critical points identified by \ac{QTAIM} topology analysis. ]{All critical points identified by \ac{QTAIM} topology analysis. 
Nuclear critical points (\acs{NCP} are located at atomic nuclear sites, 
bonding critical points (\ac{BCP}), ( \textcolor{orange}{\CIRCLE} orange spots) lie on chemical bonds, 
or on intramolecular bonding paths (\textcolor{orange}{\emph{\textbf{---}}} shown in orange); 
ring critical points (\acs{RCP}) (\textcolor{yellow}{\CIRCLE} yellow spots) denote centres of steric interaction and the 
cage critical point (\ac{CCP}) (\textcolor{green}{\CIRCLE} green spot) shows the centre of a cage-like system. }
  \centering
  \begin{subfigure}[b]{0.9\linewidth}
	\caption{Critical points idenitified for the \acs{CH3/CH3} dimer.}
  \includegraphics[width=\linewidth]{CH3_CH3-dislin_crop}
  \end{subfigure}
  \begin{subfigure}[b]{0.93\linewidth}
    \caption{Critical points for the \acs{CH3/OH} dimer.}
    \includegraphics[width=\linewidth]{CH3_OH-dislin_crop}
  \end{subfigure}
%    \caption{All critical points identified by \ac{QTAIM} topology analysis. Nuclear critical points (\acs{NCP} are located at atomic nuclear sites, bonding critical points (\ac{BCP}), (orange spots) lie on chemical bonds, or on intramolecular bonding paths shown in orange; ring critical points (\acs{RCP}) denote centres of steric interaction (yellow points) and the cage critical point (\ac{CCP}) (green spot) shows the centre of a cage-like system.}
  \label{fig:cp-all}
\end{figure}
\begin{figure}[H]
\ContinuedFloat
\caption{Continued.}
      \begin{subfigure}[b]{0.9\linewidth}
	  \caption{Critical points for \acs{OH/CH3} dimer.}
    \includegraphics[width=\linewidth]{OH_CH3-dislin_crop}
  \end{subfigure}  
  \label{fig:cp-all}
\end{figure}

\begin{table}
\begin{center}
\caption{Topology analysis of dimers with capping groups. * \acs{CH3/CH3} and \acs{CH3/OH} only; ** \acs{CH3/CH3} only ; *** Intramolecular steric centres located between rings, and between nitrate groups and nearest $\alpha$-hydrogen on the ring.}
\begin{tabular}{ l c c c l} 
\toprule
Dimer & CP type & Number & Interaction centres & Intramolecular bonding\\
\hline
\multirow{4}{3em}{\acs{CH3/CH3}} & NCP & 63 & All atomic nuclei & - \\ \cline{2-5}
& BCP &75 & \makecell{64 chemical bonds\\11 intramolecular bonds} &  \makecell[l]{\textbf{R1 - R1}\\R1C2H - R1N1Ot\\\textbf{R2 - R2}\\R2C3H - R2N2Ot\\R2C2H - R2N1Ot\\\textbf{R1 - R2}\\R1C3H - R1N2Ot\\R1C6H - R2N2Ot\\R1C5H - R2H2Ot\\R1O1 - R2N2*\\R1O1 - R2N2Ox\\R1C1H - R2N2Ox\\R2C6H - R1N1Ot\\R1N1Ot - R2N3Ot**}\\ \cline{2-5}
& RCP & 14 & \makecell[l]{R1, R2 ring centres, \\ Steric centres***} & -  
%R1C6 - R2N2, R1N1 - RC6, R1O1 - 2Ox and steric centres N1Ot - C2H, N2Ot - C3H}
%\textsubscript{n}Ot and nearest H on the ring***}
\\ \cline{2-5}
& CCP & 1 & R1C5 - R1C6 - R2N2 & -  \\ 
\bottomrule
\multirow{4}{4em}{\acs{CH3/OH}} & NCP & 60 & All atomic nuclei  & -  \\ \cline{2-5}
& BCP & 71 &  \makecell{61 chemical bonds \\10 intramolecular bonds} & \makecell[l]{\textbf{R1 - R1}\\R1C2H - R1N1Ot\\\textbf{R2 - R2}\\R2C3H - R2N2Ot\\R2C2H - R2N1Ot\\\textbf{R1 - R2}\\R1C3H - R1N2Ot\\R1C6H - R2N2Ot\\R1C5H - R2H2Ot\\R1O1 - R2N2*\\R1O1 - R2N2Ox\\R1C1H - R2N2Ox\\R2C6H - R1N1Ot}\\\cline{2-5} 
& RCP & 14 & R1, R2 ring centres,***  & - \\ \cline{2-5}
& CCP & 1 & R1C5 - R1C6 - R2N2  & -  \\
\bottomrule
\multirow{4}{4em}{\acs{OH/CH3}} & NCP & 60 & All atomic nuclei  & -  \\ \cline{2-5}
& BCP & 70 & \makecell{61 chemical bonds \\9 intramolecular bonds} &  \makecell[l]{\textbf{R1 - R1}\\R1C2H - R1N1Ot\\\textbf{R2 - R2}\\R2C3H - R2N2Ot\\R2C2H - R2N1Ot\\\textbf{R1 - R2}\\R1C3H - R1N2Ot\\R1C6H - R2N2Ot\\R1C5H - R2H2Ot\\R1O1 - R2N2Ox\\R1C1H - R2N2Ox\\R2C6H - R1N1Ot}\\ \cline{2-5}
& RCP & 13 & R1, R2 ring centres,***  & -  \\ \cline{2-5}
& CCP & 1 & R1C5 - R1C6 - R2N2  & -  \\ 
\bottomrule
\end{tabular}
\label{tab:table_cp}
\end{center}
\end{table}

Hydrogen bonding was observed between the nitrate groups directly connected to the ring and their associated $\alpha$-hydrogens (figure \ref{fig:alpha}). N1 and N2 nitrates are orientated in the eclipsed conformation relative to the  $\alpha$-hydrogen, bringing the terminal nitrate oxygen within H-bonding distance and generating a steric critical point between the oxygen and $\alpha$-hydrogen (table \ref{tab:Hconformation}). 
This was not the case for C6 nitrates, which are not directly connected to the ring. The flexibility of the C6 alkyl chain allows for relaxation of the nitrate to the gauche orientation, relative to the \ce{CH2} group. The increased distance between the terminal oxygen of the nitrate and the $\alpha$-hydrogens exceeded H-bonding distance and reduced steric crowding.% in the region.

\begin{figure}[H]
\centering
\includegraphics[width=0.3\linewidth]{alpha_a}
\caption{$\alpha$-hydrogen relative to the nitrate group, highlighted in red; the terminal oxygen (Ot) - $\alpha$-H interaction is highlighted in blue.}
\label{fig:alpha}
\end{figure}

\begin{table}[ht]
\begin{center}
\caption{Distances between terminal oxygen of nitrates and nearest $\alpha$-hydrogen.}
\begin{tabular}{ l c c c l } 
\toprule
Nitrate oxygen & Nearest $\alpha$-H & Distance / \si{\angstrom} & Dihedral & Conformation\\
\hline
 \textbf{R1} & & & \\
  R1N1Ot & R1C2H & 2.2 & -3.5 & Eclipsed \\
 R1N2Ot & R1C3H & 2.1 & 2.6 & Eclipsed\\
 R1N3Ot & R1C6H & 2.5 & 25.4 & Gauche\\
 \textbf{R2} & & & \\
 R2N1Ot & R2C2H & 2.2 & 3.4 & Eclipsed\\
 R2N2Ot & R2C3H & 2.2 & 9.4 & Eclipsed\\
 R2N3Ot & R2C6H & 2.5 & -24.9 & Gauche \\
% R1N1Ot & R1C2H & 2.1667 & -3.4599 & Eclipsed \\
% R1N2Ot & R1C3H & 2.1410 & 2.5662 & Eclipsed\\
% R1N3Ot & R1C6H & 2.5332 & 25.3615 & Gauche\\
% \textbf{R2} & & & \\
% R2N1Ot & R2C2H & 2.1788 & 3.3742 & Eclipsed\\
% R2N2Ot & R2C3H & 2.1729 & 9.3912 & Eclipsed\\
% R2N3Ot & R2C6H & 2.5185 & -24.9355 & Gauche \\
%
\bottomrule
\end{tabular}
\label{tab:Hconformation}
\end{center}
\end{table}
Inter-ring H-bonding occurred between C6 hydrogens and the terminal oxygens of the nearest nitrate on the other ring (at the C3 position), in addition to N - O interactions and O - O interactions. 
%For ring 1, the interaction was between R1C6H and R2N2Ot, and for ring 2, between R1N1Ot and R2C6H.
Upon replacement of a hydroxyl capping group for a bulkier methoxy, the geometry of the nitrate groups adjusted to accommodate. It can be seen in the case of \acs{CH3/CH3} in figure \ref{fig:cp-all}.a), that there is an additional \ac{BCP} and intramolecular bonding path between R1N1Ot and R2N3Ot that was not present in the structures with only one methoxy group, %QTAIM / Laplacian to find out the nature of this interaction?
 and an R1O1 - R2N2 interaction that is not present i n \acs{OH/CH3}. 
Inspection of nitrate-nitrate distances show that with substitution to methoxy capping groups, the inter-nitrate distances did not change significantly (table \ref{tab:geomety_cap}). Despite a lower number of intramolecular bonding interactions, \acs{OH/CH3} did not exhibit larger bonding distances between interaction centres. 
%laplacian on the R1C5 - N2 region. esp atoms H48, C5, N35, O6  to explain this? Also 20 and 41
%
%(or put his with QTAM)
%table of all bonds and angles in the appendices, expansion on the one below? Optional
%possible changes: relative ring planarity, forced closer interaction between nearby nitrate groups, twisting of the ring, and intramolecular interactions
\begin{table}[b]
\begin{center}
\caption{Distances between nitrate groups across different capping group combinations.}
\begin{tabular}{ l c c c c} 
\toprule
Capping group & Nitrate-nitrate pair &  Distance / \si{\angstrom} \\
\hline
 \multirow{4}{4em}{\acs{CH3/CH3}} 	& R1N1 - R1N2 & 4.0 \\
															& R1N1 - R2N3 & 4.8 \\
															& R1N3 - R2N2 & 5.2\\
															& R2N2 - R2N1 & 4.1 \\
\hline
 \multirow{4}{4em}{\acs{CH3/OH}} 		& R1N1 - R1N2 & 4.0 \\
															& R1N1 - R2N3 & 4.8 \\
 															& R1N3 - R2N2 & 5.2\\
															& R2N2 - R2N1 & 4.1 \\
 \hline
 \multirow{4}{4em}{\acs{OH/CH3}} 		& R1N1 - R1N2 & 4.0 \\
 															& R1N1 - R2N3 & 4.8 \\
 															& R1N3 - R2N2 & 5.3\\
 															& R2N2 - R2N1 & 4.1 \\
% \multirow{4}{4em}{\acs{CH3/CH3}} & R1N1 - R1N2 & 3.9871 \\
% & R1N1 - R2N3 & 4.8328 \\
% & R1N3 - R2N2 & 5.2424\\
% & R2N2 - R2N1 & 4.1421 \\
%\hline
% \multirow{4}{4em}{\acs{CH3/OH}} & R1N1 - R1N2 & 3.9887 \\
% & R1N1 - R2N3 & 4.8327 \\
% & R1N3 - R2N2 & 5.2371\\
% & R2N2 - R2N1 & 4.1325 \\
% \hline
% \multirow{4}{4em}{\acs{OH/CH3}} & R1N1 - R1N2 & 3.9359 \\
% & R1N1 - R2N3 & 4.8208 \\
% & R1N3 - R2N2 & 5.2868\\
% & R2N2 - R2N1 & 4.1297 \\
\bottomrule
\end{tabular}
\label{tab:geomety_cap}
\end{center}
\end{table}

\begin{figure}[htp!]
\caption{Critical points identified by \ac{QTAIM} topology analysis of NC dimer nitrated at C2 position for \textbf{a)} \ac{CH3/CH3} dimer, \textbf{b)} \acs{CH3/OH} dimer and \textbf{c)} \acs{OH/CH3} dimer.}
\centering
  \begin{subfigure}[b]{0.9\linewidth}
	\caption{}
  \includegraphics[width=\linewidth]{NC2_0-dislin_crop}
  \end{subfigure}
    \begin{subfigure}[b]{0.9\linewidth}
	\caption{}
  \includegraphics[width=\linewidth]{NC2_0_CH3_OH-dislin_Crop}
  \end{subfigure}
  \begin{subfigure}[b]{0.9\linewidth}
    \caption{}
\end{figure}
\ContinuedFloat
\caption{Continued.}
\centering
\begin{figure}[H]
    \includegraphics[width=\linewidth]{adj_NC2_0-OH_CH3_dislin_crop}
  \end{subfigure}
\label{fig:2_0}
\end{figure}
%
%wait for NC2_0 to be done.
%Here I want to comment on the interaction with diff cappign groups - why do I continue with CH3?
%Does this segway into something about the model size in the next section?
Topology analysis on dimers with lower levels of nitration (\acs{DOS}=0.5) show an interaction between unsubstituted hydroxyl side chains with the capping group on ring 1 (figure \ref{fig:2_0}). 
% Whilst not directly analogous with the H-bonding between the hydrogen of the C6 and the glycosidic oxygen seen in glucose \cite{Jebber1996}, it can be seen that the interaction of side chains with the polymeric oyxgen is present in other sugar structures. 
\ac{NC} dimers nitrated at only the R1C2 position exhibit H-bonding and H-H interaction between the capping group hydrogens and those of C6, and O-O interactions between R1O1 and hydroxyl oxygens at R2C3. %and C6 in the case of CH3CH3 and CH3OH dimers. 
The variation in interaction site is dependent on the geometry. Changes in the capping group partially influence the change of geometry around the interacting sites (table \ref{tab:torsion}). %, though the site of nitration is most crucial.
%In the case of the R1C2 dimers, the change in orientation of the rings does not give any indication of angle or bonds. 
%
\begin{table}[t]
\begin{center}
\caption{Torsion angles between the two rings in the R1C2 nitrated dimer (DOS=1), when capping groups are modified.}
\begin{tabular}{ l c } %{S[table-format=3.2]}
\toprule
Dimer & Ring torsion angle \\
\midrule
\ac{CH3/CH3} & -177.1370 \\
\acs{CH3/OH} & -177.1120 \\
\acs{OH/CH3} & -176.4909 \\
\bottomrule
\end{tabular}
\label{tab:torsion}
\end{center}
\end{table}
%
%split into two columns and have the CPS on the right hand side. 
\begin{figure}[htp!]
  \centering
	\includegraphics[width=0.8\linewidth]{scale_bar}
    \begin{subfigure}[b]{0.6\linewidth}
   		\caption{ }
	    \includegraphics[width=\linewidth]{CH3_CH3-NC3_3_ESP_surface_w-5-287e-2}
	\end{subfigure}
    \begin{subfigure}[b]{0.7\linewidth}
   		\caption{ }
	    \includegraphics[width=\linewidth]{CH3_OH_NC3_3_ESP_surface_w-5-287e-2}
    \end{subfigure}
    \begin{subfigure}[b]{0.65\linewidth}
	   	\caption{ }
	    \includegraphics[width=\linewidth]{OH_CH3_NC3_3_ESP_surface_w-5-287e-2}
  \end{subfigure}
    \caption{Electrostatic potential maps of \textbf{a)} \ac{CH3/CH3}, \textbf{b)} \acs{CH3/OH}, \textbf{c)} \acs{CH3/OH}.} % *UNITS?*
  \label{fig:ESP}
\end{figure}
%
%On comparing surfaces - you need to compare ESPs directly, rather than visually (this is last resort) - you need a table. These aren't the same as partial charges. Can I get them from the cube files? 

The \ac{ESP} maps % for the bi-capped methoxy dimer (\ac{CH3/CH3}), methoxy-hydroxyl capped dimer (\acs{CH3/OH}) and hydroxy-methoxy capped dimer (\acs{OH/CH3}) 
represent the charge density around the molecule (figure \ref{fig:ESP}). A more negative value (increasingly red) indicates an area of higher electron density, whilst a more positive value (increasingly blue) indicates an area of lower electron density. It can be seen that the areas with lowest electron density lie around the centre of the glucopyranose rings and at the C6 position of the side branches. This can be explained by the number of adjacent oxygens drawing the electron density away from the less electronegative %but electron-rich INDUCTIVE EFFECT: polarisation of a sigma bond due to differences in electronegativity - either electron donating or withdrawing. 
alkyl groups. The most electron rich areas are the outermost oxygens of the nitrate groups (Ot positions). The terminal oxygens draw electron density from the nitrogen lone pairs and are resonance stabilised. %could be BS
The presence of the hydroxyl capping group promotes a small decrease in the concentration of electron density around the adjacent ring. This is illustrated by the slight deepening of the blue shading around the ring directly bonded to the hydroxyl. The hydroxyl group itself presents the lowest concentration of electron density, suggesting that the terminating hydrogen may be %is susceptible 
prone to nucleophilic attack. 
For the extended polymer, it is expected that the electron-density profile of each glycosidic oxygen between monomers is analogous. Capping groups should therefore mimic the charge density around the central oxygen of the dimer, as closely as possible. 
The oxygen of both methoxy and hydroxyl capping groups exhibit a lower concentration of electron density than that of the bridging oxygen between the two rings. 
This deficiency is particularly pronounced in the case of the hydroxyl capping group, suggesting that the methoxy group provides a more representative approximation for the extended polymer with respect to charge density. 
 %reinforced by partial charges?
%The trimer model's ESPs should reinforce that CH3 is a more chemically similar approximation - no concentration of charge 

%Shukla’s work identified the nitrate group attached to carbon three (C3) as the most susceptible to denitration and the first to be removed. This is supported by the distribution of partial charges in the molecule, with disregard of the capping groups. %Thus, the nitrate group on C3 was used as the target site for degradation studies.

\subsection{Model size}%%%%%%%%%%%%%%%%%%%%%%%%%%%%%%%%%%%%%%%%%%%%%%%%%%%
% Graphics I need:
	% Laplacian of C2 nitrated monomer, dimer and trimer structures
	% ESP of each
	% Partial charges of each

% Want to say that partial charges / ESP / Laplacian / CP's do not look so different, across each of the model systems, but that the dimer is the best model for capturing all the interactions based on the payoff between computational efficiency and accuracy of resutls. The interaction between the rings is missing for the monomer, and this is a crucial aspect of the chemistry. 

The electrostatic potential and topology analysis of the partially nitrated monomer, dimer and trimer were compared in order to explore the limitations of each truncation. % and determine which model to proceed with further investigations.  % Work out DOS for each of these
The monomer was nitrated at C2, the dimer at R1C2 and trimer model nitrated at R1C2 and R3C2 sites. 
Analysis of the \acs{CP}s in the monomer revealed steric centres and a non-bonding interaction with the capping group. 
%Analysis of the \acs{CP}s in the monomer revealed a steric centre in the ring and between the C3 hydroxyl in a non-bonding interaction with the capping group. 
\begin{figure}[ht]
\centering
	\includegraphics[width=0.8\linewidth]{scale_bar} 
	\par\bigskip
    \begin{subfigure}[b]{0.45\linewidth}
	   	\caption{ }
	    \includegraphics[width=\linewidth]{C2_ESP_surface_w}
    \end{subfigure}
    \begin{subfigure}[b]{0.5\linewidth}
	   	\caption{ }
	    \includegraphics[width=\linewidth]{C2_dislin_3_crop}
    \end{subfigure}
\caption{\textbf{a)} Electrostatic potential map and \textbf{b)} critical point analysis of NC monomer nitrated at C2 position.  Bonding critical points (\ac{BCP}), ( \textcolor{orange}{\CIRCLE} orange spots) lie on chemical bonds; intramolecular bonding paths (\textcolor{orange}{\emph{\textbf{---}}} shown in orange); 
ring critical points (\acs{RCP}) (\textcolor{yellow}{\CIRCLE} yellow spots) denote centres of steric interaction and the 
cage critical point (\ac{CCP}) (\textcolor{green}{\CIRCLE} green spot) shows the centre of a cage-like system. }
\end{figure}
\begin{figure}[H]
\ContinuedFloat
\centering
	\includegraphics[width=0.8\linewidth]{scale_bar} 
	\par\bigskip
%\label{fig:monomer_top}
%    \caption{\textbf{a)} Electrostatic potential map and \textbf{b)} critical point analysis of NC monomer nitrated at C2 position.}
%\label{fig:monomer_top}
%\end{figure}
%\begin{figure}[h!]
%\centering
%	\includegraphics[width=0.7\linewidth]{scale_bar}
   \begin{subfigure}[b]{0.7\linewidth}
	   	\caption{ }
	    \includegraphics[width=\linewidth]{NC2_0_ESP_surface_w}
    \end{subfigure}
    \begin{subfigure}[b]{0.8\linewidth}
	   	\caption{ }
	    \includegraphics[width=\linewidth]{NC2_0-dislin_crop}
    \end{subfigure}
%    \caption{\textbf{c)} Electrostatic potential map and \textbf{d)} critical point analysis of NC dimer nitrated at R1C2 position }
%\label{fig:monomer_top}
%\label{fig:dimer_top}
%\end{figure}
%%
%\begin{figure}[htp]\ContinuedFloat
\centering 
%%	\includegraphics[width=0.8\linewidth]{scale_bar} 
    \begin{subfigure}[b]{0.9\linewidth}
   		\caption{ }
   		\label{fig:trimer_top}
	    \includegraphics[width=\linewidth]{adj_NC2_0_2_ESP_surface-w}
    \end{subfigure}
    \begin{subfigure}[b]{1\linewidth}
   		\caption{ }
	    \includegraphics[width=\linewidth]{adj_NC2_0_2-dislin_crop}
	\end{subfigure}
\caption{ 
\textbf{c)} Electrostatic potential map and \textbf{d)} critical point analysis of NC dimer nitrated at R1C2 position. 
\textbf{e)} Electrostatic potential map and \textbf{f)} critical point analysis of NC trimer nitrated at R1C2, R3C2 
}
%\caption{Cont.}
 %   \caption{\textbf{e)} Electrostatic potential map and \textbf{f)} critical point analysis of NC trimer nitrated at R1C2, R3C2}
\end{figure}
%
The terminal oxygen - $\alpha$-hydrogen interaction at the C2 nitrate was observed, as was seen in the fully nitrated systems (figure \ref{fig:trimer_top} a)).

The trimer unit displayed a more extensive network of hydrogen bonding between rings, facilitated by the higher number of unsubstituted hydroxyl groups on the second ring (figure \ref{fig:trimer_top}).
Compared with the dimer and trimer models, the monomer presents all the intra-ring interactions observed in the larger units, omitting only the between-ring interactions,
%but does not possess inter-ring interactions, as would be expected due to the absence of additional glucopyranose rings.  

\begin{figure}[htp]
\centering 
	\includegraphics[width=0.9\linewidth]{mulliken_scale_label} 
    \begin{subfigure}[b]{0.4\linewidth}
   		\caption{NC monomer nitrated at C2 }
	    \includegraphics[width=\linewidth]{monomer_symb}
    \end{subfigure}
    \begin{subfigure}[b]{0.4\linewidth}
   		\caption{ Shaded by Mulliken charge}
	    \includegraphics[width=\linewidth]{NC2-CH3-CH3_mulliken_no_label}
    \end{subfigure}
        \begin{subfigure}[b]{0.6\linewidth}
   		\caption{Mulliken charge-shaded NC dimer nitrated at R1C2  }
	    \includegraphics[width=\linewidth]{NC2_0-CH3-CH3_mulliken_1}
    \end{subfigure}
        \begin{subfigure}[b]{0.7\linewidth}
   		\caption{Mulliken charge-shaded NC trimer nitrated at R1C2 and R3C2}
	    \includegraphics[width=\linewidth]{adj_NC2_0_2-CH3-CH3_mulliken_3}
    \end{subfigure}
    \caption{\textbf{a)} \ac{NC} monomer nitrated at C2. \textbf{b)} Monomer a) shaded by Mulliken charges. More negative atoms are shaded in red whilst more positive atoms are shaded in green.}
  \label{fig:mulliken}
\end{figure}

%this is the Adj one, as matching the diagrams given (missing the CH3)
\begin{table}[ht]
\begin{center}
\caption{Comparison of Mulliken charges for the \acs{CH3/CH3} monomer and the first glucopyranose ring (R1) of the dimer and trimer, all nitrated in the C2 position. Charges are expressed in \acs{au}. For explanantion of labelling scheme, see section \ref{sect:labellingsystem}. Shaded values indicate charge differences greater than \num{+-0.1} a.u.}
\begin{tabular}{ l S S S| S S} %{S[table-format=3.2]}
\toprule
{R1 Atom} & {Monomer} & {Dimer} & {Trimer} & {$\Delta$Trimer-Dimer} & {$\Delta$Trimer-Monomer} \\
\midrule
%_adj
%O1 & -0.08859 & -0.03121 & -0.01065 & 0.02056 & 0.07793 \\
%C1 & 0.18272 & -0.30481 & 0.01888 & \color{purple} 0.32369 & \color{orange}  -0.16384 \\
%C1H & 0.13625 & 0.13965 & 0.14254 & 0.00289 & 0.00629 \\
%C1 O-capping & -0.15073 & 0.15084 & 0.20754 & 0.05670 & \color{orange}  0.35827 \\
%C2 & -0.62807 & -0.32240 & -0.58978 &\color{purple}  -0.26738 & 0.03829 \\
%C2H & 0.22606 & 0.23654 & 0.24347 & 0.00693 & 0.01741 \\
%N1Ox & 0.16183 & 0.28293 & 0.19557 & -0.08735 & 0.03374 \\
%N1 & -0.27555 & -0.38563 & -0.30544 & 0.08020 & -0.02989 \\
%N1Ot1 & 0.02659 & 0.05080 & 0.05265 & 0.00185 & 0.02606 \\
%N1Ot2 & 0.05490 & 0.08088 & 0.08833 & 0.00746 & 0.03343 \\
%C3 & -0.17080 & -0.22297 & -0.33778 & \color{purple} -0.11481 &\color{orange}  -0.16698 \\
%C3H & 0.19218 & 0.19362 & 0.19379 & 0.00017 & 0.00161 \\
%C3O & -0.24529 & -0.19800 & -0.20326 & -0.00526 & 0.04203 \\
%C3OH & 0.25402 & 0.24757 & 0.24985 & 0.00228 & -0.00417 \\
%C4 & -0.09334 & -0.44991 & -0.29786 & \color{purple} 0.15205 &\color{orange}  -0.20452 \\
%C4H & 0.15766 & 0.19473 & 0.19580 & 0.00107 & 0.03814 \\
%C4 O-capping & -0.09986 & -0.11619 & -0.07430 & 0.04189 & 0.02556 \\
%C5 & -0.29708 & 0.14055 & 0.03670 & \color{purple} -0.10385 &\color{orange}  0.33378 \\
%C5H & 0.17896 & 0.16438 & 0.16666 & 0.00229 & -0.01230 \\
%C6 & -0.23716 & -0.43021 & -0.35941 & 0.07080 &\color{orange}  -0.12225 \\
%C6H1 & 0.15456 & 0.16628 & 0.17585 & 0.00957 & 0.02129 \\
%C6H2 & 0.16665 & 0.16315 & 0.15878 & -0.00436 & -0.00787 \\
%C6O & -0.30631 & -0.25866 & -0.25610 & 0.00257 & 0.05022 \\
%C6OH & 0.24562 & 0.26001 & 0.25645 & -0.00356 & 0.01083 \\
O1 & -0.08859 & -0.03121 & -0.00662 & 0.02459 & 0.08197 \\
C1 & 0.18272 & -0.30481 & 0.01734 & \color{purple} 0.32216 & \color{purple}-0.16538 \\
C1H & 0.13625 & 0.13965 & 0.14343 & 0.00378 & 0.00718 \\
C1 Ocap & -0.15073 & 0.15084 & 0.21080 & 0.05996 &\color{purple} 0.36153 \\
C2 & -0.62807 & -0.32240 & -0.58425 & \color{purple} -0.26184 & 0.04382 \\
C2H & 0.22606 & 0.23654 & 0.24326 & 0.00672 & 0.01720 \\
N1Ox & 0.16183 & 0.28293 & 0.19402 & -0.08890 & 0.03219 \\
N1 & -0.27555 & -0.38563 & -0.30347 & 0.08217 & -0.02792 \\
N1Ot1 & 0.02659 & 0.05080 & 0.05392 & 0.00312 & 0.02733 \\
N1Ot2 & 0.05490 & 0.08088 & 0.08682 & 0.00594 & 0.03191 \\
C3 & -0.17080 & -0.22297 & -0.34537 & \color{purple} -0.12240 & \color{purple}-0.17457 \\
C3H & 0.19218 & 0.19362 & 0.19384 & 0.00022 & 0.00166 \\
C3O & -0.24529 & -0.19800 & -0.20316 & -0.00515 & 0.04214 \\
C3OH & 0.25402 & 0.24757 & 0.24983 & 0.00226 & -0.00419 \\
C4 & -0.09334 & -0.44991 & -0.29491 &  \color{purple}0.15499 &\color{purple} -0.20157 \\
C4H & 0.15766 & 0.19473 & 0.19542 & 0.00069 & 0.03776 \\
C4 Ocap & -0.09986 & -0.11619 & -0.07415 & 0.04204 & 0.02571 \\
C5 & -0.29708 & 0.14055 & 0.03937 & \color{purple} -0.10118 & \color{purple}0.33644 \\
C5H & 0.17896 & 0.16438 & 0.16634 & 0.00197 & -0.01262 \\
C6 & -0.23716 & -0.43021 & -0.36154 & 0.06868 &\color{purple} -0.12438 \\
C6H1 & 0.15456 & 0.16628 & 0.17551 & 0.00923 & 0.02095 \\
C6H2 & 0.16665 & 0.16315 & 0.15838 & -0.00476 & -0.00827 \\
C6O & -0.30631 & -0.25866 & -0.25654 & 0.00212 & 0.04978 \\
C6OH & 0.24562 & 0.26001 & 0.25692 & -0.00308 & 0.01131 \\
\bottomrule
\end{tabular}
\label{tab:mulliken}
\end{center}
\end{table}

Though Mulliken charges are highly variable with basis set and the inclusion of diffuse functions, %REF
they may offer a general overview of the relative partial charge distribution in the molecule. 
Figure \ref{fig:mulliken} and table \ref{tab:mulliken} display the Mulliken charges of each of the different sized models. The large difference in charges for the C1 capping oxygen between monomer and the larger models is due to the absence of a second connected ring. Carbon sites within the glucopyranose ring present the greatest variation between each model. 
Attention to the changes in orientation of the side chains offer explanation for the observed charge distribution. The orientation of the C4 capping group is slightly altered for the dimer, as compared to the monomer and trimer, leading to the more negative charge at C4 for the dimer. % of the C6 hydroxyl group also  
The geometry of the nitrate at the C2 site is also most similar for the monomer and trimer, explaining the contrast between the charge concentration at C2 for these two species and the dimer. The Mulliken charges are therefore extremely sensitive to fine alterations in the geometry and should not be taken as a reliable measure for the charge distribution.

%Mulliken charges are made worse by diffuse basis functions. Appaz AIM is more reliable.
Whilst the monomer lacks the essential interactions between rings and side chains present in the polymer, when exploring the chemistry of a localised group of atoms - such as in the case of the denitration reaction - it is a suitable approximation for a primary investigation into the mechanistic details at the reaction site. The validated mechanisms can then be applied to the dimer and wider systems to re-introduce ring-ring interactions and higher steric effects. 


%If the other trimer I get has a similar energy but completely different geomtry, can tabulate it and comment on how many degrees of freedom - difficult to find local minimum, even with initial MM optimisation. This counts for dimer too


\section{Summary}
In this section the polymer structure of \ac{NC} was truncated to a system size suitable for application of density functional theory, for use in subsequent investigation into the decomposition mechanisms. Methoxy and hydroxyl capping group approximations were compared. \ac{ESP} maps showed that the methoxy group performed better, exhibiting a charge density more similar to that of the glycosidic oxygen in larger polymer chain units. %, than that of the hydroxyl group. 
Topological analysis using \ac{QTAIM} highlighted non-bonding interactions between both hydroxyl and methoxy chain endings with side chains of the dimer. Findings showed that the methoxy capping group provided a more sterically and chemically similar approximation for the extended polymer, and was adopted for work in further chapters.

Topological analysis of monomer, dimer and trimer models found that the 2-ring and 3-ring systems were the most consistent %with chemistry of the polysaccharide, 
with respect to charge density distribution and profile of intramolecular interactions. It was assumed that the largest model would be most representative of the polysaccharide, as suggested in by Shukla \textit{et al.}\cite{Shukla2012a}.
%This suggested that the larger models were more representative of the wider polymer. %%%%%%%%%%
Despite the absence of inter-ring non-bonding interactions, the monomer model was deemed sufficient for initial investigations into the chemistry of \ac{NC}. Further studies aimed to probe the chemistry at individual nitrate sites on the ring by analysis of reaction energies and transition states. Medium-to-long range interactions present less significance during these initial mechanistic investigations at localised sites. By contrast, the additional time required for exploration of more intricate \ac{PES} would be significant for the larger models. 
%This is of particular importance when considering transition state searches and reaction-complex geometry optimisations. 
The exclusion of secondary non-bonding interactions %from the additional sites on larger models, 
and reduction in the degrees of freedom due to truncation of the model size simplifies and speeds-up structure searches and identification of the \ac{MEP} from reactants to products. Thus the monomer model will be used for further studies involving single-site reaction mechanisms and transition state searching.

%When considering electroststic potential and to a lesser extent steric considerations, methoxy groups were found to provide a better approximation for the extended polysaccharide with respect to both electronic and geometric properties. 

