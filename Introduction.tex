\chapter{Introduction}
\label{chapterlabel1}
\graphicspath{ {./Intro_pics/} }

\section{History of Nitrocellulose}
What is \ac{NC}.

Background story and industrial applications and properties of \ac{NC}.
Briefly explain why we need to know more about its degradation processes. 

yada yada yada
Some stuff about things.\cite{example-citation} Some more things. 
Inline citation: \bibentry{example-citation}

\begin{figure}[htp]
\centering
%\includegraphics[width=\textwidth]{NC}
\includegraphics[scale=0.5]{NC}
\caption{The structure of Nitrocellulose.}
\label{fig:NCstructure}
\end{figure}

Taken from Rigas, Sebos, Doulia, 1997:
"The nitration of organic compounds is a potentially
dangerous process, since nitrations are exothermic
reactions producing under controlled conditions explosive
substances. Many accidental explosions of nitration
industrial plants have been reported in the literature
(Evans et al., 1977; Automatic Process of Handling
Spent Acids in the Manufacture of EGDN, 1987; Urbanski,
1965; Van Dolah, 1963; Fritz, 1969). Nitric
esters (e.g., nitroglycerine and nitroglycol), unlike other
common explosives such as TNT, are prone to turn very
unstable on prolonged contact with acids. Thus, many
accidents have occurred during the manufacture of
nitroglycerine and nitroglycol, particularly while handling
or storing spent acid (residual acid after the
nitration process) (Automatic Process of Handling Spent
Acids in the Manufacture of EGDN, 1987; Urbanski,
1965).
The separation of nitric esters from spent acid and
their accumulation in unexpected places (e.g., transfer
lines, drains, and collecting areas), building up dangerous
quantities over a period of time, are the main
reasons of accidents taking place at nitric ester production
units. Since acidic nitric esters are chemically
unstable and sensitive mixtures, all possibilities of
unintentional accumulation must be avoided. The accumulation
of nitroglycerine (NG) or nitroglycol (EGDN)
may be attributed to a variety of reasons (Federation
of European Explosives Manufacturers, 1988), namely,
the following...."

\section{Synthesis and structure}
Include the structure of cellulose. 
% This just dumps some pseudolatin in so you can see some text in place.
%\blindtext
Include spectroscopic studies here
Include structural studies of nitrocellulose here.
See https://www.sciencedirect.com/science/article/pii/0032386190900484, 
https://www.sciencedirect.com/science/article/pii/0032386184900120 and https://www.sciencedirect.com/science/article/pii/0032386187902539\hspace{4pt}.

\section{Reactions of Nitrate Esters}

\section{Nitrocellulose degradation}
Review of Degradation studies
Intro to experimental studies - When, why, what, where, who and how. 
In subsequent sections, make sure to go into the "so what?"


\subsection{Industrial methods}
Include  the industrial decomposition methodologies here (biological etc.)
Include an overview of all, but keep the alkaline and acid 

Discuss Alkaline hydrolysis. However we don't want this as a big thing, as we want this lit review tailored to the study, with the equivalent emphasis.
More detail on the studies that look at this.
Summary of all alkaline hydrolysis reactions, including kinetics.
Include computational studies where relevant.

\subsection{Acid hydrolysis}
All the detail in literature that you can find on this. 
Summary of all acid hydrolysis reactions, including kinetics.
Include computational studies where relevant.
Elaborate on how acid is always present in the system and may be the main contributor to accelerated ageing.
Why is this important to know about?



\subsection{Post-denitration reactions}
What do the hydrolysi products go off to do. 
Include computational studies where relevant.

\section{Motivation}
Point out the areas that are lacking in our knowledge, why we hit the limitations of experiment and why computational studies are required. 