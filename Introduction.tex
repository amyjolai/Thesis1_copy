\chapter{Introduction}
\label{chapterlabel1}
\graphicspath{ {./Intro_pics/} }

\section{History of Nitrocellulose}
What is \ac{NC}.

Background story and industrial applications and properties of \ac{NC}.
Briefly explain why we need to know more about its degradation processes. 

Some stuff about things.\cite{example-citation} Some more things. 
%Inline citation: \bibentry{example-citation}

\begin{figure}[htp]
\centering
%\includegraphics[width=\textwidth]{NC}
\includegraphics[scale=0.5]{NC}
\caption{The structure of Nitrocellulose.}
\label{fig:NCstructure}
\end{figure}

Taken from Rigas, Sebos, Doulia, 1997:
"The nitration of organic compounds is a potentially
dangerous process, since nitrations are exothermic
reactions producing under controlled conditions explosive
substances. Many accidental explosions of nitration
industrial plants have been reported in the literature
(Evans et al., 1977; Automatic Process of Handling
Spent Acids in the Manufacture of EGDN, 1987; Urbanski,
1965; Van Dolah, 1963; Fritz, 1969). Nitric
esters (e.g., nitroglycerine and nitroglycol), unlike other
common explosives such as TNT, are prone to turn very
unstable on prolonged contact with acids. Thus, many
accidents have occurred during the manufacture of
nitroglycerine and nitroglycol, particularly while handling
or storing spent acid (residual acid after the
nitration process) (Automatic Process of Handling Spent
Acids in the Manufacture of EGDN, 1987; Urbanski,
1965).
The separation of nitric esters from spent acid and
their accumulation in unexpected places (e.g., transfer
lines, drains, and collecting areas), building up dangerous
quantities over a period of time, are the main
reasons of accidents taking place at nitric ester production
units. Since acidic nitric esters are chemically
unstable and sensitive mixtures, all possibilities of
unintentional accumulation must be avoided. The accumulation
of nitroglycerine (NG) or nitroglycol (EGDN)
may be attributed to a variety of reasons (Federation
of European Explosives Manufacturers, 1988), namely,
the following...."

\section{Synthesis and structure}
Nitrocellulose is dervied from cellulose. Cellulose is a naturally occurring polysaccharide found primarily in plant cell walls. It is the main component of cotton (FIGX). %cotton plant
The polysaccharide chain is comprised of glucose units linked via \textbeta-1,4-glycosidic bonds, resulting in the alternating orientation of the individual monomers (FIGX). A single strand of cellulose is estimated to contain between 100 - 200 glucose units, representing a \ac{MW} of 20,000-40,000 daltons.
% REF A review of the synthesis, chemistry and analysis of nitrocellulose, Saunders, 1990
%Include spectroscopic studies here
%Include structural studies of nitrocellulose here.
%See https://www.sciencedirect.com/science/article/pii/0032386190900484, 
%https://www.sciencedirect.com/science/article/pii/0032386184900120 and \\
%https://www.sciencedirect.com/science/article/pii/0032386187902539\hspace{4pt}.

\section{Reactions of Nitrate Esters}

\section{Nitrocellulose degradation}
Review of Degradation studies
Intro to experimental studies - When, why, what, where, who and how. 
In subsequent sections, make sure to go into the "so what?"


\subsection{Industrial methods}
Include  the industrial decomposition methodologies here (biological etc.)
Include an overview of all, but keep the alkaline and acid 

Discuss Alkaline hydrolysis. However we don't want this as a big thing, as we want this lit review tailored to the study, with the equivalent emphasis.
More detail on the studies that look at this.
Summary of all alkaline hydrolysis reactions, including kinetics.
Include computational studies where relevant.

\subsection{Acid hydrolysis}
All the detail in literature that you can find on this. 
Summary of all acid hydrolysis reactions, including kinetics.
Include computational studies where relevant.
Elaborate on how acid is always present in the system and may be the main contributor to accelerated ageing.
Why is this important to know about?



\subsection{Post-denitration reactions}
What do the hydrolysis products go off to do. 
Include computational studies where relevant.

\section{Motivation}
Despite its long history, \ac{NC} is still an essential ingredient in many propellant and lacquer formulations. 
%Its versatility means that we come into contact with it in some shape or form everyday, in printing ink we use, tupperware etc. 
%NOTE: it might be that they substituted NC for other reasons too. Bohn didnt say, but find out. 
Efforts have been made to substitute it with other polymeric binders in attempt to reduce the manufacturing risk it poses due to its volatility, but this has only been partly successful. %(REF 11-dr-m-bohn-ict-stability-decomp-ageing.pdf (Bohn, 2007) Give example where possible? Maybe these should be saved for the text above.). 
Insufficient understanding of the internal processes within nitrocellulose has lead to accidents in the past, sometimes resulting in lives lost. %(REF Tianjin, and AWE fire report). 
It is therefore imperative that we seek to clarify our understanding of the ageing mechanisms to inform the reduction of associated risks, whilst more effectively preserving existing \ac{NC} stock.
%, strive to eliminate associated risks,  

Experimental studies over the past 100 years %REF%
have shed light on the macroscopic degradation behaviour of bulk \ac{NC}. However, the fine mechanistic details of degradation have only been alluded to, and are as yet unvalidated. 

In this study we will elucidate the dominant degradation schemes in \ac{NC} with scrutiny of previously proposed decomposition pathways, and present new mechanistic considerations.
% what about generating our own new mechanisms?
%In this study we will examine / scrutinise previously proposed decomposition pathways in order to elucidate the dominant degradation schemes in \ac{NC}.
This will be achieved via the application of computational techniques to give insight where it has been restricted by the limitations of laboratory experimentation in the past. %the resolution of


%Point out the areas that are lacking in our knowledge, why we hit the limitations of experiment and why computational studies are required. 

