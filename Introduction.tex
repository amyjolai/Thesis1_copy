\chapter{Introduction}
\label{chapterlabel1}
\graphicspath{ {./Intro_pics/} }

\section{The modern history of Nitrocellulose}

%Re-write all this better. And make it fun please. \\
\ac{NC}, cellulose nitrate, or "guncotton" is a nitrated cellulose derivative that has been widely utilised in the manufacture of plastics, inks, propellant formulations and thin films
%"micro-electronics as a resist material" 
%a low-order explosive 
since it's discovery in 1833. [PICTURE!] %(REF R Shorts' thesis, see references 1-5). 

Cellulose is the primary component of plant cell walls (REF). It was found (by who?) the when cellulose fibres, in the form of sawdust, cotton (linen) and paper were treated with nitric acid, (it was discovered that) the product burned rapidly and in the absence of the thick black smoke that was characteristic of gunpowder of the era.%REF E. C. Wordon and E. C. Worden, Nitrocellulose Industry, 1911
%OR
In his Berlin laboratory in 1845, Dr. Schönbein serendipitously discovered that acid treated cotton wool burned violently without the accompanying black smoke that was typical of gunpowders of the time. %REF A. S. Greenberg and S. Broder, J. Pat. Off. Soc., 1925
He was granted a U.S. patent the following year. 

Today \ac{NC} is present in adhesives, paints and lacquer coatings. Varieties of high percentage nitrogen by weight (\%NMax) are used for its explosive properties, such as in rocket and gun propellants.  Since  Schönbein’s day \ac{NC} has seen applications in dynamite, artificial silks, printing inks and reels of old film, which were famously flammable. %REF M. Edge, N. S. Allen, 1990 
Collodin, a solution of \ac{NC} in either ether or ethanol, is used both in surgical dressings and in theatrical make-up. The non-flexible variety is applied to skin, which puckers as the solvent evaporates, effectively creating the appearance of scarred tissue. [PICTURE!]


Cellulose is abundant in nature, however its biological variation has implications on the quality and properties of the \ac{NC}it produces. %[REF of miscanthus cellulose SEM - caption "SEM of (Italic) Miscanthus (un_    Italic) cellulose fibres (a) and following nitration (b). The synthesized NC   
During synthesis the backbone of cellulose is largely preserved (Figure conversion of C to NC).%REF A. S. Greenberg and S. Broder, J. Pat. Off. Soc., 1925 and H. J. Prask, C. S. Choi, R. Strecker and E. Turngren, Woodpulp Crystal Structure 
Wood pulp and cotton linters are two major sources of starting material for \ac{NC} production; though more expensive, cotton linters are preferred for their uniformity. 


\ac{NC} of above 12.5 \%NMax is considered explosive. The theoretical maximum level of nitration is 14.14 \%NMax, corresponding to a \ac{DOS} of three, where every free hydroxyl group has been replaced by a nitrate ester. In practice only 13.6 \%NMax has be achieved. %REF Dow Wolff Cellulosics, Walsroder® Nitrocellulose Essential 1998
When \ac{NC} is the only energetic component in an explosive mixture, it is termed a single-base propellant. \ac{NC} mixtures with one or more energetic materials such as \ac{NG}, where \ac{NG} also acts as a plasticiser, are classed as double or triple-base propellants. Due to the degradation of \ac{NC} over time, stabilisers are necessary to neutralise the decomposition products that facilitate further reactions.


% Background story and industrial applications and properties of \ac{NC}.
% Briefly explain why we need to know more about its degradation processes. 

Some stuff about things.\cite{example-citation} Some more things. 
%Inline citation: \bibentry{example-citation}

\begin{figure}[htp]
\centering
%\includegraphics[width=\textwidth]{NC}
\includegraphics[scale=0.5]{NC}
\caption{The structure of Nitrocellulose.}
\label{fig:NCstructure}
\end{figure}

% Taken from Rigas, Sebos, Doulia, 1997:
% "The nitration of organic compounds is a potentially
% dangerous process, since nitrations are exothermic
% reactions producing under controlled conditions explosive
% substances. Many accidental explosions of nitration
% industrial plants have been reported in the literature
% (Evans et al., 1977; Automatic Process of Handling
% Spent Acids in the Manufacture of EGDN, 1987; Urbanski,
% 1965; Van Dolah, 1963; Fritz, 1969). Nitric
% esters (e.g., nitroglycerine and nitroglycol), unlike other
% common explosives such as TNT, are prone to turn very
% unstable on prolonged contact with acids. Thus, many
% accidents have occurred during the manufacture of
% nitroglycerine and nitroglycol, particularly while handling
% or storing spent acid (residual acid after the
% nitration process) (Automatic Process of Handling Spent
% Acids in the Manufacture of EGDN, 1987; Urbanski,
% 1965).
% The separation of nitric esters from spent acid and
% their accumulation in unexpected places (e.g., transfer
% lines, drains, and collecting areas), building up dangerous
% quantities over a period of time, are the main
% reasons of accidents taking place at nitric ester production
% units. Since acidic nitric esters are chemically
% unstable and sensitive mixtures, all possibilities of
% unintentional accumulation must be avoided. The accumulation
% of nitroglycerine (NG) or nitroglycol (EGDN)
% may be attributed to a variety of reasons (Federation
% of European Explosives Manufacturers, 1988), namely,
% the following...."

%--------------------------------------------------------------------------------------------------------------------------------------------------------------------------------------------%
\section{Applications}
*Talk about the renewed interest in NC - see:

https://medium.com/@pandevedashri/global-nitrocellulose-market-is-estimated-to-grow-at-a-cagr-of-6-5-during-the-period-2019-2025-800e2942f316 

and

https://medium.com/photography-and-film/thoughts-on-celluloid-c1a6744312bb

Draw a nice info graphic of stats all the uses of NC, and a paper interest graph demonstrating the increased mention of Nitrocellulose in publications, since the war. 

(thought the latter is more a niche thing)*
\ac{NC} has been used widely in manufacturing, as a feedstock for lacquers, coatings, propellants and ....., It continues to be used in industry today, due to it's low cost and versatility.  However, new 
\subsection{(Explosive,  non-explosive}
Cf. plane seat ejector propellant formulation - important to know whether it can withstand humidity and temperature cycling. 


%--------------------------------------------------------------------------------------------------------------------------------------------------------------------------------------------%
\section{Synthesis and structure}
Nitrocellulose is dervied from cellulose. Cellulose is a naturally occurring polysaccharide found primarily in plant cell walls. It is the main component of cotton (FIGX). %cotton plant
The polysaccharide chain is comprised of glucose units linked via \textbeta-1,4-glycosidic bonds, resulting in the alternating orientation of the individual monomers (FIGX). A single strand of cellulose is estimated to contain between 100 - 200 glucose units, representing a \ac{MW} of 20,000-40,000 daltons.
% REF A review of the synthesis, chemistry and analysis of nitrocellulose, Saunders, 1990
%Include spectroscopic studies here
%Include structural studies of nitrocellulose here.
%See https://www.sciencedirect.com/science/article/pii/0032386190900484, 
%https://www.sciencedirect.com/science/article/pii/0032386184900120 and \\
%https://www.sciencedirect.com/science/article/pii/0032386187902539\hspace{4pt}.

cf James Tucker - A whole life assessment of extruded double base rocket propellants. - NC chapter on synthesis

\section{Reactions of Nitrate Esters}
Reactions of all nitrate esters in general, not just NC.

\subsection{Thermolytic reactions}
Incorporate all computational, experimental studies, with mechanistic detail where possible and reference to other industrially methods at the end.

\subsection{Hydrolysis reactions}

\section{Nitrocellulose degradation}
Now, which of the above are relevant to NC, and what additional ones happen in NC / are unique to NC?

Think - secondary reactions, synergy between different nitrate groups and other properties linked to the degradation of sugar / organic polymers. 

Review of Degradation studies
Intro to experimental studies - When, why, what, where, who and how. 
In subsequent sections, make sure to go into the "so what?"

\ac{NC} is extremely flammable when dry but is largely insoluble in water, and is usually kept in solvent to prevent detonation when under storage.18,19 Mixed with at least 25 \% of water or alcohol, \ac{NC} is completely stabilised.%REF F.-J. Kim, Byung J. ; Hsieh, Hsin-Neng ; Tai, Anaerobic Digestion
Accidents [He He Liu 2017 has a reference to an accident at TianJin, but you can find the news article]
This has facilitated the use of aqueous solvents as dispersion and transport mediums in manufacturing processes, leading to small \ac{NC} fibres, or “fines”, in output streams. Demilitarisation activities have resulted in large volumes of \ac{NC} waste which regulatory action now prevents the disposal of via incineration.
%REF J. Wilkinson and D. Watt, 2006, J. E. Alleman, B. J. Kim, D. M. Quivey 1994, N. Auer, J. N. Hedger and C. S. Evans, Biodegradation 2005, J. Kim, J. K. Park and L. W. Clapp, Characterization of Nitrocellulose Fine
A high level of side group substitution relative to cellulose, in addition to its insolubility, contributes to resistance to microbial degradation.
%REF no idea what:3,939,068, 1975, 6–9, M. Duran, B. J. Kim and R. E. Speece, Waste Manag., 1994,
As a result, \ac{NC} exhibits poor mobility and a long lifetime in the environment. 
A detailed understanding of the degradation mechanisms is imperative in the design of efficient and economical processes for \ac{NC} disposal. Whilst many studies in the past century have shed light on the various decomposition schemes, a review of the literature (section 2.1) [CLEAN THIS UP] reveals conflicting conclusions on the general mechanisms of decomposition via alkaline hydrolysis, acidic hydrolysis, thermolysis and biodegradation treatments. Most notably, a distinct lack of mechanistic detail hinders optimisation of the practised treatments, or effective redirection of degradation products to more useful substances, such as plant fertiliser. %REF  J. Wilkinson and D. Watt, 2006,F. H. Bissett and L. A. Levasseur, Chemical Conversion of Nitrocellulose

\subsection{Industrial disposal methods}
This can be a tack on, at the end to highlight the relevance of the NC system, applications and current modes of usage.  

Include  the industrial decomposition methodologies here (biological etc.)
Include an overview of all, but keep the alkaline and acid 

Discuss Alkaline hydrolysis in the context of industrial process. However we don't want this as a big thing, as we want this lit review tailored to the study, with the equivalent emphasis.
More detail on the studies that look at this.
Include computational studies where relevant(maybe making reference to the computational section) .

[HAVE TO CLEAN UP]
The alkaline hydrolysis reactions of NC have undergone thorough study due to their central role in large-scale NC disposal from manufacturing waste streams. It was realised early on by Kenyon and Gray that the action of alkalis on NC did not just yield cellulose and the corresponding metal nitrate salt, but a whole mixture of highly variable organic and inorganic products.28 Miles observed that whilst acid and alkaline hydrolysis both produced similar products, the rate of the latter reaction was much greater.29 
Typically, NC is treated with concentrated sodium hydroxide and heating. Though other alkalis maybe used, such as. barium hydroxide, calcium hydroxide or sodium carbonate, these alternative alkalis require longer contact and heating times in addition to a larger measure of alkali to solubilise the NC.8 In 1976 Wendt and Kaplan reported that a sodium hydroxide solution of 3 \% by weight at 95 oC was able to effectively degrade NC of 12.6-13.4 \%NMax to completion, in only 30 minutes. Su and Christodoulatos observe a similar rate, with 90 \% of digestion occurring within 35 minutes, in 2 \% sodium hydroxide at 70 oC for NC of 12.2 \% NMax.30  
Multiple studies demonstrate that microbial biodegradation is an effective method of degrading NC.31 Wendt and Kaplan’s study presented a bench-scale continuous treatment process involving initial degradation by alkaline hydrolysis followed by biological digestion using naturally occurring microbes in raw wastewater, for which they acquired a patent a year earlier25
It was noted by Fan et al. that highly substituted cellulose derivatives are resistant to direct biodegradation and require pre-treatment. Work on the enzymatic hydrolysis of cellulose revealed that increasing the crystallinity of the substrate reduced its digestibility.32 This is reinforced in the study by Mittal et al. involving alkaline and liquid-ammonia treatments on crystalline cellulose.33 Substitution of hydroxyl groups to increase solubility of cellulose derivatives lead to increased enzymatic hydrolysis, up to complete solubilisation. After this point, susceptibility slowly decreased. As solubility was no longer the limiting property, further increase in substitution began to inhibit digestion. It may be that as the levels of substitution increase, polymer geometries become less suited to the active sites of enzymes involved in degradation. 
Kim et al. explored the possibility of combining acid hydrolysis with biodegradation on an industrial scale. Feasibility of the hydrolysis reaction using hydrochloric acid with heat, was thoroughly investigated.20 The study assumes that the nitrate group of C6 reacts the fastest, reasoning that groups on C2 and C3 experience more steric hindrance. A much later computational study by Shukla disagrees, finding that the C3 nitrate is the first to be liberated.3 Kim achieved denitration by treatment with acid more dilute than that initially used to synthesise the NC sample. A lower concentration of acid promoted shift of the equilibrium towards denitration. Though glucose was the major product of the monitored denitration process, it was stated that the rate of denitration was more rapid than the peeling-off reaction. In theory, it should be possible to regenerate strands of cellulose under the appropriate conditions. However, results only focussed on the retrieval of end-stage hydrolysis products such as glucose, nitrates, nitrites and ammonia. In addition, acid catalysed side reactions may introduce complications. 
Though less studied than alkaline hydrolysis, the acid hydrolysis mechanism is extremely important for the understanding of decomposition and ageing processes of NC, as a residual amount of acid always remains in the system. Products liberated via other breakdown routes also go on to form acidic species within the environment and contribute to further decomposition via secondary reactions and autocatalysis.34  

\subsection{(Computational Studies)}
A review of published works yields very few NC mechanistic degradation studies involving computational methods. Due to the nature of the topic, many of the referenced works above are linked with private or military organisations. Thus, their release may be restricted until declassified many years later, or completely withheld. The timeline of much of the material reviewed here ranges from mid to late 1900s, before the widespread use of computational methods in chemical research. Consequently, it is to be expected that more NC computational studies will appear in the near future.
Shukla et al. performed a series of mechanistic studies on 2,3,6-trinitro-β-glucopyranose (FIG. 3), exploring its validity as monomer model for the NC polymer during alkaline hydrolysis via the saponification route.3,4,35 If the hydrolysis reaction were to proceed via the saponification route, in practice it is unlikely that all nitro groups are liberated simultaneously. 
The work examined the possible sequences of nitrate group removal for a fully nitrated NC polymer in the hydrolysis reaction, in addition to whether denitration, depolymerisation or rupture of the ring would initiate. It was reported that nitrate groups would undergo an SN2 substitution reaction whereby each nitrate ester group is replaced by the incoming hydroxide nucleophile, liberating a nitrate ion. 
Calculated enthalpies and free energies indicated a C3 $\rightarrow$  C2 $\rightarrow$  C6 sequence of denitration for the monomeric unit. Their later work involving the dimer and trimer models (FIG. 4.) suggests a significant change in behaviour when scaling up from the monomer. Dimer and trimer activation energies proved comparable, but showed an inconsistency with the monomer, instead exhibiting a C3 $\rightarrow$  C6 $\rightarrow$  C2 sequence.
There is the problem that the latter sequence was derived only from the energies of the initial nitrate removal step, whereas the monomer sequence was derived from energies calculated at each of the three stages of denitration. As there is little disparity between the dimer and trimer energies, but a large difference to the monomer results, it was said that the dimer should be the smallest unit used to describe the alkaline hydrolysis behaviour of NC.
Though not a study involving NC, detailed work on the nitrate ester degradation routes in pentaerythritol tetranitrate (PETN) was conducted by Tsyshevsky, Sharia and Kuklja (Figure 4).10 PETN is an energetic material possessing four nitroester moieties. Similarly to NG, it is used as both an explosive and in medicine.

Existing experimental activation energies for decomposition were scattered in the range of 30 to 70 kJ mol-1. The group attributed the dispersion in data to inconsistent procedures and experimental conditions between studies. 
The most common degradation products recorded were CO, CO2, NO, N2O, CH2O, HCN, and HNCO. Echoing earlier NC and other nitroester thermolysis studies,36 the first degradation step was assumed to be nitrate ester homolytic fission. The seven mechanisms explored, corresponding to the labels in Figure 4, are detailed in equations (1) to (7):
 (1)  (2) (3) (4) (5) (6) 						 (7)
 (1) homolytic cleavage of the O-NO2 bond, (2) the elimination of nitrous acid (HONO) which is usually considered as a competing reaction to homolytic fission, (3) the nitro-nitrite rearrangement (OONO), (4) γ-attack, the encounter of a peripheral oxygen atom and the central C atom (5) the homolytic C-O bond cleavage and (6), (7) two variations of the homolytic C-C bond cleavage. 
Activation barriers calculated using PBE, PBE0 and wB97XD were compared, finding that reaction (1) proceeded without a barrier and was therefore most favourable. Reaction (2) possessed an activation barrier of only approximately 6 kcal mol-1. However, this second reaction was also exothermic, and through bulk calculations it was determined the secondary reaction would accelerate global processes via self-heating. 
Analyses of the functionals used found that PBE consistently underestimated barrier height by 12 to 14 kcal mol-1 on average, with respect to the remaining methods used. Transition states were optimised within GAUSSIAN, followed by an intrinsic reaction co-ordinate computation for each mechanism.


Ring fission vs denitration / post denitration reactions. 

\subsection{Acid hydrolysis}
"Acid catalysed decomposition of sugar molecules in aqueous medium is initiated by the protonation of the hydroxyl groups." - Assary, Kim 2012
All the detail in literature that you can find on this. 
Summary of all acid hydrolysis reactions, including kinetics.
Include computational studies where relevant.
Elaborate on how acid is always present in the system and may be the main contributor to accelerated ageing.
Why is this important to know about?

\subsubsection{Autocatalysis}

[REF] suggests it is NO2+ that is the catalytic species, which is reinforce by Knight and Barry [REF] who found that the elevated catalytic degradation rate was not observed for cellulose acetate. (Though it is postulated to undergo similar hydrolysis steps? [REF]) 


\subsection{Post-denitration reactions}
What do the hydrolysis products go off to do. 
Include computational studies where relevant.

\section{Motivation}
Despite its long history, \ac{NC} is still an essential ingredient in many propellant and lacquer formulations. 
%Its versatility means that we come into contact with it in some shape or form everyday, in printing ink we use, tupperware etc. 
%NOTE: it might be that they substituted \ac{NC}for other reasons too. Bohn didnt say, but find out. 
Efforts have been made to substitute it with other polymeric binders in attempt to reduce the manufacturing risk it poses due to its volatility, but this has only been partly successful. %(REF 11-dr-m-bohn-ict-stability-decomp-ageing.pdf (Bohn, 2007) Give example where possible? Maybe these should be saved for the text above.). 
Insufficient understanding of the internal processes within nitrocellulose has lead to accidents in the past, sometimes resulting in lives lost. %(REF Tianjin, and AWE fire report). 
It is therefore imperative that we seek to clarify our understanding of the ageing mechanisms to inform the reduction of associated risks, whilst more effectively preserving existing \ac{NC} stock.
%, strive to eliminate associated risks,  

Experimental studies over the past 100 years %REF%
have shed light on the macroscopic degradation behaviour of bulk \ac{NC}. However, the fine mechanistic details of degradation have only been alluded to, and are as yet unvalidated. 

In this study we will elucidate the dominant degradation schemes in \ac{NC} with scrutiny of previously proposed decomposition pathways, and present new mechanistic considerations.
% what about generating our own new mechanisms?
%In this study we will examine / scrutinise previously proposed decomposition pathways in order to elucidate the dominant degradation schemes in \ac{NC}.
This will be achieved via the application of computational techniques to give insight where it has been restricted by the limitations of laboratory experimentation in the past. %the resolution of


%Point out the areas that are lacking in our knowledge, why we hit the limitations of experiment and why computational studies are required. 

