\chapter{Dominant decomposition mechanisms}
\label{chapterlabel3}

\section{Introduction}

Intro to the topic
Stepwise denitration, peeling off, or alternative reactions dominate? Which in which conditions?

Camera's equations and most others agree that first stage of degradation in the presence of "spent" (?) acids involves hydrolysis, removing the nitrate group and forming the alcohol. [Here just reiterate the reactions you will be looking at. The actual literature should already have been covered in the lit review in the intro.] 


\section{Computational details}
\ac{NC} monomer and dimer structures underwent an initial \ac{MM} geometry optimisation using the COMPASSII forcefield(REF), performed in % Materials materials modelling package. Talk about cycles, and other settings etc?
Materials Studio (version?). Subsequent \ac{QM} geometry optimisations to the level of 6-31+g(2df,p) / wB9X-D, except where otherwise stated, were performed in Gaussian 09 D.01 (REF). 

Programme, method (e.g. functional/ FF), settings(basis sets, thermostats, ensembles etc)

Include further calculation details - such as transition state theory etc, in another subsection (e.g. Transtion state theory) under this header.

Mention QTAIM if included in subsequent sections.

\section{Truncating the polymer model}
obvs mention Shukla's studies here.


Looked at QTAIM for interaction with capping groups

\section{Thermal decomposition mechanisms}
i.e not in acid
Whilst the primary focus of this study is to explore the action of acids in the aging processes of  \ac{NC} , the thermolytic degradation routes must also be considered. These pathways are confluent with the reactions in the acid hydrolysis pathway and will dominate at elevated temperatures, due to their intramolecular character and therefore rapid, nature.

\subsection{Homolytic fission}
Comment on how you did it for PETN first, as a tester (but put the results in the appendices?)
\subsection{Elimination of \ce{HNO2}}

\section{Acid hydrolysis mechanism}
\subsection{Protonation site}
Denitration vs peeling off reaction, based on protonation site - it may be that peeling off is more favourable, but doesn't lead to mixed level of nitration - you can explain this and take it into account. Just proceed with having explained this / full understanding on future considerations.
Thermodynamic numbers (in table)
Scans of water approaching NC with the H coming off



\subsection{Denitration}
Different DFT functionals and HF and MP2 (?)
Scans of the nitrate leaving the protonated NC. 2D scans of the water donating proton and nitrate leaving.



\section{Effects of acid concentration on the Degree of Substitution}
Phase diagram of monomer and dimer of acid conc vs nitrocellulose conc, for the denitration direction AND the nitration direction.
Come up with a sequence for nitration and denitration.
Comment on the effect of acid on the denitration / nitration behaviour of NC.

\section{Summary}
Found that Homolytic fission faster than elimination of HNO2, but the latter is more likely at etc... same as for PETN
Phase diagram shows that with the increase in acid concentration, the degree of substitution increases by some modelling factor
