% whilst you're still writing up
\setcounter{chapter}{1}

\chapter{Nitration and Denitration Sequence of Nitrocellulose}
\label{chapterlabel4}
\graphicspath{ {./R_chap_2_pics/} }

[Added as preamble for Chapter 4 below]

\section{Introduction}
\label{chapter4:intro}

%This section utlises the numbering scheme introduced in section \ref{sect:labellingsystem}.
Moniruzzaman \textit{et al.} used the UV absorption of an anthraquinone dye to determine the activation energies for the removal of the nitrate at C2, C3, C6 sites on \ac{NC} (figure \ref{fig:anthraquinone_SB59})\cite{moniruzzaman2014}.
The reaction of the released NO\textsubscript{x} with the \ac{SB59} mimics the action of stabilisers within \ac{NC} formulations. The dye consumes any nitrates released in the system, eliminating the possibility of further reactions generating acidic species. The presence of acids has been linked to autocatalytic rates of degradation in later stages of \ac{NC} decomposition\cite{Edge1990,FernandezdelaOssa2011,Baker1952,Binke1999}. %REF autocatalytic processes, and possibly evidence that stabilisers behave by absorbing NOx sepcies. Maybe even some examples of stabilisers that do this?
Un-aged \ac{NC} thin films and films aged at 40\textdegree C, 50\textdegree C, 60\textdegree C and 70\textdegree C for timescales of up to 2000hrs for 40\textdegree C, were compared.
%include their graph?
%[CHECK THIS - they also mention the nitration of the dye - they may have used the nitration of the dye at the marker] 
The \ac{NC} starting material was 12.15\%N by mass, indicating that individual glucopyranose rings were of non-uniform, mixed nitration level.  Mean \ac{DOS} was 2.307. 

\begin{figure}[ht!]
  \centering
	\includegraphics[width=0.35\linewidth]{anthraquinone_SB59}
\caption{\acf{SB59} used to probe the release of nitrates from \ac{NC} using UV-Vis spectroscopy and \textsuperscript{1}H NMR\cite{moniruzzaman2014}. %The action of nitrate absorption by the dye imitates that of stabilisers commonly used with nitrate ester formulations. 
Following reaction with NO\textsubscript{x} the UV absorption peak of the dye shifts, indicating the extent of denitration and the presence of secondary reaction products.}
\label{fig:anthraquinone_SB59}
\end{figure}

%Initial conclusions from literature considering both experimental and QM studies
The study found that the nitrate at the C3 position would be most reactive, possessing the lowest activation barrier to removal \textit{via} thermolysis. This was followed by C2 and C6. %, with the highest activation energy and deemed most stable to removal.  %,  based on what data/ conclusions? Gotta click through the literature rabbit holes 
The findings contrast with the work of Shukla \textit{et al.} which determined that denitration via alkyaline hydrolysis followed the order of C3$\rightarrow$C6$\rightarrow$C2 \cite{Shukla2012,Shukla2012a}. However, Shukla's study considered only the fully nitrated system. There is evidence that nitration and denitration is influenced by the presence of nitrate groups at adjacent positions. Matveev \textit{et al.} demonstrated that for polynitro esters the rate of liquid-phase decomposition did not increase linearly with number of nitrate reaction centres.\cite{Matveev2003}. It was suggested that the trend in reactivity could be explained by the inductive effect of nitrate groups. %, which influences reactivity based on the proximity of the groups. 
The inductive effect arises when a difference in the electronegativity between atoms connected by a $\textsigma$-bond causes a polarisation, or permanenet dipole, in the bond. Electron donating groups% are of lower electronegativity than  
increase the $\textdelta$- partial charge on neighbouring atoms through the release of electrons, and electron withdrawing groups pull electron density away from neighbouring atoms generating a $\textdelta$+ charge. %pushing electron density towards the  %HOW DOES THIS LINK WITH DESHIELDLING - NMR shifts
%The characteristic step-wise removal of nitrate groups was also attributed to this property.   (Don't really understand why this is)
This suggests that the denitration order could vary with the mechanism of denitration, \ac{DOS} of the \ac{NC}, and distribution of the nitrate groups along the polymer.

%% Note: nitration is via electrophilic subsitition of :OH by NO2+, and denitration via the reverse - hydrolysis / nucleophilic subsitition. We do not consider the other mechanisms of denitraiton here. 
%In this section the nitration and denitration order of a fully nitrated \ac{NC} monomer at thermal equilibrium is determined. The results are compared against experimental observations using simluated NMR and IR spectra, in order to deconflict contradictory views presented in literature on the order of subsitution. 

%\section{Methodology}
%\subsection{Computational details}
%%Programme, method (e.g. functional/ FF), settings(basis sets, thermostats, ensembles etc)
%
%%Include further calculation details - such as transition state theory etc, in another subsection (e.g. Transtion state theory) under this header.
%
%All chemical species underwent an initial \ac{QM} geometry optimisation to the level of $\omega$B97X-D / 6-31G(2df,p) and except where otherwise stated, were performed in Gaussian 09 D.01 (REF). %[REMEMBER TO INCLUDE ANY NMR AND FREQ OPTIONS I USED]
%
%All studies were performed in the gaseous phase and initial structures were geometry optimised with B3LYP/6-311+G(d,p) and tight convergence criteria. 
%Any incomplete or unconverged optimisations were restarted with generation of new internal co-ordinates via the geom=(newdefinition) keyword.  
%% //REJIGG THE BELOW
%The fully nitrated dimer structure was used for MECH 1-2. For the mechanism involving protonation (MECH3), a hydronium cation was independently optimised to the same level. The dimer+cation complex was then optimised with and without CP correction for comparison. For the starting geometry of the intramolecular SN2 reaction (MECH4), the first ring of original dimer was manually adjusted to a boat conformation. Substituents were adjusted to appropriate axial and equatorial positions. 
%All geometry scans were performed at 6-31+G(d) using either UB3LYP or ROB3LYP. Transition state searches were performed using UB3LYP/6-31+G(d). IRC calculations were performed using UB3LYP/6-31+G(d) and either the Hessian-based Predictor-Corrector (HPC), or the Euler integration predictor with the HPC corrector (EPC) algorithm.46
%
%Following each successful scan, a low-level frequency calculation was performed on the obtained transition state. If singular imaginary vibration matching the key bond transformation for the reaction step persisted, then a transition state search was performed using this geometry.
%Where possible, the intermediate “product” geometry obtained from the successful scan was also optimised to B3LYP/6-311+G(d,p) for use in transition state searching using QST2 and QST3 methods.
%
%\section{Results \& Discussion}
%
%\section{Summary}
