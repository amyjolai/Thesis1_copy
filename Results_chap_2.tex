% whilst you're still writing up
\setcounter{chapter}{1}

\chapter{Nitration and Denitration Sequence of Nitrocellulose}
\label{chapterlabel4}
\graphicspath{ {./R_chap_2_pics/} }

[Added as preamble for Chapter 4 below]\newline
[Review usage of NO\textsubscript{x} for consistency with the whole text. Maybe just stick to "nitrous oxides" for the whole document.]

% [ Fix decimal places etc https://tex.stackexchange.com/questions/111670/scientific-notation-only-for-large-numbers
% and table formatting https://tex.stackexchange.com/questions/425414/missing-number-treated-as-zero-using-booktabssiunits?noredirect=1&lq=1]

\section{Introduction}
\label{chapter4:intro}
% Read through an introduce more, the relevant peak intensities, etc. 
%This section utlises the numbering scheme introduced in section \ref{sect:labellingsystem}.
Moniruzzaman \textit{et al.} used the \acs{UV} absorption of an anthraquinone dye to determine the activation energies for the removal of the nitrate at C2, C3, C6 sites on \ac{NC} (figure \ref{fig:anthraquinone_SB59})\cite{Moniruzzaman2008,Moniruzzaman2014}. \acs{UVVis} was chosen as an efficient, non-destructive method of monitoring the decomposition process.
The reaction of the \ac{SB59} with NO\textsubscript{x} released by denitration, mimics the action of stabilisers such as \ac{DPA} and \ac{2NDPA} commonly used in \ac{NC} formulations. 
%https://www.islandpyrochemical.com/nitrocellulose-based-propellants/
The secondary amine groups of the dye consume any nitrates in the system, eliminating the possibility of successive reactions generating acidic species. The presence of acid has been linked to autocatalytic rates of degradation during later stages of \ac{NC} degradation\cite{Edge1990,FernandezdelaOssa2011,Baker1952,Binke1999}. %REF autocatalytic processes, and possibly evidence that stabilisers behave by absorbing NOx sepcies. Maybe even some examples of stabilisers that do this?

\begin{figure}[ht!]
  \centering
	\includegraphics[width=0.35\linewidth]{anthraquinone_SB59}
\caption{\acf{SB59} used to probe the release of nitrates from \ac{NC} using \ac{UVVis} and \textsuperscript{1}H NMR spectroscopy\cite{Moniruzzaman2014}. %The action of nitrate absorption by the dye imitates that of stabilisers commonly used with nitrate ester formulations. 
%Following reaction with NO\textsubscript{x}, the UV absorption peak of the dye is depleted.
}
\label{fig:anthraquinone_SB59}
\end{figure}

\begin{figure}[h]
  \centering
	\includegraphics[width=0.75\linewidth]{monirazzuman-UV}
\caption{\acs{UV} spectra of \ac{NC} thin films with and without accelerated ageing at 60 °C, over a total time of 37 days, from the work of Moniruzzaman \textit{et al.}\cite{Moniruzzaman2008}. The unaged peak demonstrates a strong absorbance in the region corresponding to the \ac{SB59} dye (. This decreases as the dye is consumed in the reaction of NO\textsubscript{x} released upon \ac{NC} denitration.}
\label{fig:monirazzuman-UV}
\end{figure}

Un-aged \ac{NC} thin films and films aged at 40\textdegree C, 50\textdegree C, 60\textdegree C and 70\textdegree C for timescales of up to 2000hrs for 40\textdegree C, were compared. Decreasing absorption peak intensity of the dye and appearance of new absorption regions gave insight into the extent of denitration and the presence of secondary reaction products (figure \ref{fig:monirazzuman-UV}).  
%include their graph?
%[CHECK THIS - they also mention the nitration of the dye - they may have used the nitration of the dye at the marker] 
The \ac{NC} starting material was 12.15\%N by mass,  with mean \ac{DOS}=2.307, indicating that individual glucopyranose rings were of mixed nitration level with non-uniform distribution of nitrate groups along the polymer. % they could all be uniformly nitrated at say, C6, an C2, but some would have C3 and some woudl not, etc.
%Initial conclusions from literature considering both experimental and QM studies
The study found that the nitrate at the C3 position would be most reactive, possessing the lowest activation barrier to removal. % \textit{via} thermolysis.  (Is it really only? And do we need to mention it?)
This was followed by C2 and C6. %, with the highest activation energy and deemed most stable to removal.  %,  based on what data/ conclusions? Gotta click through the literature rabbit holes 
The findings contrast with the computational work of Shukla \textit{et al.}, who determined that denitration via alkyaline hydrolysis followed the order of C3$\rightarrow$C6$\rightarrow$C2 \cite{Shukla2012,Shukla2012a}. 
In this case, the study only considered the fully nitrated system. 
%%%%%Shifted to Chap 3%%%%%%%%%%%%%%%%
There is evidence that nitration and denitration are influenced by the presence of nitrate groups at adjacent positions. Matveev \textit{et al.} demonstrated that for polynitro esters the rate of liquid-phase decomposition did not increase linearly with number of nitrate reaction centres, but was mainly dependent on individual structures {table \ref{tab:reactions}}\cite{Matveev2003}. 
%Explain the pattern of behaviour you see in the table, and what each of the columns mean - remove the ones that don't concern you! / you don't understand or know how to explain!
It was suggested that the trend in reactivity could be explained by the inductive effect of nitrate groups \cite{Oxley2003}.%, which influences reactivity based on the proximity of the groups. 
Hu \textit{et al.} found that the presence of adjacent OH groups hampered the rate of hydrolyis for aerosol dispersed organonitrates \cite{Hu2011}. It is therefore ambiguous whether the apparent rate increase due to adjacent nitrate groups arises as a result of the inductive effect of the nitrate, or whether it is solely due to the absence of the neighbouring hydroxyl. 

%%%%%%%%%%%%%%%%%%%%%%%%%%%%%
% Note, in Monirazz's paper, Bohn also did some modelling, to reach teh C6 > C2 > C3 conclusion (in addition to rate calcs.)
% Make sure that you compare the modelling they did to Shukla - method, structures,  and explain the differences in results.
%%%%%%%%%%%%%%%%%%%%%%%%%%%%%



% Check the units for log A
 %*{3}{S[table-format = 2.4]}
%T\textsubscript{C}, \degree C &
\begin{table}[htp]
\begin{center}
\caption{Comparison of rate constants of decomposition for various polynitrate esters at 140\degree C. Collated from literature sources by Matveev \textit{et al.}\cite{Matveev2003}.
% $\Delta$T is the aging temperature range (check this),  $E$ is the energy barrier to ...  log$A$ is the pre-expoenential factor (Check this), $k$\textsubscript{expt} is the rate constant for denitration.
}
%\begin{tabular}{ l c {S[table-format = 2.1]} {S[table-format = 2.1]} {S[table-format = 2.1]}} 
\begin{tabular}{p{15em} c c c c} 
\toprule
\multirow{2}{10em}{Compound	}	&	 $\Delta T$	&  $E$							&  log$A$ 				& $k$\textsubscript{expt} \\ %$k\subscript{exp}
 										&	/ \degree C & / kcal mol\textsuperscript{-1}	&[s\textsuperscript{-1}]	&  / \SI{e-6}s\textsuperscript{-1} \\
\midrule
\ce{O2NOCH2CH2CH2ONO2}					& 72–140 	& 39.1 	& 14.9 	& 1.7 \\
\ce{O2NOCH2CH2CH2CH2ONO2}			& 100–140 	& 39.0 	& 14.7 	& 1.1 \\
\ce{O2NOCH(CH3)CH(CH3)ONO2}			& 72–140 	& 40.3	& 14.9 	& 5.0 \\
%\ce{O2NOCH2CH2(NNO2)CH2CH2ONO2}		& 80–140 	& 41.5	& 16.5 	& 3.5 \\
\\
\ce{O2NOCH2CH2OCH2CH2ONO2}			& 80–140 	& 42.0 	& 16.5 	& 1.9 \\
\ce{O2NOCH2CH(OH)(CH2ONO2)}			& 80–140 	& 42.4 	& 16.8 	& 2.3 \\
\\
\ce{O2NOCH2CH(ONO2)(CH3)}				& 72–140 	& 40.3	& 15.8 	& 3.0 \\ 
\ce{[(O2NOCH2)CH(ONO2)CH(ONO2)]2}		& 80–140 	& 38.0	& 15.9 	& 63.0 \\
	 (hexanitromannite)							&			&		&		&	\\
%j)	& \ce{(O2NOCH2)4C (TEN)} 					& 145–171	& 39.0 & 15.6 	& 9.3 \\
%k)	& \ce{(O2NOCH2)2CHONO2} 					&		-	&	-	&	-	& 13.0 \\
%%%%%%%%%%%%%%%%%%%%%%%%%%%%%%%%%%%%%%%%%%%%%%%%%%%%%%%%%%%%%%%%%%%%%%%%
%%a)	& \ce{O2NOCH2CH2ONO2}						&	-	 	& 	-	& 	-	& 4.7 \\
%% 	& (nitroglycol) 									&			&		&		&	\\
%%b)
%	& \ce{O2NOCH2CH2CH2ONO2}					& 72–140 	& 39.1 	& 14.9 	& 1.7 \\
%%e)
%	& \ce{O2NOCH2CH2CH2CH2ONO2}			& 100–140 	& 39.0 	& 14.7 	& 1.1 \\
%%f)
%	& \ce{O2NOCH(CH3)CH(CH3)ONO2}			& 72–140 	& 40.3	& 14.9 	& 5.0 \\
%%h)	& \ce{O2NOCH2CH2(NNO2)CH2CH2ONO2}		& 80–140 	& 41.5	& 16.5 	& 3.5 \\
%\\
%%g)
%	& \ce{O2NOCH2CH2OCH2CH2ONO2}			& 80–140 	& 42.0 	& 16.5 	& 1.9 \\
%%d)
%	& \ce{O2NOCH2CH(OH)(CH2ONO2)}			& 80–140 	& 42.4 	& 16.8 	& 2.3 \\
%\\
%%c)
%	& \ce{O2NOCH2CH(ONO2)(CH3)}				& 72–140 	& 40.3	& 15.8 	& 3.0 \\ 
%%i)
%	& \ce{[(O2NOCH2)CH(ONO2)CH(ONO2)]2}		& 80–140 	& 38.0	& 15.9 	& 63.0 \\
% 	&	 (hexanitromannite)							&			&		&		&	\\
%%j)	& \ce{(O2NOCH2)4C (TEN)} 					& 145–171	& 39.0 & 15.6 	& 9.3 \\
%%k)	& \ce{(O2NOCH2)2CHONO2} 					&		-	&	-	&	-	& 13.0 \\
\bottomrule
\end{tabular}
\label{tab:reactions}
\end{center}
\end{table}

The inductive effect arises when a difference in the electronegativity between atoms connected by a $\sigma$ bond leads to a polarisation, or permanent dipole, in the bond. Electron donating groups % are of lower electronegativity than  
increase the $\delta$- partial charge on neighbouring atoms through the release of electrons, whilst electron withdrawing groups pull electron density away from neighbouring atoms generating a $\delta$+ charge on connected atoms. %pushing electron density towards the  %HOW DOES THIS LINK WITH DESHIELDLING - NMR shifts
%The characteristic step-wise removal of nitrate groups was also attributed to this property.   (Don't really understand why this is)
%Nitrates are electron donating? whilst O are withdrawing? Thus the presence of a second nitrate group in the vicinity would push electron density towards the C linking the nitrate to the ring (or perhaps the linking O), leading to stabilisation of charge in the form of a carbocation or at least increased nucleophilicity, for subsequent reactions / taking OH from  water, etc. 
From the studies above, it is seen that the denitration order of \ac{NC} can be influenced by the location, distribution and saturation of the nitrated sites along the polymer, as well as mechanistic differences in thermal and chemical degradation.
Crucially, a synergistic effect leading to facile removal of the C3 group when the C2 site is also nitrated may play an important part in the decomposition pathway. The \ac{DOS} value of the source \ac{NC} would therefore give an indication of the likelihood of adjacent nitrates.


%in this section... probe the energies of (de)nitration at different nitrate sites in order to garner the (magnitude of the) effect of adjacent nitrate groups on the ease of nitration or denitration at a specific location on a monomer ring. Results will be compared with experimental compositions of aged samples \textit{via} \ac{IR} and \ac{NMR} spectra. 

%% Note: nitration is via electrophilic subsitition of :OH by NO2+, and denitration via the reverse - hydrolysis / nucleophilic subsitition. We do not consider the other mechanisms of denitraiton here. 
%In this section the nitration and denitration order of a fully nitrated \ac{NC} monomer at thermal equilibrium is determined. The results are compared against experimental observations using simluated NMR and IR spectra, in order to deconflict contradictory views presented in literature on the order of subsitution. 

%\section{Methodology}
%\subsection{Computational details}
%%Programme, method (e.g. functional/ FF), settings(basis sets, thermostats, ensembles etc)
%
%%Include further calculation details - such as transition state theory etc, in another subsection (e.g. Transtion state theory) under this header.
%
%All chemical species underwent an initial \ac{QM} geometry optimisation to the level of $\omega$B97X-D / 6-31G(2df,p) and except where otherwise stated, were performed in Gaussian 09 D.01 (REF). %[REMEMBER TO INCLUDE ANY NMR AND FREQ OPTIONS I USED]
%
%All studies were performed in the gaseous phase and initial structures were geometry optimised with B3LYP/6-311+G(d,p) and tight convergence criteria. 
%Any incomplete or unconverged optimisations were restarted with generation of new internal co-ordinates via the geom=(newdefinition) keyword.  
%% //REJIGG THE BELOW
%The fully nitrated dimer structure was used for MECH 1-2. For the mechanism involving protonation (MECH3), a hydronium cation was independently optimised to the same level. The dimer+cation complex was then optimised with and without CP correction for comparison. For the starting geometry of the intramolecular SN2 reaction (MECH4), the first ring of original dimer was manually adjusted to a boat conformation. Substituents were adjusted to appropriate axial and equatorial positions. 
%All geometry scans were performed at 6-31+G(d) using either UB3LYP or ROB3LYP. Transition state searches were performed using UB3LYP/6-31+G(d). IRC calculations were performed using UB3LYP/6-31+G(d) and either the Hessian-based Predictor-Corrector (HPC), or the Euler integration predictor with the HPC corrector (EPC) algorithm.46
%
%Following each successful scan, a low-level frequency calculation was performed on the obtained transition state. If singular imaginary vibration matching the key bond transformation for the reaction step persisted, then a transition state search was performed using this geometry.
%Where possible, the intermediate “product” geometry obtained from the successful scan was also optimised to B3LYP/6-311+G(d,p) for use in transition state searching using QST2 and QST3 methods.
%
%\section{Results \& Discussion}
%
%\section{Summary}
