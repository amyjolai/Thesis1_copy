% whilst you're still writing up
\setcounter{chapter}{3}

\chapter{Post-Denitration Reactions}
\label{chapterlabel6}
\graphicspath{ {./R_chap_4_pics/} }

% In this chapter:
% Which reactions happen post denitration, and which don't (based on pure thermodynamics)
% Which ones contribute to the end products seen in literature, and which much be consumed in subsequent reactions
% {Mention that this list is not likely to be exhaustive, but cover some of the more ''obvious'' ones}

\section{Introduction}%%%%%%%%%%%%%%%%%%%%%%%%%%%%%%%%%%%%%
%secondary reactions
%autocatalytic rate 
	
Following the primary denitration step, products are evolved as gases or remain in the \ac{NC} matrix. 
%Reactive species include %the acidic products from the elimination of \ce{HNO2} and 
Reactive nitrogen dioxide radicals generated from homolysis of the O-N bond %not evolved as \ce{NO2} gas, 
are likely to migrate within the bulk and attack other sites in the polymer 
% and free species
. Nitrous acid released from intramolecular reactions within the polymer contribute to the acidity of the overall system, lowering the pH and further stimulating hydrolysis processes. %(who? et al

Neutral and alkaline hydrolysis reactions follow a pseudo-first order process. %The degradation of cellulose also follows a pseudo-first order rate\cite{Calvini2008}.
As described in section \ref{chapter4:intro}, Moniruzzaman \textit{et al.} used the reaction of nitrates with anthraquinone dye \acs{SB59} to probe the reactivity at each of the C2, C3 and C6 sites on \ac{NC}, using \ac{UVVis} and \textsuperscript{1}H NMR spectroscopy (figure \ref{fig:SB59_NOx}).
High concentrations of secondary reaction products following the liberation of the nitrate group were observed. Samples with longer ageing time presented spectra dominated by consecutive products. Though figure \ref{fig:SB59_NOx} illustrates the reaction of the dye with NO groups, the study makes no indication of the source of NO\textsubscript{x}, except that they are products of thermolysis of \ac{NC}. %Whether these are 

\begin{figure}[h]
  \centering
	\includegraphics[width=0.5\linewidth]{SB59_NOx_reaction}
\caption{Proposed reaction pathway for anthraquinone dye (\ac{SB59}) with NOx released as a result of denitration,  from the work of Moniruzzaman \textit{et al.}\cite{moniruzzaman2014}. }
\label{fig:SB59_NOx}
\end{figure}


%Discuss the work of Camera, Aellig and Chin 2007

In this section, secondary and extended reaction schemes for the low temperature ageing of \ac{NC} are explored. Mechanisms proposed by Camera \textit{et al.} and Aellig \textit{et al.} are probed to determine the reactions responsible for the experimentally observed degradation products.

\section{Methodology}%%%%%%%%%%%%%%%%%%%%%%%%%%%%%%%%%%%%%
The species reactions proposed by Camera and Aellig \textit{et al.} were geometry optimised using \ac{wb97xd}, and \ac{B3LYP} functionals. The reactions were initally modelled using ethyl nitrate as a test system before expansion to the full C2 
monomer. 

\subsection{Computational details}%%%%%%%%%%%%%%%%%%%%%%%%%%%%%%%
All geometry optimisations were performed in \ac{G09}, using the \ac{wb97xd} and \ac{B3LYP} functionals. Optimisations were repeated with \ac{PCM} to introduce solvent effects. 

\section{Results and Discussion}%%%%%%%%%%%%%%%%%%%%%%%%%%%%%%%%%%
\subsection{Thermodynamics of Ethyl Nitrate reactions}%%%%%%%%%%%%%%%%%%%%%%%%%
%For an in ital comparison of the methods you used, you could do a diagram like the one Kuklja did (kuja2014.pdf, page 89, fig 3.7).
%She also makes mention of the overestimation of activation barriers for pure DFT methods. Make sure you know the background surrounding this - why does this occur, and what is done to remedy it?
 
%\begin{figure}[h]
%  \centering
%	\includegraphics[width=0.8\linewidth]{name}
%\caption{\cite{}.}
%\end{figure}
%$\Delta$-G\textsubscript{r}
%$\textDelta$H\textsubscript{r}

\begin{table}[htp]
\begin{center}
\caption{Free energies of protonation at different oxygens sites on ethyl nitrate.}
\begin{tabular}{ l *{4}{S[table-format = 2.4]}} 
\toprule
\multirow{2}{*}{protonation site } & \multicolumn{2}{c}{$\Delta$G\textsubscript{r}} & \multicolumn{2}{c}{$\Delta$H\textsubscript{r} }\\\cline{2-5}
  & \acs{wb97xd} & PCM & \acs{B3LYP} & PCM\\
\hline
 Terminal (up) O \ce{CH3CH3ONO2H+} &  0.007232 & 0.008311 &  0.00635  & 0.007715\\
 Terminal (down) O  & -0.019487 & 0.014003 & -0.021877 & 0.010113 \\ 
 Bridging O & -0.019487 & 0.014003 & -0.021877 & 0.010113 \\
% \ce{CH3CH3ONOOH+} 
% \ce{CH3CH3ONO(H+)O} 
% \ce{CH3CH3O(H+)NOO} 
\bottomrule
\end{tabular}
\label{tab:reactions}
\end{center}
\end{table}


%\begin{table}[htp]
%\begin{center}
%\caption{Free energies of protonation at different oxygens sites on ethyl nitrate.}
%\begin{tabular}{ c l *{4}{S[table-format = 2.4]}} 
%\toprule
% \multicolumn{2}{c}{\multirow{2}{*}{\ce{CH3CH2ONO2 + H3O+ <-> CH3CH2ONO2H+ + H2O}}} & \multicolumn{2}{c}{$\Delta$G\textsubscript{r}} & \multicolumn{2}{c}{$\Delta$H\textsubscript{r} }\\\cline{3-6}
%  &  & \acs{wb97xd} & PCM & \acs{B3LYP} & PCM\\
%\hline
%1 & Terminal O &  0.007232 & 0.008311 &  0.00635  & 0.007715\\
%2 & Bridging O & -0.019487 & 0.014003 & -0.021877 & 0.010113 \\
%3 & Bridging O & -0.019487 & 0.014003 & -0.021877 & 0.010113 \\ 
%\bottomrule
%\end{tabular}
%\label{tab:reactions}
%\end{center}
%\end{table}

%\begin{table}[htp]
%\begin{center}
%\caption{Reaction energies of reactions following denitration. Calculations carried out at 6-31+G(2df,p) level and solvent \acs{PCM} was implemented.}
%\begin{tabular}{ c l *{4}{S[table-format = 2.4]}} 
%\toprule
%\multirow{2}{*}{}& \multirow{2}{*}{Reaction} & \multicolumn{2}{c}{$\Delta$G\textsubscript{r}} & \multicolumn{2}{c}{$\Delta$H\textsubscript{r} }\\\cline{3-6} 
%  &  & \acs{wb97xd} & PCM & \acs{B3LYP} & PCM\\
%\hline
%1 & \ce{CH3CH2ONO2 + H2O -> CH3CH2OH + HNO3} &  0.007232 & 0.008311 & 0.00635 & 0.007715\\
%2 & \ce{CH3CH2ONO2 + H3O+ <-> CH3CH2ONO2H+ + H2O} &	-0.019487 & 0.014003 & -0.021877 & 0.010113 \\
%3 & 
%\bottomrule
%\end{tabular}
%\label{tab:reactions}
%\end{center}
%\end{table}



Camera' equations \\
Hydrolysis \\
\ce{CH3CH2ONO2 + H+ <=>[\text{fast}] CH3CH2ONO2H+} \\
\ce{CH3CH2ONO2H+ <=>[\text{slow}] CH3CH2H + NO2+} \\
\ce{NO2+ + 2H2O <=>[\text{fast}] HNO3 + H3O+} \\

Initiation \\
\ce{CH3CH2OH + HNO3  CH3CH=O + H2O + HNO2} \\

Propagation \\
\ce{HNO2 + HNO3 <-> N2O4 + H2O} \\
\ce{N2O4 <-> 2 .NO2} \\
\ce{CH3CH2OH + ^{.}NO2 -> CH3^{.}CHOH + HNO2} \\
\ce{CH3^{.}CHOH + HNO3 -> CH3CHO + H2O + ^{.}NO2} \\

Aellig's equations\\
Initiation \\
\ce{4HNO3 <=> 4NO2 + 2H2O + O2} \\
\ce{2NO2 + H2O <=> N2O4 + H2O} \\
\ce{N2O4 + H2O -> HNO3 + HNO2} \\

Propagation\\
\ce{CH3CH2OH + HNO2 <-> CH3CH2ONO + H2O} \\
\ce{CH3CH2ONO -> CH3CH=O + HNO} \\
\ce{^{.}NO2 + HNO -> HNO2 + ^{.}NO} \\
\ce{2 ^{.}NO + O2 -> 2 ^{.}NO2} \\

Termination \\
\ce{HNO + HNO -> HON=NOH} \\
\ce{HON=NOH -> N2O + H2O} \\

%Ionic reactions
%
%\ce{NO2+ + 2 H2O <-> HNO3 + H3O+}
%
%
%Acid reactions
%
%\ce{CH3CH2OH + HNO3 -> CH3CH=O + H2O + HNO2}
%
%\ce{HNO2 + HNO3 <-> N2O4 + H2O}
%\ce{N2O4 <-> 2 ^{.}NO2}
%
%\ce{CH3CH2OH + HNO2 <-> CH3CH2ONO + H2O}
%\ce{CH3CH2ONO -> CH3CH=O + HNO}
%
%\ce{HNO + HNO -> HON=NOH}
%
%\ce{HON=NOH -> N2O + H2O}
%
%Radical reactions
%
%\ce{^{.}NO2 + HNO -> HNO2 + ^{.}NO}
%\ce{2 ^{.}NO + O2 -> 2 ^{.}NO2}
%
%Didn't go:
%
%\ce{CH3CH2OH + ^{.}NO2 -> CH3^{.}CHOH + HNO2}
%\ce{CH3^{.}CHOH + HNO3 -> CH3CHO + H2O + ^{.}NO2}



\subsubsection{Radical mechanistic route}

\subsubsection{Ionic mechanistic route}

\subsection{Reactions of Nitrocellulose Monomer}

%\section{ \textit{(Kinetics of Ethyl Nitrate)}}
%\subsection{Radical mechanistic route}
%\subsection{Ionic mechanistic route}
%
%\section{ \textit{(Kinetics of Nitrocellulose Monomer)}}



\section{Summary}