% whilst you're still writing up
\setcounter{chapter}{3}

\chapter{Post-Denitration Reactions}
\label{chapterlabel6}
\graphicspath{ {./R_chap_4_pics/} }


% In this chapter:
% Which reactions happen post denitration, and which don't (based on pure thermodynamics)
% Which ones contribute to the end products seen in literature, and which much be consumed in subsequent reactions
% {Mention that this list is not likely to be exhaustive, but cover some of the more ''obvious'' ones}

\section{Introduction}%%%%%%%%%%%%%%%%%%%%%%%%%%%%%%%%%%%%%
%secondary reactions
 
%Following the preliminary denitration step, products of intramolecular or hydrolytic nitrate elmination are evolved as gases or remain trapped in the \ac{NC} matrix. 
Products of the preliminary denitration step of \ac{NC} can be evolved as gases or remain trapped in the polymer matrix. 
% IS there any implicit evidence of species being trapped though? Or do they either escape, or continue reacting? Perhaps there is no state of just being “trapped” and latent in the NC matrix.
%
%Reactive species include %the acidic products from the elimination of \ce{HNO2} and 
Reactive nitrous oxide radicals generated from homolysis of the O-N bond %not evolved as \ce{NO2} gas, 
are likely to migrate within the bulk and attack other sites on the polysaccharide. %REF
% and free species
Nitrous and nitric acids released directly from denitration, or via transformation of released NO\textsubscript{x} species, contribute to the acidity of the overall system, lowering the pH and stimulating further hydrolysis processes. %

%As described in section \ref{chapter4:intro}, Moniruzzaman \textit{et al.} used the reaction of nitrates with an anthraquinone dye (\acs{SB59}) to probe the reactivity at each of the C2, C3 and C6 sites on \ac{NC}, using \acs{UVVis} and \textsuperscript{1}H NMR spectroscopy (figure \ref{fig:SB59_NOx}).
When studying the ageing of \ac{NC} using  \acs{UVVis} spectroscopy, Moniruzzaman \textit{et al.} observed %identified 
increasing concentrations of secondary reaction products following heat treatment over extended timescales\footnote{First introduced in section \ref{chapter4:intro}} (figure \ref{fig:UV_all}) \cite{Moniruzzaman2008,Moniruzzaman2014}. 
%Samples with higher ageing temperatures presented spectra dominated by consecutive products (figure \ref{fig:UV_all}). 
\acs{UV} absorbances at 600 nm and 650 nm were characteristic of the \ac{SB59} dye used to indicate the presence of NO\textsubscript{x}, released by the denitration of \ac{NC}. The isosbestic point identified at 552 nm showed that as the concentration of \acs{SB59} decreased, the concentration of the [\acs{SB59} + \ac{NC}] product increased. % accordingly. %though not necessarily proportionally
%the relative concentrations of unreacted \acs{SB59} and \acs{SB59} following reaction with \ac{NC} were proportional.
For measurements $>$40\degree C, the isosbestic point demostrated a shift downwards. %, until it was eventually lost. 
This was most clearly illustrated by the 70\degree C case, whereby the final measurement (indicated by the royal-blue line in bold) deviated from the isosbestic point entriely, and presented more than 81\% consumption of the original dye concentration. The drift of the isosbestic point with the appearance of new peaks below 400 nm suggests the presence of additional species in the reaction mixture. It is likely that these arise from the continued reaction of \acs{SB59} derivatives with \ac{NC} degradation products, or further derivatives thereof, as suggested in scheme \ref{sch:SB59_NOx}.

%The \ac{NC} films with higher ageing temperatures demonstrated a greater loss of \ac{SB59} dye absorbances, and more pronounced peaks corresponding to secondary reaction products below 400 nm. 
%
% were changing, but that the idenity of the complexes in the system remained the same. 
%Though figure \ref{fig:SB59_NOx} illustrates the reaction of the dye with NO groups, the study makes no indication of the source of NO\textsubscript{x}, except that they are products of thermolysis of \ac{NC}. 


% Insert blank line between top and bottom rows, if you can - LATER
\begin{figure}[ht]
\centering
  \begin{subfigure}[b]{0.42\linewidth}
	\caption{\ac{NC} film aged at 40\degree C.}
  \includegraphics[width=\linewidth]{40-monirazzuman-UV-2014}
  \end{subfigure}
    \begin{subfigure}[b]{0.4\linewidth}
	\caption{\ac{NC} film aged at 50\degree C.}
  \includegraphics[width=\linewidth]{50-monirazzuman-UV-2014}
  \end{subfigure}
  \begin{subfigure}[b]{0.4\linewidth}
    \caption{\ac{NC} film aged at 60\degree C.}
    \includegraphics[width=\linewidth]{60-monirazzuman-UV-2014}
  \end{subfigure}
  \begin{subfigure}[b]{0.4\linewidth}
    \caption{\ac{NC} film aged at 70\degree C.}
    \includegraphics[width=\linewidth]{70-monirazzuman-UV-2014}
  \end{subfigure}
\caption{\acs{UVVis} spectra of aged \ac{NC}-based film, from the work of Moniruzzaman \textit{et al.}\cite{Moniruzzaman2014}. The peaks at 600 nm and 650 nm are attributed to the $\pi$ - $\pi$* transitions in the anthraquinone dye (\ac{SB59}). Spectral lines with highest absorbance in this region correspond to the sample prior to heat treatment. Peaks below 400 nm indicate the formation of \acs{SB59} derivatives due to secondary reactions.
%The spectral lines with the highest absoprtions at the double peak region (with crests at 600 nm and 650 nm) correspond to the sample before heat treatment, with the highest concentration of \acs{SB59} dye.
}
\label{fig:UV_all}
\end{figure}

\begin{scheme}[h!]
  \centering
	\includegraphics[width=0.55\linewidth]{SB59_NOx_reaction}
\caption{Proposed reaction pathway for the the \acs{SB59} dye with NOx released as a result of denitration of \ac{NC} \cite{Moniruzzaman2014}. }
\label{sch:SB59_NOx}
\end{scheme}

Following the possible denitration routes outlined in %section \ref{chapter5:intro}
Chapter 3, the remaining residues are available for further reaction with the polymer or other free molecules in the system. Chin \textit{et al.} proposed schemes for the propagation of such reactions initated by both thermolysis and hydrolysis of nitrate esters \cite{Chin20007} :
\\
\\
% Fix alignment left, please
%%Chin's equations \\
\noindent\textbf{Thermolysis}

\textit{Initiation:}
\begin{equation}
%\textit{Initiation}
%\begin{block_indent}{1cm}
\ce{RONO2 -> RO^{.} + ^{.}NO2 }
%\end{block_indent}
\label{equ:chinT1}
\end{equation}

\textit{Propagation:}
%\textit{Propagation}
%\begin{block_indent}{1cm}
\begin{equation}
\begin{split}
%\begin{multline}
%\ce{RO^{.} + ^{.}NO2 ->[\ce{RONO2}] RONO2 + NO + N2O4 + ^{.}NO2}\\
%								\ce{ + H2O  + N2O + CO2 + CO } \\ 
%								\ce{+ C2H2O +} \text{other organic fragments}
\ce{RO^{.} + ^{.}NO2 ->[\ce{RONO2}]} & \ce{ RONO2 + NO + N2O4 + ^{.}NO2}\\
								& \ce{ + H2O  + N2O + CO2 + CO } \\
								& \ce{+ C2H2O +} \text{other organic fragments}
%\end{multline}
\end{split}
\label{equ:chinT2}
\end{equation}


\begin{equation}
\ce{2^{.}NO + O2 -> 2^{.}NO2} 
\label{equ:chinT3}
\end{equation}
\begin{equation}
\ce{2^{.}NO2 <-> N2O4}
\label{equ:chinT4}
%\end{block_indent}
\end{equation}

\noindent\textbf{Hydrolysis}
%\subsubsection*{Hydrolysis}
%\textit{Initial}

\textit{Initiation:}
%\begin{block_indent}{1cm}
\begin{equation}
\ce{R-ONO2 + H2O -> ROH + HNO3} 
\label{equ:chinH1}
\end{equation}
\begin{equation}
\ce{R-OH + HNO3 -> R-CHO + HNO2 + H2O}
\label{equ:chinH2}
\end{equation}
%\end{block_indent}

\textit{Propagation:}
%\textit{Propagation}
%\begin{block_indent}{1cm}
\begin{equation}
\ce{HNO3 + HNO2 -> N2O4 + H2O} 
\label{equ:chinH3}
\end{equation}
\begin{equation}
\ce{N2O4 -> 2^{.}NO2}
\label{equ:chinH4}
\end{equation}
\begin{equation}
\ce{R-OH + ^{.}NO2 -> ^{.}R-OH + HNO2}
\label{equ:chinH5}
\end{equation}
\begin{equation}
\ce{^{.}R-OH + HNO3 -> RCH=O + H2O + NO2}
\label{equ:chinH6}
\end{equation}
%\end{block_indent}

%%Chin's equations \\
%\noindent Thermolysis
%
%\textit{Initiation:}
%%\begin{equation}
%\begin{block_indent}{1cm}
%\ce{RONO2 -> RO^{.} + ^{.}NO2 } \\
%\end{block_indent}
%%\end{equation}
%
%\textit{Propagation:}
%\begin{block_indent}{1cm}
%%\begin{equation}
%%\begin{aligned}
%\ce{RO^{.} + ^{.}NO2 ->[\ce{RONO2}] RONO2 + NO + N2O4 + ^{.}NO2 + H2O  + N2O + CO2 + CO + C2H2O +} other organic fragments \\
%%\end{aligned}
%%\end{equation}
%%\begin{equation}
%\ce{2^{.}NO + O2 -> 2^{.}NO2} \\
%%\end{equation}
%%\begin{equation}
%\ce{2^{.}NO2 <-> N2O4}\\
%\end{block_indent}
%%\end{equation}
%
%\noindent Hydrolysis
%%\subsubsection*{Hydrolysis}
%%\textit{Initial}
%
%\textit{Initiation:}
%\begin{block_indent}{1cm}
%%\begin{equation}
%\ce{R-ONO2 + H2O -> ROH + HNO3} \\
%%\end{equation}
%%\begin{equation}
%\ce{R-OH + HNO3 -> R-CHO + HNO2 + H2O}\\
%%\end{equation}
%\end{block_indent}
%
%\textit{Propagation:}
%\begin{block_indent}{1cm}
%%\begin{equation}
%\ce{HNO3 + HNO2 -> N2O4 + H2O} \\
%%\end{equation}
%%\begin{equation}
%\ce{N2O4 -> 2^{.}NO2}\\
%%\end{equation}
%%\begin{equation}
%\ce{R-OH + ^{.}NO2 -> ^{.}R-OH + HNO2}\\
%%\end{equation}
%%\begin{equation}
%\ce{^{.}R-OH + HNO3 -> RCH=O + H2O + NO2}\\
%%\end{equation}
%\end{block_indent}

The hydrolysis equations were modified from an earlier work by Camera \textit{et al.}, where the scheme was presented with ethyl nitrate as the organonitrate (where R = \ce{CH3CH2}) \cite{Camera1982}. The rate-determining nature of equation \ref{equ:chinH2}




%autocatalytic rate 
%%%%%%%%%%%%%%%% FIX / CHECK THIS SECTION %%%%%%%%%%%%%%%%%%%%%

% confused rates of thermolytic degradation: \cite{brill1997}
	% Despite good understanding about the initial step of the process, the kinetics of thermolysis of NC are complicated by the existence of rapid secondary reactions, autocatalysis, and self-heating. As a result, further kinetic descriptions of the process are fragmented and somewhat contradictory. To organize this subject, the past studies can be grouped according to the temperature range of study.''
% info on rates of hyrolysis

% what I actually want to focus on here, are the reactions AFTER intial hydrolysis.
% So knowledge on the rates quoted in literature are good, but I need to know what they correspond to:
	% All reactions, with just an increase in rate as the system gets hotter / more reactions splinter off? (ie more reactions happening, so more exothermic heat release - self heating
	% Does the sudden change to autocatalytic arise from a global change in the mechanism? / All the of the “first order” reaction stuff is done / it’s much less important, and that the 

It is widely agreed that first-stage decomposition follows a first-order process (or pseudo-first order, with respect to hydrolysis reactions). %REF for first & psuedo first order rate for NC
A number of studies observe catalytic rate of decay for the longer-term aging processes. %REF the studies in DAUERMAN's who quoted an autocatalytic rate. 
Dauerman \cite{Dauerman1968} observed that when \ac{NC} was treated with \ce{NO2} gas before heating, the time required for sample ignition halved. He suggested that the \ce{NO2} adsorbed onto the surface acted as a catalysing agent. 

% Now, how do the above rate observations relate to the post-denitration phases of reaction?

Neutral and alkaline hydrolysis reactions follow a pseudo-first order process, however it has been suggested that the presence of acid faciltiatles a catalyic rate of degradation after an initial incubation period. %Incubation is not correct - what I want to say is that the reaction starts off first order, but at an obeservable inflection point, becomes catalyitc. 
%Check whether i am jsut repeating myself from chap 2?
%The degradation of cellulose also follows a pseudo-first order rate\cite{Calvini2008}.

% Introducing the decondary reaction products %%%%%%%%%%%%%%%%%%%%
%Discuss the work of Camera, Aellig and Chin 2007
Multiple studies have addressed the decomposition reactions of nitrate esters following the initial scission of the nitrate group \cite{,Baker1952,Camera1982,Camera1983,
Matveev2003,Hu2011} %ETC refine this list later

%Baker1950
%\subsection*{Camera's equations}
%\subsubsection*{Hydrolysis}
%\begin{block_indent}{1cm}
%\ce{CH3CH2ONO2 + H+ <=>[\text{fast}] CH3CH2ONO2H+} \\
%\ce{CH3CH2ONO2H+ <=>[\text{slow}] CH3CH2H + NO2+} \\
%\ce{NO2+ + 2H2O <=>[\text{fast}] HNO3 + H3O+} \\
%\end{block_indent}

\textit{Initiation}
\begin{block_indent}{1cm}
\ce{CH3CH2OH + HNO3  -> CH3CH=O + H2O + HNO2} \\
\end{block_indent}

\textit{Propagation}
\begin{block_indent}{1cm}
\ce{HNO2 + HNO3 <-> N2O4 + H2O} \\
\ce{N2O4 <-> 2 .NO2} \\
\ce{CH3CH2OH + ^{.}NO2 -> CH3^{.}CHOH + HNO2} \\
\ce{CH3^{.}CHOH + HNO3 -> CH3CH=O + H2O + ^{.}NO2} \\
\end{block_indent}

\subsection*{Aellig's equations}
\textit{Initiation}
\begin{block_indent}{1cm}
\ce{4HNO3 <=> 4NO2 + 2H2O + O2} \\
\ce{2^{.}NO2 + H2O <=> N2O4 + H2O} \\
\ce{N2O4 + H2O -> HNO3 + HNO2} \\
\end{block_indent}

\textit{Propagation}
\begin{block_indent}{1cm}
\ce{CH3CH2OH + HNO2 <-> CH3CH2ONO + H2O} \\
\ce{CH3CH2ONO -> CH3CH=O + HNO} \\
\ce{^{.}NO2 + HNO -> HNO2 + ^{.}NO} \\
\ce{2 ^{.}NO + O2 -> 2 ^{.}NO2} \\
\end{block_indent}

\textit{Termination}
\begin{block_indent}{1cm}
\ce{HNO + HNO -> HON=NOH} \\
\ce{HON=NOH -> N2O + H2O} \\
\end{block_indent}



%%%%%%%%%%%%%%%%%%%%%%%%%%%%%%%%%%%%%%%%%%%%%%%%%%

In this section, secondary and extended reaction schemes for the low temperature ageing of \ac{NC} are explored. Decompostition pathways defined by Camera \textit{et al.} and Aellig \textit{et al.} are probed to determine the reactions responsible for the experimentally observed degradation products. 
The reactions found to be energetically feasible from the proposed routes will be scrutinised to determine whether an autocatalytic pathway can be formed from the energitcally validated reaction schemes. 

\section{Methodology}%%%%%%%%%%%%%%%%%%%%%%%%%%%%%%%%%%%%%
The species reactions proposed by Camera and Aellig \textit{et al.} were geometry optimised using \ac{wb97xd}, and \ac{B3LYP} functionals, in both vacuum and solvent. The reactions were modelled using ethyl nitrate as a test system before expansion to the full C2 monomeric model. 
The \ac{DG} were used to determine the feasibility of a reaction. 
% Free energy paper with equations: Effect of nitrate content on thermal decomposition of nitrocellulose. Pourmortazavi,2009
Where the choice of method lead to a variation in the result 
%Energetically feasible reactions were added to the decomposition pathway 

\subsection{Computational details}%%%%%%%%%%%%%%%%%%%%%%%%%%%%%%%
All geometry optimisations were performed in \ac{G09}, using the \ac{wb97xd} and \ac{B3LYP} functionals. Optimisations were repeated with \ac{PCM} to introduce solvent effects. 

\section{Results and Discussion}%%%%%%%%%%%%%%%%%%%%%%%%%%%%%%%%%%
\subsection{Thermodynamics of Ethyl Nitrate reactions}%%%%%%%%%%%%%%%%%%%%%%%%%
%For an in ital comparison of the methods you used, you could do a diagram like the one Kuklja did (kuja2014.pdf, page 89, fig 3.7).
%She also makes mention of the overestimation of activation barriers for pure DFT methods. Make sure you know the background surrounding this - why does this occur, and what is done to remedy it?
 
%\begin{figure}[h]
%  \centering
%	\includegraphics[width=0.8\linewidth]{name}
%\caption{\cite{}.}
%\end{figure}
%$\Delta$-G\textsubscript{r}
%$\textDelta$H\textsubscript{r}

%Change units to kcal/mol
\begin{table}[htp]
\begin{center}
\caption{Free energies of protonation at different oxygens sites on ethyl nitrate.}
\begin{tabular}{ l *{4}{S[table-format = 2.4]}} 
\toprule
\multirow{2}{*}{Protonation site } & \multicolumn{2}{c}{$\Delta$G\textsubscript{r}} & \multicolumn{2}{c}{$\Delta$H\textsubscript{r} }\\\cline{2-5}
  & \acs{wb97xd} & PCM & \acs{B3LYP} & PCM\\
\hline
 Terminal (up) O \ce{CH3CH3ONO2H+} &  0.007232 & 0.008311 &  0.00635  & 0.007715\\
 Terminal (down) O  & -0.019487 & 0.014003 & -0.021877 & 0.010113 \\ 
 Bridging O & -0.019487 & 0.014003 & -0.021877 & 0.010113 \\
% \ce{CH3CH3ONOOH+} 
% \ce{CH3CH3ONO(H+)O} 
% \ce{CH3CH3O(H+)NOO} 
\bottomrule
\end{tabular}
\label{tab:reactions}
\end{center}
\end{table}


%\begin{table}[htp]
%\begin{center}
%\caption{Free energies of protonation at different oxygens sites on ethyl nitrate.}
%\begin{tabular}{ c l *{4}{S[table-format = 2.4]}} 
%\toprule
% \multicolumn{2}{c}{\multirow{2}{*}{\ce{CH3CH2ONO2 + H3O+ <-> CH3CH2ONO2H+ + H2O}}} & \multicolumn{2}{c}{$\Delta$G\textsubscript{r}} & \multicolumn{2}{c}{$\Delta$H\textsubscript{r} }\\\cline{3-6}
%  &  & \acs{wb97xd} & PCM & \acs{B3LYP} & PCM\\
%\hline
%1 & Terminal O &  0.007232 & 0.008311 &  0.00635  & 0.007715\\
%2 & Bridging O & -0.019487 & 0.014003 & -0.021877 & 0.010113 \\
%3 & Bridging O & -0.019487 & 0.014003 & -0.021877 & 0.010113 \\ 
%\bottomrule
%\end{tabular}
%\label{tab:reactions}
%\end{center}
%\end{table}

%\begin{table}[htp]
%\begin{center}
%\caption{Reaction energies of reactions following denitration. Calculations carried out at 6-31+G(2df,p) level and solvent \acs{PCM} was implemented.}
%\begin{tabular}{ c l *{4}{S[table-format = 2.4]}} 
%\toprule
%\multirow{2}{*}{}& \multirow{2}{*}{Reaction} & \multicolumn{2}{c}{$\Delta$G\textsubscript{r}} & \multicolumn{2}{c}{$\Delta$H\textsubscript{r} }\\\cline{3-6} 
%  &  & \acs{wb97xd} & PCM & \acs{B3LYP} & PCM\\
%\hline
%1 & \ce{CH3CH2ONO2 + H2O -> CH3CH2OH + HNO3} &  0.007232 & 0.008311 & 0.00635 & 0.007715\\
%2 & \ce{CH3CH2ONO2 + H3O+ <-> CH3CH2ONO2H+ + H2O} &	-0.019487 & 0.014003 & -0.021877 & 0.010113 \\
%3 & 
%\bottomrule
%\end{tabular}
%\label{tab:reactions}
%\end{center}
%\end{table}





%Ionic reactions
%
%\ce{NO2+ + 2 H2O <-> HNO3 + H3O+}
%
%
%Acid reactions
%
%\ce{CH3CH2OH + HNO3 -> CH3CH=O + H2O + HNO2}
%
%\ce{HNO2 + HNO3 <-> N2O4 + H2O}
%\ce{N2O4 <-> 2 ^{.}NO2}
%
%\ce{CH3CH2OH + HNO2 <-> CH3CH2ONO + H2O}
%\ce{CH3CH2ONO -> CH3CH=O + HNO}
%
%\ce{HNO + HNO -> HON=NOH}
%
%\ce{HON=NOH -> N2O + H2O}
%
%Radical reactions
%
%\ce{^{.}NO2 + HNO -> HNO2 + ^{.}NO}
%\ce{2 ^{.}NO + O2 -> 2 ^{.}NO2}
%
%Didn't go:
%
%\ce{CH3CH2OH + ^{.}NO2 -> CH3^{.}CHOH + HNO2}
%\ce{CH3^{.}CHOH + HNO3 -> CH3CHO + H2O + ^{.}NO2}



\subsubsection{Radical mechanistic route}

\subsubsection{Ionic mechanistic route}

\subsection{Reactions of Nitrocellulose Monomer}

%\section{ \textit{(Kinetics of Ethyl Nitrate)}}
%\subsection{Radical mechanistic route}
%\subsection{Ionic mechanistic route}
%
%\section{ \textit{(Kinetics of Nitrocellulose Monomer)}}



\section{Summary}