% whilst you're still writing up
\setcounter{chapter}{3}

\chapter{Post-Denitration Reactions}
\label{chapterlabel6}
\graphicspath{ {./R_chap_4_pics/} }

% In this chapter:
% Which reactions happen post denitration, and which don't (based on pure thermodynamics)
% Which ones contribute to the end products seen in literature, and which much be consumed in subsequent reactions
% {Mention that this list is not likely to be exhaustive, but cover some of the more ''obvious'' ones}

%[Fix the arrows of the reaction schemes? So that eluting products leave with a u shaped arrow?]

\section{Introduction}%%%%%%%%%%%%%%%%%%%%%%%%%%%%%%%%%%%%%
% Talk about the prevalence and importance of secondary + products. 

% Statement - it is these secondary reactions that do the majority of the decomposition, and breakdown of the polymer such that the experimental observables are produced. 
 
%Following the preliminary denitration step, products of intramolecular or hydrolytic nitrate elmination are evolved as gases or remain trapped in the \ac{NC} matrix. 
Products of the preliminary denitration step of \ac{NC} can be evolved as gases or remain trapped in the polymer matrix. 
% IS there any implicit evidence of species being trapped though? Or do they either escape, or continue reacting? Perhaps there is no state of just being “trapped” and latent in the NC matrix.
%
%Reactive species include %the acidic products from the elimination of \ce{HNO2} and 
Reactive nitrous oxide radicals generated from homolysis of the O-N bond %not evolved as \ce{NO2} gas, 
are likely to migrate within the bulk and attack other sites on the polysaccharide. %REF
% and free species
Nitrous and nitric acids released directly from denitration, or via transformation of released NO\textsubscript{x} species, contribute to the acidity of the overall system, lowering the pH and stimulating further hydrolysis processes \cite{Hu2011}. 

%As described in section \ref{chapter4:intro}, Moniruzzaman \textit{et al.} used the reaction of nitrates with an anthraquinone dye (\acs{SB59}) to probe the reactivity at each of the C2, C3 and C6 sites on \ac{NC}, using \acs{UVVis} and \textsuperscript{1}H NMR spectroscopy (figure \ref{fig:SB59_NOx}).
%
% Insert blank line between top and bottom rows, if you can - LATER
\begin{figure}[htp!]
\centering
  \begin{subfigure}[b]{0.42\linewidth}
	\caption{\ac{NC} film aged at 40\degree C.}
  \includegraphics[width=\linewidth]{40-monirazzuman-UV-2014}
  \end{subfigure}
    \begin{subfigure}[b]{0.4\linewidth}
	\caption{\ac{NC} film aged at 50\degree C.}
  \includegraphics[width=\linewidth]{50-monirazzuman-UV-2014}
  \end{subfigure}
  \begin{subfigure}[b]{0.4\linewidth}
    \caption{\ac{NC} film aged at 60\degree C.}
    \includegraphics[width=\linewidth]{60-monirazzuman-UV-2014}
  \end{subfigure}
  \begin{subfigure}[b]{0.4\linewidth}
    \caption{\ac{NC} film aged at 70\degree C.}
    \includegraphics[width=\linewidth]{70-monirazzuman-UV-2014}
    \label{fig:boldUV}
  \end{subfigure}
\caption{\acs{UVVis} spectra of aged \ac{NC}-based film, from the work of Moniruzzaman \textit{et al.}\cite{Moniruzzaman2014}. The peaks at 600 nm and 650 nm are attributed to the $\pi$ - $\pi$* transitions in the anthraquinone dye (\ac{SB59}). Spectral lines with highest absorbance peaks in this region correspond to the sample prior to heat treatment. Peaks below 400 nm indicate the formation of \acs{SB59} derivatives due to secondary reactions.
%The spectral lines with the highest absoprtions at the double peak region (with crests at 600 nm and 650 nm) correspond to the sample before heat treatment, with the highest concentration of \acs{SB59} dye.
}
\label{fig:UV_all}
\end{figure}


When studying the ageing of \ac{NC} using  \acs{UVVis} spectroscopy, Moniruzzaman \textit{et al.} observed %identified 
increasing concentrations of secondary reaction products following heat treatment over extended timescales%\footnote{First introduced in section \ref{chapter4:intro}} (figure \ref{fig:UV_all})
 \cite{Moniruzzaman2008,Moniruzzaman2014}. 
Samples exposed to higher ageing temperatures presented spectra dominated by consecutive products (figure \ref{fig:UV_all}). 
\acs{UV} absorbances at 600 nm and 650 nm were characteristic of the \ac{SB59} dye used to indicate the presence of NO\textsubscript{x}, released by the denitration of \ac{NC}. The isosbestic point identified at 552 nm showed that as the concentration of \acs{SB59} decreased, the concentration of the [\acs{SB59} + \ac{NC}] product increased. % accordingly. %though not necessarily proportionallys
%the relative concentrations of unreacted \acs{SB59} and \acs{SB59} following reaction with \ac{NC} were proportional.
For sample aged at temperatures $>$40\degree C, the isosbestic point demonstrated a downwards shift. %, until it was eventually lost. 
%This was most clearly illustrated by 
In the case of the 70\degree C treated run, the final measurement (indicated by the royal-blue line in bold, figure \ref{fig:boldUV}) deviated from the isosbestic point entirely, and showed more than 81\% consumption of the original dye concentration. The drift from the isosbestic point, in addition to the appearance of new absorbance peaks below 400 nm, alludes to the presence of new species in the reaction mixture not generated by the primary reaction of \acs{SB59} and \ac{NC}. It is likely that these arise from the continued reaction of \acs{SB59} derivatives with \ac{NC} degradation products, or further derivatives thereof, as suggested in scheme \ref{sch:SB59_NOx}. 

%The \ac{NC} films with higher ageing temperatures demonstrated a greater loss of \ac{SB59} dye absorbances, and more pronounced peaks corresponding to secondary reaction products below 400 nm. 
%
% were changing, but that the idenity of the complexes in the system remained the same. 
%Though figure \ref{fig:SB59_NOx} illustrates the reaction of the dye with NO groups, the study makes no indication of the source of NO\textsubscript{x}, except that they are products of thermolysis of \ac{NC}. 


\begin{scheme}[hbp]
  \centering
	\includegraphics[width=0.55\linewidth]{SB59_NOx_reaction}
\caption{Proposed pathway for the reaction of \acs{SB59} dye with $^{.}$NO released as a result of denitration of \ac{NC} \cite{Moniruzzaman2014}. }
\label{sch:SB59_NOx}
\end{scheme}

% Here is where we start talking about the NATURE and IDENTITY of the secondary reactions. 
%Following the possible denitration routes outlined in section \ref{chapter5:intro}Chapter 3, 
Following cleavage of the nitrate ester via homolytic fission, elimination of nitrous acid, or hydrolysis, the resulting residues are available for further reaction with the polymer or other free molecules in the system. Chin \textit{et al.} proposed schemes for the propagation of such reactions initated by both the thermolysis and hydrolysis of nitrate esters \cite{Chin2007}: \\



% Fix alignment left, please
%%Chin's equations \\
\begin{scheme}[h]
\noindent\textbf{Thermolytic initiation}
%\textit{Initiation:}
\begin{equation}
%\textit{Initiation}
%\begin{block_indent}{1cm}
\ce{R-ONO2 -> R-O^{.} + ^{.}NO2 }
%\end{block_indent}
\label{equ:chinT1}
\end{equation}

\textit\textbf{Propagation}
%\textit{Propagation}
%\begin{block_indent}{1cm}
\begin{equation}
\begin{split}
%\begin{multline}
%\ce{RO^{.} + ^{.}NO2 ->[\ce{RONO2}] RONO2 + NO + N2O4 + ^{.}NO2}\\
%								\ce{ + H2O  + N2O + CO2 + CO } \\ 
%								\ce{+ C2H2O +} \text{other organic fragments}
\ce{R-O^{.} + ^{.}NO2 ->[\ce{R-ONO2}]} & \ce{ R-ONO2 + ^{.}NO + ^{.}NO2} + \ce{N2O4}\\
								& \ce{ + H2O  + N2O + CO2 + CO } \\
								& \ce{+ C2H2O +} \text{other organic fragments}
%\end{multline}
\end{split}
\label{equ:chinT2}
\end{equation}


\begin{equation}
\ce{2^{.}NO + O2 -> 2^{.}NO2} 
\label{equ:chinT3}
\end{equation}
\begin{equation}
\ce{2^{.}NO2 <=> N2O4}
\label{equ:chinT4}
%\end{block_indent}
\end{equation}

\noindent\textbf{Hydrolytic initiation}
%\subsubsection*{Hydrolysis}
%\textit{Initial}
%\textit{Initiation:}
%\begin{block_indent}{1cm}
\begin{equation}
\ce{R-ONO2 + H2O -> R-OH + HNO3} 
\label{equ:chinH1}
\end{equation}
\begin{equation}
\ce{R-OH + HNO3 -> R=O + HNO2 + H2O}
\label{equ:chinH2}
\end{equation}
%\end{block_indent}

\textit\textbf{Propagation}
%\textit{Propagation}
%\begin{block_indent}{1cm}
\begin{equation}
\ce{HNO3 + HNO2 <=> N2O4 + H2O} 
\label{equ:chinH3}
\end{equation}
\begin{equation}
\ce{N2O4 <=> 2^{.}NO2}
\label{equ:chinH4}
\end{equation}
\begin{equation}
\ce{R-OH + ^{.}NO2 -> ^{.}R-OH + HNO2}
\label{equ:chinH5}
\end{equation}
\begin{equation}
\ce{^{.}R-OH + HNO3 -> R=O + H2O + ^{.}NO2}
\label{equ:chinH6}
\end{equation}
\label{sch:chinschema}
\end{scheme}
%\end{block_indent}

Termination reactions were not emphasised in the schemes for either of these cases. The hydrolysis scheme was adapted from an earlier work by Camera \textit{et al.} involving the nitrate ester decomposition and subsequent reactions of ethyl nitrate (where R = \ce{CH3CH2} for the scheme above) \cite{Camera1982}. The original study included an expansion of the hydrolysis step (equation \ce{PhCH2ONO2}, ), where the involvement of \ce{NO2+} is illustrated: \\ %pertaining to ethyl nitrate:



%\subsection*{Camera's equations}
\begin{scheme}[h!]
\textbf{Hydrolysis scheme for ethyl nitrate}
%\begin{block_indent}{1cm}
%\ce{CH3CH2ONO2 + H+ <=>[\text{fast}] CH3CH2ONO2H+} \\
%\ce{CH3CH2ONO2H+ <=>[\text{slow}] CH3CH2H + NO2+} \\
%\ce{NO2+ + 2H2O <=>[\text{fast}] HNO3 + H3O+} \\
%\end{block_indent}
\begin{equation}
\ce{CH3CH2ONO2 + H+ <=>[\text{fast}] CH3CH2ONO2H+} \\
\end{equation}
\begin{equation}
\ce{CH3CH2ONO2H+ <=>[\text{slow}] CH3CH2OH + NO2+} \\
\end{equation}
\begin{equation}
\ce{NO2+ + 2H2O <=>[\text{fast}] HNO3 + H3O+} \\
\end{equation}
\label{sch:cameraschema}
\end{scheme}

It was highlighted by Camera, that the oxidation of alcohol by nitric acid (equation \ref{equ:chinH2}) is slow and thus rate-limiting. The mechanism is likely to occur \textit{via} a series of intermediate reactions of which the details are not known.  Following the generation of nitrous acid, subsequent oxidations occur rapidly. %why and how? 
According to Rigas \textit{et al.}, alcohols are more susceptible to wet oxidation than esters \cite{Rigas1997}. A higher concentration of unsubstituted hydroxyl groups in the system, and therefore a fewer nitrate ester groups (or a lower \ac{DOS} value), decreases overall stabililty. % What is the mechanism, and why? This would suggest that below a certain threshold \ac{DOS} value, the oxidation of hydroxyl groups (equation \ref{equ:chinH2}) would dominate over the hydrolytic denitration reaction (equation \ref{equ:chinH2}). therefore indicates that a higher concentration of -OH groups present in the system decreases stabililty. 
% Check that this doesn’t contradict the point made in Chap 2, about the inductive effect of adjacent nitrates on the reactivity/ predisposition to denitrate easier, or if it does, ref the two contrasting papers. 

Equations \ref{equ:chinH3} - \ref{equ:chinH6} describe a possible branched radical chain mechanism, fed by the nitrous and nitric acids produced during the hydrolysis and alchohol oxidation reactions during the initiation stage. By contrast, the propagation reactions in the branched radical chain mechanism for thermolysis are poorly characterised (equation \ref{equ:chinT2}), defined only by the observable products. This is likely due to their rapid and varied nature, rendering it difficult to follow spectroscopically. %REF

%but eventually, everything feeds back into that equation of just radical deg, esp for the species contianing carbon 
%Who actually did the measureing? 
%measured by \ac{IR}, liquid \cite{Bluhm1977}, ion \cite{lopezlopez2011} and gas chromatography \cite{Huwei1988},  ignition loss determination \cite{Bluhm1977}. %and others

Aellig \textit{et al.} presented an alternative scheme for the decomposition of benzyl nitrate (R = \ce{PhCH2}), involving more interaction with the solvent  \cite{Aellig2011}:\\ %\ce{PhCH2ONO2}, 

%{Aellig's equations}
\begin{scheme}[h]
\noindent\textbf{\ce{HNO3} decomposition initiated}
%\textit{Initiation}
\begin{equation}
\ce{4HNO3 <=> 4^{.}NO2 + 2H2O + O2}
\end{equation}
\begin{equation}
\ce{2^{.}NO2 + H2O <=> N2O4 + H2O}
\end{equation}
\begin{equation}
\ce{N2O4 + H2O -> HNO3 + HNO2}
\end{equation}

\textit{Propagation}
\begin{equation}
\ce{R-OH + HNO2 <=> R-ONO + H2O}
\end{equation}
\begin{equation}
\ce{R-ONO -> R=O + HNO}
\end{equation}
\begin{equation}
\ce{^{.}NO2 + HNO -> HNO2 + ^{.}NO}
\end{equation}
\begin{equation}
\ce{2 ^{.}NO + O2 -> 2 ^{.}NO2}
\end{equation}

\textit{Termination}
\begin{equation}
\ce{2HNO -> HON=NOH}
\end{equation}
\begin{equation}
\ce{HON=NOH -> N2O + H2O}
\end{equation}
\label{sch:aelligschema}
\end{scheme}


Both the Camera/Chin and Aellig schemes above produce final end products observed in the decomposition of \ac{NC}. In particular, Aellig’s scheme accounts for the production of \ce{N2O}, which forms a significant part of the decomposition eluent \cite{Buelow2002}. %Whislt not a full-resolution, exhaustive depiction the full spectrum of reactions that take place in the \ac{NC} matrix during it’s slow ageing, the presented reactions encapsulate the early major reactions of the most prevalent and active species in the system.
Whilst the schemes do not a propose an exhaustive description of the full spectrum of reactions that take place in the \ac{NC} matrix during its slow ageing, the early stage reactions of the key species responsible for decomposition are  encapsulated.

%autocatalytic rate 
%%%%%%%%%%%%%%%% FIX / CHECK THIS SECTION %%%%%%%%%%%%%%%%%%%%%

% confused rates of thermolytic degradation: \cite{brill1997}
	% Despite good understanding about the initial step of the process, the kinetics of thermolysis of NC are complicated by the existence of rapid secondary reactions, autocatalysis, and self-heating. As a result, further kinetic descriptions of the process are fragmented and somewhat contradictory. To organize this subject, the past studies can be grouped according to the temperature range of study.''
% info on rates of hyrolysis

% what I actually want to focus on here, are the reactions AFTER intial hydrolysis.
% So knowledge on the rates quoted in literature are good, but I need to know what they correspond to:
	% All reactions, with just an increase in rate as the system gets hotter / more reactions splinter off? (ie more reactions happening, so more exothermic heat release - self heating
	% Does the sudden change to autocatalytic arise from a global change in the mechanism? / All the of the “first order” reaction stuff is done / it’s much less important, and that the 

It is widely agreed that first-stage decomposition follows a first-order process (or pseudo-first order, with respect to hydrolysis reactions). %REF for first & psuedo first order rate for NC
A number of studies observe catalytic rate of decay for the longer-term aging processes. %REF the studies in DAUERMAN's who quoted an autocatalytic rate. 
Dauerman \cite{Dauerman1968} observed that when \ac{NC} was treated with \ce{NO2} gas before heating, the time required for sample ignition halved. He suggested that the \ce{NO2} adsorbed onto the surface acted as a catalysing agent. 

% Now, how do the above rate observations relate to the post-denitration phases of reaction?

Neutral and alkaline hydrolysis reactions follow a pseudo-first order process, however it has been suggested that the presence of acid facilitates a catalytic rate of degradation after an initial incubation period. %Incubation is not correct - what I want to say is that the reaction starts off first order, but at an obeservable inflection point, becomes catalyitc. 
%Check whether i am jsut repeating myself from chap 2?
%The degradation of cellulose also follows a pseudo-first order rate\cite{Calvini2008}.
%
% Introducing the decondary reaction products %%%%%%%%%%%%%%%%%%%%
%Discuss the work of Camera, Aellig and Chin 2007
Multiple studies have addressed the decomposition reactions of nitrate esters following the initial scission of the nitrate group \cite{Baker1952,Camera1982,Camera1983,Matveev2003,Hu2011} %ETC refine this list later

%Baker1950


%%%%%%%%%%%%%%%%%%%%%%%%%%%%%%%%%%%%%%%%%%%%%%%%%%

In this section, secondary and extended reaction schemes for the low temperature ageing of \ac{NC} are explored. Decompostition pathways defined by Chin, Camera and Aellig \textit{et al.} are probed to determine the reactions responsible for the experimentally observed degradation products. 
The reactions found to be energetically feasible from the proposed routes will be scrutinised to determine whether an autocatalytic pathway can be formed from the thermodynamically validated reaction schemes. 

\section{Methodology}%%%%%%%%%%%%%%%%%%%%%%%%%%%%%%%%%%%%%
The reactions proposed by Chin, Camera and Aellig \textit{et al.} were used to construct degradation routes for \ac{NC}. %Proceeding on from the initial denitration step, 
The products of homolytic fission, elimination of \ce{HNO2} and acid hydrolysis of \ac{NC} were used as starting points.%, generating a for each proposed degradation pathway. 
Schemes were constructed based on the propagation of the given reactions in a step-wise fashion; subsequent reactions were dependent on the products generated in prior steps, in addition to the assumed availability of other reactants in the system. 
An abundance of water and oxygen were assumed present in the system, attributed to air exposure or the wetted storage conditions of \ac{NC}. Unsubsitutued alcohol moieties (R-OH) were also presumed available, due to incomplete nitration during the synthesis of \ac{NC} \cite{Wolf1997}, or re-generation following denitration \textit{via} hydrolysis. 
The schemes were modelled with both ethyl nitrate and the \ac{NC} monomer. Free energies of reaction ($\Delta$ G) were used to determine the feasibility of a reaction. %Where the choice of method lead to a variation in the result of a reaction, the geometries around the reaction centres were further scrutinised in order to ensure no spurious behaviour due to artefacts from functional choice. [Bit of a dodgy sentence, how would you know what to say was right / wrong?]

%The species reactions  were geometry optimised using \ac{wb97xd}, and \ac{B3LYP} functionals, in both vacuum and solvent. The reactions were modelled using ethyl nitrate as a test system before expansion to the full C2 monomeric model. 
%The \ac{DG} were used to determine . 
% Free energy paper with equations: Effect of nitrate content on thermal decomposition of nitrocellulose. Pourmortazavi,2009
%Where the choice of method lead to a variation in the result 
%Energetically feasible reactions were added to the decomposition pathway 

\subsection{Computational details}%%%%%%%%%%%%%%%%%%%%%%%%%%%%%%%
All geometry optimisations were performed in \ac{G09}, using the \acs{wb97xd} and \acs{B3LYP} functionals. Optimisations were to the level of 6-31+G(2df,p) with tight convergence criteria (table \ref{tab:convergence}). Chemical species were constructed using \ac{GView} and for molecules of more than 3 atoms, the “Clean” function was used to re-order atoms to a preliminary reasonable geometry. Optimisations were performed in both vacuum and with \ac{PCM} to introduce implicit solvent effects. Energies of optimised structures were checked against values listed on NIST Computational Chemistry Comparison and Benchmark Database \cite{JohnsonIII2018} where available.

\section{Results and Discussion}%%%%%%%%%%%%%%%%%%%%%%%%%%%%%%%%%%
%Collation of the above schemes to fit the starting products from denitration:
%The reaction energies for the proposed schemes 

Simplified schemes for the ageing reactions of \ac{NC} beginning from homolytic fission, elimination of \ce{HNO2} or acid hydrolysis are illustrated in schemes \ref{sch:homolytic} - \ref{sch:hydrolysis}. For the reactions starting from the products of homolytic fission, the propagation reactions are dominated by radical interactions. \ce{^{.}NO2} and \ce{HNO2} are consumed and regenerated, supporting the theory that these may be species contributing to the observed autocatalytic rate of decomposition, following a first-order rate induction period \cite{Rodger1963,Lindblom2002,Volltrauer1981}.  \ce{R=O} and \ce{N2O} are terminating species, which may go on to participate in wider reactions outside the scope of the proposed reactions. %why are they terminating species? - they do not go on to react and regenerate HNO2 and .NO2?

\textcolor{red}{Still to mention:\\
 - Describe the other two schemes\\
 - Describe the energies in the table\\
 - Discuss why some of the values may be positive. \\
 - Include enthalpies of reaction, zero point energies, and any experimental proxies I can find for the reaction enthalpies too. \\
 NOTE: ZPE energy correction means that you REMOVE the ZPE, so that you only compare the actual energy available for the reaction. 
 -  ?}

\begin{scheme}[htp!]
\centering
\schemestart
\textbf{\ce{R-ONO2}}\arrow(xx--aa){->[\textbf{homolysis}]}\textbf{\ce{R - O^{.} + ^{.}NO2}}
\arrow(aa--bb)[-25,2]Thermolyic fragmentation (equation \ref{equ:chinT2})
\arrow(@aa--cc)[-90,2]\ce{2^{.}NO2 + H2O <=> N2O4 + H2O}
\arrow(cc--dd){0}[-90,0.2]\ce{^{.}NO2 + R-OH -> ^{.}R-OH + HNO2}
\arrow(@dd--ee)[-90,1.5]\ce{N2O4 + H2O -> HNO3 + HNO2}
\merge>(dd)(ee)--(ff)\ce{^{.}R-OH + HNO3}
\arrow(@ff--nn)[90,1]\ce{R=O + H2O + ^{.}NO2}
%\arrow(@dd--ff)[-10,2.5]\ce{^{.}R-OH + HNO3 -> R=O + H2O + ^{.}NO2}
%\arrow(@ee--@ff)
\arrow(@ee--gg)[-90,1]\ce{HNO2 + R-OH <=> R-ONO + H2O}
\arrow(@ee--hh)[-25,3.5]\ce{HNO3 + R-OH -> R=O + HNO2 + H2O}
%\arrow(hh--ii){0}[-90,0.2]\ce{HNO3 + HNO2 <=> N2O4 + H2O} 
\arrow(@gg--jj)[-90,1.5]\ce{R-ONO -> R=O + HNO}
\arrow(jj--kk)[-90,1]\ce{2HNO -> HON=NOH}
\arrow(kk--ll){0}[-90,0.2]\ce{HNO + ^{.}NO2 -> HNO2 + ^{.}NO}
\arrow(ll--oo)[-90,1]\ce{2 ^{.}NO + O2 -> 2 ^{.}NO2}
%\arrow(@kk--mm)\ce{HON=NOH ->N2O + H2O}
\arrow(@kk--mm)\ce{N2O + H2O}
\schemestop 
\caption{Proposed degradation pathway starting from the homolysis products of a nitrate ester, derived from the schemes presented by Camera \cite{Camera1982} and Aellig\cite{Aellig2011}.}
\label{sch:homolytic}
\end{scheme}

%\subsection{Thermodynamics of Ethyl Nitrate reactions}%%%%%%%%%%%%%%%%%%%%%%%%%
%
%For an initial comparison of the methods you used, you could do a diagram like the one Kuklja did (kuja2014.pdf, page 89, fig 3.7).
%She also makes mention of the overestimation of activation barriers for pure DFT methods. Make sure you know the background surrounding this - why does this occur, and what is done to remedy it?
 
%\begin{figure}[h]
%  \centering
%	\includegraphics[width=0.8\linewidth]{name}
%\caption{\cite{}.}
%\end{figure}
%$\Delta$-G\textsubscript{r}
%$\textDelta$H\textsubscript{r}
 
%note, terminal up is right, terminal down is left
%\begin{scheme}[htp]
%\begin{table}
\begin{table}[htp]
\begin{center}
%\centering
\caption{Free energies of protonation for each oxygen site on ethyl nitrate.}
\begin{tabular}{ l l *{4}{S[table-format = 2.4]}} 
\toprule
\multicolumn{2}{l}{\multirow{2}{*}{Protonated site}} & \multicolumn{4}{c}{$\Delta$G\textsubscript{r} /kcal mol\textsuperscript{-1}} %& \multicolumn{2}{c}{$\Delta$H\textsubscript{r} }
\\\cline{3-6}
  & & \acs{wb97xd} & PCM & \acs{B3LYP} & PCM\\
\midrule
% Right is up, left is down, with ethanol in an “m” shape. 
 Terminal (upper) & \ce{CH3CH3ONO2H+} & -12.276810	& 8.821890 & -13.782510 & 5.625270\\
 Terminal (lower) & \ce{CH3CH3ONO2H+}  &-9.475200 &	9.459450	&-11.132100	&5.646060 \\ 
 Bridging & \ce{CH3CH3O(H+)NO2} &-9.322740	& 9.058140	& -15.309630 &	6.673590 \\
\bottomrule
\end{tabular}
\label{tab:reactions}
\end{center}
\end{table}

%replace pics with the atom labelled ones
%\begin{figure}
\begin{figure}[htp]
\centering
\begin{subfigure}[b]{0.3\linewidth}
\centering
\includegraphics[width=\linewidth]{terminal_r_up}
 \caption{Upper terminal oxgen.\\
 Bond(O--\ce{NO2}): \num[round-mode=places,round-precision=2]{1.27787} \AA} 
 \label{fig:t_l_up}
\end{subfigure}
\begin{subfigure}[b]{0.3\linewidth}
\centering
\includegraphics[width=\linewidth]{terminal_l_down}
\caption{Lower terminal oxgen.\\
 Bond(O--\ce{NO2}): \num[round-mode=places,round-precision=2]{1.27863} \AA} 
 \label{fig:t_r_down}
\end{subfigure}
\begin{subfigure}[b]{0.3\linewidth}
\centering
\includegraphics[width=\linewidth]{bridging}
\caption{Bridging oxgen.\\
 Bond(O--\ce{NO2}): \num[round-mode=places,round-precision=2]{1.97967} \AA} 
 \label{fig:bridge}
\end{subfigure}
 \caption{Optimised geometries of the possible protonation sites on ethyl nitrate.}
 \label{fig:en_protonation}
\end{figure}
%\end{scheme}

Due to the availability of oxygen sites on the ethyl nitrate molecule, the optimal site for protonation was determined for inclusion in the reaction scheme for the first stage of hydrolysis. Table \ref{tab:reactions} shows the protonation energies for the three different oxygen sites on ethyl nitrate. Despite the upper terminal oxygen possessing the most thermodynamically favourable energy of protonation, inspection of the reaction geometries shows that the bridging structure most resembles that expected for the liberation of the \ce{NO2+} group at the next step. Though appearing less thermodynamically favourable when compared to protonation at the terminal upper oxygen site, the higher energy of reaction likely arises from the instability of the protonated complex. The elongation of the O--\ce{NO2} bond allows to stabilisation of the proton at the bridging site, such that the departure of \ce{NO2+} is easily facilitated. Subsequent calculation involving the energy of the protonated ethyl nitrate will employ the values associated with the protonated bridging site. %The \ce{CO---NO2} bond lengths are (a) \num[round-mode=places,round-precision=4]{1.27787} \AA (b) \num[round-mode=places,round-precision=4]{1.27863} \AA (c) \num[round-mode=places,round-precision=4]{1.97967} \AA
%Comment on the diff between wb97xd and B3LYP.%

%something about why the energy of the HNO3 reaction looks so rubbish (but that it stilll degrades at room temp, appaz. But check the literature)
For the decomposition of \ce{HNO3} to \ce{{.}NO2},  \ce{2H2O} and \ce{O2}, Aellig prescribes the use of an amberlyst catalyst (amberlyst-15).
%\cite{Ellis2007,Robertson1955}
%The propagation reactions are acid catalysed by \ce{HNO2}. 
%
%\subsubsection{Radical mechanistic route}
%
%\subsubsection{Ionic mechanistic route}
%
%\subsection{Reactions of Nitrocellulose Monomer}
%\section{ \textit{(Kinetics of Ethyl Nitrate)}}
%\subsection{Radical mechanistic route}
%\subsection{Ionic mechanistic route}
%
%\section{ \textit{(Kinetics of Nitrocellulose Monomer)}}

%Insights into the reaction energies
%“HNO cannot be stored or concentrated and is typically studied using donor species that release HNO as a decomposition product.” (paper since retracted)

\begin{scheme}[ht!]
\centering
\schemestart
\textbf{\ce{R-ONO2}}\arrow(xx--aa){->[\textbf{Elimination}]}[,1.5]\textbf{\ce{R=O + HNO2}}
\arrow(@aa--bb)[-140,2]\ce{HNO2 + R-OH <=> R-ONO + H2O}
\arrow(bb--cc)[-90]\ce{R-ONO -> R=O + HNO}
\arrow(cc--dd)[-90]\ce{2HNO -> HON=NOH}
%\arrow(dd--gg){0}[-90,0.2]\ce{HNO + ^{.}NO2 -> HNO2 + ^{.}NO}
\arrow(dd--ee)[-90]\ce{HON=NOH -> N2O + H2O}
%\arrow(@aa--ff)[-35,2]\ce{R=O}
%\arrow(@gg--hh)\ce{2 ^{.}NO + O2 -> 2 ^{.}NO2}
\schemestop 
\caption{Proposed degradation pathway starting from the elimination of \ce{HNO2} from a nitrate ester, derived from the schemes presented by Camera \cite{Camera1982} and Aellig\cite{Aellig2011}.}
\label{sch:elimination}
\end{scheme}

\begin{scheme}[ht!]
\centering
\schemestart
\textbf{\ce{R-ONO2 + H+}}\arrow(xx--aa){->[\textbf{Hydrolysis}]}[,1.5]\textbf{\ce{R-OH + NO2+}}
\arrow(@aa--bb)[-150,2]\ce{NO2+ + 2H2O <=> HNO3 + H3O+}
%\merge>(@aa--@bb)
%\arrow(@aa--cc)[-30,2]\ce{R-OH + HNO3 -> R=O + HNO2 + H2O}
\arrow(@bb--cc)[-30,2]\ce{HNO3 + R-OH -> R=O + HNO2 + H2O}
\arrow(@aa--@cc)[-90]
\arrow(@cc--dd)[-90]\ce{HNO2 + HNO3 <=> N2O4 + H2O} 
\arrow(@dd--ee){0}[-90,0.2]\ce{HNO2 + R-OH <=> R-ONO + H2O}
%\arrow(@dd--ff)[180]\ce{N2O4 + H2O -> HNO3 + HNO2}
%\arrow(ff--gg){0}[-90,0.2]\ce{N2O4 <=> 2^{.}NO2}
\arrow(@dd--gg)[180]\ce{N2O4 <=> 2^{.}NO2}
\arrow(@ee--hh)[-90]\ce{R-ONO -> R=O + HNO}
\arrow(hh--ii)[-90]\ce{HNO + ^{.}NO2 -> HNO2 + ^{.}NO}
\arrow(ii--kk){0}[-90,0.2]\ce{2HNO -> HON=NOH}
\arrow(@gg--ii)[-50,3.3]
\arrow(@kk--jj)[-90]\ce{HON=NOH -> N2O + H2O}
\arrow(@ii--ll)[180,1]\ce{2 ^{.}NO + O2}
\arrow(@ll--mm)[180,0.5]\ce{2^{.}NO2}
\schemestop 
\caption{Proposed degradation pathway starting from the acid hydrolysis of a nitrate ester, derived from the schemes presented by Camera \cite{Camera1982} and Aellig\cite{Aellig2011}.}
\label{sch:hydrolysis}
\end{scheme}

\begin{table}[h]
\begin{center}
\caption{Energies of nitrate ester decomposition reactions proposed by Camera \cite{Camera1982}, Chin \cite{Chin2007} and Aellig \cite{Aellig2011}. R = \ce{CH3CH2} for ethyl nitrate, and R = \ce{(H3CO)2C6H9O3} (bi-methoxy capped glucopyraonse monomer unit).}
\begin{tabular} { l *{4}{S[table-format = 2.4]}} 
\toprule
\multirow{2}{*}{Reaction}	& \multicolumn{4}{c}{$\Delta$G\textsubscript{r} /kcal mol\textsuperscript{-1}}
\\\cline{2-5}
			& \acs{wb97xd} & PCM & \acs{B3LYP} & PCM\\
\midrule
%\ce{CH3CH2ONO2 + H+ <=>[\text{fast}] CH3CH2ONO2H+} &
%\ce{CH3CH2ONO2H+ <=>[\text{slow}] CH3CH2H + NO2+} & 14.930370&-2.247210&16.012710&-4.038930\\
%[\text{fast}]
\ce{NO2+ + 2H2O <=> HNO3 + H3O+}& -0.896490 & -1.338750 & 1.770300 & 2.464560\\
%\ce{R-ONO2 -> R-O^{.} + ^{.}NO2 } \\ Homolysis
\ce{2^{.}NO + O2 -> 2^{.}NO2} &-20.77047&	-21.97314&-21.16044&-22.15899
 \\
\ce{2^{.}NO2 <=> N2O4} &-0.122220&	-1.310400	&0.541800	&0.155610
\\
\ce{HNO3 + HNO2 <=> N2O4 + H2O} 	&	-2.251620	&	-1.854090	&	-5.131350	&	-4.180050 \\
\ce{N2O4 <=> 2^{.}NO2}	&	0.123480&	1.461600&	-0.539280&	-0.155610
 \\
\ce{4HNO3 <=> 4NO2 + 2H2O + O2}&53.35029&58.36446&42.60942&46.93563\\
\ce{2^{.}NO2 + H2O <=> N2O4 + H2O}&-0.12222&-1.4616&0.53928&0.15561\\
\ce{N2O4 + H2O -> HNO3 + HNO2}&2.25162&1.85409&5.13135&4.18005\\
\ce{^{.}NO2 + HNO -> HNO2 + ^{.}NO}&-28.21644&-28.66815&-27.32688&-27.6255\\
\ce{2 ^{.}NO + O2 -> 2 ^{.}NO2}&-59.89473&-60.47244&-60.46866&-60.99597\\
\ce{2HNO -> HON=NOH}&-38.96928&-39.71583&-36.62757&-37.40814\\
\ce{HON=NOH -> N2O + H2O}&-48.08286&-48.18429&-50.55309&-50.74902\\
\midrule
Ethyl nitrate ( R = \ce{CH3CH2} )\\
\hline
%\ce{R-ONO2 + H2O -> R-OH + HNO3}	&	4.556160	&	5.235930	&	4.000500	&	4.860450 \\ Hydrolysis
\ce{R-OH + HNO3 -> R=O + HNO2 + H2O}	&	-34.062210	&	-38.427480	&	-37.593990	&	-41.770260 \\
\ce{R-OH + ^{.}NO2 -> ^{.}R-OH + HNO2}	&	16.376220	&	13.923000	&	15.887340	&	13.699350 \\
\ce{^{.}R-OH + HNO3 -> R=O + H2O + ^{.}NO2}&-50.438430&-52.35048&-53.48133&-55.4715\\
\ce{R-OH + HNO2 <=> R-ONO + H2O}&-3.20544&-3.276&-2.64096&-2.94903\\
\ce{R-ONO -> R=O + HNO}&-1.49625&-5.82183&-4.36716&-8.50122\\
\midrule
\ac{NC} monomer ( R = \ce{(H3CO)2C6H9O3} )\\
\hline
\ce{R-ONO2 + H2O -> R-OH + HNO3}	&	0.67536	&	5.63094	&	0.61236	&	-0.70119 \\
%RONO2 &+&H3O+& <->&RONO2H+&+&H2O&(see protonation section)&&&-30.87756&3.5464&-31.98636&-0.24507
%\ce{R-OH + HNO3 -> R=O + HNO2 + H2O} &	-36.72522&	-38.34306	&	-41.71419	&	-41.70411 \\ Hydrolysis
\ce{R-OH + ^{.}NO2 -> ^{.}R-OH + HNO2}	&	14.71302	&	11.15163	&	13.03407	&	23.20983 \\
\ce{^{.}R-OH + HNO3 -> R=O + H2O + ^{.}NO2} &
-51.438240	&-49.494690	&-54.748260 &	-56.369250 \\
\ce{R-OH + HNO2 <=> R-ONO + H2O}&-4.43142&-7.30233&-4.30605&-0.17829\\
\ce{R-ONO -> R=O + HNO}&-2.93328&	-1.71108	&-6.82227&	-11.20581
\\
\bottomrule
\end{tabular}
\end{center}
\end{table}

\section{Summary}
%
%A disadvantage is that \ce{N2O} is a significant greenhouse gas, and cannot be re-converted back to \ce{HNO3}, so much be considered win experimental design when observing industrial and environmental impact. 