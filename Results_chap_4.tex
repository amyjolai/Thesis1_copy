%% whilst you're still writing up
%\setcounter{chapter}{3}

\chapter{Post-Denitration Reactions}
\label{chapterlabel6}
\graphicspath{ {./R_chap_4_pics/} }

% In this chapter:
% Which reactions happen post denitration, and which don't (based on pure thermodynamics)
% Which ones contribute to the end products seen in literature, and which much be consumed in subsequent reactions
% {Mention that this list is not likely to be exhaustive, but cover some of the more ''obvious'' ones}

%[Fix the arrows of the reaction schemes? So that eluting products leave with a u shaped arrow?]

\section{Introduction}%%%%%%%%%%%%%%%%%%%%%%%%%%%%%%%%%%%%%
% Talk about the prevalence and importance of secondary + products. 
% Statement - it is these secondary reactions that do the majority of the decomposition, and breakdown of the polymer such that the experimental observables are produced. 
 %Following the preliminary denitration step, products of intramolecular or hydrolytic nitrate elmination are evolved as gases or remain trapped in the \ac{NC} matrix. 
%The full mechanism of \ac{NC} decomposition is complex. 
%
% Cover the NMR of the product species that I observed, and see if I can match it with any experimental NMR and IR observations. 
% Also consider the denitration order here, if there is time - include diagram of different DOS subsitution options, as in Saunders1990, pg 187 
Products of the preliminary denitration step of \ac{NC} can be evolved as gases or remain trapped in the polymer matrix. 
Reactive nitrogen dioxide radicals generated from homolysis of the O-N bond %not evolved as \ce{NO2} gas, 
are likely to migrate within the bulk and attack other sites on the polysaccharide, initiating branched radical chain reactions. %REF % and free species  by a radical chain reaction
These lead to deeper decomposition of the polymer \textit{via} % at later stages of decomposition
chain scission and rupture of glucose rings, with eventual complete disintegration of the molecule, assisted by products released by ongoing acid hydrolysis. %REF
Nitrous and nitric acids are released directly from denitration or via transformation of released NO\textsubscript{x} species. In addition to catalysing hydrolysis, they increase the acidity of the overall system, lowering the pH and stimulating further hydrolytic processes \cite{Hu2011}.  %Plus the self heating, more bond dissociations, etc. 
The final product mixture is dictated by the numerous side reactions involving autocatalysis, radical reactions and product interactions. 

When studying the ageing of \ac{NC} using \ac{UVVis}, Moniruzzaman \textit{et al.} observed increasing concentrations of reaction products, beyond those generated from first-stage decomposition, with increasing heat treatment and over longer timescales \cite{Moniruzzaman2008,Moniruzzaman2014}. %over extended
%\acs{UV} absorption of an anthraquinone dye was used to determine the activation energies for the removal of a nitrate. 
The studies used the reaction of nitrates with an anthraquinone dye (\acs{SB59}) (figure \ref{fig:anthraquinone_SB59}) to probe the reactivity at each of the C2, C3 and C6 sites of \ac{NC}, using \textsuperscript{1}H NMR spectroscopy and \acs{UVVis} (figure \ref{fig:UV_all}).
\begin{figure}%[ht!]
  \centering
	\includegraphics[width=0.35\linewidth]{anthraquinone_SB59}
\caption[\acf{SB59} used to probe the release of nitrates from \ac{NC} using \ac{UVVis} and \textsuperscript{1}H NMR spectroscopy]{\acf{SB59} used to probe the release of nitrates from \ac{NC} using \ac{UVVis} and \textsuperscript{1}H NMR spectroscopy \cite{Moniruzzaman2014}. The action of nitrate absorption by the dye imitates that of stabilisers commonly used with nitrate ester formulations. %Following reaction with NO\textsubscript{x}, the UV absorption peak of the dye is depleted.
}
\label{fig:anthraquinone_SB59}
\end{figure}
The reaction of \acs{SB59} with \ce{NO_{x}} released by denitration (figure \ref{fig:SB59_NOx}) mimics the action of stabilisers such as \ac{DPA} and \ac{2NDPA} commonly used in \ac{NC} formulations. 
The secondary amine groups of the dye consume any nitrates in the system, eliminating the possibility of successive reactions generating acidic species. 
\begin{figure}%[htp!]
\centering
  \begin{subfigure}[b]{0.55\linewidth}
	\caption{\ac{NC} film aged at 40\degree C.}
  \includegraphics[width=\linewidth]{40-monirazzuman-UV-2014}
  \end{subfigure}
    \begin{subfigure}[b]{0.55\linewidth}
	\caption{\ac{NC} film aged at 50\degree C.}
  \includegraphics[width=\linewidth]{50-monirazzuman-UV-2014}
  \end{subfigure}
  \begin{subfigure}[b]{0.55\linewidth}
    \caption{\ac{NC} film aged at 60\degree C.}
    \includegraphics[width=\linewidth]{60-monirazzuman-UV-2014}
  \end{subfigure}
  \begin{subfigure}[b]{0.55\linewidth}
    \caption{\ac{NC} film aged at 70\degree C.}
    \includegraphics[width=\linewidth]{70-monirazzuman-UV-2014}
    \label{fig:boldUV}
  \end{subfigure}
\caption[\acs{UVVis} spectra of aged \ac{NC}-based film]{\acs{UVVis} spectra of aged \ac{NC}-based film, from the work of Moniruzzaman \textit{et al.}\cite{Moniruzzaman2014}. The peaks at 600 nm and 650 nm are attributed to the $\pi$ - $\pi$* transitions in the anthraquinone dye (\ac{SB59}). Spectral lines with highest absorbance peaks in this region correspond to the sample prior to heat treatment. Appearance of peaks below 400 nm indicate the formation of new \acs{SB59} derivatives due to secondary reactions. 
\added{Adapted with permission from the publisher. } %The spectral lines with the highest absoprtions at the double peak region (with crests at 600 nm and 650 nm) correspond to the sample before heat treatment, with the highest concentration of \acs{SB59} dye.
}
\label{fig:UV_all}
\end{figure}
Un-aged \ac{NC} thin films, and films aged at 40\textdegree C, 50\textdegree C, 60\textdegree C and 70\textdegree C for timescales of up to 2000 hrs for 40\textdegree C, were compared. 

\acs{UV} absorbances at 600 nm and 650 nm were characteristic of the \ac{SB59} dye before reaction with \ce{NO_{x}}. 
The isosbestic point identified at 552 nm showed a proportional relationship between the decrease in concentration of \acs{SB59} as it was consumed%decreased
, with the concentration of the [\acs{SB59} + \ac{NC}] product as it increased. % accordingly. %though not necessarily proportionally

For samples aged at temperatures $>$40\degree C, the isosbestic point demonstrated a downwards shift with increasing dye consumption. The drift from the isosbestic point, in addition to the appearance of new absorbance peaks below 400 nm, allude to the presence of new species in the reaction mixture not generated by the primary reaction of \acs{SB59} and \ac{NC}. It is likely that these arise from the continued reaction of \acs{SB59} derivatives with \ac{NC} degradation products, or further derivatives thereof, as suggested in figure \ref{fig:SB59_NOx} III, IV and V. 
%
\begin{figure}%[hbp]
  \centering
	\includegraphics[width=0.55\linewidth]{SB59_NOx_reaction}
\caption[Proposed pathway for the reaction of \acs{SB59} dye with ${\rad}$NO released as a result of denitration of \ac{NC}.]{Proposed pathway for the reaction of \acs{SB59} dye with ${\rad}$NO released as a result of denitration of \ac{NC} \cite{Moniruzzaman2014}. 
\added{Reproduced with permission from the publisher.}}
\label{fig:SB59_NOx}
\end{figure}
In the case of the 70\degree C treated run, the final measurement (indicated by the royal-blue line in bold in figure \ref{fig:boldUV})) deviated from the isosbestic point entirely, with more than 81\% consumption of the original dye concentration. This suggests that the samples exposed to the higher ageing temperatures presented spectra dominated by products formed \textit{via} secondary reactions. 
%The \ac{NC} films with higher ageing temperatures demonstrated a greater loss of \ac{SB59} dye absorbances, and more pronounced peaks corresponding to secondary reaction products below 400 nm. 
%Though figure \ref{fig:SB59_NOx} illustrates the reaction of the dye with NO groups, the study makes no indication of the source of NO\textsubscript{x}, except that they are products of thermolysis of \ac{NC}. 

Following cleavage of the nitrate ester \textit{via} homolytic fission, elimination of \ce{HNO2}, or hydrolysis, the resulting residues are available for further reaction with the polymer or other free molecules in the system. 
%The thermolytic scheme highlights the key nitrous oxide reations; a number of second-order degradation routes depend on the ...
In the study of \ac{PETN} ageing at high temperatures (115\degree C - 135\degree C) in vacuum, and low temperatures (20\degree C - 65\degree C) in acetonitrile solution, Shepodd \textit{et al.} commented that thermolysis produced a more complex and varied mixture, due to deeper degradation and recombination of radicals \cite{Shepodd1997}. By contrast, the low temperature hydrolytic process emphasised formation of \ac{PETRIN} was followed by side reactions with reduced likelihood of radical recombination in solution compared to in a solid, as \rad\ce{NO2} would be more likely to diffuse and react elsewhere. 

Chin \textit{et al.} proposed schemes for the propagation of secondary reactions initated by both the thermolysis (scheme \ref{sch:chinschema1}) and hydrolysis of nitrate esters (scheme \ref{sch:chinschema2}) \cite{Chin2007}. %(Say something about the thermolyti scheme.)
%Termination reactions were not emphasised in either of the schemes for these cases. 
The hydrolysis scheme was adapted from an earlier work by Camera \textit{et al.}  %(first referenced in section \ref{equ:camera_first}%\ref{chapter5:intro}
involving the nitrate ester decomposition and subsequent reactions of \ac{EN} (where R = \ce{CH3CH2} for the scheme above) \cite{Camera1982}. The original study included an expansion of the hydrolysis step (equation \ref{equ:protonation}), where the involvement of \ce{NO2+} was illustrated (scheme \ref{sch:cameraschema_full}).

% Fix alignment left, please												
\begin{scheme}%[h]
%\noindent\textbf{Thermolytic initiation}
\caption{\textbf{Thermolytic initiation} 
\added{proposed by Chin \textit{et al.} \cite{Chin2007}}
}
%\textit{Initiation:}
\begin{equation}
%\textit{Initiation}
%\begin{block_indent}{1cm}
\ce{R-ONO2 -> R-O{\rad} + {\rad}NO2 }
%\end{block_indent}
\label{equ:chinT1}
\end{equation}
%
\textit\textbf{Propagation}
%\textit{Propagation}
%\begin{block_indent}{1cm}
\begin{equation}
\begin{split}
%\begin{multline}
%\ce{RO{\rad} + {\rad}NO2 ->[\ce{RONO2}] RONO2 + NO + N2O4 + {\rad}NO2}\\
%								\ce{ + H2O  + N2O + CO2 + CO } \\ 
%								\ce{+ C2H2O +} \text{other organic fragments}
\ce{R-O{\rad} + {\rad}NO2 ->[\ce{R-ONO2}]} & \ce{ R-ONO2 + {\rad}NO + {\rad}NO2}\\
								& \ce{ + N2O4 + H2O  + N2O} \\
								& \ce{+ CO2 + CO + C2H2O} \\
								& \ce{+} \text{other organic fragments}
%\end{multline}
\end{split}
\label{equ:chinT2}
\end{equation}
\begin{equation}
\ce{2{\rad}NO + O2 -> 2{\rad}NO2} 
\label{equ:chinT3}
\end{equation}
\begin{equation}
\ce{2{\rad}NO2 <=> N2O4}
\label{equ:chinT4}
%\end{block_indent}
\end{equation}
\label{sch:chinschema1}
\end{scheme}
%
It was highlighted by Camera, that the oxidation of alcohol by nitric acid (equation \ref{equ:chinH2}) is slow and thus rate-limiting. This mechanism is likely to occur \textit{via} a series of intermediate reactions of which the details are not known.
However, following the generation of nitrous acid, subsequent oxidations occur rapidly. %why and how? 
According to Rigas \textit{et al.}, alcohols are more susceptible to wet oxidation than esters \cite{Rigas1997}. A higher concentration of unsubstituted hydroxyl groups in the system, and therefore a fewer nitrate ester groups (or a lower \ac{DOS} value), decreases overall stabililty. % What is the mechanism, and why? This would suggest that below a certain threshold \ac{DOS} value, the oxidation of hydroxyl groups (equation \ref{equ:chinH2}) would dominate over the hydrolytic denitration reaction (equation \ref{equ:chinH2}). therefore indicates that a higher concentration of -OH groups present in the system decreases stabililty. 
% Check that this doesn’t contradict the point made in Chap 2, about the inductive effect of adjacent nitrates on the reactivity/ predisposition to denitrate easier, or if it does, ref the two contrasting papers. 

Equations \ref{equ:chinH3} - \ref{equ:chinH6} describe a possible branched radical chain mechanism, fed by the nitrous and nitric acids produced by the hydrolysis and alchohol oxidation reactions during the initiation stage. By contrast, the propagation reactions in the branched radical chain mechanism for thermolysis are poorly characterised (equation \ref{equ:chinT2}) and defined only by the observable products. Due to their rapid and varied nature, these reactions have been difficult to follow spectroscopically. %REF
%but eventually, everything feeds back into that equation of just radical deg, esp for the species contianing carbon 
%Who actually did the measureing?   \ce{PhCH2ONO2}
%measured by \ac{IR}, liquid \cite{Bluhm1977}, ion \cite{lopezlopez2011} and gas chromatography \cite{Huwei1988},  ignition loss determination \cite{Bluhm1977}. %and others
%
\begin{scheme}%[h]
%\begin{scheme}
%{\noindent\textbf{Hydrolytic initiation}}
\caption{\textbf{Hydrolytic initiation} 
\added{proposed by Chin \textit{et al.} \cite{Chin2007}}
}
%\subsubsection*{Hydrolysis}
%\textit{Initial}
%\textit{Initiation:}
%\begin{block_indent}{1cm}
\begin{equation}
\ce{R-ONO2 + H2O -> R-OH + HNO3} 
\label{equ:chinH1}
\end{equation}
\begin{equation}
\ce{R-OH + HNO3 -> R=O + HNO2 + H2O}
\label{equ:chinH2}
\end{equation}
%\end{block_indent}
\textit\textbf{Propagation}
%\textit{Propagation}
%\begin{block_indent}{1cm}
\begin{equation}
\ce{HNO3 + HNO2 <=> N2O4 + H2O} 
\label{equ:chinH3}
\end{equation}
\begin{equation}
\ce{N2O4 <=> 2{\rad}NO2}
\label{equ:chinH4}
\end{equation}
\begin{equation}
\ce{R-OH + {\rad}NO2 -> {\rad}R-OH + HNO2}
\label{equ:chinH5}
\end{equation}
\begin{equation}
\ce{{\rad}R-OH + HNO3 -> R=O + H2O + {\rad}NO2}
\label{equ:chinH6}
\end{equation}
%\end{scheme}
\label{sch:chinschema2}
\end{scheme}
% 
\begin{scheme}%[h!] 
%\textbf{Hydrolysis scheme for ethyl nitrate}
\caption{\textbf{Hydrolysis scheme for ethyl nitrate} 
\added{from the work of Camera \textit{et al.} \cite{Camera1982}}
}
\label{sch:cameraschema_full}
%\begin{block_indent}{1cm}
%\ce{CH3CH2ONO2 + H+ <=>[\text{fast}] CH3CH2ONO2H+} \\
%\ce{CH3CH2ONO2H+ <=>[\text{slow}] CH3CH2H + NO2+} \\
%\ce{NO2+ + 2H2O <=>[\text{fast}] HNO3 + H3O+} \\
%\end{block_indent}
\begin{equation}
\ce{CH3CH2ONO2 + H+ <=>[\text{fast}] CH3CH2ONO2H+} \\
\label{equ:protonation}
\end{equation}
\begin{equation}
\ce{CH3CH2ONO2H+ <=>[\text{slow}] CH3CH2OH + NO2+} \\
\end{equation}
\begin{equation}
\ce{NO2+ + 2H2O <=>[\text{fast}] HNO3 + H3O+} \\
\end{equation}
\end{scheme}
%\end{block_indent}

Aellig \textit{et al.} presented an alternative scheme for the decomposition of benzyl nitrate (R = \ce{PhCH2}) in scheme \ref{sch:aelligschema}, involving more interaction with the solvent \cite{Aellig2012}. %:\\ %\ce{PhCH2ONO2}, 
Both the Camera/Chin and Aellig schemes above produce final end products observed in the decomposition of \ac{NC}. In particular, Aellig’s scheme accounts for the production of \ce{N2O}, which forms a significant part of the decomposition eluent \cite{Buelow2002}. %Whislt not a full-resolution, exhaustive depiction the full spectrum of reactions that take place in the \ac{NC} matrix during it’s slow ageing, the presented reactions encapsulate the early major reactions of the most prevalent and active species in the system.
Whilst the schemes do not a propose an exhaustive description of the full spectrum of reactions that take place in the \ac{NC} matrix during its slow ageing, the early stage reactions of the key species responsible for decomposition are encapsulated.
%{Aellig's equations}
\begin{scheme}[h!]
%\noindent\textbf{\ce{HNO3} decomposition}% initiated}
\caption{\textbf{\ce{HNO3} decomposition initiation} 
\added{proposed by Aellig \textit{et al.} \cite{Aellig2012}.}
}
%\textit{Initiation}
\begin{equation}
\ce{4HNO3 <=> 4{\rad}NO2 + 2H2O + O2}
\end{equation}
\begin{equation}
\ce{2{\rad}NO2 + H2O <=> N2O4 + H2O}
\end{equation}
\begin{equation}
\ce{N2O4 + H2O -> HNO3 + HNO2}
\end{equation}

\textit{Propagation}
\begin{equation}
\ce{R-OH + HNO2 <=> R-ONO + H2O}
\end{equation}
\begin{equation}
\ce{R-ONO -> R=O + HNO}
\end{equation}
\begin{equation}
\ce{{\rad}NO2 + HNO -> HNO2 + {\rad}NO}
\end{equation}
\begin{equation}
\ce{2 {\rad}NO + O2 -> 2 {\rad}NO2}
\end{equation}

\textit{Termination}
\begin{equation}
\ce{2HNO -> HON=NOH}
\end{equation}
\begin{equation}
\ce{HON=NOH -> N2O + H2O}
\end{equation}
\label{sch:aelligschema}
\end{scheme}
%autocatalytic rate 
%%%%%%%%%%%%%%%% FIX / CHECK THIS SECTION %%%%%%%%%%%%%%%%%%%%%
% confused rates of thermolytic degradation: \cite{brill1997}
	% Despite good understanding about the initial step of the process, the kinetics of thermolysis of NC are complicated by the existence of rapid secondary reactions, autocatalysis, and self-heating. As a result, further kinetic descriptions of the process are fragmented and somewhat contradictory. To organize this subject, the past studies can be grouped according to the temperature range of study.''
% info on rates of hyrolysis
%
% what I actually want to focus on here, are the reactions AFTER intial hydrolysis.
% So knowledge on the rates quoted in literature are good, but I need to know what they correspond to:
	% All reactions, with just an increase in rate as the system gets hotter / more reactions splinter off? (ie more reactions happening, so more exothermic heat release - self heating
	% Does the sudden change to autocatalytic arise from a global change in the mechanism? / All the of the “first order” reaction stuff is done / it’s much less important, and that the 

It is widely agreed that first-stage decomposition follows a first-order process (or pseudo-first order, with respect to hydrolysis reactions). %REF for first & psuedo first order rate for NC
A number of studies observe a catalytic rate of decay for the longer-term aging processes. %REF the studies in DAUERMAN's who quoted an autocatalytic rate. 
Dauerman and Tajima \cite{Dauerman1968} observed that when \ac{NC} was treated with \ce{NO2} gas before heating, the time required for sample ignition halved. He suggested that the \ce{NO2} adsorbed onto the surface acted as a catalysing agent. 
% Now, how do the above rate observations relate to the post-denitration phases of reaction?

Neutral and alkaline hydrolysis reactions follow a pseudo-first order process, however it has been suggested that the presence of acid facilitates a catalytic rate of degradation after an initial incubation period. %Incubation is not correct - what I want to say is that the reaction starts off first order, but at an obeservable inflection point, becomes catalyitc. 
%Check whether i am jsut repeating myself from chap 2?
%The degradation of cellulose also follows a pseudo-first order rate\cite{Calvini2008}.
%
% Introducing the decondary reaction products %%%%%%%%%%%%%%%%%%%%
%Discuss the work of Camera, Aellig and Chin 2007
Multiple studies have addressed the decomposition reactions of nitrate esters following the initial scission of the nitrate group \cite{Hu2011,Camera1982,Camera1983,Baker1952,Matveev2003}. %ETC refine this list later
%Kuklja made mention of some peroxy intermediates / transition states in the denitration of PETN.
In their work looking into the atmospheric reactions of methynitrate and methylperoxy nitrate Arenas \textit{et al.} suggested it was possible for the homolytic denitration reaction of methynitrate to share a common peroxy intermediate with the peroxide \cite{Arenas2008}. This could account for some of the lower order NO\textsubscript{x} generated. %(just \ce{{\rad}NO}, tbh, or maybe perhaps how NO3 may get generated?).
%Baker1950
%%%%%%%%%%%%%%%%%%%%%%%%%%%%%%%%%%%%%%%%%%%%%%%%%%
In this section, secondary and extended reaction schemes for the low temperature ageing of \ac{NC} are explored. Decomposition pathways defined by Chin, Camera \textit{et al.} and Aellig \textit{et al.} are probed to determine the reactions responsible for the experimentally observed degradation products. 
The reactions found to be energetically feasible from the proposed routes will be scrutinised to determine whether an autocatalytic pathway can be formed from the thermodynamically validated reaction schemes. 

\section{Methodology}%%%%%%%%%%%%%%%%%%%%%%%%%%%%%%%%%%%%%
The reactions proposed by Chin \textit{et al.}, Camera \textit{et al.} and Aellig \textit{et al.} were used to construct possible degradation routes for \ac{NC}, each route starting with either the %Proceeding on from the initial denitration step, 
products of homolytic fission, elimination of \ce{HNO2} or acid hydrolysis. 
\added{The energies of each reaction in the schemes above were determined after optimisation of the individual reactant and product species. }
%, generating a for each proposed degradation pathway. 
Pathways were constructed based on propagation of the given reactions in a step-wise fashion; subsequent reactions were dependent on the products generated in preceding steps. %, in addition to the assumed availability of other reactants in the system. %These were at constant concentration REF
An abundance of water and oxygen were assumed present in the system, attributed to air exposure under the wetted storage conditions of \ac{NC}. %REF
Unsubstituted alcohol moieties (R-OH) were also presumed abundant due to incomplete nitration during the synthesis of \ac{NC} \cite{Wolf1997}, and re-generation following denitration \textit{via} hydrolysis. 
The schemes were modelled with \acf{EN} as a smaller test system, then expanded to the \ac{NC} monomer. Free energies of reaction ($\Delta G_{r}$) at 298.15 K were used to determine the feasibility of each reaction:
\begin{equation}
\added{\Delta G_{r}=\Delta G_{product}-\Delta G_{reactant}}
\end{equation}
\added{Where computational or experimental literature values for molecular energies or for reactions energies were available, these were compared with the results generated here. For the cases where the $\Delta G_{r}$ was large and positive, the reaction was omitted from the constructed reaction schemes as it would be unlikely to occur spontaneously under ambient ageing conditions, even when considering the possibility of increased heating in the system as degradation progressed. }
%Where the choice of method lead to a variation in the result of a reaction, the geometries around the reaction centres were further scrutinised in order to ensure no spurious behaviour due to artefacts from functional choice. [Bit of a dodgy sentence, how would you know what to say was right / wrong?]
%The species reactions  were geometry optimised using \acs{wb97xd} , and \acs{B3LYP} functionals, in both vacuum and solvent. The reactions were modelled using \ac{EN} as a test system before expansion to the full C2 monomeric model. 
%The \ac{DG} were used to determine . 

\subsection{Computational details}%%%%%%%%%%%%%%%%%%%%%%%%%%%%%%%
%What you should’ve - geometry optimisations with whatever method, to a reasonably high level, then taken whichever geometry had the lowest energy. Then done the thermochemistry to an even higher level, using whatever methods you wanted, using the geometry of whatever method opt’d previously. 
All geometry optimisations were conducted in \ac{G09}, using the \acs{wb97xd} and \acs{B3LYP} functionals. Optimisations and thermochemistry calculations were performed to the level of 6-31+G(2df,p) with tight convergence criteria \added{(max. force 1.5$\times 10^{-5}$ H/Bohr, \acs{RMS} force 1.0$\times 10^{-5}$, max. displacement 6.0$\times 10^{-5}$ H/Bohr and \acs{RMS} displacement 4.0$\times 10^{-5}$ H/Bohr, chapter \ref{chapterlabel3} table \ref{tab:convergence}) and zero point energy corrected. }
Calculations were repeated in both vacuum and with a \ac{PCM} \added{using water ($\epsilon$=78.4) } to introduce implicit solvent effects. Chemical species were constructed using \acf{GView} \cite{GV5} and for molecules of more than 3 atoms, the \ac{GView} “Clean” function was used to re-order atoms to a preliminary starting geometry. Energies of optimised structures were checked against literature values listed on NIST Computational Chemistry Comparison and Benchmark Database \cite{NCCCBD2020} if analogous molecules to a similar level of theory existed. 
\added{In most cases, the exact same level of theory was not available but a similar level could still be meaningfully referenced; for example, \acs{wb97xd}/ 6-31G(d,p) instead of \acs{wb97xd}/ 6-31+G(2df,p) was available for most of the common small molecules. %Where database reaction energies could be determined, 
These comparisons are detailed alongside the calculated energies in the text. }% and included in table \ref{tab:post_energies}.} %OR in the table with a big caveat, if thats the ballgame!}

\section{Results and Discussion}%%%%%%%%%%%%%%%%%%%%%%%%%%%%%%%%%%% Free energy paper with equations: Effect of nitrate content on thermal decomposition of nitrocellulose. Pourmortazavi,2009
\added{Individual molecules were optimised and their energies were calculated to evaluate the $\Delta G$ for each of the reactions in schemes \ref{sch:chinschema1}-\ref{sch:aelligschema} proposed by Chin \textit{et al.}, Camera \textit{et al.} and Aellig \textit{et al.}. }
%MAybe add that table of compariosn to the supp info. 
The protonation of \ac{EN} was inspected to determine whether the protonation site matched that of \ac{NC}, where the bridging site was the most likely to lead to hydrolytic denitration. 
This would indicate that \ac{EN} followed the same hydrolytic mechanism, and was therefore likely to share the same extended decomposition scheme initiated by hydrolysis. 
The proton was placed at each of the possible terminal and bridging sites around the nitrate, and the energy for each isomer was calculated. 
%Due to the availability of oxygen sites on the \ac{EN} molecule, the optimal site for protonation was determined for inclusion in the reaction scheme for the first stage of hydrolysis. 
Table \ref{tab:reactions} shows the protonation energies for the three different oxygen sites on \ac{EN}. Despite the terminal (syn) oxygen possessing the most thermodynamically favourable energy of protonation, inspection of the reaction geometries (figure \ref{fig:en_protonation}) shows that the bridging structure most resembles that expected for the liberation of the \ce{NO2+} group at the next step, \added{as was observed in the case of \ac{NC} (section \ref{AH_Protonation})}. 
The higher $\Delta G_{proton.}$ arises from the elongation of the \ce{O-NO2} bond that allows stabilisation of the bridging site, whilst preparing to lose the \ce{NO2+}.
%
%Though appearing less thermodynamically favourable when compared to protonation at the terminal (syn) oxygen site, the higher energy of reaction likely arises from the instability of the protonated complex. 
%The elongation of the \ce{O-NO2} bond allows stabilisation of the proton at the bridging site, but 
Further studies involving protonated \ac{EN} used the values and geometry associated with the protonated bridging site. 
%note, terminal up is right, terminal down is left
%\begin{scheme}[htp]
%\begin{table}
\begin{table}[htp]
\begin{center}
%\centering
\caption{Free energies of protonation for each oxygen site on \ac{EN}.}
\begin{tabular}{ l l *{4}{S[table-format = 2.1]}} 
\toprule
\multicolumn{2}{l}{\multirow{2}{*}{Protonated site}} & \multicolumn{4}{c}{$\Delta$G\textsubscript{proton.} /kcal mol\textsuperscript{-1}} %& \multicolumn{2}{c}{$\Delta$H\textsubscript{r} }
\\\cline{3-6}
  								& 							& \acs{wb97xd} & PCM & \acs{B3LYP} & PCM\\
\midrule
% Right is up, left is down, with ethanol in an “m” shape. 
 Terminal ({\added{syn}}) 	& \ce{CH3CH3ONO2H+} 	& -12.276810	& 8.821890 & -13.782510 & 5.625270\\
 Terminal ({\added{anti}}) 	& \ce{CH3CH3ONO2H+} 	&-9.475200 	& 9.459450	&-11.132100	&5.646060 \\ 
 Bridging 						& \ce{CH3CH3O(H+)NO2} &-9.322740	& 9.058140	& -15.309630 	&	6.673590 \\
\bottomrule
\end{tabular}
\label{tab:reactions}
\end{center}
\end{table}

%replace pics with the atom labelled ones
%\begin{figure}
\begin{figure}[htp]
\centering
\begin{subfigure}[b]{0.3\linewidth}
\centering
\includegraphics[width=\linewidth]{terminal_r_up}
 \caption{Syn terminal oxgen.\\
 Bond(O--\ce{NO2}): \num[round-mode=places,round-precision=2]{1.27787} \AA} 
 \label{fig:t_l_up}
\end{subfigure}
\begin{subfigure}[b]{0.3\linewidth}
\centering
\includegraphics[width=\linewidth]{terminal_l_down}
\caption{Anti terminal oxgen.\\
 Bond(O--\ce{NO2}): \num[round-mode=places,round-precision=2]{1.27863} \AA} 
 \label{fig:t_r_down}
\end{subfigure}
\begin{subfigure}[b]{0.3\linewidth}
\centering
\includegraphics[width=\linewidth]{bridging}
\caption{Bridging oxgen.\\
 Bond(O--\ce{NO2}): \num[round-mode=places,round-precision=2]{1.97967} \AA} 
 \label{fig:bridge}
\end{subfigure}
 \caption{Optimised geometries of the possible protonation sites on \ac{EN}.}
 \label{fig:en_protonation}
\end{figure}
%\end{scheme}

\begin{table}
\begin{center}
\caption{Calculated energies for the nitrate ester decomposition reactions proposed by Camera \textit{et al.}, Chin \textit{et al.} and Aellig \textit{et al.} \cite{Camera1982,Chin2007,Aellig2012}. R = \ce{CH3CH2} for \acf{EN}, and R = \ce{(H3CO)2C6H9O3} (bi-methoxy capped glucopyraonse monomer unit) for \ac{NC}. \added{The \ac{PCM} model applied was water ($\epsilon = 78.4$)}.}
\label{tab:post_energies}
\begin{tabular} { l *{4}{S[table-format = 2.1]}} 
\toprule
\multirow{2}{*}{Reaction}								& \multicolumn{4}{c}{$\Delta$G\textsubscript{r} /kcal mol\textsuperscript{-1}}
\\\cline{2-5}
															& \acs{wb97xd} 	& PCM 			& \acs{B3LYP} & PCM\\
\midrule
\ce{N2O4 + H2O -> HNO3 + HNO2}						&2.25162			&1.85409		&5.13135		&4.18005\\
\ce{N2O4 <=> 2{\rad}NO2}								&	0.123480		&	1.461600	&	-0.539280	&-0.155610 \\
\toprule
Radical reactions \\
\midrule
%\ce{CH3CH2ONO2 + H+ <=>[\text{fast}] CH3CH2ONO2H+} &
%\ce{CH3CH2ONO2H+ <=>[\text{slow}] CH3CH2H + NO2+} & 14.930370&-2.247210&16.012710&-4.038930\\
%[\text{fast}]
%\ce{R-ONO2 -> R-O{\rad} + {\rad}NO2 } \\ Homolysis
%\ce{2{\rad}NO2 <=> N2O4} &-0.122220&	-1.310400	&0.541800	&0.155610\\
%\ce{2{\rad}NO2 + H2O <=> N2O4 + H2O}&-0.12222&-1.4616&0.53928&0.15561\\
\ce{{\rad}NO2 + HNO -> HNO2 + {\rad}NO}			&-28.21644		&-28.66815	&-27.32688	&-27.6255\\
 \ce{2{\rad}NO + O2 -> 2{\rad}NO2} 					&-20.77047		&	-21.97314	&-21.16044	&-22.15899 \\
%\ce{2 {\rad}NO + O2 -> 2 {\rad}NO2}&-59.89473&-60.47244&-60.46866&-60.99597\\
\toprule
Acid reactions\\
\midrule
\ce{HNO3 + HNO2 <=> N2O4 + H2O} 					&	-2.24269		&-1.84673		&-5.11099		& -4.16346x\\
\ce{4HNO3 <=> 4{\rad}NO2 + 2H2O + O2}				&53.35029			&58.36446		&42.60942		&46.93563\\
\ce{2HNO -> HON=NOH}									&-38.96928		&-39.71583	&-36.62757	&-37.40814\\
\ce{HON=NOH -> N2O + H2O}							&-48.08286		&-48.18429	&-50.55309	&-50.74902\\
\toprule
Ionic reactions \\
\midrule
\ce{NO2+ + 2H2O <=> HNO3 + H3O+}					& -0.896490 		& -1.338750 	& 1.770300 	& 2.464560\\
%Literature values: 
\midrule
\ac{EN} ( R = \ce{CH3CH2} )\\
\hline
\ce{R-ONO2 + H2O -> R-OH + HNO3}					&	4.556160		&	5.235930	&	4.000500	&	4.860450 \\ %Hydrolysis
\ce{R-OH + HNO3 -> R=O + HNO2 + H2O}				&	-34.062210		&	-38.427480	&	-37.593990	&	-41.770260 \\
\ce{R-OH + {\rad}NO2 -> {\rad}R-OH + HNO2}			&	16.376220		&	13.923000	&	15.887340	&	13.699350 \\
\ce{{\rad}R-OH + HNO3 -> R=O + H2O + {\rad}NO2}	&-50.438430		&-52.35048	&-53.48133	&-55.4715\\
\ce{R-OH + HNO2 <=> R-ONO + H2O}					&-3.20544			&-3.276		&-2.64096		&-2.94903\\
\ce{R-ONO -> R=O + HNO}								&-1.49625			&-5.82183		&-4.36716		&-8.50122\\
\midrule
%include a graphic of the molecule instead of this rubbish text
\ac{NC} monomer ( R = \ce{(H3CO)2C6H9O3} )\\
\hline
\ce{R-ONO2 + H2O -> R-OH + HNO3}					&	0.67536		&	5.63094	&	0.61236	&	-0.70119 \\%Hydrolysis
%RONO2 &+&H3O+& <->&RONO2H+&+&H2O&(see protonation section)&&&-30.87756&3.5464&-31.98636&-0.24507
\ce{R-OH + HNO3 -> R=O + HNO2 + H2O} 				&	-36.72522		&	-38.34306	&	-41.71419	&	-41.70411 \\ %super unconverged?
\ce{R-OH + {\rad}NO2 -> {\rad}R-OH + HNO2}			&	14.71302		&	11.15163	&	13.03407	&	23.20983 \\
\ce{{\rad}R-OH + HNO3 -> R=O + H2O + {\rad}NO2} 	&-51.438240		&-49.494690	&-54.748260 	&	-56.369250 \\
\ce{R-OH + HNO2 <=> R-ONO + H2O}					&-4.43142			&-7.30233		&-4.30605		&-0.17829\\
\ce{R-ONO -> R=O + HNO}								&-2.93328			&	-1.71108	&-6.82227		&	-11.20581\\
\bottomrule
\end{tabular}
\end{center}
\end{table}
%
\added{The combined list of calculated energies for the reactions in all schemes, for both \ac{EN} and the \ac{NC} monomer, are listed in table \ref{tab:post_energies}.} 
\added{%
%reactions to talk about:
% Look at the std errors between all calcs!!
% can also compare and contrast individual bond lengths and angles and IR frequencies like Awasthi did too
%In case I decide to look at any NMR for the degradatino products, and need to explain how NMR values are derived compuationally; \cite{Gunko2014}. (And also teh Gaussian textbok)
The obtained energies for the formation of \ce{N2O4} from 2\rad\ce{NO2} were are out of the experimentally recorded range of 4.7 - 5.9 kcal mol$^{-1}$ \cite{Awasthi2013} (from calculation: \acs{wb97xd}/ 6-31+G(2df,p) $\Delta G_{r}$ = 0.1 kcal mol$^{-1}$ in vacuum and \mbox{1.5 kcal mol$^{-1}$} in water, and \acs{B3LYP}/ 6-31+G(2df,p), $\Delta G_{r}$ = -0.5 kcal mol$^{-1}$ in vacuum and -0.2 kcal mol$^{-1}$ in water). %,Roscoe1993}. %CCSD(T) calculated
The \acs{B3LYP} result incorrectly predicts that the reaction is spontaneous under ambient conditions. 
This error may arise from a number of factors, including the limitation to short-range interactions in \acs{B3LYP}, the general tendency of density functional methods to under predict reaction energies, or the geometry optimisation procedure, whereby a small variation or imperfect minimisation in the obtained optimised structures for the reaction species is amplified when combined for the 
% exacerbated
calculation of reaction energies. 
% good-ish

For the reaction of {\ce{HNO3 + HNO2 <=> N2O4 + H2O}} in the gas phase, \acs{wb97xd}/ \mbox{6-31+G(2df,p)} gave $\Delta G_{r}$ = {-2.2 kcal mol$^{-1}$} and \acs{B3LYP}/ \mbox{6-31+G(2df,p)} gave $\Delta G_{r}$ = {-5.1 kcal mol$^{-1}$}. 
The corresponding published theoretical reaction energies calculated at \acs{wb97xd}/ 6-31+G(d,p) and \acs{B3LYP}/ \mbox{6-31+G(d,p)} were {-1.0 kcal mol$^{-1}$} and {-4.3 kcal mol$^{-1}$} respectively. 
The literature and calculated values cannot be directly cross-referenced as the applied basis sets differ, however it can be seen that in either set there is a large relative disparity between the two functionals. 
Values are lower for \acs{B3LYP} likely as a result of neglect of mid to long-range correlations that have been included in \acs{wb97xd}; it is therefore expected for the \acs{B3LYP} result to more heavily underpredict reaction energies when these interactions are signficant in the studied system \cite{Chai2008a}.   
This is illustrated in the energies of \ce{N2O4}, where the long-range interaction of each nitrogen with oxygens of the other nitrogen group are missed, and the \acs{B3LYP}/ \mbox{6-31+G(2df,p)} energy falls below the \acs{wb97xd} energy by \mbox{89 kcal mol$^{-1}$}. }

% way off but can explain something, or at least talk about it
%It was expected that the \ac{NC} monomer species would not be readily available in literature.
%discarded ones:
\added{
Simplified schemes for the possible ageing reactions of nitrate esters beginning from homolytic fission, elimination of \ce{HNO2} or acid hydrolysis are illustrated in schemes \ref{sch:homolytic} - \ref{sch:hydrolysis}. %It was assumed that the reactions took place in a closed system with only the 
The reaction of {\ce{4HNO3 <=> 4{\rad}NO2 + 2H2O + O2}} has a large, positive $\Delta G_{r}$ of \mbox{53.4 kcal mol$^{-1}$} (\acs{wb97xd}/ 631+G(2df,p) in vacuum) indicating that this process is highly unlikely to proceed in given ambient temperatures, and so was omitted from the drawn reaction schemes. 

When starting with the products of homolytic fission (scheme \ref{sch:homolytic}), the reaction pathway split into a branched radical chain mechanism and separate acid driven pathway. }
%
\begin{scheme}[htp!]
\centering
\schemestart
\textbf{\ce{R-ONO2}}\arrow(xx--aa){->[\textbf{homolysis}]}\textbf{\ce{R - O{\rad} + {\rad}NO2}}
\arrow(aa--bb)[-25,2]Thermolyic fragmentation (equation \ref{equ:chinT2})
%\arrow(@aa--cc)[-90,2]\ce{2{\rad}NO2 + H2O <=> N2O4 + H2O}
% Keep the water in , as it makes sense for the next stage?
\arrow(@aa--cc)[-90,2]\ce{2{\rad}NO2 <=> N2O4}
\arrow(cc--dd){0}[-90,0.2]\ce{{\rad}NO2 + R-OH -> {\rad}R-OH + HNO2}
\arrow(@dd--ee)[-90,1.5]\ce{N2O4 + H2O -> HNO3 + HNO2}
\merge>(dd)(ee)--(ff)\ce{{\rad}R-OH + HNO3}
\arrow(@ff--nn)[90,1]\ce{R=O + H2O + {\rad}NO2}
%\arrow(@dd--ff)[-10,2.5]\ce{{\rad}R-OH + HNO3 -> R=O + H2O + {\rad}NO2}
%\arrow(@ee--@ff)
\arrow(@ee--gg)[-90,1]\ce{HNO2 + R-OH <=> R-ONO + H2O}
\arrow(@ee--hh)[-25,3.5]\ce{HNO3 + R-OH -> R=O + HNO2 + H2O}
%\arrow(hh--ii){0}[-90,0.2]\ce{HNO3 + HNO2 <=> N2O4 + H2O} 
\arrow(@gg--jj)[-90,1.5]\ce{R-ONO -> R=O + HNO}
%Feed into elimhono’s chain
\arrow(jj--kk)[-90,1]\ce{2HNO -> HON=NOH}
\arrow(kk--ll){0}[-90,0.2]\ce{HNO + {\rad}NO2 -> HNO2 + {\rad}NO}
% The {\rad}NO can also then combine with the R-O{\rad} of the first step too
\arrow(ll--oo)[-90,1]\ce{2 {\rad}NO + O2 -> 2 {\rad}NO2}
%\arrow(@kk--mm)\ce{HON=NOH ->N2O + H2O}
\arrow(@kk--mm)\ce{N2O + H2O}
\schemestop 
\caption[Proposed degradation pathway starting from the homolytic fission of the nitrate ester.]{Proposed degradation pathway starting from the homolytic fission of the nitrate ester, derived from the schemes presented by Camera \textit{et al.} \cite{Camera1982} and Aellig\cite{Aellig2012}.}
\label{sch:homolytic}
\end{scheme}%
%the propagation reactions are dominated by radical interactions. 
\added{
Reaction of \rad\ce{NO2} with a hydroxyl group (on either the \ac{NC} monomer or denitrated \ac{EN} molecule) leads to the formation of an alkyl radical which goes on to react with \ce{HNO3} present in the environment from formation \textit{via} other steps in the pathway, formed during the reaction of \ce{NO2+} in the hydrolytic scheme, or residual in the system from incomplete washing following \ac{NC} synthesis, to regenerate \rad\ce{NO2}. 
The \rad\ce{NO2} attack on \ce{R-OH} is expected to occur rapidly, dominating the reaction scheme but leading to increased concentration of \rad\ce{NO2} in the system to fuel other pathways. 
Formation of \ce{N2O4} from 2\rad\ce{NO2} and subsequent decomposition into \ce{HNO2} and \ce{HNO3} drives the remaining acid reactions. 
\ce{HNO3} reactions lead to further formation of \rad\ce{NO2}, however, the attack of \ce{HNO2} on hydroxyl groups on the \ac{NC} backbone generates the experimentally observed end product \ce{N2O} whilst the hydroxyl group is converted to a ketone \cite{Dauerman1968,Adams1949}. }

\added{%The action of \ce{HNO2} 
The generation and re-generation of \rad\ce{NO2} and \ce{HNO2} in the above scheme supports the theory that these may be the species responsible for the autocatalytic rate of decomposition that is observed following a first-order rate induction period \cite{Rodger1963,Lindblom2002,Volltrauer1981}. 
The first-order rate is likely attributed to the denitration step, as the concentration of these species slowly increases and self-heating occurs, leading to the rapid speed-up of the post-denitration reactions.  
If the liberated \ce{HNO2} does not go on to attack hydroxyl groups on the \ac{NC} backbone, it is then free to undergo a conversion to generate further \rad\ce{NO2} in the reactions \ce{HNO2 + HNO3 -> N2O4 + H2O  <=> 2{\rad}NO2} +H2O}, 
%\begin{equation}
%\added{\ce{HNO2 + HNO3 -> N2O4 + H2O}}\\
%\added{\ce{N2O4 <=> 2{\rad}NO2}}
%\end{equation}
\added{indicating that the fate of all nitrogen in the system is towards the formation of \rad\ce{NO2} until it is captured in \ce{N2O}. 
This is supported by scheme \ref{sch:elimination} that begins with the elimination of \ce{HNO2}. The reactions of \ce{HNO2} with available hydroxyl groups drives the formation of \ce{N2O}, ketone and water as the only end products.
 The hydrolytic scheme (scheme \ref {sch:hydrolysis}) involves the early generation of \ce{HNO3}, conversion to \ce{HNO2} and then enters a similar radical chain mechanism to that observed in the homolytic pathway concluding in formation of \ce{N2O}, \rad\ce{NO2} and hydroxyl conversion to ketone. }
 \added{
For all schemes, \ce{R=O} and \ce{N2O} were terminating species. Whilst \ce{N2O} is released into the environment, \ce{R=O} remains in the system may go on to participate in further decomposition reactions leading to ring opening. %outside the scope of the proposed reactions. %why are they terminating species? - they do not go on to react and regenerate HNO2 and .NO2?
It is important to note that these schemes are not exhaustive, but rather a limited view of the possible steps that may take place during ageing based on the presence of nitrate and acid species expected to be in the closed system. Deeper degradation of the \ac{NC} backbone has not been considered here and should be the next step for future work. }
 %
 \begin{scheme}[hb]
\centering
\schemestart
\textbf{\ce{R-ONO2}}\arrow(xx--aa){->[\textbf{Elimination}]}[,1.5]\textbf{\ce{R=O + HNO2}}
\arrow(@aa--bb)[-140,2]\ce{HNO2 + R-OH <=> R-ONO + H2O}
\arrow(bb--cc)[-90]\ce{R-ONO -> R=O + HNO}
\arrow(cc--dd)[-90]\ce{2HNO -> HON=NOH}
%\arrow(dd--gg){0}[-90,0.2]\ce{HNO + {\rad}NO2 -> HNO2 + {\rad}NO}
\arrow(dd--ee)[-90]\ce{HON=NOH -> N2O + H2O}
%\arrow(@aa--ff)[-35,2]\ce{R=O}
%\arrow(@gg--hh)\ce{2 {\rad}NO + O2 -> 2 {\rad}NO2}
\schemestop 
\caption{Proposed degradation pathway starting from the elimination of \ce{HNO2} from a nitrate ester, derived from the schemes presented by Camera \textit{et al.} \cite{Camera1982} and Aellig\cite{Aellig2012}.}
\label{sch:elimination}
\end{scheme}
%
\begin{scheme}[hbp!]
\centering
\schemestart
\textbf{\ce{R-ONO2 + H+}}\arrow(xx--aa){->[\textbf{Hydrolysis}]}[,1.5]\textbf{\ce{R-OH + NO2+}}
\arrow(@aa--bb)[-150,2]\ce{NO2+ + 2H2O <=> HNO3 + H3O+}
%\merge>(@aa--@bb)
%\arrow(@aa--cc)[-30,2]\ce{R-OH + HNO3 -> R=O + HNO2 + H2O}
\arrow(@bb--cc)[-30,2]\ce{HNO3 + R-OH -> R=O + HNO2 + H2O}
\arrow(@aa--@cc)[-90]
\arrow(@cc--dd)[-90]\ce{HNO2 + HNO3 <=> N2O4 + H2O} 
\arrow(@dd--ee){0}[-90,0.2]\ce{HNO2 + R-OH <=> R-ONO + H2O}
%\arrow(@dd--ff)[180]\ce{N2O4 + H2O -> HNO3 + HNO2}
%\arrow(ff--gg){0}[-90,0.2]\ce{N2O4 <=> 2{\rad}NO2}
\arrow(@dd--gg)[180]\ce{N2O4 <=> 2{\rad}NO2}
\arrow(@ee--hh)[-90]\ce{R-ONO -> R=O + HNO}
\arrow(hh--ii)[-90]\ce{HNO + {\rad}NO2 -> HNO2 + {\rad}NO}
\arrow(ii--kk){0}[-90,0.2]\ce{2HNO -> HON=NOH}
\arrow(@gg--ii)[-50,3.3]
\arrow(@kk--jj)[-90]\ce{HON=NOH -> N2O + H2O}
\arrow(@ii--ll)[180,1]\ce{2 {\rad}NO + O2}
\arrow(@ll--mm)[180,0.5]\ce{2{\rad}NO2}
\schemestop 
\caption{Proposed degradation pathway starting from the acid hydrolysis of a nitrate ester, derived from the schemes presented by Camera \textit{et al.} \cite{Camera1982} and Aellig\cite{Aellig2012}.}
\label{sch:hydrolysis}
\end{scheme}
%
%\textcolor{red}{Still to mention:\\
% - Describe the other two schemes\\
% - Describe the energies in the table - what are the experimental energies for them, and how do my values compare?\\
% - Discuss why some of the values may be positive. \\
% - Include enthalpies of reaction, zero point energies, and any experimental proxies I can find for the reaction enthalpies too. \\
% NOTE: ZPE energy correction means that you REMOVE the ZPE, so that you only compare the actual energy available for the reaction. 
% -  Look a little more into (pics of the species / reactions, etc on a few specific funny cases, and explain what you see - ie. different resulst for different funcitonals. - explain why
% - }
%It is noted that oxidation of any nitrate is never via 
 %Need ZPE corrected values etc
 
%
%\subsection{Thermodynamics of \ac{EN} reactions}%%%%%%%%%%%%%%%%%%%%%%%%%
%
%For an initial comparison of the methods you used, you could do a diagram like the one Kuklja did (kuja2014.pdf, page 89, fig 3.7).
%She also makes mention of the overestimation of activation barriers for pure DFT methods. Make sure you know the background surrounding this - why does this occur, and what is done to remedy it?
 %
%\begin{figure}[h]
%  \centering
%	\includegraphics[width=0.8\linewidth]{name}
%\caption{\cite{}.}
%\end{figure}
%$\Delta$-G\textsubscript{r}
%$\textDelta$H\textsubscript{r}
 %
%The \ce{CO---NO2} bond lengths are (a) \num[round-mode=places,round-precision=4]{1.27787} \AA (b) \num[round-mode=places,round-precision=4]{1.27863} \AA (c) \num[round-mode=places,round-precision=4]{1.97967} \AA
%Comment on the diff between wb97xd and B3LYP.%
%
%something about why the energy of the HNO3 reaction looks so rubbish (but that it stilll degrades at room temp, appaz. But check the literature)
%For the decomposition of \ce{HNO3} to \ce{{.}NO2},  \ce{2H2O} and \ce{O2}, Aellig prescribes the use of an amberlyst catalyst (amberlyst-15).
%\cite{Ellis2007,Robertson1955}
%The propagation reactions are acid catalysed by \ce{HNO2}. 
%
%\subsubsection{Radical mechanistic route}
%
%\subsubsection{Ionic mechanistic route}
%
%\subsection{Reactions of Nitrocellulose Monomer}
%\section{ \textit{(Kinetics of \ac{EN})}}
%\subsection{Radical mechanistic route}
%\subsection{Ionic mechanistic route}
%
%\section{ \textit{(Kinetics of Nitrocellulose Monomer)}}
%
%Insights into the reaction energies
%“HNO cannot be stored or concentrated and is typically studied using donor species that release HNO as a decomposition product.” (paper since retracted)
%
% Comparison of reaction energies for \ac{EN} and the monomer shows that most processes are more thermodynamically favourable in the case of \ac{NC}. % explain why. Something to do with teh groups / removal of nitrate / existence as acohol is most favourable? (i think i saw that in a paper)
% %Reaction energies for hydrolysis, 
% %Only the reaction energies for 
%
\section{Summary}
\added{
In this section, the energies ($\Delta G_{r}$) for each of the reactions in the combined nitrate ester decomposition schemes proposed by Camera \textit{et al.}, Chin \textit{et al.} and Aellig \textit{et al.} were evaluated using the \acs{wb97xd} and \acs{B3LYP} functionals \cite{Camera1982,Chin2007,Aellig2012}. 
Obtained energy values were compared to literature where available \cite{NCCCBD2020}, confirming that individual molecular free energy and reaction energy values fell within the expected theoretical and experimental bounds. 
Decomposition reactions involving an alkyl nitrate were initially performed with \acf{EN} and then repeated for \ac{NC}. }
%It as found that except in the case of nitric acid decomposition, \ce{4HNO3 <=> 4NO2 + 2H2O + O2}, all reactions were feasible at atmospheric conditions o. 
%
%The proposed reactions following the denitration of \ac{NC} were validated by calculating energies of reaction. % The reaction energies were contrasted 
%The reactions were used to create rough schemes of the possible routes degradation routes, with increasing number of the \ac{NC} decay. 

\added{Starting from the three denitration mechanisms explored in chapter \ref{chapterlabel5}, using the reactions listed in the schemes above, three different degradation pathways were constructed. 
It can be seen that it is the \ce{HNO2} species conversion of \ce{R-OH} $\rightarrow$ \ce{R-ONO} $\rightarrow$ \ce{R=O} that drives oxidation in all three schemes, playing a central role in the extended degradation scheme of \ac{NC}. The regeneration of \rad\ce{NO2} indicated that this was the species most likely responsible for the autocatalytic rate of degradation observed experimentally \cite{Binke1999,Wang2015,Chai2019}. }

\added{Decomposition further than formation of the ketone species was not studied here. To understand the full ageing behaviour of \ac{NC} and comprehensively attribute experimentally observed products to individual reactions, the relative rate of each of the mechanistic schemes above, in addition to ring fission and peeling-off reactions that disrupt the \ac{NC} chain, must be probed. 
A %non-trivial
limitation to this work is that the reaction energies were calculated from the molecules optimised in isolation. In practice, there are energy barriers associated with complexation of reactant compounds and solvation energies as newly liberated species depart from the \ac{NC} backbone to freely move in the solvent. To account for these, a more detailed study to evaluate complexation and solvation energies is needed. The diffusion behaviour of individual reactant species through the \ac{NC} bulk both in vacuum and solvent, and as the reaction mixtures evolves, should also be considered. }
%
%A disadvantage is that \ce{N2O} is a significant greenhouse gas, and cannot be re-converted back to \ce{HNO3}, so much be considered win experimental design when observing industrial and environmental impact. 