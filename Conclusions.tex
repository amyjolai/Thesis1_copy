\chapter{Conclusion and future work}
\label{chapterlabel7}
\graphicspath{ {./Conclusion_pics/} }

\section{Conclusion}
\added{Since winning a medal at the World's Fair in London in 1862 as the first man-made plastic \cite{Rossell2002,Parkes1862}, \ac{NC} has become a central component in the manufacturing of everyday items across a whole spectrum of critical applications, from kitchenware to rocket fuel \cite{Greenberg1925}. 
As has been demonstrated by this study and the extensive years of research since its first discovery, the degradation behaviour of \ac{NC} is multi-staged and subject to high variation. The interplay of thermal triggers, hydrolytic initiation, as well as other factors not discussed here such as \ac{UV} initiation \cite{Berthumeyrie2014,Khryachkov2017} and physical shock \cite{Taylor1971}, are all subject to the unique composition of each sample; variation owed to its biological origins and preparation method. Adding to this, the ageing reactions that occur are at the mercy of the precise environmental conditions under which the \ac{NC} is stored. The final result is that the true, exhaustive reaction scheme for full decomposition has remained elusive. }% and efficacy of washing procedures.

In this thesis, the degradation processes in \ac{NC} were explored using computational methods to elucidate the dominant processes and key reactants involved in ambient ageing. 
In the first section, the polymeric structure of \ac{NC} was introduced. Different sized truncations and capping group approximations for polymer chain endings were tested as models for the polysaccharide. 
This was achieved by inspection of the partial charges, \ac{ESP} and critical interaction points for monomeric, dimeric and trimeric \textbeta-glucopyranose structures. 
The dimer was found to be the minimum structure required to reproduce the full properties of \ac{NC} within a repeat unit. 

Methoxy and hydroxyl capping groups were compared; the methoxy groups provided a more sterically and chemically similar proxy for the extended polymer, following examination of charges %\ac{ESP}
and %the geometry profiles.
geometry dependent interactions.  
Comparison of the charge densities and intramolecular interactions around the monomer and dimer revealed that the former exhibited an acceptable level of 
%discrepancy 
deviation from the dimer behaviour%properties
, particularly with reference to further investigations concentrating only on localised reaction interactions. 
The bi-methoxy monomer was implemented as the model for \ac{NC} in later studies. %syn
%Hydroxyl and methoxyl capping group ends were also compared, finding that the methoxyl groups provided a more similar charge and geometry profile to the extended polymer than the small hydroxyl groups. 

Using the monomer model, the primary steps of decomposition were explored in Chapter \ref{chapterlabel5}. Thermolytic denitration reactions were investigated; homolytic fission of the nitrate \ce{O-NO2} bond, and elimination of \ce{HNO2} were tested for both the PETN test case, and the \ac{NC} monomer model. 
%Whilst the reaction energy for homolysis did not agree with the computational values in the 
Good agreement with literature values was found for the reaction energies and activation energies, in case of \ce{HNO2} elimination in both \ac{PETN} and \ac{NC}. The loss of \rad\ce{NO2} \textit{via} homolysis was confirmed. %And in the HNO2 case for the monomer?
For the acid hydrolysis pathway %facilitated denitration
, possible protonation sites in the monomer were analysed. It was found that the proton 
%sat most favourably at the bridging oxygen site of the nitrate. % for the monomer.
site most amenable to denitration was the bridging oxygen position of the nitrate. 
%Inspection of the optimised geometry showed that it very closely resembled the same geometry of the capping group site. 
Further investigations considered denitration routes beginning from isomers protonated at both the terminal (upper) and bridging sites. %why - energy diff? means it will spend less time there, but would still happen
The denitration step was then explored \textit{via} a series of \ac{PES} scans. %probing 
The stability of different possible \ac{TS} ring structures involving both pre-protonation and concerted protonation-denitration was examined, in addition to the 
%identity
nature of the \ce{NO2} leaving group. 
No stable \ac{TS} structures presenting the correct vibration for denitration were isolated, however scans confirmed that the \ce{NO2} was released as \ce{NO2+}, with possible formation of \ce{HNO2} at greater separations. 
%It is likely that the formation of \ce{HNO2} and stabilisation of the \ac{TS} requires an explicit solvent shell around the monomer. 

Proposed decomposition routes originating from the primary denitration step were collated from nitrate ester reactions in literature. Using \ac{EN} as an initial test case, the energies of each reaction were evaluated to determine whether it were a viable %step in the extended network of secondary reactions 
secondary reaction step following liberation of the \rad\ce{NO2}, \ce{NO2+} or \ce{HNO2} following first stage decomposition. Possible decomposition schemes were constructed, mapping from the point of \ce{NO2} liberation to the oxidation of the alcohol group. 
% on the sugar ring to a ketone. 
The reaction energies were determined for the \ac{NC} monomer. It was found that the energies were largely favourable from a thermodynamic equilibrium perspective. 
%Except in the case of Hno2 formation - 
The fate of the released nitrogen species was in the accumulation of \ce{N2O} or regeneration of \rad\ce{NO2}, suggesting \rad\ce{NO2} as the species responsible for autocatalytic processes in the system. Consumption of \rad\ce{NO2} in the formation of acids proved to be thermodynamically unfavourable.%, as was \ce{HNO2} formation. 
\ce{HNO2} routes lead to the formation of \ce{N2O} without self-regeneration and \ce{HNO3} routes lead primarily to formation of \rad\ce{NO2}. This indicates that \ce{HNO2} was unlikely to be a direct contributor to catalysis, and that \ce{HNO3} was the precursor to the \rad\ce{NO2} catalytic species, acknowledging experimental observations that \ce{HNO3} appeared to facilitate autocatalysis \cite{Baker1952}. 

\added{Following the successful application of the monomer model in the investigation of denitration and protonation reactions of \ac{NC}, further studies repeating the mechanisms explored here using the dimer (and trimer models, where feasible) would be extremely valuable. Possible synergistic effects of neighbouring nitrate groups on adjacent rings, in addition to increased steric factors, are likely to change the energetics and favourability of attack and protonation sites, thus altering the likelihood of denitration at each ring position. 
Inter-chain hydrogen bonding, largely dependent on unsubstituted hydroxyl groups, is also likely to alter the contribution to hydrolysis in particular. Porosity to solvents is influenced by local crystallinity and packing \cite{French1978,Clark1982}, and is sensitive to the spatial arrangements of individual hydroxyl groups \cite{Zhbankov1992,Akerholm2004,Nishiyama2002}. Molecular packing will determine the ease at which liberated degradation products can diffuse through and escape the \ac{NC} matrix. It can be surmised that there exists certain decomposition pathways with corresponding reactions that are solely structure dependent. }
%Things I haven't mentioned:
% Modelling the protonation step and rates (Lure1991), and its contribution to rate of hydrolysis
% The resulting interplay of reaction rates of homofiss, hydro and elimhono, and under diff concs of water / acid / temps
% Possible QM/MM with waters

% The affect of changing reaction mixture as ageing proceeds in a closed system `If denitration proceeds via protonation of the nitrate ester and then ionization, to give the nitronium ion, then as the 'polarity' of the nitrating mix decreases this mechanism will become less favourable.' Short1996
% Quantify effect of diff storage environs

Whilst this work has not exhaustively explored the myriad reactions that may occur in the complex ageing procedures of \ac{NC}, it has established the key reactions %shed light 
of the early stages of degradation, with presentation of an effective %approximation 
truncation of the polymeric structure %suitable 
applicable for further study in the topic. Key competing reactions for the denitration step, the identity of nitrogen species released and their role in the longer-range decomposition process has been presented. 
% brought to attention. %This project takes pause / set the stage for 
The conclusion of this project sets the scope for subsequent investigations into the later-stage 
%secondary
reaction processes that lead to deeper degradation of the \ac{NC} backbone.

%%%%%%%%%%%%%%%%%%%%%%%%%%%%%%%%%%%%%%%%%%%%%%%%%%%%%%%%%%%%%%%%%%%%%%%%%
\section{Further Work}
\added{In addition to expansion of the \ac{NC} model to larger dimer and trimer units, the }existing \ac{NC} model may be refined by conducting a more rigorous examination of the %partial charges and the 
subtle variations in geometry. 
%This includes explicit calculation of charges, perhaps using \acs{NBO} methods \cite{Xie2012,Santos-Carballal2013} or \acs{RESP} charges \cite{Wang2000,Woods2000,Dupradeau2010}, as has been applied to other saccharide and hetercyclic organic compounds. 
%dodgy
Conformational scans, in particular for the C6 chain and for the orientation of the %floppy side chains and branches of 
rings within the trimer, would benefit identification of other low energy structures likely to be present in the natural polymer.
Here, only the denitration schemes for the singly nitrated \ac{NC} monomer were documented. 
The differing stabilities of \ac{NC} at varying levels of nitration will undoubtedly affect the reactivity at each site. 
Propagation of different nitration level and conformational structures through the 
%prescribed
denitration and secondary decomposition schemes may reveal alternative reactions, or alter the balance of products obtained. 

Classical \ac{MD} techniques would also provide further insight into the diffusion of the released products, and their interaction with the wider polymer. Studies involving the interaction of \ac{NC} with plasticisers has effectively probed the diffusion rates of plasticiser migration, which is of key interest in the preservation of stable \ac{NC} product formulations \cite{Richards2018}. 
%but I didn't look at the ratio. I mean product interactions?

Another avenue of interest is in the exploration of other transition structures for the denitration stage, and for further degradation following formation of the ketone. 
The inclusion of additional explicit water molecules or water clusters is likely required to stabilise \ac{TS} that were previously \added{not viable (chapter \ref{chapterlabel5} figure \ref{fig:all_da_TS}), highlighting the need for further understanding of \ac{NC}-solvent interactions. A starting reference work would be the modelling of pentahydrate complexes around glucose by Momany \textit{et al.} \cite{Momany2004}, with extension to a hybrid quantum mechanics/molecular mechanics (\acs{QM}/\acs{MM}) approach to treatment of solvation shells. }
% starting with reproduction of the primary solvent shell  \cite{Momany2005}. %cite water cluster glucose work 
Ab-initio molecular dynamics (\acs{AIMD}) techniques may be effective for investigation into the interactions of both water and acids with \ac{NC}, offering possible insight into the effect of increasing acid concentration on the protonation behaviour and water clustering around monomer, dimer and trimer structures at different \ac{DOS} \cite{Ardura2009}. 
%A cyclic transition state involving the proton located on the capping group should be probed in the possibility that it leads to a facile route to the peeling off individual glucopyraonse units, perhaps even preserving the nitrate group. 
%
%- Rates of each reaction (homolysis, vs hno2 vs acid)
%- explore the effect of water coordination around NC
%- explore mechanism of OH retardation of hydrolysis (?) rate, for teritary nitrates. And why this doesn't happen for primary / secondary (does the ph need to really be that low for reactions?)


A natural extension to the study of the secondary reactions driving decomposition is the expansion to a wider range of possible reaction pathways. These may include the widely documented mechanisms studied for glucose, \added{such as the conversion to hydroxymethyfurfural (\acs{HMF}), whereby the 6-membered ring is converted to a 5-membered ring} (figure \ref{fig:glucose_hmf}) \cite{Qian2005a,Qian2005,Qian2010}\added{; furan and other aromatic species have been observed in \ac{NC} degradation residues \cite{Jensen2014}}.  
This is in addition to further studies of possible ring opening mechanisms and chain scission reactions, in order to fully account for the broad spectrum of experimentally observed degradation products in \ac{IR} and \ac{NMR} measurements \cite{Huwei1988,Clark1981,Dauerman1968,Kovalenko1994,Wu1980}.  
\begin{figure}[htb!]
\centering
\caption{The conversion of glucose to \ac{HMF} with a) showing the proposed reaction  scheme, and b) displaying a possible mechanistic pathway, from the \ac{AIMD} study by Qian \textit{et al.} \cite{Qian2011}. }
\par\bigskip
%\caption{The conversion of glucose to \ac{HMF} with \ref{fig:hmf}) showing the proposed reaction  scheme, and  \ref{fig:hmf_mech}) displaying a possible mechanistic pathway, from the \ac{AIMD} study by Qian \textit{et al.} \cite{Qian2011}. }
\begin{subfigure}[b]{\linewidth}
\centering
\caption{Conversion of glucose (I) to \ac{HMF} (III) \textit{via} a furan aldehyde intermediate (II).}
\includegraphics[width=\linewidth]{conv_glucos_furan}
\label{fig:hmf}
\par\bigskip
\end{subfigure}\\

%
\begin{subfigure}[b]{\linewidth}
\centering
\par\bigskip
\caption{(1): Protonation of \ce{C{2}-OH} on \textbeta-D-glucose, (2): breakage of the \ce{C{2}-O{2}} Bond, (3): the formation of the \ce{C{2}-O{5}} bond during glucose conversion to \ac{HMF}.}
\includegraphics[width=0.4\linewidth]{glucos_mech}
%\caption{Protonation of C2–OH on \textbeta-d-glucose (CV3), the subsequent breakage of the \ce{C2–O2} Bond (CV2), and the formation of the C2–O5 bond during glucose conversion to \acs{HMF} (CV1).}
\label{fig:hmf_mech}
\end{subfigure}
\label{fig:glucose_hmf}
\end{figure}
%