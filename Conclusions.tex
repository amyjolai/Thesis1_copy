\chapter{Conclusion and future work}
\label{chapterlabel7}
\graphicspath{ {./Conclusion_pics/} }

\section{Conclusion}
\section{Future Work}

Other transition structures including more waters or water clusters, and cyclic transition states involving the proton located on the capping group, possibly leading to the peeling off reaction. 

A natural extension to the study of secondary reactions driving decomposition is the expansion to a wider range of possible reactions, including the mechansims studied for glucose.  
\begin{figure}[htp]
\centering
\begin{subfigure}[b][0.4\linewidth]
\centering
\includegraphics{conv_glucos_furan}
\caption{Conversion of glucose A to HMF C via a furan aldehyde intermediate B.}
\label{fig:hmf}
\end{subfigure}
\hfill
\begin{subfigure}[b][0.4\linewidth]
\centering
\includegraphics{glucos_mech}
\caption{The protonation of C2–OH on \textbeta-d-glucose (CV3), the subsequent breakage of the C2–O2 Bond (CV2), and the formation of the C2–O5 bond during glucose conversion to \ac{HMF} (CV1).}
\label{fig:hmf_mech}
\end{subfigure}
\caption{The conversion of glucose to \ac{HMF} (hydroxymethylfurfurl) with \ref{fig:hmf} showing the proposed reaction scheme, and  \ref{fig:hmf_mech} displaying a possible mechanistic pathway, from the \textit{ab initio} \ac{MD} study by Qian \cite{Qian2011}.}
\label{fig:glucose_hmf}
\end{figure}


