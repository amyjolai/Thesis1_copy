\addcontentsline{toc}{chapter}{Appendices}
\graphicspath{ {./Appendicies_pics/} }
% The \appendix command resets the chapter counter, and changes the chapter numbering scheme to capital letters.

%Split appendicies by theme. E.g. One section is extra data, one section is working scripts, final one is notes and details about the working details of the calculations I did. 

\appendix

%
%\chapter{Physical constants}
%\begin{table}[htp]
%\centering

\chapter{Supplementary Data}
\label{appendixlabel1}
\section{Energies of \ac{NC} species with different capping groups}
Extra data tables etc.

\section{Chapter 4}
\label{AH_Protonation_QTAIM}
\begin{figure}[htp]
\centering
\begin{subfigure}[t]{0.4\linewidth}
\centering
\includegraphics[width=\linewidth]{corr_s_H_terminal_up}
%\caption{Terminal(Upper) optimised with \ac{wb97xd}/6-31+G(2df,p) with implicit solvent.  }
%\label{fig:proton_site_terminal}
\end{subfigure}
\begin{subfigure}[t]{0.4\linewidth}
\centering
\includegraphics[width=\linewidth]{B3LYP_s_C2_NC3-CH3_corr-OH_up-hi}
%\caption{Terminal(Upper) optimised with \ac{B3LYP}/6-31+G(2df,p) with implicit solvent.}
%\label{fig:proton_site_cap}
\caption{\ac{CP} analysis of \ac{NC} monomers protonated at the terminal oxygen (upper) site, with \ac{PCM} implicit solvent.}
\label{fig:terminal_upper_same}
\end{subfigure}
\begin{subfigure}[t]{0.4\linewidth}
\centering
\includegraphics[width=\linewidth]{s_C2_NC3-CH3_corr-OH_up-hi-CP}
\caption{Terminal(Upper) optimised with \ac{wb97xd}/6-31+G(2df,p) with implicit solvent.  }
%\label{fig:proton_site_terminal}
\end{subfigure}
\begin{subfigure}[t]{0.4\linewidth}
\centering
\includegraphics[width=\linewidth]{B3LYP_s_C2_NC3-CH3_corr-OH_up-hi-CP}
\caption{Terminal(Upper) optimised with \ac{B3LYP}/6-31+G(2df,p) with implicit solvent.}
%\label{fig:proton_site_cap}
\end{subfigure}
\caption{\{CP} analysis of \ac{NC} monomers protonated at the terminal oxygen (upper) site, with \ac{PCM} implicit solvent.}
\label{fig:terminal_upper_same}
\end{figure}
Figure \ref{fig:terminal_upper_same} shows the \ac{QTAIM} \ac{CP} analysis for the \ac{NC} monomer nitrated at the C2 position, protonated at the terminal (upper) oxygen of the nitrate group. Though the two different functionals produced significantly different values for reaction energy (\ac{wb97xd} 12.79 kcal mol$^{-1}$ and 3.72 kcal mol$^{-1}$) 
the difference was not reflected in the geometry of the optimised structures or \ac{CP} analysis, which was extremely similar for both. 

\chapter{Working scripts}
%\chapter{Appendix}
\label{appendixlabel2}
Analysis \& working scripts.
(Optional?)
%Just literally copypasta.

\chapter{Computational architecture}
\label{appendixlabel3}
%\textit{This is a description of the tools you used to make your thesis. It helps people make future documents, reminds you, and looks good.}

(Tabulate some details about the architecture I used.)

Slater (now out of commission)
Legion (now out of commission)
Grace
Huygens 1 (now out of commission)
Hugyens 2
Myriad
Hawk

% \textit{(example)} This document was set in the Times Roman typeface using \LaTeX\ and Bib\TeX , composed with a text editor. 
 % description of document, e.g. type faces, TeX used, TeXmaker, packages and things used for figures. Like a computational details section.
% e.g. http://tex.stackexchange.com/questions/63468/what-is-best-way-to-mention-that-a-document-has-been-typeset-with-tex#63503

% Side note:
%http://tex.stackexchange.com/questions/1319/showcase-of-beautiful-typography-done-in-tex-friends

%INCLUDE any little tips and notes for the next person - the thing about formchk between diff versions of Gaussian.
% - The thing about tweaking the last line in an output to make it work and visalise. 
